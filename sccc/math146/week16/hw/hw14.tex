% no answer key
\documentclass[letterpaper]{exam}

% answer key
% \documentclass[letterpaper, landscape]{exam}
% \usepackage{2in1, lscape} 
% \printanswers

% for the cent symbol
\usepackage{textcomp}

% the textcent command eats the space following the symbol
\usepackage{xspace}
\newcommand{\cent}{\textcent\xspace}

\usepackage{units} 
\usepackage{xfrac} 
\usepackage[fleqn]{amsmath}
\usepackage{cancel}
\usepackage{float}
\usepackage{mdwlist}
\usepackage{booktabs}
\usepackage{cancel}
\usepackage{polynom}
\usepackage{caption}
\usepackage{fullpage}
\usepackage{comment}
\usepackage{enumerate}
\usepackage{graphicx}
\usepackage{parskip}

\everymath{\displaystyle}


\title{Statistics \\ Homework Fourteen}
\date{\today}
\author{}

\begin{document}

  \maketitle

  \section{Homework}
  Chapter 15: 34-38, 41-43, 45-46

  \ifprintanswers
    \begin{description}

      \item[34] 
        \begin{enumerate}[(a)]
          \item 
            \[
              137 \pm 2.576 \cdot \frac{65}{\sqrt{269}} \approx 137 \pm 10.2
            \]
            The 99\% confidence interval is from \fbox{126.8 to 147.2}.

          \item The students selected must be a simple random sample.

        \end{enumerate}

  \end{description}

  \else
    \vspace{12 cm}
    \begin{quote}
      \begin{em}
        Everything goes, everything returns; eternally rolls the wheel of
        being.
      \end{em}
    \end{quote}
    \hspace{1 cm}--Frederich Nietzsche
  \fi

\end{document}

