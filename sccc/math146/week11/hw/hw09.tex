% no answer key
% \documentclass[letterpaper]{exam}

% answer key
\documentclass[letterpaper, landscape]{exam}
\usepackage{2in1, lscape} 
\printanswers

\usepackage{units} 
\usepackage{xfrac} 
\usepackage[fleqn]{amsmath}
\usepackage{cancel}
\usepackage{float}
\usepackage{mdwlist}
\usepackage{booktabs}
\usepackage{cancel}
\usepackage{polynom}
\usepackage{caption}
\usepackage{fullpage}
\usepackage{comment}
\usepackage{enumerate}
\usepackage{graphicx}
\usepackage{parskip}

\everymath{\displaystyle}


\title{Statistics \\ Homework Nine}
\date{\today}
\author{}

\begin{document}

  \maketitle

  \section{Homework}
    \begin{itemize*}
      \item read Chapter 10 
      \item take a look at the ``Check Your Skills'' exercises
      \item exercises: TO DO
    \end{itemize*}

  \ifprintanswers
    \begin{description}

      \item[31] 
        \begin{enumerate}[(a)]
          \item If you are recording the sequence, the order matters.  If 1 is a
            make and 0 a miss, the sample space would be 
            
            \{ 0000, 0001, 0010, 0011, 0100, 0101, 0110, 0111, 1000, 1001, 1010,
               1011, 1100, 1101, 1110, 1111 \}

          \item If you don't care about the order, the sample space is just the
            number of shots made, or {0, 1, 2, 3, 4 }

        \end{enumerate}

      \item[32]
        \begin{enumerate}[(a)]
          \item The probabilities are all less than 1 and add up to 1, so they
            satisfy the rules of probability.  They aren't the correct
            probabilities for an actual die, however.

          \item The probabilities are all less than 1 and add up to 1, so they
            satisfy the rules of probability.  They aren't the correct
            probabilities for an actual card deck, however.

          \item These probabilities don't add up to exactly 1, so they don't
            satisfy the rules of probability.  
        \end{enumerate}

      \item[33]
        \begin{enumerate}[(a)]
          \item This is the only probability not provided, so it is:
            \[
              1 - 0.13 - 0.29 - 0.30 = \boxed{ 0.28 }
            \]

          \item This is everybody except the people that didn't complete high
            school:
            \[
              1 - 0.13 = \boxed{ 0.87 }
            \]
        \end{enumerate}

      \item[34]
        \begin{enumerate}[(a)]
          \item $\frac{4,176,000}{9,094,000} \approx 0.46$
          \item $1 - \frac{4,176,000}{9,094,000} \approx 0.54$
        \end{enumerate}

      \item[35]
        \begin{enumerate}[(a)]
          \item The probabilities are mutually exclusive and add up to 1.
          \item $1 - 0.59 = \boxed{ 0.41 }$
          \item $0.26 + 0.09 + 0.03 = \boxed{ 0.38 }$
        \end{enumerate}

      \item[36]
        \begin{enumerate}[(a)]

          \item The probabilities listed add up to 0.9, so the probability of
            any other color is \boxed{ 0.1 }.

          \item The probability of silver or white is: $0.19 + 0.18 = 0.37$. The
            probability of any other color is: $1 - 0.37 = \boxed{ 0.63 }$.

        \end{enumerate}

      \item[37]
        \begin{enumerate}[(a)]
          \item $3/7 \approx \boxed{ 0.43 } $
          \item $2/7 \approx \boxed{ 0.29 } $
          \item $5/7 \approx \boxed{ 0.71 } $
        \end{enumerate}

      \item[38]
        For an unloaded die, the probability for any face coming up is:
        \[
          1/6 \approx 0.17
        \]

        The probability of something other than a 1 or a 6 is:
        \[
          4/6 \approx 0.6667
        \]

        Since this probability is unaffected, the probability of getting a 1 for
        the loaded die is:
        \[
          1 - 0.2 - 0.6667 \approx 0.1333
        \]

        The probabilities for all the numbers are:
        \begin{itemize*}
          \item 1: 0.1333
          \item 2, 3, 4, 5: 0.1667
          \item 6: 0.2
        \end{itemize*}

      \item[39]
        With 48 men and 42 women, there are 90 people at the party. 42 of the 90
        are women, so the chance of a woman winning the prize is:
        \[
          42 / 90 \approx \boxed{ 0.4667 }
        \]

      \item[40]
        \begin{enumerate}[(a)]
          \item They add up to 1, so they are a legitimate set of probabilities.

          \item The sum of the Hispanic probabilities is $\boxed{ 0.149 }$

          \item $1 - 0.674 = \boxed{ 0.326 }$
        \end{enumerate}

      \item[41]
        \begin{enumerate}[(a)]
          \item The probabilities add up to 1.

          \item $\boxed{ 0.04 }$

          \item $0.11 + 0.04 + 0.03 = \boxed{ 0.18 }$

          \item $0.04 + 0.09 + 0.13 + 0.16 = \boxed{ 0.42 }$

        \end{enumerate}

      \item[42]
        \begin{enumerate}[(a)]
          \item The 19 year old column and the ``Own Place'' row.

          \item The value for 19 year olds who live in their own place would be
            counted twice.

        \end{enumerate}  

      \item[43]
        \begin{enumerate}[(a)]
          \item $0.11 + 0.13 + 0.03 + 0.11 + 0.16 + 0.02 = 0.56$

          \item The probability of living with parents is:
            \[
              0.11 + 0.13 + 0.11 + 0.11 = 0.46
            \]

            The probability of not living with parents is:
            \[
              1 - 0.46 = \boxed{ 0.54 }
            \]
        \end{enumerate}  

      \item[44]
        \begin{enumerate}[(a)]
          \item A word is either right or wrong, (it can't be half right) so the
            variable is discrete.

          \item $P(X \geq 1) = 0.2 + 0.3 + 0.3 + 0.1 = 0.9$

          \item This is the probability that there were 2 or fewer errors.
          \[
            P(X \leq 2) = 0.1 + 0.2 + 0.3 = 0.6
          \]
        \end{enumerate}  

      \item[45]
        \begin{enumerate}[(a)]
          \item Each choice has a $1/9 \approx \boxed{ 0.11 }$ chance of being selected.
          \item $P(X \geq 6) = 4/9 \approx \boxed{ 0.44 }$
        \end{enumerate}  

      \item[47]
        \begin{enumerate}[(a)]
          \item GGG, GGB, GBG, GBB, BGG, BGB, BBG, BBB
          \item $P(X = 2) = 3/8 = \boxed{ 0.375 }$

          \item
            \begin{tabular}[H]{lrrrr}
              \toprule
              Value of X  & 0     & 1     & 2     & 3 \\
              \midrule
              Probability & 0.125 & 0.375 & 0.375 & 0.125 \\
              \bottomrule
            \end{tabular}
        \end{enumerate}  

      \item[48]
        All of the numbers between 1 and 12 have an equal probability of
        approximately $\boxed{ 0.083 }$

      \item[49]
        \begin{enumerate}[(a)]
          \item Continuous, since any number may be selected.
          \item Since the area is 1 and the base is 2, the height is 0.5.

          \item $P(X \leq 1) = 0.5$

        \end{enumerate}  

      \item[50]
        \begin{enumerate}[(a)]
          \item $P(0.5 < Y \leq 1.3) = 0.5 \cdot 1.3 - 0.5 \cdot 0.5 = \boxed{ 0.4 }$
          \item $P(Y \geq 0.8) = 1 - 0.8 \cdot 0.5= \boxed{ 0.6 }$
        \end{enumerate}  

      \item[51]
        \begin{enumerate}[(a)]
          \item Convert 0.52 and 0.60 to z-scores:
            \begin{align*}
              z_1 & = \frac{0.52 - 0.56}{0.019} \\
                  & \approx -2.1 \\
              z_2 & = \frac{0.60 - 0.56}{0.019} \\
                  & \approx 2.1 \\
            \end{align*}

            Use table A to find the fraction below each of these:
            \begin{align*}
              p_1 &\approx 0.0176 \\
              p_2 &\approx 0.9824 \\
            \end{align*}

            Subtract:
            \[
              P(0.52 \leq V \leq 0.60) \approx 0.9824 - 0.0174 = \boxed{ 0.9647 }
            \]

          \item Convert 0.72 to a z-score:
            \begin{align*}
              z & = \frac{0.72 - 0.56}{0.019} \\
                & \approx 8.4 \\
            \end{align*}

            Table A doesn't go this high, so $P(V \geq 0.720) \approx 0$
        \end{enumerate}  

      \item[52]
        \begin{enumerate}[(a)]
          \item Convert to z-scores:
            \begin{align*}
              z_1 & = \frac{8.9 - 9}{0.075} \\
                  & \approx -1.33 \\
              z_2 & = \frac{9.1 - 9}{0.075} \\
                  & \approx 1.33 \\
            \end{align*}

            Use table A to find the fraction below each of these:
            \begin{align*}
              p_1 &\approx 0.0912 \\
              p_2 &\approx 0.9088 \\
            \end{align*}

            Subtract:
            \[
              P(8.9 \leq \bar{X} \leq 9.1) \approx 0.9088 - 0.0912 = \boxed{ 0.8176 }
            \]
          \end{enumerate}

        \item[53]
          \begin{enumerate}[(a)]
            \item $1/10,000 = \boxed{ 0.0001 }$

            \item There are 4 different choices for the first digit, 3 choices
              for the second digit, once you've selected the first digit, 2
              choices for the third digit once you've selected the first 2, and
              only one choice for the remaining digit. The total number of ways
              to rearrange the digits is:
              \[
                4 \times 3 \times 2 = 24
              \]

              The chance of those digits coming up in any order is:
              \[
                24 / 10,000 = \boxed{ 0.0024 }
              \]
          \end{enumerate}

  \end{description}

  \else
    \vspace{10 cm}
    \begin{quote}
      \begin{em}
        Is a democracy, such as we know it, the last improvement possible in
        government? Is it not possible to take a step further towards
        recognizing and organizing the rights of man? There will never be a
        really free and enlightened State until the State comes to recognize the
        individual as a higher and independent power, from which all its own
        power and authority are derived, and treats him accordingly. 
      \end{em}
    \end{quote}
    \hspace{1 cm} --Henry David Thoreau
  \fi

\end{document}

