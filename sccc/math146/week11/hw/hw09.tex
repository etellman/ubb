% no answer key
% \documentclass[letterpaper]{exam}

% answer key
\documentclass[letterpaper, landscape]{exam}
\usepackage{2in1, lscape} 
\printanswers

\usepackage{units} 
\usepackage{xfrac} 
\usepackage[fleqn]{amsmath}
\usepackage{cancel}
\usepackage{float}
\usepackage{mdwlist}
\usepackage{booktabs}
\usepackage{cancel}
\usepackage{polynom}
\usepackage{caption}
\usepackage{fullpage}
\usepackage{comment}
\usepackage{enumerate}
\usepackage{graphicx}
\usepackage{parskip}

\everymath{\displaystyle}


\title{Statistics \\ Homework Nine}
\date{\today}
\author{}

\begin{document}

  \maketitle

  \section{Homework}
    \begin{itemize*}
      \item read Chapter 10 
      \item take a look at the ``Check Your Skills'' exercises
      \item exercises: TO DO
    \end{itemize*}

  \ifprintanswers
    \begin{description}

      \item[31] 
        \begin{enumerate}[(a)]
          \item If you are recording the sequence, the order matters.  If 1 is a
            make and 0 a miss, the sample space would be 
            
            \{ 0000, 0001, 0010, 0011, 0100, 0101, 0110, 0111, 1000, 1001, 1010,
               1011, 1100, 1101, 1110, 1111 \}

          \item If you don't care about the order, the sample space is just the
            number of shots made, or {0, 1, 2, 3, 4 }

        \end{enumerate}

      \item[32]
        \begin{enumerate}[(a)]
          \item The probabilities are all less than 1 and add up to 1, so they
            satisfy the rules of probability.  They aren't the correct
            probabilities for an actual die, however.

          \item The probabilities are all less than 1 and add up to 1, so they
            satisfy the rules of probability.  They aren't the correct
            probabilities for an actual card deck, however.

          \item These probabilities don't add up to exactly 1, so they don't
            satisfy the rules of probability.  
        \end{enumerate}

      \item[33]
        \begin{enumerate}[(a)]
          \item This is the only probability not provided, so it is:
            \[
              1 - 0.13 - 0.29 - 0.30 = \boxed{ 0.28 }
            \]

          \item This is everybody except the people that didn't complete high
            school:
            \[
              1 - 0.13 = \boxed{ 0.87 }
            \]
        \end{enumerate}

      \item[34]
        \begin{enumerate}[(a)]
          \item $\frac{4,176,000}{9,094,000} \approx 0.46$
          \item $1 - \frac{4,176,000}{9,094,000} \approx 0.54$
        \end{enumerate}

  \end{description}

  \else
    \vspace{10 cm}
    \begin{quote}
      \begin{em}
        Is a democracy, such as we know it, the last improvement possible in
        government? Is it not possible to take a step further towards
        recognizing and organizing the rights of man? There will never be a
        really free and enlightened State until the State comes to recognize the
        individual as a higher and independent power, from which all its own
        power and authority are derived, and treats him accordingly. 
      \end{em}
    \end{quote}
    \hspace{1 cm} --Henry David Thoreau
  \fi

\end{document}

