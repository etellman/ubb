% no answer key
\documentclass[letterpaper]{exam}

% answer key
% \documentclass[letterpaper, landscape]{exam}
% \usepackage{2in1, lscape} 
% \printanswers

% for the cent symbol
\usepackage{textcomp}

% the textcent command eats the space following the symbol
\usepackage{xspace}
\newcommand{\cent}{\textcent\xspace}

\usepackage{units} 
\usepackage{xfrac} 
\usepackage[fleqn]{amsmath}
\usepackage{cancel}
\usepackage{float}
\usepackage{mdwlist}
\usepackage{booktabs}
\usepackage{cancel}
\usepackage{polynom}
\usepackage{caption}
\usepackage{fullpage}
\usepackage{comment}
\usepackage{enumerate}
\usepackage{graphicx}
\usepackage{parskip}

\everymath{\displaystyle}


\title{Statistics \\ Homework Twelve}
\date{\today}
\author{}

\begin{document}

  \maketitle

  \section{Homework}
  \ifprintanswers
  \else
    \begin{itemize*}
      \item read Chapter 13 
      \item take a look at the ``Check Your Skills'' exercises
      \item exercises: TO DO
    \end{itemize*}
  \fi

  \ifprintanswers
    \begin{description}

      \item[27] 
        The probability of each bet losing is 0.75. Since the bets are
        independent, the probability of all 8 losing is
        \[
          p_{fail} = 0.75^8 \approx 0.1
        \]
        The probability of winning at least once is the probability of not
        losing all 8 or:
        \[
          p_{win} = 1 - p_{fail} \approx 1 - 0.1 = \boxed{ 0.9 } 
        \]

  \end{description}

  \else
    \vspace{11 cm}
    \begin{quote}
      \begin{em}
        A wrongdoer is often a man who has left something undone, not always one
        who has done something.
      \end{em}
    \end{quote}
    \hspace{1 cm}--Marcus Aurelius
  \fi

\end{document}

