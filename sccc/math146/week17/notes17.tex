\documentclass[landscape]{exam}

\usepackage{2in1, lscape} 
\usepackage{units} 
\usepackage[fleqn]{amsmath}
\usepackage{float}
\usepackage{mdwlist}
\usepackage{booktabs}
\usepackage{caption}
\usepackage{fullpage}
\usepackage{enumerate}
\usepackage{graphicx}
\usepackage{parskip}

\printanswers

\everymath{\displaystyle}

\printanswers

\title{Statistics \\ Week Sixteen}
\date{\today}
\author{}

\begin{document}

  \maketitle
  \tableofcontents

  \section{Gate Design}
  \begin{itemize*}
    \item transistors
    \item not, nor and nand gates
    \item two ways to make an xor gate
    \item truth tables
  \end{itemize*}

  \section{Binary Numbers}
  \begin{itemize*}
    \item base 2 
    \item adding and multiplying in base 2
  \end{itemize*}

  \section{Adders}

  half adder:
  \begin{itemize*}
    \item two inputs
    \item sum bit is $A xor B$
    \item carry bit is $A \cdot B$
  \end{itemize*}

  full adder is two half adders

  multi-bit adder out of multiple full-adders with $C_0 = 0$.

  \section{Multiplier}

  \section{Computer}
  \begin{itemize*}
    \item CPU
    \item Program Counter
    \item Instruction Register
    \item Memory
    \item Disk
    \item Cache
  \end{itemize*}

  \section{Algorithms}

  \subsection{Power}

  show in assembly language, C, and Java.

  \begin{verbatim}
    r0 <- power
    r1 <- 1
    r2 <- base
    
    loop: 
      if r0 == 0 go to done
      r1 <- r1 * r2
      r0 <- r0 - 1
      go to loop

    done:
      // r1 holds the result
  \end{verbatim}

  \begin{verbatim}
    // compute powers
    int power(int base, int p) {
      int result = 1;
      for (int i = 0; i < p; i++) {
        result *= base;
      }

      return result;
    }

    // compute powers
    int power(int base, int p) {
      return p == 0 ? 1 : base * power(base, p - 1);
    }

  \end{verbatim}

  \subsection{Quicksort}

  \begin{verbatim}
    sort(numbers)
      if numbers is not empty
        pick a random number
        partition into two sets of numbers
        recursively sort each set
        merge

      return numbers
  \end{verbatim}

  draw tree for how many get sorted at each level to show why sorting is $n \log n$ ($2^{10} \cdot 10$)

  \subsection{NP Complete}
  \begin{itemize*}
    \item traveling salesman
    \item shipping methods and order items
  \end{itemize*}

  \section{Audio}
  \begin{itemize*}
    \item sampling
    \item FFT and MP3
  \end{itemize*}
\end{document}

