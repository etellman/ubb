\documentclass[landscape]{exam}

\usepackage{2in1, lscape} 
\usepackage{units} 
\usepackage[fleqn]{amsmath}
\usepackage{float}
\usepackage{mdwlist}
\usepackage{booktabs}
\usepackage{caption}
\usepackage{fullpage}
\usepackage{enumerate}
\usepackage{graphicx}
\usepackage{parskip}

\printanswers

\everymath{\displaystyle}

\printanswers

\title{Statistics \\ Week Fourteen}
\date{\today}
\author{}

\begin{document}

  \maketitle
  \tableofcontents

  \begin{description}
    \item[14.1] 
      \begin{enumerate}[(a)]

        \item $ sd = \frac{60}{\sqrt{840}} \approx \boxed{ 2.07 } $

        \item 4.14

        \item 267.86 to 276.14

      \end{enumerate}

    \item[14.3]
      $(1 - 0.975)/2 = 0.0125$

      From table A, the $z^*$ values are $\pm 2.2414$

    \item[14.4]
      For a 90\% confidence interval and Table A, $z^* \approx 1.645$. None of
      the measurements should be more than 1.645 standard deviations from the
      mean.

      In conductivity units, this corresponds to $\pm 1.645 \cdot 0.2 = \pm
      0.329$. All of the measurements should be between 4.67 and 5.33.

      They all are, so they can say with 95\% confidence that everything is
      fine with what they are providing to the customers.

      If any of the samples had been outside this range, we would not have had
      95\% confidence that the liquid actually had a mean of 5.

    \item[14.5]
      The mean for the sample is 105.84. 

      For a 99\% confidence interval, $z^* = 2.576$. Converting this to IQ
      scores gives:
      \[
        iq = 2.576 \frac{15}{\sqrt{31}} \approx 6.94
      \]

      We can be 99\% confident that the actual mean is between and 98.90 and 112.78

  \end{description}
\end{document}

