\documentclass[landscape]{exam}

\usepackage{2in1, lscape} 
\usepackage{units} 
\usepackage[fleqn]{amsmath}
\usepackage{float}
\usepackage{mdwlist}
\usepackage{booktabs}
\usepackage{caption}
\usepackage{fullpage}
\usepackage{enumerate}
\usepackage{graphicx}

\printanswers

\everymath{\displaystyle}

\printanswers

\title{Statistics \\ Week Six}
\date{\today}
\author{}

\begin{document}

  \maketitle
  \tableofcontents

  \section{Berkeley Sex Discrimination}

  \begin{table}[H]
    \centering
    \begin{tabular}{rrr}
      \toprule
      admitted & rejected & total \\
      \midrule
      1749     & 2676     & 4425 \\
      \bottomrule
    \end{tabular}
  \end{table}

  The overall admission rate for the university is 40 \%.

  Break down by sex:
  \begin{table}[H]
    \centering
    \begin{tabular}{rlrrr}
      \toprule
                & sex   & admitted & rejected & (all) \\
      \midrule
      1         & M     & 1192     & 1398     & 2590 \\
      2         & F     & 557      & 1278     & 1835 \\
      3         & (all) & 1749     & 2676     & 4425 \\
       \bottomrule
    \end{tabular}
    \caption{By sex}
  \end{table}

  The totals are ``marginal distributions.'' In this case, they tell you:
  \begin{itemize*}
    \item total number of men
    \item total number of women
    \item total number of admitted
    \item total number of rejected
    \item total number of applicants
  \end{itemize*}

  A table of proportions is more useful:
  \begin{table}[H]
    \centering
    \begin{tabular}{rlrr}
      \toprule
                & sex & admitted & rejected \\
      \midrule
      1         & M   & 0.46     & 0.54 \\
      2         & F   & 0.30     & 0.70 \\
       \bottomrule
    \end{tabular}
  \end{table}

  Each row is a ``conditional distribution.'' All the values for a row have the
  same value for the categorical variable.

  Things looked suspicious, so they decided to do a breakdown by department:

  \begin{table}[H]
    \centering
    \begin{tabular}{rllrrr}
      \toprule
                 & department & sex   & admitted & rejected & (all) \\
      \midrule
      1          & A          & M     & 511      & 314      & 825 \\
      2          & A          & F     & 89       & 19       & 108 \\
      3          & A          & (all) & 600      & 333      & 933 \\
      \midrule
      4          & B          & M     & 353      & 207      & 560 \\
      5          & B          & F     & 17       & 8        & 25 \\
      6          & B          & (all) & 370      & 215      & 585 \\
      \midrule
      7          & C          & M     & 120      & 205      & 325 \\
      8          & C          & F     & 202      & 391      & 593 \\
      9          & C          & (all) & 322      & 596      & 918 \\
      \midrule
      10         & D          & M     & 139      & 278      & 417 \\
      11         & D          & F     & 131      & 244      & 375 \\
      12         & D          & (all) & 270      & 522      & 792 \\
      \midrule
      13         & E          & M     & 53       & 138      & 191 \\
      14         & E          & F     & 94       & 299      & 393 \\
      15         & E          & (all) & 147      & 437      & 584 \\
      \midrule
      16         & F          & M     & 16       & 256      & 272 \\
      17         & F          & F     & 24       & 317      & 341 \\
      18         & F          & (all) & 40       & 573      & 613 \\
      \midrule
      19         & (all)      & (all) & 1749     & 2676     & 4425 \\
      \bottomrule
    \end{tabular}
  \end{table}

  The percentages are more useful;

  \begin{table}[H]
    \centering
    \begin{tabular}{rlrr}
      \toprule
               & department & M    & F \\
      \midrule
      1        & A          & 0.62 & 0.82 \\
      2        & B          & 0.63 & 0.68 \\
      3        & C          & 0.37 & 0.34 \\
      4        & D          & 0.33 & 0.35 \\
      5        & E          & 0.28 & 0.24 \\
      6        & F          & 0.06 & 0.07 \\
      \bottomrule
    \end{tabular}
    \caption{Admissions proportions by department and sex.}
  \end{table}

  Department A has a bias towards women and all the other departments are
  evenly balanced

  Which departments are the most selective:

  \begin{table}[H]
    \centering
    \begin{tabular}{rlr}
      \toprule
               & department & admitted \\
      \midrule
      1        & A          & 0.64 \\
      2        & B          & 0.63 \\
      3        & C          & 0.35 \\
      4        & D          & 0.34 \\
      5        & E          & 0.25 \\
      6        & F          & 0.07 \\
      \bottomrule
    \end{tabular}
    \caption{Proportion admitted for each department}
  \end{table}

  A is the easiest and F is the toughest.

  Where do men and women most frequently apply:

  \begin{table}[H]
    \centering
    \begin{tabular}{rrrr}
      \toprule
               & department & women & men \\
      \midrule
      1        & A          & 0.06  & 0.32 \\
      2        & B          & 0.01  & 0.21 \\
      3        & C          & 0.32  & 0.12 \\
      4        & D          & 0.20  & 0.16 \\
      5        & E          & 0.21  & 0.07 \\
      6        & F          & 0.19  & 0.10 \\
      \bottomrule
    \end{tabular}
  \end{table}

  \begin{itemize*}
    \item 62 \% of the men apply to the two easiest departments.
    \item 17 \% of the men apply to the two hardest departments.
    \item 7 \% of the women apply to the two easiest departments.
    \item 40 \% of the women apply to the two hardest departments.
  \end{itemize*}

  \section{Apply Your Knowledge}

  \begin{description}
    \item[1] 
      \begin{table}[H]
        \centering
        \begin{tabular}{rlrrrr}
          \toprule
                   & consumer & higher & same & lower & (all) \\
          \midrule
          1        & buyer    & 20     & 7    & 9     & 36 \\
          2        & nonbuyer & 29     & 25   & 43    & 97 \\
          3        & (all)    & 49     & 32   & 52    & 133 \\
          \bottomrule
        \end{tabular}
      \end{table}

    81 out of 133 or 61\% are in the ``same'' or ``higher'' categories.

    \item[2]
      \begin{table}[H]
        \centering
        \begin{tabular}{rlrrr}
          \toprule
                   & age    & female & male & (all) \\
          \midrule
          1        & 15to17 & 116    & 61   & 177 \\
          2        & 18to24 & 5470   & 4691 & 10161 \\
          3        & 25to34 & 1319   & 824  & 2143 \\
          4        & over34 & 1075   & 616  & 1691 \\
          5        & (all)  & 7980   & 6192 & 14172 \\
          \bottomrule
        \end{tabular}
      \end{table}

      $10161 / 14172 = 0.72$ are in the 18 to 24 age group

    \item[3]
      \begin{table}[H]
        \centering
        \begin{tabular}{rlrrr}
          \toprule
                   & consumer & higher & same & lower \\
          \midrule
          1        & buyer    & 0.56   & 0.19 & 0.25 \\
          2        & nonbuyer & 0.30   & 0.26 & 0.44 \\
          \bottomrule
        \end{tabular}
      \end{table}
                                
      \begin{table}[H]
      \centering
      \begin{tabular}{rlrr}
        \toprule
                 & inclination & buyer & nonbuyer \\
        \midrule
        1        & higher      & 0.41  & 0.59 \\
        2        & same        & 0.22  & 0.78 \\
        3        & lower       & 0.17  & 0.83 \\
        \bottomrule
      \end{tabular}
      \end{table}

    \item[4]
      \begin{table}[H]
        \centering
        \begin{tabular}{rlrr}
          \toprule
                   & age    & female & male \\
          \midrule
          1        & 15to17 & 0.66   & 0.34 \\
          2        & 18to24 & 0.54   & 0.46 \\
          3        & 25to34 & 0.62   & 0.38 \\
          4        & over34 & 0.64   & 0.36 \\
           \bottomrule
        \end{tabular}
      \end{table}

      \begin{table}[H]
        \centering
        \begin{tabular}{rlrrrr}
          \toprule
                   & sex    & 15to17 & 18to24 & 25to34 & over34 \\
          \midrule
          1        & female & 0.01   & 0.69   & 0.17   & 0.13 \\
          2        & male   & 0.01   & 0.76   & 0.13   & 0.10 \\
           \bottomrule
        \end{tabular}
      \end{table}
  \end{description}

\end{document}

