\documentclass[landscape]{exam}

\usepackage{2in1, lscape} 
\usepackage{units} 
\usepackage[fleqn]{amsmath}
\usepackage{float}
\usepackage{mdwlist}
\usepackage{booktabs}
\usepackage{caption}
\usepackage{fullpage}
\usepackage{enumerate}
\usepackage{graphicx}
\usepackage{parskip}

\printanswers

\everymath{\displaystyle}

\printanswers

\title{Statistics \\ Week Eleven}
\date{\today}
\author{}

\begin{document}

  \maketitle
  \tableofcontents

  \section{Combinations and Permutations}

  For 8 choose 3, for example, numerator is $8 \times 7 \times 6 = 336$.
  There are 336 different combinations of the letters A-H, without
  repeating letters.

  If you don't care about the order, for each triplet, there are 3 choices for
  the first letter, 2 choices for the second letter, and only one choice for the
  third letter. The number of times each triplet of letters occurs is: 
  $3 \times 2 \times 1 = 6$.

  The total number of different combinations, not paying attention to order is:
  \[
    \frac{336}{6} = 56
  \]

  \section{Collins Case}

  \subsection{Cast}
  \begin{description*}
    \item[Juanita Brooks] elderly victim
    \item[John Bass] witness
    \item[Malcolm and Janet Collins] suspects
  \end{description*}

  \subsection{Investigation}
  Brooks' son visited nearby gas stations until he found one that mentioned
  interracial couple.

  \subsection{Evidence}
  Eyewitness descriptions of suspects:
  \begin{itemize*}
    \item woman with hair ``between light and dark blond'' and ponytail
    \item yellow car
    \item bearded and mustached black man
    \item nobody could positively identify suspects in lineups
    \item In trial, Bass claimed that Collins was person he saw in car, despite
      failure to identify him earlier.
  \end{itemize*}

  Actual appearance of suspects:
  \begin{itemize*}
    \item woman light blond hair sometimes in ponytail
    \item yellow car
    \item clean shaven black man
  \end{itemize*}

  The police arrested, interrogated, harassed suspects in unsuccessful attempt
  to extract confession.

  \subsection{Alibi}

  Janet was picked up by Malcolm from work about time crime had been committed
  but no one was too sure about either time.

  Both claimed to have gone to visit a friend, but exact time couldn't be
  established.

  Hair color, ponytail, and beard evidence all based on things easily changed.

  \subsection{Trial}
  Entire case based on probabilities:
  \begin{itemize*}
    \item Black man with a beard: 1 in 10
    \item Man with a mustache: 1 in 4
    \item White woman with blond hair: 1 in 3
    \item Woman with ponytail: 1 in 10
    \item Interracial couple in car: 1 in 1000
    \item Yellow car: 1 in 10
  \end{itemize*}

  Chance of all of the above: 1 in 12,000,000.

  Guilty verdict after 8 hours and 5 ballots.

  \subsection{Appeals}

  \subsubsection{Notes}
  Malcolm lost first appeal.

  In second appeal, 25 year old law clerk Lawrence Tribe pointed out flaws in
  arguments:

  \begin{itemize*}
    \item all statistics made up
    \item hair color, ponytail, beard, and mustache all easily altered
    \item no reason to assume actual perpetrators were married 
    \item product rule doesn't apply when probabilities aren't independent

    \item 40\% probability of at least one other couple in LA area with similar
      characteristics, even if you accept the original made-up prosecution
      figures. 

  \end{itemize*}

  \subsubsection{Fabricated Statistics}
  \begin{itemize*}
    \item all statistics made up
    \item hair color, ponytail, beard, and mustache all easily altered
    \item no reason to assume actual perpetrators were married 
  \end{itemize*}

  From judgement:
  \begin{quote}
    \begin{em}
      The prosecution produced no evidence whatsoever showing, or from which it could
      be in any way inferred, that only one out of every ten cars which might have
      been at the scene of the robbery was partly yellow, that only one out of every
      four men who might have been there wore a mustache, that only one out of every
      ten girls who might have been there wore a ponytail, or that any of the other
      individual probability factors listed were even roughly accurate.
    \end{em}
  \end{quote}

  \subsubsection{Independence}
  Product rule doesn't apply when probabilities aren't independent

  From Tribe:
  \begin{quote}
    \begin{em}
      There was another glaring defect in the prosecution’s technique, namely an
      inadequate proof of the statistical independence of the six factors. No
      proof was presented that the characteristics selected were mutually
      independent , even though the witness himself acknowledged that such
      condition was essential to the proper application of the ``product rule''
      or ``multiplication rule.'' \dots To the extent that the traits or
      characteristics were not mutually independent (e.g., Negroes with beards
      and men with mustaches obviously represent overlapping categories) the
      ``product rule'' would inevitably yield a wholly erroneous and exaggerated
      result even if all of the individual components had been determined with
      precision.
    \end{em}
  \end{quote}

  \subsubsection{Wrong Probability to Investigate}
    40\% probability of at least one other couple in LA area with similar
    characteristics, even if you accept the original made-up prosecution
    figures. 

    1/12,000,000 represents supposed probability randomly selected couple will
    have this particular set of characteristics. Correct probability to look at
    is probability that the suspects are the correct pair of people with this
    set of statistics, given there are probably multiple people like this in the
    LA area.


  \subsection{Aftermath}
  Tribe went on to:
  \begin{itemize*}
    \item Defend Gore in Gore vs. Bush
    \item Become one of Obama's instructors
    \item Serve in the Obama administration
    \item Write numerous papers on use of law in trials
  \end{itemize*}

  \section{Monty Hall Game}
  Select one of 3 doors, etc., etc.

  \section{Probability}

  \subsection{Definitions}

  \begin{description*}
    \item[random] coin toss, etc. individual outcome uncertain, but long term
      average predictable

    \item[probability] proportion of times something happens over many
      repetitions

  \end{description*}

  \subsection{Probability Models}

  \begin{description*}
    \item[sample space] all possible outcomes

    \item[event] a particular outcome

    \item[probability model] sample space and way of assigning probabilities to
      events

  \end{description*}

  \subsection{Probability Rules}

  \begin{itemize}
    \item probability between 0 and 1
    \item all possible outcomes add up to probability one
    \item if events are disjoint (no outcomes in common, can never occur
      together) probability of either one is sum of individual probabilities
    \item ``mutually exclusive'' is another term for disjoint. If one event
      occurs, the other event is excluded from occurring.
    \item if events are independent, probability of both is product of
      individual probabilities
  \end{itemize}

  \section{Discrete Probability Models}
  Thins like coin tosses, cards, etc., where you can enumerate all possible
  outcomes.

  \subsection{Pairs}
  Draw table with 1-6 labeling rows and 1-6 labeling columns and cells sum of
  row/column labels. 36 ways for dice to land. Show getting a particular number
  is diagonal going up and to the right.

  \begin{itemize*}
    \item what's the most common roll?
    \item how many ways are there to get 5?
    \item how many ways are there to get 12?
  \end{itemize*}

  \subsection{Three Dice/Sum 9 or 10}

  \begin{itemize*}
    \item combinations for 9: 126, 135, 144, 234, 225, 333
    \item combinations for 10: 145, 136, 226, 235, 244, 334
    \item $6^3 = 216$ ways for dice to fall
  \end{itemize*}

  Galileo analysis:
  \begin{itemize*}
    \item combinations with all different numbers can come up 
      $3 \times 2 \times 1 = 6$ ways
    \item combinations with two different numbers can come up 
      $3 \times 1 = 3$ ways
    \item 333 can only come up one way
    \item actually 25 ways to get 9 and 27 ways to get 10 
      (because of 333 is only 1 and 334 is 3 and all others match up)
  \end{itemize*}

  \subsection{Dice vs. Cards}
  A die is rolled twice; a deck of cards is shuffled.

  \begin{parts}
    \part What's the probability that either the first roll is a one or the
      second roll is a one?

      \begin{solution}
      The first roll is a one or the second roll is a 1, subtracting the time
      when they are both one to avoid counting it twice:
        \[
          \frac{1}{6} + \frac{1}{6} - \frac{1}{36} = \frac{11}{36}
        \]

        The first roll is a one and the second one isn't, or the first roll
        isn't a one and the second one is, or both are ones:
        \[
          \frac{1}{6} \cdot \frac{5}{6} + \frac{5}{6} \cdot \frac{1}{6} +
            \frac{1}{6} \cdot \frac{1}{6} = \frac{11}{36} 
        \]

        One minus the probability that neither is a one:
        \[
          1 - \frac{5}{6} \cdot \frac{5}{6} = \frac{11}{36} 
        \]

      \end{solution}

    \part What's the probability that the first card is the ace of spades or the
      last card is the ace of spades?
      \begin{solution}
        Mutually exclusive events: $\frac{1}{52} + \frac{1}{52} = \frac{1}{26}$
      \end{solution}

    \part What's the probability that the first roll is a 1 and the second roll
      is a 1?
      \begin{solution}
        Independent events: $\frac{1}{36}$
      \end{solution}

    \part What's the probability that the first card is the ace of spades and
    the last card is the ace of spades?
      \begin{solution}
        Mutually exclusive events: $0$
      \end{solution}
  \end{parts}

  \subsection{Card Drawing}
  Contestants are given a shuffled deck of cards. They win a prize if the first
  card is the ace of hearts or the second card is the king of hearts.

  \begin{parts}
    \part What's the probability of winning with the king of hearts?: $1/52$
    \part What's the probability of winning with the ace of hearts?: $1/52$

    \part What's the overall probability of winning? 
    
    \begin{solution}
      \begin{align*}
        \frac{1}{52} + \frac{1}{52} - \frac{1}{52^2} & = \frac{103}{51^2} \\
                                                     & \approx 0.0381
      \end{align*}
      
      If you don't account for the possibility of getting both winning cards,
      your odds would be about 0.0384.
      
      You need to subtract the possibility of getting both cards. The events are
      not mutually exclusive.

      Win only on the first card or only on the second card or on both cards:
      \begin{align*}
        \frac{1}{52} \cdot \frac{51}{52} + \frac{1}{52} \cdot \frac{51}{52} 
            + \frac{1}{52^2} & = \frac{103}{51^2} \\
                             & \approx 0.0381
      \end{align*}

      1 minus chance of losing on both attempts:
      \begin{align*}
            1 - \frac{51^2}{52^2} & = \frac{103}{51^2} \\
                                  & \approx 0.0381
      \end{align*}

    \end{solution}
  \end{parts}

  \subsection{Paradox of Chevalier de Moivre}

  Which is more likely:
  \begin{itemize}
    \item Getting at least one one with four rolls of a single die.
    \item Getting at least one pair of ones with 24 rolls of a pair of dice.
  \end{itemize}

  The naive approach is that the probability of getting at least one one is
  \[
    4 \cdot \frac{1}{6} = \frac{2}{3}
  \]

  Probability of getting at least one pair of ones is
  \[
    24 \cdot \frac{1}{36} = \frac{2}{3}
  \]

  This is incorrect because the events are not mutually exclusive.

  \begin{solution}
    The probability of getting at least one one is:
    \[
      1 - \left( \frac{5}{6} \right)^4 \approx 0.5177
    \]

    The probability of not getting at least one pair of ones is:
    \[
      1 - \left( \frac{35}{36} \right)^{24} \approx 0.4914
    \]
  \end{solution}

  \subsection{Roulette}
  In a roulette style game from the 1700's, with a wheel with 32 slots, the
  players win 27 pounds plus the original pound on a 1 pound bet if their number
  comes up and they lose otherwise. The players complained that the payoff
  should have been closer to 31. The casino ``proved'' them wrong by betting
  even money that any number the player picked would come up within 21 rolls.
  What is the probability of the casino winning this second bet:

  \begin{solution}
    \[
      1 - \left( \frac{1}{32} \right)^{22} \approx 0.5027
    \]
  \end{solution}

  \subsection{WW II Pilots}
  If a pilot has a 2\% chance of getting shot down, what's the chance of him
  surviving 50 missions?

  \begin{solution}
    \[
      0.98^{50} \approx 0.36
    \]
  \end{solution}

  \section{Random Variables}
  \begin{itemize*}
    \item Like regular variable but value is determined by probability distribution
    \item Usually upper case letters near end of alphabet
  \end{itemize*}

  \section{Continuous Probability Models}
  \subsection{Notes}
  \begin{itemize*}
    \item Pick a random number between 0 and 1, etc.

    \item Probability density curve from previous chapter. 
  
    \item Area under curve is chance of getting outcome from in range.

    \item {\em Uniform Distribution} is flat

    \item {\em Normal Distribution}

    \item $P(X < 2)$, etc. notation

  \end{itemize*}

  \subsection{Packaging Laws}
  \begin{itemize*}
    \item too heavy expensive for manufacturer
    \item too light expensive for consumer
    \item impossible to be 100\% accurate
    \item normal distribution around target weight
  \end{itemize*}


  \begin{enumerate}
    \item If target weight is 1 kg and machine with standard deviation of 40 g,
      what is the probability the actual weight is less than 950 g?

    \begin{solution}
      Convert to z-score:
      \[
        z = \frac{950 - 1000}{40} = -1.25
      \]

      Use table A:
      \[
        P(X < 980) \approx 0.1056
      \]
    \end{solution}

    \item If the same machine is set at a target weight of 1020 g instead, what
      is the probability the actual weight is less than 1000 g?

    \begin{solution}
      Convert to z-score:
      \[
        z = \frac{1000 - 1020}{40} = -0.5
      \]

      Use table A:
      \[
        P(X < 1000) \approx 0.3085
      \]
    \end{solution}

    In Australia, law is that if a package is found to be underweight,
    everything is cool as long as the average of 12 similar items is not
    underweight.

    \begin{itemize*}
      \item Average also normal curve
      \item Same area (1, of course) but steeper and narrower graph
      \item same mean, $sd = \frac{ sd_{original} }{\sqrt{n}}$
    \end{itemize*}

    Strategy is to set machine slightly high, but chance of violating law is
    much lower with this strategy.

    What is the probability of violating the law when setting the target weight
    to 1020 with the same machine?

    \begin{solution}
      Calculate new sd:
      \[
        sd = \frac{40}{\sqrt{12}} \approx 11.55
      \]
      Convert to z-score:
      \[
        z = \frac{1000 - 1020}{11.55} = -1.73
      \]

      Use table A:
      \[
        P(X < 1000) \approx 0.0417
      \]
    \end{solution}

  \end{enumerate}

  \subsection{Exercise 10.16}
    \begin{parts}
      \part 
        \begin{solution}
          $P(X \geq 10)$
        \end{solution}

      \part 
        \begin{solution}
          Convert 10 to a z-score:
          \[
            z = \frac{10 - 6.8}{1.6} = 2
          \]

          $P(Z < 2) = 0.9772$
        \end{solution}
    \end{parts}

\end{document}

