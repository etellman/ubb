\documentclass[landscape]{exam}

\usepackage{2in1, lscape} 
\usepackage{units} 
\usepackage[fleqn]{amsmath}
\usepackage{float}
\usepackage{mdwlist}
\usepackage{booktabs}
\usepackage{caption}
\usepackage{fullpage}
\usepackage{enumerate}
\usepackage{graphicx}
\usepackage{parskip}

\printanswers

\everymath{\displaystyle}

\printanswers

\title{Statistics \\ Week Eleven}
\date{\today}
\author{}

\begin{document}

  \maketitle
  \tableofcontents

  \section{Multiplication Rule}

  For independent events:
  \[
    P(\text{A and B}) = P(A) P(B)
  \]

  independence:
  \begin{itemize*}
    \item outcome of A doesn't influence B
    \item dice and roulette are independent
    \item drawing cards without replacement isn't--cards already drawn are gone
      from the deck
  \end{itemize*}

  \section{Addition Rule}

  \[
    P(\text{A or B}) = P(A) + P(B) - P(\text{A and B})
  \]

  \begin{itemize*}
    \item if $P(A and B) = 0$, events are {\em disjoint} or {\em mutually
      exclusive}
    \item for disjoint events, if A occurs B can't occur
    \item draw Venn diagram to explain why subtracting $P(\text{A and B})$ is
      necessary
  \end{itemize*}

  examples:
  \begin{itemize*}
    \item red cars, compact cars, red compact cars
    \item people with beards, people with mustaches, people with both
  \end{itemize*}

  \section{Independence vs. Disjoint}
  \begin{itemize*}
    \item two independent events may both happen. event one has no effect on
      event two.

    \item to disjoint events can never both happen. if event one happens, event
      two can't happen

  \end{itemize*}

  \section{DNA}
  99\% of DNA is same between all people.

  DNA tests select 13 (TO DO) genes that are often different from person to person. The
  chance of all 13 matching by chance is (TO DO).

  If any of them don't match, it's definitely a different person. 

  If all of them match, is probably the same person.
  
  \section{Conditional Probability}

  \subsection{OJ Case}
  Prosecutors said that there is a natural progression from abuse to murder.

  Prosecutors countered with 
  \begin{itemize*}
    \item 14,000,000 women battered annually by spouse or boyfriend
    \item 1,432 of these, or 1 in 2,500 murdered by spouse
    \item chance of OJ having murdered Nicole only 1 in 2,500
  \end{itemize*}

  Defense should have countered with conditional probability. Murder is a very
  rare event, murder by spouse also very rare. If an abused woman has been
  murdered, 90\% of the time the spouse is the murderer.

  The interesting probability is the {\em Conditional Probability} that OJ did
  it, given that she was in fact murdered.

  Of course, a 90\% probability isn't evidence that OJ did it, but it is
  reasonable to consider OJ as a suspect and look at DNA, etc.

  \begin{align*}
    P(\text{A and B}) &= P(A) P(B | A) \\
    \\
    P(B | A) &= \frac{P(\text{A and B})}{P(A} \\
  \end{align*}

  \begin{tabular}[H]{lrrr}
    \toprule
    species & white & black & total\\
    \midrule \\
    dog     & 30    & 70    & 100\\
    cat     & 25    & 25    & 50\\
    \midrule
    total   & 45    & 105   & 150\\
    \bottomrule
  \end{tabular}

  \begin{align*}
    P(\text{black dog}) & = P(dog) \cdot P(black | dog) \\
                        & = \frac{2}{3} \cdot 0.7 \\
                        & \approx 0.4667 \\
  \end{align*}
  
  verify:
  \[
    \frac{70}{150} \approx 0.4667 \\
  \]

  \begin{align*}
    P(\text{black | dog}) & = \frac{P(\text{black dog})}{P(dog)} \\
                          & \approx \frac{0.4667}{0.6667} \\
                          & = 0.7 \\
  \end{align*}

  verify:
  \[
    \frac{70}{100} = 0.8 \\
  \]

  \begin{align*}
    P(\text{dog | black}) & = \frac{P(\text{black dog})}{P(black)} \\
                          & \approx \frac{0.4667}{0.7} \\
                          & = 0.6667 \\
  \end{align*}

  verify:
  \[
    \frac{70}{105} \approx 0.6667 \\
  \]

  \begin{table}[H]
    \centering
    \begin{tabular}{rlrr}
      \toprule
          & species & white  & black \\
      \midrule
      1   & dog     & 0.2000 & 0.4667 \\
      2   & cat     & 0.1667 & 0.1667 \\
      \bottomrule
    \end{tabular}
  \end{table}

  \begin{align*}
    P(\text{black | dog}) & = \frac{P(\text{black dog})}{P(dog)} \\
                          & \approx \frac{0.4667}{0.6667} \\
                          & = 0.7 \\
  \end{align*}

  \begin{align*}
    P(\text{dog | black}) & = \frac{P(\text{black dog})}{P(black)} \\
                          & \approx \frac{0.4667}{0.7} \\
                          & = 0.6667 \\
  \end{align*}

  verify:
  \[
    \frac{70}{105} \approx 0.6667 \\
  \]


\end{document}

