% no answer key
% \documentclass[letterpaper]{exam}

% answer key
\documentclass[letterpaper, landscape]{exam}
\usepackage{2in1, lscape} 
\printanswers

% for the cent symbol
\usepackage{textcomp}

% the textcent command eats the space following the symbol
\usepackage{xspace}
\newcommand{\cent}{\textcent\xspace}

\usepackage{units} 
\usepackage{xfrac} 
\usepackage[fleqn]{amsmath}
\usepackage{cancel}
\usepackage{float}
\usepackage{mdwlist}
\usepackage{booktabs}
\usepackage{cancel}
\usepackage{polynom}
\usepackage{caption}
\usepackage{fullpage}
\usepackage{comment}
\usepackage{enumerate}
\usepackage{graphicx}
\usepackage{parskip}

\everymath{\displaystyle}


\title{Statistics \\ Homework Ten}
\date{\today}
\author{}

\begin{document}

  \maketitle

  \section{Homework}
  \ifprintanswers
  \else
    \begin{itemize*}
      \item read Chapter 11 
      \item take a look at the ``Check Your Skills'' exercises
      \item exercises: 22-27, 29, 31-33, 35-40
    \end{itemize*}
  \fi

  \ifprintanswers
    \begin{description}

      \item[22] Since this is the mean of the sample and not the total
        population, it is a \fbox{ statistic }.

      \item[23] These are \fbox{ statistics }.

      \item[24] Making repeated identical bets is increasing the sample size.
        As the sample size increases, the mean of the sample will approach the
        mean of the population. In the long run, the gambler should expect to
        lose about \fbox{ 5.3\cent } per bet.

      \item[25] The mean of the sampling distribution is the same as the mean of
        the population, or $\boxed{ \unit[6]{strikes/year} }$.

        The standard deviation for a sample size of 10 is:
        \[
          s = \frac{2.4}{\sqrt{10}} \approx \boxed{ \unit[0.7589]{strikes/year} }
        \]

      \item[26]
        The parameters of the population are:
        \begin{align*}
          \sigma & = \unit[2.8]{in} \\
          \mu    & = \unit[70]{in} \\
        \end{align*}

        \begin{enumerate}[(a)]
          \item Find the two z-scores:
            \begin{align*}
              z_{69} &= \frac{69 - 70}{2.8} \approx -0.3571 \\
              z_{71} &= \frac{71 - 70}{2.8} \approx 0.3571 \\
            \end{align*}

            Use table A and subtract:
            \begin{align*}
              P(69 \leq X \leq 71) & \approx 0.6395 - 0.3605 \\
                                   & = \boxed{ 0.2790 } \\
            \end{align*}

          \item 
            The mean of the $\bar{x}$ values is the same as the mean of the
            population ($\mu$) or $\boxed{ 70 }$

            The standard deviation is:
            \[
              s = \frac{2.8}{\sqrt{25}} = \boxed{ \unit[0.56]{in} }
            \]

          \newpage

          \item Find the two z-scores:
            \begin{align*}
              z_{69} &= \frac{69 - 70}{0.56} \approx -1.7858 \\
              z_{71} &= \frac{71 - 70}{0.56} \approx 1.7858 \\
            \end{align*}

            Use table A and subtract:
            \begin{align*}
              P(69 \leq \bar{x} \leq 71) & \approx 0.9629 - 0.0371 \\
                                         & = \boxed{ 0.9259 } \\
            \end{align*}

        \end{enumerate}

      \item[27]
        The parameters are:
        \begin{align*}
          \mu    & = \unit[125]{mg/dl} \\
          \sigma & = \unit[10]{mg/dl} \\
        \end{align*}

        \begin{enumerate}[(a)]
          \item 
            Converting to z-score and using table A:
            \[
              P(X > 140) \approx \boxed{ 0.0668 }
            \]

          \item 
            With four measurements, the standard deviation is:
            \[
              s = \frac{10}{\sqrt{4}} = \unit[5]{mg/dl}
            \]

            Converting to z-score and using table A:
            \[
              P(\bar{x} > 140) \approx \boxed{ 0.0013 }
            \]
        \end{enumerate}

      \item[29]
        The parameters are:
        \begin{align*}
          \mu    & = \unit[125]{mg/dl} \\
          \sigma & = \unit[10]{mg/dl} \\
        \end{align*}

        Use table A to find the z-score for 0.95:
        \[
          z_{0.95} \approx 1.6449
        \]

        With four measurements, the standard deviation is:
        \[
          s = \frac{10}{\sqrt{4}} = \unit[5]{mg/dl}
        \]

        Convert the z-score to a glucose level:
        \begin{align*}
          1.6449 & = \frac{x - 125}{5} \\
          x      & \approx \boxed{ \unit[133.2]{mg/dl} } \\
        \end{align*}

      \item[31]
        \begin{enumerate}[(a)]
          \item The sample mean is the same as the population mean and the
            standard deviation is:
            \[
              s = \frac{1.4}{\sqrt{52}} \approx 0.1941
            \]

            The distribution is: \fbox{ N(2.2, 0.1941) }

          \item $P(\bar{x} < 2) \approx \boxed{ 0.1515 }$

          \item 100 accidents per year would lead to a weekly rate of:
            \[
              r = \frac{100}{52} \approx 1.923
            \]

            The probability of a rate less than this is:
            \[
              P(\bar{x} < 1.923) \approx \boxed{ 0.0769 }
            \]
        \end{enumerate}

      \newpage

      \item[32]
        The parameters are:
        \begin{align*}
          \mu    & = \unit[0.2]{g/mi} \\
          \sigma & = \unit[0.05]{g/mi} \\
        \end{align*}

        Use table A to find the z-score for 0.99:
        \[
          z_{0.99} \approx 2.3263
        \]

        With 25 samples, the standard deviation is:
        \[
          s = \frac{0.05}{\sqrt{25}} = \unit[0.01]{g/mi}
        \]

        Convert the z-score to an NOX rate:
        \begin{align*}
          2.3263 & = \frac{x - 0.2}{0.01} \\
          x     & \approx \boxed{ \unit[0.2233]{g/mi} } \\
        \end{align*}

      \item[33]
        The parameters are:
        \begin{align*}
          \mu    & = 8.7\% \\
          \sigma & = 20.2\% \\
        \end{align*}

        With a sample size of 40, the standard deviation is:
        \[
          s = \frac{20.2}{\sqrt{40}} \approx 3.1939\%
        \]

        Convert to z-scores:
        \begin{align*}
          z_{10} & = \frac{10 - 8.7}{3.1939} \approx 0.4070 \\
          z_{5}  & = \frac{5 - 8.7}{20.2} \approx -1.1585 \\
        \end{align*}

        Use table A:
        \begin{align*}
          P(r > 10\%) & = 1 - P(r < 10\%) \approx \boxed{ 0.3420 } \\
          P(r < 5\%)  & \approx \boxed{ 0.1233 } \\
        \end{align*}

      \item[35]
        \begin{align*}
          0.5      & = \frac{2.8}{\sqrt{n}} \\
          \sqrt{n} & = \frac{2.8}{0.5} \\
                   & = 5.6 \\
          n        & \approx 31.86 \\
        \end{align*}

        Since you need a whole number, you'll need a sample size of 
        \fbox{ 32 students }.

      \item[36]
        \begin{enumerate}[(a)]
          \item 99.7\% of the samples are within two standard deviations of the
            mean:
            \begin{align*}
              2s & = 0.5 \\
              s  & = \boxed{ \unit[0.25]{in} } \\
            \end{align*}

          \item 
            \begin{align*}
              0.25 & = \frac{-1.8}{\sqrt{n}} \\
              n    & \approx 125.44 \\
            \end{align*}

            Since you need a whole number, you'll need a sample size of 
            \fbox{ 126 students }.
        \end{enumerate}

      \item[37] Each bet costs him about \fbox{ 40\cent }.

      \item[38]
        \begin{enumerate}[(a)]
          \item 
            \begin{align*}
              m &= \boxed{ 0.6 } \\
              s &= \frac{18.96}{\sqrt{14,000}} \approx \boxed{ 0.1602 } \\
            \end{align*}
            
          \item
            convert to z-scores:
            \begin{align*}
              z_{.50} &= \frac{0.5 - 0.6}{0.1602} = -0.6242 \\
              z_{.70} &= \frac{0.7 - 0.6}{0.1602} = 0.6242 \\
            \end{align*}

            use table A:
            \begin{align*}
              p_{.50} &= 0.2662 \\
              p_{.70} &= 0.7337 \\
            \end{align*}

            subtract:
            \[
              p = 0.7337 - 0.2662 = \boxed{ 0.4675 }
            \]

            There is only about a 47\% chance that Joe will average between
            50\cent and 70\cent after 40 years.

        \end{enumerate}

      \item[39]
        \begin{enumerate}[(a)]
          \item 
            \begin{align*}
              m &= \boxed{ 0.6 \text{\cent} } \\
              s &= \frac{18.96}{\sqrt{150,000}} \approx 
                \boxed{ 0.0490 \text{\cent}} \\
            \end{align*}
            
          \item
            convert to z-scores:
            \begin{align*}
              z_{.30} &= \frac{0.3 - 0.4}{0.049} \approx -2.0408 \\ 
              z_{.50} &= \frac{0.5 - 0.4}{0.049} \approx 2.0408 \\
            \end{align*}

            use table A:
            \begin{align*}
              p_{.30} &= 0.2662 \\
              p_{.50} &= 0.7337 \\
            \end{align*}

            subtract:
            \[
              p = 0.7337 - 0.2662 = \boxed{ 0.9587 }
            \]

            There is about a 96\% chance that Casper will make about 40\cent per
            bet on any particular week.

        \end{enumerate}

      \newpage

      \item[40]
        \begin{enumerate}[(a)]
          \item The difference between the two values is:
            \[
              \Delta = 0.4961 - 0.4675 = \boxed{ 0.0286 }
            \]

          \item
            \begin{align*}
              s &= \frac{18.96}{\sqrt{3500}} \approx \boxed{ 0.3205 } \\
            \end{align*}
            
          \item
            convert to z-scores:
            \begin{align*}
              z_{.50} &= \frac{0.5 - 0.6}{0.3205} = -0.3120 \\
              z_{.70} &= \frac{0.7 - 0.6}{0.3205} = 0.3120 \\
            \end{align*}

            use table A:
            \begin{align*}
              p_{.50} & \approx 0.3775 \\
              p_{.70} & \approx 0.6225 \\
            \end{align*}

            subtract:
            \[
              p = 0.6225 - 0.3775 = 0.2450
            \]

            This estimate is off from the actual value by:
            \[
              \Delta = 0.4048 - 0.245 \approx \boxed{ 0.16 }
            \]
            
          \item Using the 150,000 bet value from exercise 39:
            \[
              \Delta = 0.9629 - 0.9587 = \boxed{ 0.0042 }
            \]

        \end{enumerate}
  \end{description}

  \else
    \vspace{10 cm}
    \begin{quote}
      \begin{em}
        Because they don't teach the truth about the world, schools have to rely on
        beating students over the head with propaganda about democracy. If schools were,
        in reality, democratic, there would be no need to bombard students with
        platitudes about democracy. They would simply act and behave democratically, and
        we know this does not happen. The more there is a need to talk about the ideals
        of democracy, the less democratic the system usually is.
      \end{em}
    \end{quote}
    \hspace{1 cm} --Noam Chomsky
  \fi

\end{document}

