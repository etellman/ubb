% no answer key
% \documentclass[letterpaper]{exam}

% answer key
\documentclass[letterpaper, landscape]{exam}
\usepackage{2in1, lscape} 
\printanswers

\usepackage{units} 
\usepackage{xfrac} 
\usepackage[fleqn]{amsmath}
\usepackage{cancel}
\usepackage{float}
\usepackage{mdwlist}
\usepackage{booktabs}
\usepackage{cancel}
\usepackage{polynom}
\usepackage{caption}
\usepackage{fullpage}
\usepackage{comment}
\usepackage{enumerate}
\usepackage{graphicx}
\usepackage{parskip}

\everymath{\displaystyle}


\title{Statistics \\ Homework Ten}
\date{\today}
\author{}

\begin{document}

  \maketitle

  \section{Homework}
  \ifprintanswers
  \else
    \begin{itemize*}
      \item read Chapter 11 
      \item take a look at the ``Check Your Skills'' exercises
      \item exercises: TO DO
    \end{itemize*}
  \fi

  \ifprintanswers
    \begin{description}

      \item[22] Since this is the mean of the sample and not the total
        population, it is a \fbox{ statistic }.

      \item[23] Since these are the means for the entire population of the
        students in each category, they are \fbox{ parameters }.

      \item[24] Making repeated identical bets is increasing the sample size.
        As the sample size increases, the mean of the sample will approach the
        mean of the population. In the long run, the gambler should expect to
        lose about 5.3 cents per bet.

        In 1000 bets, he should expect to lose about \$53. 

      \item[25] The mean of the sampling distribution is the same as the mean of
        the population, or \fbox{ 6 }.

        The standard deviation for a sample size of 10 is:
        \[
          s = \frac{2.4}{\sqrt{10}} \approx \boxed{ 0.7589 }
        \]

      \item[26]
        The parameters of the population are:
        \begin{align*}
          \sigma & = 2.8 \\
          \mu    & = 70 \\
        \end{align*}

        \begin{enumerate}[(a)]
          \item Find the two z-scores:
            \begin{align*}
              z_{69} &= \frac{69 - 70}{2.8} \approx -0.3571 \\
              z_{71} &= \frac{71 - 70}{2.8} \approx 0.3571 \\
            \end{align*}

            Use table A and subtract:
            \begin{align*}
              P(69 \leq X \leq 71) & \approx 0.6395 - 0.3605 \\
                                   & = \boxed{ 0.2790 } \\
            \end{align*}

          \item 
            The mean of the $\bar{x}$ values is the same as the mean of the
            population ($\mu$) or $\boxed{ 70 }$

            The standard deviation is:
            \[
              s = \frac{2.8}{\sqrt{25}} = \boxed{ 0.56 }
            \]

          \item Find the two z-scores:
            \begin{align*}
              z_{69} &= \frac{69 - 70}{0.56} \approx -1.7858 \\
              z_{71} &= \frac{71 - 70}{0.56} \approx 1.7858 \\
            \end{align*}

            Use table A and subtract:
            \begin{align*}
              P(69 \leq \bar{x} \leq 71) & \approx 0.9629 - 0.0371 \\
                                         & = \boxed{ 0.9259 } \\
            \end{align*}

        \end{enumerate}
  \end{description}

  \else
    \vspace{10 cm}
    \begin{quote}
      \begin{em}
        TO DO
      \end{em}
    \end{quote}
    \hspace{1 cm} --Henry David Thoreau
  \fi

\end{document}

