\documentclass[landscape]{exam}

\usepackage{2in1, lscape} 
\usepackage{units} 
\usepackage[fleqn]{amsmath}
\usepackage{float}
\usepackage{mdwlist}
\usepackage{booktabs}
\usepackage{caption}
\usepackage{fullpage}
\usepackage{enumerate}
\usepackage{graphicx}
\usepackage{parskip}

\printanswers

\everymath{\displaystyle}

\printanswers

\title{Statistics \\ Week Ten}
\date{\today}
\author{}

\begin{document}

  \maketitle
  \tableofcontents

  \section{Tea/Milk Experiment}

  \subsection{Terminology}
  
  Goal: determine if subject can tell whether tea was made milk/tea or tea/milk.

  \begin{description*}
    \item[treatment] thing you are trying to find out (adding cream first)
    \item[control] no change (usual order)
    \item[factor] tea/milk order
  \end{description*}

  \subsection{Experiments}
  With 2 cup experiment, subject has 50\% of success by chance.

  With 4 cup experiment:
  \begin{itemize*}
    \item $4 \times 3 = 12$ ways to choose 2 cups.  4 choices for first cup and
      3 choices for second cup.  
    \item since the order doesn't matter, 6 of the choices are identical and
      there are 6 different ways to choose 2 out of 4. aabb, abab, abba, bbaa,
      baba, baab.
    \item taster has 1 out of 6 or 17\% chance of success by luck
  \end{itemize*}

  With 6 cup experiment:
  \begin{itemize*}
    \item $6 \times 5 \times 4 = 120$ ways to choose 3 cups.  
    \item $3 \times 2 \times 1 = 6$ duplicates
    \item taster has 1 out of 20 or 5\% chance of success by luck
  \end{itemize*}

  With 8 cup experiment:
  \begin{itemize*}
    \item $8 \times 7 \times 6 \times 5 = 1680$ ways to choose 3 cups.  
    \item $4 \times 3 \times 2 \times 1 = 24$ duplicates
    \item taster has 1 out of 70 or 1.4\% chance of success by luck
  \end{itemize*}

  With 10 cup experiment 1 out of 252: 0.4\% chance of success by luck.

  \begin{itemize*}
    \item no number of chances gives 100\% confidence taster can always tell
    \item usual acceptance of success is 5\% chance of success by luck
  \end{itemize*}


  \part{Experiments}

  \section{Salk Polio Vaccine}
  \subsection{History}
  \begin{itemize*}
    \item 1916-54, thousands of cases
    \item 1954 trial of new Jonas Salk vaccine for polio
    \item Polio affects richer kids more than poorer kids. Poorer kids tend to
      be exposed to polio early in life and get very mild cases which allows
      them to develop immunity.
    \item parental permission required for vaccination
    \item richer parents were more likely to give permission for vaccination
    \item disease spreads through proximity with other people with the disease
    \item vaccine looked promising but it was uncertain whether it would
      actually help or possibly hurt
  \end{itemize*}

  \subsection{Original Design}
  The original design was to vaccinate all grade 2 kids whose parents would
  consent, leaving grades 3 and 4 kids as controls.  This would allow any second
  grader who wanted the vaccine to get it.

  problems:
  \begin{itemize}

    \item Parents who consented likely to be richer than parents who didn't.
      This biased the study against the vaccine since the polio rate is higher
      among this group.

    \item Disease was spread through contact.  Since kids tend to hang out with
      their own grade level, the incidence of the disease in a particular grade
      level was likely to be different from the incidence of the disease in the
      other grade levels. This might bias the study for or against the vaccine,
      based on chance.
  \end{itemize}

  Problem: both of these problems are that the control group is different from the
  treatment group.

  Solution: treatment and control have to be drawn from the same population

  \subsection{Final Design}
  Another problem was how to make the treatment and control groups similar.  One
  option would be to look at all the characteristics of the kids (race, income,
  gender, health, number of friends, etc.) and try to get similar kids in each
  group (stratified sample).  Since it was hard to identify everything that
  might matter, they used a simple random sample.

  \begin{itemize*}
    \item 2 million kids
    \item 500,000 vaccinated (treatment)
    \item 500,000 selected for vaccination but declined to participate
    \item 1,000,000 unvaccinated (controls)
  \end{itemize*}<++>


  \part{Miscellaneous Examples}
  \section{Breast Cancer Study}
  Health Insurance Plan of NY (HIP) study in 1963.

  Conclusions:
  \begin{itemize*}
    \item screening didn't affect diseases other than breast cancer
    \item poor less likely to accept screening
    \item most diseases other than breast cancer affect poor more than rich
  \end{itemize*}

  \begin{table}
    \centering
    \begin{tabular}{lrrrrr}
      & & \multicolumn{2}{c}{Breast Cancer} & \multicolumn{2}{c}{All Other} \\
                               \cmidrule(r){3-4} \cmidrule(r){5-6}    
                      &        & number & rate & number & rate \\
      Treatment group \\
      examined        & 20,200 & 23     & 1.1  & 428    & 21 \\
      refused         & 10,800 & 16     & 1.5  & 409    & 38 \\
      total           & 31,000 & 39     & 1.3  & 837    & 27 \\
    \midrule
      control group   & 31,000 & 63     & 2.0  & 879    & 28 \\
    \end{tabular}
    \caption{Breast cancer study: cause of death}
    \label{tab:breast.cancer}
  \end{table}

  \begin{itemize*}
    \item does screening save lives?
    \item why is rate for ``all other'' same for treatment and control groups?
    \item why is rate for all other causes higher for ``refused'' than
      ``examined'' groups?

    \item breast cancer affects rich more than poor.  which numbers confirm
      this?

      \begin{solution}
        Rich people tend to accept.  Income of groups, in order, is examined,
        control, refused, since control has mix of rich and poor and
        examined/refused are segregated by income.  Death rate among control
        group is higher than death rate among refused, and only thing different
        is income.
      \end{solution}

    \item death rate (all causes) among accepted is about half death rate among
      refused.  Did screening cut the death rate in half?  What explains the
      difference in death rate?

    \item Is comparing accepted vs. refused a fair comparison?  Is it biased for
      or against screening?

      \begin{solution}
        Since the people who accepted the screening tend to have more money, and
        money is positively correlated with breast cancer incidence, this
        comparison would be biased against screening.
      \end{solution}

    \item Someone claims that encouraging women to come in for breast cancer
      screening encourages their health conciousness so they live longer for
      that reason.  Is the data consistent or inconsistent with that claim?

      \begin{solution}
        Inconsistent.  The only thing affected by the screening is the breast
        cancer rate.
      \end{solution}

    \item In the first year of the study, 67 breast cancers were found in the
      ``examined'' group, 12 in the refused group, and 58 in the control group.
      True or false: screening causes breast cancer.

      \begin{solution}
        Comparing raw numbers doesn't make sense, since there are different
        numbers in each group.  It makes more sense to look at the percentages
        with cancer detected in each group:
        \begin{itemize*}
          \item examined: 0.33\%
          \item refused: 0.11\%
          \item control: 0.19\%
        \end{itemize*}

        More cancer is detected among the examined because:
        \begin{itemize*}
          \item they have more money, so they are more likely to get breast
            cancer
          \item the regular exams detect cases that go undetected in the other
            groups
        \end{itemize*}

        Control has more than refused because control is slightly richer.
      \end{solution}

  \end{itemize*}
\end{document}

