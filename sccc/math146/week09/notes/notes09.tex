\documentclass[landscape]{exam}

\usepackage{2in1, lscape} 
\usepackage{units} 
\usepackage[fleqn]{amsmath}
\usepackage{float}
\usepackage{mdwlist}
\usepackage{booktabs}
\usepackage{caption}
\usepackage{fullpage}
\usepackage{enumerate}
\usepackage{graphicx}
\usepackage{parskip}

\printanswers

\everymath{\displaystyle}

\printanswers

\title{Statistics \\ Week Nine}
\date{\today}
\author{}

\begin{document}

  \maketitle
  \tableofcontents

  \section{Statistical Inference vs. Exploratory Data Analysis}
  
  Exploratory:
  \begin{itemize*}
    \item given data, draw conclusions
    \item conclusions only apply to data you actually have
  \end{itemize*}

  Statistical Inference:
  \begin{itemize*}
    \item try to gather data to answer a question
    \item conclusions apply to more than just data actually gathered
    \item answers come with a statement of how confident we are that they are
      accurate
  \end{itemize*}

  \section{Sample vs. Population}

  \subsection{Notes}
  \begin{description}
    \item[population] entire group
    \item[sample] group you actually get data from
    \item[sample design] plan for getting a sample that represents population
  \end{description}

  Planning sample survey:
  \begin{itemize*}
    \item What is the population we want to find out about? 
    \item What do we want to find out from each observation? (What are the variables?)
  \end{itemize*}

  \subsection{Examples}

  \subsubsection{US Census}
  \begin{itemize*}
   \item variables: age, income, race, marital, status, number of children, etc.
   \item population: US
  \end{itemize*}

  \subsubsection{Election Polls}
  \begin{itemize*}
   \item variables: who are you going to vote for?
   \item population: probable voters (not just registered voters)
  \end{itemize*}

  \section{Longitudinal Studies} 
  Observe changes in the same sample over a long period of time.

  Framingham Heart Study:
  \begin{itemize*}
    \item 5209 adult residents of Framingham, MA from 1948 to now
    \item height, weight, blood pressure, education, etc.
    \item information on kids 
    \item 2000 academic articles since 1950, 1000 between 2000 and 2009 on
      smoking, cholesterol, heart disease, blood pressure, 
  \end{itemize*}

  With a longitudinal study you don't have to reconstruct the past history of
  the subjects. You don't need to find someone today and ask him what his blood
  pressure was when he was 19, for example.

  \section{Difficulties}

  \subsection{Credits}
  Most of this stuff is from {\em Naked Statistics}.

  \subsection{Selection Bias}
  Sample is not representative of population.  Every member of the population
  doesn't have an equal chance of appearing in the sample.

  Example: Republican Iowa straw poll:
  \begin{itemize*}
    \item August of year before presidential election
    \item Iowa only
    \item \$30 to vote
  \end{itemize*}

  problems:
  \begin{itemize*}
    \item Iowan's may not be representative of Republicans nationwide
    \item people who are willing to pay \$30 to vote may not be representative
      of Republicans nationwide
  \end{itemize*}

  \subsection{Self-Selection Bias}
  People who sign up for AA may not be representative of drug users.  Did they
  stop drinking because of AA or did they sign up for AA because they were
  ready to stop drinking?

  \subsection{Publication Bias}
  Every study has some possibility of error.

  Suppose study is 95\% likely to be correct.  1 out of 20 times the study
  returns the wrong result. To demonstrate a new drug is effective, a drug
  company just needs to do 20 studies, discard the 19 that say the drug does
  nothing, and publish study number 20.

  The press release says: ``With 95\% certainty, New Miracle Drug cures insomnia.''

  With Prozac, 94\% of the positive studies were published and 14\% of the
  non-positive studies were published.

  \subsection{Recall Bias}
  Current events may affect how people remember the past.  

  In 1993 study of diet/breast cancer, people with breast cancer reported high
  fat diets earlier in life.  Cause was people struggling to think of
  explanation for cancer rather than actual high-fat diets.

  \begin{em}
    The diagnosis of breast cancer had not just changed a woman’s present and the
    future; it had altered her past. Women with breast cancer had (unconsciously)
    decided that a higher-fat diet was a likely predisposition for their disease
    and (unconsciously) recalled a high-fat diet.
  \end{em}

  Longitudinal study wouldn't be susceptible to this problem.

  \subsection{Survivorship Bias}
  In a longitudinal study, the sample may change over time.

  GPA of high school students who haven't dropped out goes up.  Sophomores do
  better than freshman, juniors to better than sophomores, etc. because the bad
  students are the ones that drop out.

  A mutual fund company can start 16 mutual funds.  By chance, 8 are above
  the index after year 1, 4 after year 2, and 2 after year 3.  Below average
  funds quietly shut down and above average funds heavily advertised.

  \subsection{Convenience Sample}
  Sample that is easiest to gather:

  \begin{itemize*}
    \item ask your friends about their views on something to estimate US population's opinion.
    \item observe cars in your neighborhood to find most common car in US (my
      neighborhood has disproportionate number of SUVs)
    \item UBB students as sample of US prison population (WA may not be typical.
      UBB students may not be typical)
  \end{itemize*}

  \subsection{Self-Selecting Sample}
  \begin{itemize*}
    \item web poll
    \item call in radio poll
    \item Amazon ``Did you find what you were looking for?''
  \end{itemize*}

  problems:

  For any poll, if the population is the US population:
  \begin{itemize*}
    \item People who know about the poll may not be typical (listeners to the
      show, visitors to the web site)
    \item People who actually respond may be even less typical (too much time on
      their hands, feel very strongly)
  \end{itemize*}

  \section{Polling}
  \begin{itemize*}
    \item nonresponse
    \item rigged questions
  \end{itemize*}

  \section{Simple Random Sample}
  \begin{itemize*}
    \item table of random numbers in back of book
  \end{itemize*}

  \section{Stratified Sample}

  \section{Examples}

  \subsection{Jury Selection}

\end{document}

