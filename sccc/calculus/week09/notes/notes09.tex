\documentclass[letterpaper, landscape]{exam}
\usepackage{2in1, lscape} 
\printanswers

\usepackage{units} 
\usepackage[fleqn]{amsmath}
\usepackage{float}
\usepackage{mdwlist}
\usepackage{booktabs}
\usepackage{caption}
\usepackage{fullpage}
\usepackage{enumerate}
\usepackage{graphicx}
\usepackage[justification=justified]{caption}

\setcounter{tocdepth}{1}
\everymath{\displaystyle}

\author{}
\date{\today}
\title{Calculus I \\ Week Eight}

\begin{document}

  \maketitle
  \tableofcontents

  \part{Polynomials} % (fold)
  
  \section{Multiply by Constant} % (fold)
  \begin{align*}
    \frac{d}{dx} c f(x) & = \lim_{h \to 0} \frac{c f(x + h) - cf(x)}{h} \\
                        & = c \cdot \frac{f(x + h) - f(x)}{h} \\
                        & = c f'(x) \\
  \end{align*}

  \section{Derivative of Sums and Differences} % (fold)
  
  \begin{align*}
    \frac{d}{dx} (f(x) + g(x)) & = \lim_{h \to 0} \frac{f(x + h) + g(x + h) - (f(x) + g(x)) }{h} \\
                               & = \lim_{h \to 0} \frac{f(x + h) - f(x)}{h} + \frac{g(x + h) - g(x)}{h} \\
                               & = f'(x) + g'(x) \\
  \end{align*}

  Same reasoning applies for differences. Or you can think of subtraction as multiplying by $-1$ and adding.

  \section{Derivatives of $x^n$}

  \begin{align*}
    \frac{d}{dx} x^2 & = \lim_{h \to 0} \frac{(x + h)^2 - x^2}{h} \\
                     & = \lim_{h \to 0} \frac{x^2 + 2xh + h^2 - x^2}{h} \\
                     & = \lim_{h \to 0} \frac{2xh + h^2}{h} \\
                     & = 2x \\
    \\
    \frac{d}{dx} x^3 & = \lim_{h \to 0} \frac{(x + h)^3 - x^3}{h} \\
                     & = \frac{x^3 + 3x^2h + 3xh^2 + h^3 - x^3}{h} \\
                     & = 3x^2 \\
  \end{align*}

  Talk about Pascal's Triangle.

  general rule:
  \begin{align*}
    \frac{d}{dx} x^n & = \lim_{h \to 0} \frac{(x + h)^n - x^n}{h} \\
                     & = \lim_{h \to 0} \frac{x^n + nx^{n - 1}h + \ldots - x^n}{h} \\
                     & = \lim_{h \to 0} \frac{x^{n - 1}h + \ldots}{h} \\
                     & = n x^{n-1} \\
  \end{align*}

  The same rule works for any real number $n$.

  examples
  \begin{align*}
    \frac{d}{dx} x^7     & = 7x^6 \\
    \frac{d}{dx} x^{-1}  & = -x^{-2} = - \frac{1}{x^2} \\
    \frac{d}{dx} x^{1/3} & = \frac{1}{3} x^{-2/3} = \frac{1}{3 x^{2/3}} \\
  \end{align*}

  \section{Derivatives of Polynomials} % (fold)
  
  Polynomials are just combinations of adding, subtracting, and multiplying by constants. 
  
  Do examples.

  \part{Exponentials} % (fold)
  
  \section{Interest}

  \subsection{Simple Interest} % (fold)
  \[
    A = P + Prt
  \]

  \$1,000 at 5\% interest for 10 years is:
  \[
    A = 1000 + 1000 \cdot 0.05 \cdot 10 = \$1,500
  \]

  \subsection{Compound Interest} % (fold)

  \subsubsection{Compound Yearly} % (fold)
  \begin{align*}
    A_0 &= P + Pr = P(1 + r) \\
    A_1 &= A_0 + A_0 r = A_0 (1 + r) = P(1 + r)^2 \\
    A_2 &= A_1 + A_1 r = A_1 (1 + r) = P(1 + r)^3 \\
    \vdots \\
    A(t) &= P(1 + r)^t \\
  \end{align*}
  
  \$1,000 at 5\% interest for 10 years is:
  \[
    A = 1,000 \cdot (1 + 0.05)^{10} = \$1,628.89
  \]

  \subsubsection{Compounded More Frequently} % (fold)

  $n$ compoundings per year
  \begin{align*}
    A(t) &= P \left(1 + \frac{r}{n} \right)^nt
  \end{align*}
  
  Compounded monthly
  \[
    A = 1,000 \cdot \left( 1 + \frac{0.05}{12} \right)^{10 \cdot 12} = \$1,647.01
  \]

  Compounded daily
  \[
    A = 1,000 \cdot \left( 1 + \frac{0.05}{8760} \right)^{10 \cdot 8760} = \$1,648.66
  \]

  Compounded hourly
  \[
    A = 1,000 \cdot \left( 1 + \frac{0.05}{8760} \right)^{10 \cdot 8760} = \$1,648.72
  \]

  \subsection{Continuous Compounding} % (fold)
  
  \begin{align*}
    A   & = \lim_{n \to \infty} A \left( 1 + \frac{r}{n} \right)^{nt}
    \\
    x   & = \frac{r}{n} \
    \\
    A   & = \lim_{x \to 0} A \left( 1 + x \right)^{rt/x} \\
        & = \lim_{x \to 0} A \left[ \left( 1 + x \right)^{1/x} \right]^{rt} \\
    \\
    e   & = \lim_{x \to 0} \left( 1 + x \right)^{1/x} \\
    \\
    A_t & = P e^{rt} \\
  \end{align*}

  \section{Calculating e} % (fold)
  
  \begin{align*}
    (a & + b)^n = a^n + a^{n - 1} b \cdot \frac{n}{1!} + a^{n - 2} b^2 \cdot \frac{n(n - 1)}{2!} \\
       & + a^{n - 3} b^3 \cdot \frac{n(n - 1)(n - 2)}{3!} + \ldots + b^n
  \end{align*}

  \begin{align*}
    e &= \lim_{n \to \infty} \left( 1 + \frac{1}{n} \right)^n \\
      & = \lim_{n \to \infty} \left( 1 + \frac{1}{n} \cdot n + \frac{1}{n^2} \cdot \frac{n(n - 1)}{2!} 
        + \frac{1}{n^3} \cdot \frac{n(n - 1)(n - 2)}{3!} + \ldots \right) \\
      & = 1 + 1 + \frac{1}{2!} + \frac{1}{3!} + \frac{1}{4!} + \ldots
    \\
    e^x &= \lim_{n \to \infty} \left( 1 + \frac{1}{n} \right)^{nx} \\
      & = \lim_{n \to \infty} \left( 1 + \frac{1}{n} \cdot nx + \frac{1}{n^2} \cdot \frac{nx(nx - 1)}{2!} 
        + \frac{1}{n^3} \cdot \frac{nx(nx - 1)(nx - 2)}{3!} + \ldots \right) \\
      & = 1 + x + \frac{x^2}{2!} + \frac{x^3}{3!} + \frac{x^4}{4!} + \ldots
  \end{align*}

  \section{Derivative of $a^x$} % (fold)

  approach one:
  \begin{align*}
    \frac{d}{dx} a^x & = \lim_{h \to 0} \frac{a^{x + h} - a^x}{h} \\
                     & = \lim_{h \to 0} \frac{a^x \left( a^h - 1 \right)}{h} \\
                     & = a^x \lim_{h \to 0} \frac{a^x \left( a^h - 1 \right)}{h} \\
  \end{align*}

  Find $a$ where limit is 1 for simplest result:
  \begin{align*}
    \frac{a^h - 1}{h} & = 1 \\
    a^h - 1           & = h \\
    a^h               & = h + 1 \\
    a                 & = \left( h + 1 \right)^{1/h} \\
    \lim_{h \to 0} a  & = \lim_{h \to 0} \left( h + 1 \right)^{1/h} \\
    a                 & = \lim_{h \to 0} \left( h + 1 \right)^{1/h} \\
    a                 & = e \\
    \frac{d}{dx} e^x  & = e^x \\
  \end{align*}

  approach two:
  \begin{align*}
    \frac{d}{dx} e^x &= \frac{d}{dx} \left( 1 + x + \frac{x^2}{2!} + \frac{x^3}{3!} + \ldots  \right) \\
                     &= 0 + 1 + \frac{2 x}{2!} + \frac{3x^2}{3!} + \ldots \\
                     &= 1 + x + \frac{x^2}{2!} + \frac{x^3}{3!} + \ldots \\
    \ldots
  \end{align*}
  

\end{document}

