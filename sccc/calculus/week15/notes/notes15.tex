\documentclass[letterpaper, landscape]{exam}
\usepackage{2in1, lscape} 
\printanswers

\usepackage{units} 
\usepackage[fleqn]{amsmath}
\usepackage{float}
\usepackage{mdwlist}
\usepackage{booktabs}
\usepackage{caption}
\usepackage{fullpage}
\usepackage{enumerate}
\usepackage{graphicx}
\usepackage[justification=justified]{caption}

\setcounter{tocdepth}{1}
\everymath{\displaystyle}

\author{}
\date{\today}
\title{Calculus I \\ Week Fifteen}

\begin{document}

  \maketitle
  \tableofcontents
  \section{Homework 13} % (fold)
  \label{sec:Homework 13}

  When finding the second derivative using implicit differentiation, make sure to go back and plug
  in the first derivative in the answer. Answer key didn't do this.
  
  \section{Minimum/Maximum} % (fold)

  \begin{itemize}
    \item absolute min/max is smallest/largest $f(x)$ in the domain
    \item Local max at $x = a$ means for all $b$ in some open interval around $a$, $f(a) > f(b)$
  \end{itemize}

  \section{Extreme Value Theorem} % (fold)
  
  If $f$ is continuous on closed interval $[a, b]$ then it has a maximum and minimum in that
  interval.

  Continuous and closed matter because you might have a vertical asymptote or undefined value where
  \[
    \lim_{x \to a} f(a) = L
  \]

  but $f(a)$ isn't defined so there is no maximum
  
  \section{Fermat's Theorem} % (fold)
  
  two ways of saying it:
  \begin{itemize*}
    \item If $f$ has a local min/max at $c$ and $f'(c)$ exists, then $f'(c) = 0$ 
    \item If $f$ has a local min/max at $c$ then $f'(c) = 0$ or $f'(c)$ doesn't exist.
  \end{itemize*}

  \begin{itemize*}
    \item Draw local max with $f'$ exists and $f'$ doesn't exist.
    \item Draw $f(x) = x^3$ where $f'(0) = 0$ and there is no min/max at $x = 0$
    \item Draw $f(x) = \frac{1}{x}$ where $f'(0)$ doesn't exist and there is no min/max at $x = 0$
  \end{itemize*}

\end{document}

