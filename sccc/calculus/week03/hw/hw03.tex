% no answer key
% \documentclass[letterpaper]{exam}

% answer key
\documentclass[letterpaper, landscape]{exam}
\usepackage{2in1, lscape} 
\printanswers

\usepackage{units} 
\usepackage{xfrac} 
\usepackage[fleqn]{amsmath}
\usepackage{cancel}
\usepackage{float}
\usepackage{mdwlist}
\usepackage{booktabs}
\usepackage{cancel}
\usepackage{polynom}
\usepackage{caption}
\usepackage{fullpage}
\usepackage{comment}
\usepackage{enumerate}
\usepackage{graphicx}

\newcommand{\dg}{\ensuremath{^\circ}} 
\everymath{\displaystyle}

\title{Calculus I \\ Homework Three \\ Section 1.3}
\author{}
\date{\today}

\begin{document}

  \maketitle

  \section{Homework}
    \begin{itemize*}
      \item read Section 1.3
      \item exercises: TO DO
    \end{itemize*}

  \ifprintanswers

    \section{Solutions}

    \begin{description}

      \item[31]
        \begin{align*}
          (f \circ g)(x) &= 4x^2 + 4x \\
          (g \circ f)(x) &= 2x^2 - 1 \\
          (f \circ f)(x) &= x^4 - 2x^2 \\
          (g \circ g)(x) &= 4x + 3 \\
        \end{align*}

      \item[33]
        \begin{align*}
          (f \circ g)(x) &= 1 - 3 \cos{x} \\
          (g \circ f)(x) &= \cos{ (1 - 3x) } \\
          (f \circ f)(x) &= 9x - 2 \\
          (g \circ g)(x) &= \cos { (\cos { x }) } \\
        \end{align*}

      \item[36]
        \begin{align*}
          (f \circ g)(x) &= \frac{\sin (2 x) }{\sin (2 x) +1} \\
          (g \circ f)(x) &= \sin \left( \frac{ 2x }{x+1} \right) \\
          (f \circ f)(x) &= \frac{x}{2x + 1} \\
          (g \circ g)(x) &= \sin ( 2 \sin 2x) \\
        \end{align*}

      \item[37]
        \[
          (f \circ g \circ h)(x) = 2x - 1 \\
        \]
  
      \item[38]
        \[
          (f \circ g \circ h)(x) = 2 x^2-4 x+1 \\
        \]
  
      \item[41]
        \begin{align*}
          f(x) &= x^{10} \\
          g(x) &= x^2 + 1 \\
        \end{align*}

      \item[42]
        \begin{align*}
          f(x) &= \sin x \\
          g(x) &= \sqrt{x} \\
        \end{align*}

      \item[43]
        \begin{align*}
          f(x) &= \frac{1}{1 + x} \\
          g(x) &= \sqrt[3]{x} \\
        \end{align*}

      \item[44]
        \begin{align*}
          f(x) &= \sqrt[3]{x} \\
          g(x) &= \frac{1}{1 + x} \\
        \end{align*}

      \item[45]
        \begin{align*}
          f(x) &= \sqrt{x} \\
          g(x) &= \cos x \\
        \end{align*}

      \item[46]
        \begin{align*}
          f(x) &= \frac{1}{1 + x} \\
          g(x) &= \tan x \\
        \end{align*}

      \item[47]
        \begin{align*}
          h(x) &= x^2 \\
          g(x) &= 3^x \\
          f(x) &= 1 - x \\
        \end{align*}
    \end{description}


  \else
    \vspace{11 cm}
    \begin{quote}
      \begin{em}
        TO DO
      \end{em}
    \end{quote}
    \hspace{1 cm} --Malcolm X
  \fi

\end{document}

