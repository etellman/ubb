\documentclass[letterpaper, landscape]{exam}
\usepackage{2in1, lscape} 
\printanswers

\usepackage{units} 
\usepackage[fleqn]{amsmath}
\usepackage{float}
\usepackage{mdwlist}
\usepackage{booktabs}
\usepackage{caption}
\usepackage{fullpage}
\usepackage{enumerate}
\usepackage{graphicx}

\setcounter{tocdepth}{1}
\everymath{\displaystyle}

\author{}
\date{\today}
\title{Calculus I \\ Week Three}

\begin{document}

  \maketitle
  \tableofcontents

  \section{Limit Laws}

  \subsection{Addition}
  \begin{align}
    \lim_{x \to a} \left[ f(x) + g(x) \right] 
    &= \lim_{x \to a} f(x) + \lim_{x \to a} g(x) \label{eq:addition} \\
    \lim_{x \to a} c f(x) &= c \lim_{x \to a} f(x) \label{eq:constant_factor} \\
    \lim_{x \to a} c &= c \\
  \end{align}

  \subsection{Subtraction}
  \begin{align}
    h(x)                & = -1 \cdot g(x) \\
    \lim_{x \to a} h(x) & = \lim_{x \to a} \left[ (-1) g(x) \right] \\
                        & = - \lim_{x \to a} g(x) \\
    \\
    \lim_{x \to a} \left[ f(x) + h(x) \right] 
               & = \lim_{x \to a} f(x) + \lim_{x \to a} h(x) \\
               & = \lim_{x \to a} f(x) - \lim_{x \to a} g(x) \\
               & = \lim_{x \to a} \left[ f(x) - g(x) \right]
  \end{align}

  \subsection{Multiplication and Division}
  \begin{align}
    \lim_{x \to a} \left[ f(x) \cdot g(x) \right] 
      &= \lim_{x \to a} f(x) \cdot \lim_{x \to a} g(x) \\
    \lim_{x \to a} \frac{f(x)}{g(x)} 
      &= \frac{\lim_{x \to a} f(x)}{\lim_{x \to a} g(x)} 
  \end{align}

  \subsection{Powers}
  \begin{align}
    \lim_{x \to a} \left[ f(x) \right]^n = \left[ \lim_{x \to a} [f(x)]^n \right] \label{eq:power}
    \lim_{x \to a} \sqrt[n]{x} = \sqrt[n]{x} \\
    \lim_{x \to a} \sqrt[n]{ f(x) } = \sqrt[n]{ f(x) } \\
  \end{align}

  \subsection{Polynomials and Rational}
  From Equations \ref{eq:power}, \ref{eq:addition}, and
  \ref{eq:constant_factor}, the limit for a polynomial is the sum of the limit
  of its terms.

  Do some polynomial examples with justifications for each step.

  Do some rational examples where the limit of the denominator is not zero.

  \section{Rational Functions}

  \subsection{Overview}
  If the limit of the numerator and denominator are both zero, sometimes things
  factor and you get an actual number.

  \subsection{Simple Examples}

  \begin{enumerate}
    \item 
      \[
        \lim_{x \to 3} \frac{2x^2 - 5x - 3}{x - 3} = 7 
      \]
  \end{enumerate}

  \subsection{Difference of Squares}

  \subsubsection{Formulas}
  \[
    a^2 - b^2 = (a + b)(a - b)
  \]

  \subsubsection{Examples}
  \begin{enumerate}
    \item 
      \[
        \lim_{x \to 3} \frac{x^2 - 9}{x + 3} = \boxed{ -2 }
      \]

    \item 
      \[
        \lim_{x \to 4} \frac{4 - x}{2 - \sqrt{x}} = \boxed{ 4 }
      \]

    \item 
      \[
        \lim_{x \to 4} \frac{x^4 - 16}{x + 2} = \boxed{ 4 }
      \]
  \end{enumerate}

  \subsection{Sum/Difference of Cubes}

  \subsubsection{Formulas}
  \begin{align*}
    a^3 + b^3 &= (a+b) \left(a^2-a b+b^2\right) \\
    a^3 - b^3 &= (a-b) \left(a^2+a b+b^2\right) \\
  \end{align*}

  Show how to get formulas with polynomial long division.

  \subsubsection{Examples}
  \begin{enumerate}
    \item 
      \[
        \lim_{x \to -2} \frac{x^3 + 8}{x + 2} = \boxed{ 12 }
      \]

    \item 
      \[
        \lim_{x \to 1} \frac{x - 1}{x^3 - 1} = \boxed{ \frac{1}{3} }
      \]

    \item 
      \begin{align*}
        \lim_{x \to 0} & \frac{(x + 3)^3 - 27}{x} \\
          & = \lim_{x \to 0} \frac{( (x + 3) - 3) \left( (x + 3)^2 +3(x + 3) + 9 \right)}{x} \\
          & = \lim_{x \to 0} \left( x^2 + 6x + 9 + 3x + 9 + 9 \right) \\
          & = \lim_{x \to 0} \left( x^2 + 9x + 27 \right) \\
          & = \boxed{ 27 } \\
          \\
      \end{align*}

  \end{enumerate}


  \subsection{Rationalizing the Numerator/Denominator}

  Multiply by the conjugate.

  \begin{enumerate}
    \item 
      \[
        \lim_{x \to 0} \frac{\sqrt{x + 4} - 2}{x} = \boxed{ \frac{1}{4} }
      \]

    \item 
      \[
        \lim_{x \to -1} \frac{x+1}{\sqrt{x+2}-1} = \boxed{ 2 }
      \]
  \end{enumerate}

  \subsection{Miscellaneous}

  \begin{enumerate}
    \item 
      \[
        \lim_{x \to 0} \frac{1}{2x} - \frac{1}{x^3 + 2x} = \boxed{ \frac{1}{4} }
      \]
  \end{enumerate}


  \section{Squeeze Theorem}
  
  If $f(x) \leq g(x) \leq h(x)$ near $a$ and 
  $\lim_{x \to a} f(x) = \lim_{x \to a} h(x)$, then $\lim_{x \to a} g(x) = a$.

  \begin{enumerate}
    \item 
      \[
        \lim_{x \to 0} \left( x^2 + 1 \right) = 0 
      \]

      use $\lim_{x \to 0} \left( |x| + 1 \right) = 1$

    \item
      \[
        \lim_{x \to 0} \frac{|x|}{\sqrt{x^4 + 4x^2 + 7}} = 0
      \]

      Use $f(x) = 0$ and $g(x) = |x|$

  \end{enumerate}

  \section{Absolute Value}

  Absolute value limits exist when the right and left hand limits match. Show
  how to write absolute value as two cases.

  \begin{enumerate}
    \item 
      \[
        \lim_{x \to 3} | x - 3 | = \boxed{ 0 }
      \]
  \end{enumerate}

\end{document}

