% no answer key
\documentclass[letterpaper]{exam}

% answer key
% \documentclass[letterpaper, landscape]{exam}
% \usepackage{2in1, lscape} 
% \printanswers

\usepackage{units} 
\usepackage{xfrac} 
\usepackage[fleqn]{amsmath}
\usepackage{float}
\usepackage{mdwlist}
\usepackage{booktabs}
\usepackage{cancel}
\usepackage{polynom}
\usepackage{caption}
\usepackage{fullpage}
\usepackage{comment}
\usepackage{enumerate}
\usepackage{graphicx}

\usepackage{mathtools} 

\newcommand{\dg}{\ensuremath{^\circ}} 
\newcommand{\sgn}{\operatorname{sgn}}

\everymath{\displaystyle}
\title{Calculus I \\ Homework Four \\ Section 2.4}
\author{}
\date{\today}

\begin{document}

  \maketitle

  \section{Homework}
    \begin{itemize*}
      \item read Section 2.4
      \item exercises: TO DO
    \end{itemize*}

  \ifprintanswers

    \section{Solutions}

    \begin{description}

      \item[1]
        \begin{enumerate}[(a)]
          \item $\lim_{x \to 2} \left[ f(x) + 5g(x) \right] = \boxed{ -6 }$
          \item $\lim_{x \to 2} \left[ g(x) \right]^3 = \boxed{ -8 }$
          \item $\lim_{x \to 2} \sqrt{f(x)} = \boxed{ 2 }$
          \item $\lim_{x \to 2} \frac{3 f(x)}{g(x)} = \boxed{ -6 }$
          \item $\lim_{x \to 2} \frac{g(x) h(x)}{f(x)} \boxed{ 0 }$ 
        \end{enumerate}

    \end{description}

  \else
    \vspace{11 cm}
    \begin{quote}
      \begin{em}
        Striving for peace and preparing for war are incompatible with each
        other, and in our time more so than ever. 
      \end{em}
    \end{quote}
    \hspace{1 cm} --Albert Einstein
  \fi

\end{document}

