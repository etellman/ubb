% no answer key
% \documentclass[letterpaper]{exam}

% answer key
\documentclass[letterpaper, landscape]{exam}
\usepackage{2in1, lscape} 
\printanswers

\usepackage{units} 
\usepackage{xfrac} 
\usepackage[fleqn]{amsmath}
\usepackage{float}
\usepackage{mdwlist}
\usepackage{booktabs}
\usepackage{cancel}
\usepackage{polynom}
\usepackage{caption}
\usepackage{fullpage}
\usepackage{comment}
\usepackage{enumerate}
\usepackage{graphicx}

\usepackage{mathtools} 

\newcommand{\dg}{\ensuremath{^\circ}} 
\newcommand{\sgn}{\operatorname{sgn}}

\everymath{\displaystyle}
\title{Calculus I \\ Homework Four \\ Section 2.4}
\author{}
\date{\today}

\begin{document}

  \maketitle

  \section{Homework}
    \begin{itemize*}
      \item read Section 2.4
      \item exercises: 1-4, 11, 13-22, 27-32, 36-37
    \end{itemize*}

  \ifprintanswers

    \section{Solutions}

    \begin{description}

      \item[1] $\delta = \boxed{ \frac{3}{7} }$

      \item[2] $\delta = \boxed{ 0.7 }$

      \item[3]
        \begin{align*}
          1.6^2     & \approx 2.56 \\
          2.4^2     & \approx 5.76 \\
          \\
          4 - 1.6^2 & = 1.44 \\
          2.4^2 - 4 & = 1.76 \\
          \\
          \delta    & = \boxed{ 1.44 } \\
        \end{align*}

      \item[4]
        \begin{align*}
          \sqrt{0.5}     & \approx 0.7071 \\
          \sqrt{1.5}     & \approx 1.2247 \\
          \\
          1 - \sqrt{0.5} & \approx 0.2929 \\
          \sqrt{1.5} - 1 & \approx 0.2247 \\
          \\
          \delta    & = \boxed{ 0.2247 } \\
        \end{align*}

      \item[11]
        \begin{enumerate}[(a)]
          \item 
            \begin{align*}
              \pi r^2 & = 1000 \\
              r       & \approx \boxed{ 17.8412 } \\
            \end{align*}

          \item The area must be between $\unit[999]{cm^2}$ and $\unit[1005]{cm^2}$. This
            means the radius must be between $\unit[17.1966]{cm}$ and $\unit[17.8588]{cm}$

            This can be accomplished with a tolerance of $r = 17.8412 \pm \boxed{ 0.0176 }$.

          \item 
            \begin{itemize*}
              \item $x$ is the radius
              \item $f(x)$ is the area
              \item $a$ is the target radius of $\unit[17.8412]{cm}$
              \item $L$ is the target area of $\unit[1000]{cm^2}$
              \item $\epsilon$ is $\unit[0.5]{cm^2}$
              \item $\delta$ is $\unit[0.0176]{cm}$
            \end{itemize*}
        \end{enumerate}

      \item[13]
        \begin{enumerate}[(a)]
          \item 
            \begin{align*}
              |4x - 8|    & = 4 |x - 2| \\
              4 | x - 2 | & < 0.1 \\
              |x - 2|     & < 0.025 \\
              \\
              \delta      & = \boxed{ 0.025 } \\
            \end{align*}

          \item $\delta = \boxed{ 0.0025 }$ 
        \end{enumerate}

      \item[14]
        \begin{align*}
          |5x - 7 - 3| & = 5 |x - 2| \\
          \\
          5 | x - 2 |  & < \epsilon \\
          |x - 2|      & < \frac{\epsilon}{5} \\
        \end{align*}

        \begin{tabular}[H]{rr}
          \toprule
          $\epsilon$ & $\delta$ \\
          \midrule
          $0.1$      & $0.02$ \\
          $0.05$     & $0.01$ \\
          $0.01$     & $0.002$ \\
          \bottomrule
        \end{tabular}

      \pagebreak

      \item[15] Prove that $\lim_{x \to 1} 2x + 3 = 5$

        Show that for all $\epsilon$ there is a $\delta$ such that if 
        $0 < |x - 1| < \delta$ 
        
        then $|(2x + 3) - 5| < \epsilon$

        \paragraph{preliminary investigation}
        \begin{align*}
          |(2x + 3) - 5| & = 2|x - 1| \\
          \delta         & = \frac{\epsilon}{2} \\
        \end{align*}

        \paragraph{proof}
        Show that if $\delta = \frac{\epsilon}{2}$ and $0 < |x - 1| < \delta$

        then $|(2x + 3) - 5| < \epsilon$
        \[
          |(2x + 3) - 5| = 2|x - 1| < 2 \delta = 2 \cdot \frac{\epsilon}{2} = \epsilon
        \]

      \newpage

      \item[16] Prove that $\lim_{x \to -2} \left( \frac{1}{2} x + 3 \right) = 2$

        Show that for all $\epsilon$ there is a $\delta$ such that if 
        $0 < |x + 2| < \delta$ 
        
        then $ \left| \left( \frac{1}{2} x + 3 \right) - 2 \right| < \epsilon$

        \paragraph{preliminary investigation}
        \begin{align*}
          \left| \left( \frac{1}{2} x + 3 \right) - 2 \right| & = \frac{1}{2}|x + 2| \\
          \delta         & = 2 \epsilon \\
        \end{align*}

        \paragraph{proof}
        Show that if $\delta = 2 \epsilon$ and $0 < |x + 2| < \delta$
        then $|(2x + 3) - 5| < \epsilon$
        \[
          \left| \left( \frac{1}{2} x + 3 \right) - 2 \right| = \frac{1}{2} |x + 2| < \frac{1}{2} \delta = \epsilon
        \]

      \item[17] Prove that $\lim_{x \to -3} ( 1 - 4x ) = 13$

        Show that for all $\epsilon$ there is a $\delta$ such that if 
        $0 < |x + 3| < \delta$ 
        
        then $ | (1 - 4x) - 13 | < \epsilon$

        \paragraph{preliminary investigation}
        \begin{align*}
          | (1 - 4x) - 13 | & = 4 |x + 3| \\
          \delta         & = \frac{\epsilon}{4} \\
        \end{align*}

        \paragraph{proof}
        Show that if $\delta = \frac{\epsilon}{4}$ and $0 < |x + 3| < \delta$

        then $|(1 - 4x) - 13| < \epsilon$
        \[
          |(1 - 4x) - 13| = 4 |x + 3| < 4 \delta = \epsilon
        \]

      \item[18] Prove that $\lim_{x \to 4} ( 7 - 3x ) = -5$

        Show that for all $\epsilon$ there is a $\delta$ such that if 
        $0 < |x - 4| < \delta$ 
        
        then $ | (7 - 3x) + 5 | < \epsilon$

        \paragraph{preliminary investigation}
        \begin{align*}
          | (7 - 3x) + 5 | & = 3 |x - 3| \\
          \delta         & = \frac{\epsilon}{3} \\
        \end{align*}

        \paragraph{proof}
        Show that if $\delta = \frac{\epsilon}{3}$ and $0 < |x - 4| < \delta$

        then $|(7 - 3x) + 5| < \epsilon$
        \[
          |(7 - 3x) + 5| = 3 |x - 3| < 3 \delta = \epsilon
        \]

      \newpage

      \item[19] Prove that $\lim_{x \to 3} \frac{x}{5} = \frac{3}{5}$

        Show that for all $\epsilon$ there is a $\delta$ such that if 
        $0 < |x - 3| < \delta$ then $ \left| \frac{x}{5} - \frac{3}{5} \right| < \epsilon$

        \paragraph{preliminary investigation}
        \begin{align*}
          \left| \frac{x}{5} - \frac{3}{5} \right| & = \frac{1}{5} |x - 3| \\
          \delta                                   & = 5 \epsilon \\
        \end{align*}

        \paragraph{proof}
        Show that if $\delta = 5 \epsilon$ and $0 < |x - 3| < \delta$

        then $\left| \frac{x}{5} - \frac{3}{5} \right| < \epsilon$

        \[
          \left| \frac{x}{5} - \frac{3}{5} \right| = \frac{1}{5} |x - 3| < \frac{1}{5} \delta = \epsilon
        \]

      \item[20] Prove that $\lim_{x \to 6} \left( \frac{x}{4} + 3 \right) = \frac{9}{2}$

        Show that for all $\epsilon$ there is a $\delta$ such that if 
        $0 < |x - 6| < \delta$ 
        
        then $ \left| \left( \frac{x}{4} + 3 \right) - \frac{9}{2} \right| < \epsilon$

        \paragraph{preliminary investigation}

        \begin{align*}
          \left| \left( \frac{x}{4} + 3 \right) - \frac{9}{2} \right| & = \frac{1}{4} |x - 6| \\
          \delta                                               & = 4 \epsilon \\
        \end{align*}

        \paragraph{proof}
        Show that if $\delta = 4 \epsilon$ and $0 < |x - 6| < \delta$

        then $\left| \left( \frac{x}{4} + 3 \right) - \frac{9}{2} \right| < \epsilon$
        \[
          \left| \left( \frac{x}{4} + 3 \right) - \frac{9}{2} \right| = \frac{1}{4} |x - 6| < \frac{1}{4} \delta = \epsilon
        \]

      \item[21] Prove that $\lim_{x \to 2} \left( \frac{x^2 + x - 6}{x - 2} \right) = 5$

        Show that for all $\epsilon$ there is a $\delta$ such that if 
        $0 < |x - 2| < \delta$ then $ \left| \left( \frac{x^2 + x - 6}{x - 2} \right) - 5 \right| < \epsilon$

        \paragraph{preliminary investigation}
        \begin{align*}
          \left| \left( \frac{x^2 + x - 6}{x - 2} \right) - 5 \right| & = |x - 2| \\
          \delta                                                      & = \epsilon \\
        \end{align*}

        \paragraph{proof}
        Show that if $\delta = \epsilon$ and $0 < |x - 2| < \delta$

        then $\left| \left( \frac{x^2 + x - 6}{x - 2} \right) - 5 \right| < \epsilon$

        \[
          \left| \left( \frac{x^2 + x - 6}{x - 2} \right) - 5 \right| = |x - 2| < \delta = \epsilon
        \]

      \newpage

      \item[22] Prove that $\lim_{x \to 1.5} \left( \frac{9 - 4x^2}{3 + 2x} \right) = 6$

        Show that for all $\epsilon$ there is a $\delta$ such that if 
        $0 < |x - 1.5| < \delta$ then $ \left| \left( \frac{9 - 4x^2}{3 + 2x} \right) - 6 \right| < \epsilon$

        \paragraph{preliminary investigation}
        \begin{align*}
          \left| \left( \frac{9 - 4x^2}{3 + 2x} \right) - 6 \right| & = 2 |x + 1.5| \\
          \delta                                                    & = \frac{\epsilon}{2} \\
        \end{align*}

        \paragraph{proof}
        Show that if $\delta = \frac{\epsilon}{2}$ and $0 < |x - 1.5| < \delta$

        then $\left| \left( \frac{9 - 4x^2}{3 + 2x} \right) - 6 \right| < \epsilon$

        \[
          \left| \left( \frac{9 - 4x^2}{3 + 2x} \right) - 6 \right| = 2 |x + 1.5| < 2 \delta = \epsilon
        \]

      \item[27] Prove that $\lim_{x \to 0} |x| = 0$

        Show that for all $\epsilon$ there is a $\delta$ such that if 
        $0 < |x - 0| < \delta$ then $ \left| |x| - 0 \right| < \epsilon$

        \paragraph{proof}
        Show that if $\delta = \epsilon$ and $0 < |x - 0| < \delta$
        then $| |x| - 0| < \epsilon$

        \[
          | |x| - 0| = |x| < \delta = \epsilon
        \]

      \item[28] Prove that $\lim_{x \to 9^-} \sqrt[4]{9 - x} = 0$

        Let $y = 9 - x$

        Then we need to prove that: $\lim_{y \to 0^+} \sqrt[4]{y} = 0$

        Show that for all $\epsilon$ there is a $\delta$ such that if 
        $0 < |y| < \delta$ 
        
        then $ \left| \sqrt[4]{y} \right| < \epsilon$

        \paragraph{proof}
        Show that if $\delta = \min \{ \epsilon^4, 1 \}$ and 
        $0 < |y| < \delta$ 
        
        then $| \sqrt[4]{y} | < \epsilon$

        \[
          | \sqrt[4]{y} | = \sqrt[4]{y} < \delta = \epsilon
        \]

      \item[29] Prove that $\lim_{x \to 2} \left( x^2 - 4x + 5 \right) = 1$

        Show that for all $\epsilon$ there is a $\delta$ such that if 
        $0 < |x - 1| < \delta$ then $ \left| \left( x^2 - 4x + 5 \right) - 1 \right| < \epsilon$

        \paragraph{preliminary investigation}
        \begin{align*}
          \left| \left( x^2 - 4x + 5 \right) - 1 \right| & = |(x - 2)^2| \\
          \delta                                         & = \sqrt{ \epsilon } \\
        \end{align*}

        \paragraph{proof}
        Show that if $\delta = \sqrt{ \epsilon }$ and 
        $0 < |x - 2| < \delta$ 
        
        then $\left| \left( x^2 + 4x + 5 \right) - 1 \right| < \epsilon$

        \[
          \left| \left( x^2 + 4x + 5 \right) - 1 \right| = |(x - 2)^2| < \delta^2 = \epsilon
        \]

      \newpage

      \item[30] Prove that $\lim_{x \to 3} \left( x^2 + x - 4 \right) = 8$

        Show that for all $\epsilon$ there is a $\delta$ such that if 
        $0 < |x - 3| < \delta$ then $\left| \left( x^2 + x - 4 \right) - 8 \right| < \epsilon$

        \paragraph{preliminary investigation}

        \[
          \left| \left( x^2 + x - 4 \right) - 8 \right| = |(x - 3)(x + 4)| 
        \]

        If we limit $x$ to the range: $2 \leq x \leq 4$, then $6 \leq x + 4 \leq 8$ and
        \[
          |(x - 3)(x + 4)| \leq |8 (x - 3)|
        \]

        Choose $\delta = \min \left\{1, \frac{\epsilon}{8} \right\}$

        \paragraph{proof}
        Show that if $\delta = \min \left\{ 1, \frac{\epsilon}{8} \right\}$ and 
        $0 < |x - 3| < \delta$ 
        
        then $\left| \left( x^2 + x - 4 \right) - 8 \right| < \epsilon$

        \[
          \left| \left( x^2 + x - 4 \right) - 8 \right| = |(x - 3)(x + 4)| < 8 \delta \leq \epsilon
        \]

      \item[31] Prove that $\lim_{x \to -2} \left( x^2 - 1 \right) = 3$

        Show that for all $\epsilon$ there is a $\delta$ such that if 
        $0 < |x + 2| < \delta$ then $| \left( x^2 - 1 \right) - 3 | < \epsilon$

        \paragraph{preliminary investigation}

        \[
          | \left( x^2 - 1 \right) - 3 | = |(x - 2)(x + 2)| 
        \]

        If we limit $x$ to the range: $-3 \leq x \leq -1$, then $-5 \leq x - 2 \leq -3$ and
        \[
          |(x - 2)(x + 2)| \leq |5 (x + 2)|
        \]

        Choose $\delta = \min \left\{1, \frac{\epsilon}{5} \right\}$

        \paragraph{proof}
        Show that if $\delta = \min \left\{ 1, \frac{\epsilon}{5} \right\}$ and 
        $0 < |x + 2| < \delta$ 
        
        then $| \left( x^2 - 1 \right) - 3 | < \epsilon$

        \[
          | \left( x^2 - 1 \right) - 3 | = |(x - 2)(x + 2)| < 5 \delta \leq \epsilon
        \]

      \item[32] Prove that $\lim_{x \to 2} x^3 = 8$

        Show that for all $\epsilon$ there is a $\delta$ such that if 
        $0 < |x - 2| < \delta$ then $| x^3 - 8 | < \epsilon$

        \paragraph{preliminary investigation}

        \[
          | x^3 - 8 | = |(x - 2) \left(x^2 + 2x + 4\right) | 
        \]

        If we limit $x$ to the range: $1 \leq x \leq 3$, then $7 \leq x^2 + 2x + 4 \leq 19$ and
        \[
          |(x - 2) \left(x^2 + 2x + 4\right) | \leq |19 (x - 2)|
        \]

        Choose $\delta = \min \left\{1, \frac{\epsilon}{19} \right\}$

        \paragraph{proof}
        Show that if $\delta = \min \left\{ 1, \frac{\epsilon}{19} \right\}$ and 
        $0 < |x - 2| < \delta$ 
        
        then $| x^3 - 8 | < \epsilon$

        \[
          | x^3 - 8 | = |(x - 2) \left(x^2 + 2x + 4\right) | < 19 \delta \leq \epsilon
        \]

      \newpage

      \item[36] Prove that $\lim_{x \to 2} \frac{1}{x} = \frac{1}{2}$

        Show that for all $\epsilon$ there is a $\delta$ such that if 
        $0 < |x - 2| < \delta$ 
        
        then $\left| \frac{1}{x} - \frac{1}{2} \right| < \epsilon$

        \paragraph{preliminary investigation}

        \[
          \left| \frac{1}{x} - \frac{1}{2} \right| = \left| \frac{x - 2}{2x} \right| 
        \]

        If we limit $x$ to the range: $1 \leq x \leq 3$, 
        
        then $\frac{1}{6} \leq \frac{1}{2x} \leq 1$ and

        \[
          \left| \frac{x - 2}{2x} \right| \leq |x - 2|
        \]

        Choose $\delta = \min \{ 1, \epsilon \}$

        \paragraph{proof}
        Show that if $\delta = \min \{ 1, \epsilon \}$ and 
        $0 < |x - 2| < \delta$ 
        
        then $\left| \frac{1}{x} - \frac{1}{2} \right| < \epsilon$

        \[
          \left| \frac{1}{x} - \frac{1}{2} \right| = \left| \frac{x - 2}{2x} \right| < \delta \leq \epsilon
        \]

        % An alternate and easier approach would be to let $y = \frac{1}{x}$ and prove that
        % \[
        %   \lim_{y \to 1/2} = \frac{1}{2}
        % \]

      \newpage

      \item[37] Prove that $\lim_{x \to a} \sqrt{x} = \sqrt{a}$ for $a > 0$

        Show that for all $\epsilon$ there is a $\delta$ such that if 
        $0 < |x - a| < \delta$ 
        
        then $\left| \sqrt{x} - \sqrt{a} \right| < \epsilon$

        \paragraph{preliminary investigation}

        \[
          \left| \sqrt{x} - \sqrt{a} \right| = \frac{|x - a|}{\sqrt{x} + \sqrt{a}} 
        \]

        % If we limit $x$ to the range: $0 \leq x \leq a + 1$, 
        % then $\sqrt{a} \leq \sqrt{x} + \sqrt{a} \leq \sqrt{a} + \sqrt{a + 1}$ and

        Since $\sqrt{x} \geq 0$:
        \[
          \frac{|x - a|}{\sqrt{x} + \sqrt{a}} \leq \frac{|x - a|}{\sqrt{a}}
        \]

        Choose $\delta = \epsilon$

        \paragraph{proof}
        Show that if $\delta = \epsilon$ and $0 < |x - a| < \delta$ 
        
        then $\left| \sqrt{x} - \sqrt{a} \right| < \epsilon$

        \[
          \left| \sqrt{x} - \sqrt{a} \right| = \frac{|x - a|}{\sqrt{x} + \sqrt{a}} < \frac{\delta}{\sqrt{a}} < \delta = \epsilon
        \]

    \end{description}

  \else
    \vspace{10 cm}
    \begin{quote}
      \begin{em}
        I don't believe in any form of unjustified extremism! But when a man is
        exercising extremism---a human being is exercising extremism---in
        defense of liberty for human beings it's no vice, and when one is
        moderate in the pursuit of justice for human beings I say he is a
        sinner. 
      \end{em}
    \end{quote}
    \hspace{1 cm} --Malcolm X
  \fi

\end{document}

