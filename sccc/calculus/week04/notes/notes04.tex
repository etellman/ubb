\documentclass[letterpaper, landscape]{exam}
\usepackage{2in1, lscape} 
\printanswers

\usepackage{units} 
\usepackage[fleqn]{amsmath}
\usepackage{float}
\usepackage{mdwlist}
\usepackage{booktabs}
\usepackage{caption}
\usepackage{fullpage}
\usepackage{enumerate}
\usepackage{graphicx}

\setcounter{tocdepth}{1}
\everymath{\displaystyle}

\author{}
\date{\today}
\title{Calculus I \\ Week Four}

\begin{document}

  \maketitle
  \tableofcontents

  \section{Homework One} 
  \begin{itemize}
    \item taxi problem
    \item $[]$ vs $()$
    \item $[-2, -1] \cup [-1, 3] = [-2, 3]$
    \item vertical velocity of airplane 
    \item $f(x) = 4x + 2 \neq 2x + 1$
  \end{itemize}

  \section{Limit Definition}

  Describe $\epsilon/\delta$ definition, challenge response.

  \[
    \lim_{x \to 1} 2x + 1 = 3 \\
  \]

  \begin{tabular}[H]{rrrr}
    \toprule
    $\epsilon$ & y range        & $\delta$ & x range \\
    \midrule
    1          & $(2, 4)$       & 0.5      & $(0.5, 1.5)$ \\
    0.5        & $(2.5, 3.5)$   & 0.25     & $(0.75, 1.25)$ \\
    0.1        & $(2.9, 3.1)$   & 0.05     & $(0.95, 1.05)$ \\
    0.01       & $(2.99, 3.01)$ & 0.005    & $(0.995, 1.005)$ \\
    \bottomrule
  \end{tabular}

  Note that smaller deltas always also work.

  \subsection{Finding Limits}

  \subsubsection{Lines}

  $\lim_{x \to 1} 2x + 1 = 3$ means that:

  For any $\epsilon$ there is a $\delta$ such that if $|x - 1| < \delta$ then
  $|f(x) - 3| < \epsilon$ 

  \[
    |2x + 1 - 3| = |2x - 2| = 2 |x - 1|
  \]

  If we keep $|x - 1| < \frac{\epsilon}{2}$ then $2 |x - 1| < \epsilon$.
  $\delta = \frac{\epsilon}{2}$ seems like a good choice.

  proof:
  If $\delta = \frac{\epsilon}{2}$ and $|x - 1| < \delta$ then:

  \begin{align*}
    |(2x + 1) - 3| & = |2x - 2| = 2 |x - 1| < 2 \delta = 2 \frac{\epsilon}{2} = \epsilon \\
    |(2x + 1) - 3| & < \epsilon \\
  \end{align*}

  Do other linear examples.

  \subsection{Parabolas}

  \subsubsection{Example One}
  $\lim_{x \to 0} x^2 = 0$ means that:

  For any $\epsilon$ there is a $\delta$ such that if $|x - 0| < \delta$ then
  $|x^2 - 0| < \epsilon$ 

  If we keep $|x| < \sqrt{\epsilon}$ then $x^2< \epsilon$.
  $\delta = \sqrt{\epsilon}$ seems like a good choice.

  proof:
  If $\delta = \sqrt{\epsilon}$ and $|x| < \delta$ then:

  \begin{align*}
    |x^2 - 0| & = x^2 = < \delta^2 = \epsilon \\
    |x^2 - 0| & < \epsilon \\
  \end{align*}

  \subsubsection{Example Two}
  $\lim_{x \to 2} x^2 = 4$ means that:

  For any $\epsilon$ there is a $\delta$ such that if $|x - 2| < \delta$ then
  $|x^2 - 4| < \epsilon$ 

  preliminaries:

  \[
    |x^2 - 4| = |(x + 2)(x - 2)|
  \]

  Keep $x$ near 2: $1 < x < 3$, $3 < x + 2 < 5$

  Chose $\delta = \frac{\epsilon}{5}$.

  proof: If $\delta = \min \left\{ 1, \frac{\epsilon}{5} \right\}$ 
  and $|x - 2| < \delta$ then:

  \begin{align*}
    |x^2 - 4| & = |(x + 2)(x - 2)| < 5 |x - 2| 
      < 5 \cdot \frac{\epsilon}{5} = \epsilon \\
    |x^2 - 4| & < \epsilon \\
  \end{align*}

  Do other quadratics.

  \section{Substitution}

  $\lim_{x \to 2} \sqrt{3 - x} = 1$ 

  let $y = 3 - x$

  $\lim_{y \to 1} \sqrt{y} = 1$ 

  For any $\epsilon$ there is a $\delta$ such that if $|y - 1| < \delta$ then
  $|\sqrt{y} - 1| < \epsilon$ 

  preliminaries:

  \begin{align*}
    |\sqrt{y} - 1| &= \left| \frac{y - 1}{\sqrt{y} + 1} \right| \\
  \end{align*}

  $0 < y < 1$ implies $1 < \sqrt{y} + 1 < 2$

  chose $\delta = \epsilon$

  proof: If $\delta = \min \left\{ 1, \epsilon \right\}$ 
  and $|y - 1| < \delta$ then:

  \begin{align*}
    |\sqrt{y} - 1| & = \left| \frac{y - 1}{\sqrt{y} + 1} \right| \\
                   & \leq |y - 1| \\
                   & < \delta \\
                   & = \epsilon \\
    \\
    |\sqrt{y} - 1| & < \epsilon
  \end{align*}
\end{document}

