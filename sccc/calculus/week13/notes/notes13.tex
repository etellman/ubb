\documentclass[letterpaper, landscape]{exam}
\usepackage{2in1, lscape} 
\printanswers

\usepackage{units} 
\usepackage[fleqn]{amsmath}
\usepackage{float}
\usepackage{mdwlist}
\usepackage{booktabs}
\usepackage{caption}
\usepackage{fullpage}
\usepackage{enumerate}
\usepackage{graphicx}
\usepackage[justification=justified]{caption}

\setcounter{tocdepth}{1}
\everymath{\displaystyle}

\author{}
\date{\today}
\title{Calculus I \\ Week Thirteen \\ Implicit Differentiation}

\begin{document}

  \maketitle
  \tableofcontents

  \newpage 

  \section{Logarithm Review} % (fold)
  
  \begin{itemize*}
    \item Turn multiplication into addition, division into subtraction, powers into multiplication,
      and roots into division.
    \item farmer, etc.
    \item logarithm tables
    \item guy who followed Napier and made more tables
    \item tables in use until 1920's
    \item enabled Kepler, Newton, etc.
    \item Apollo program still doing calculations with slide rules
    \item logarithmic scale used for Decibels, Richter scale, and other things where there is a wide
      range between the smallest and largest values
    \item do a few logarithm problem
  \end{itemize*}
  
  \section{Differentiating $a^x$} % (fold)
  
  \begin{align*}
    y             & = 10^x \\
                  & = \left( e^{\ln 10} \right)^x \\
                  & = e^{x \cdot \ln 10} \\
    \\
    \frac{dy}{dx} & = e^{x \cdot \ln 10} \cdot \ln 10 \\
                  & = \ln 10 \cdot 10^x \\
    \\
    \frac{d}{dx} a^x &= \ln a \cdot a^x \\
  \end{align*}

  Larger bases grow at faster 
  
  rates. The natural logarithm of numbers less than e are less than 1
  and the natural logarithms of numbers greater than e are greater than 1. 
  
  Logarithms of numbers less than 1 are negative, as and these grow at a negative rate--larger
  $x$ results in a smaller $f(x)$.

  \begin{align*}
    \ln 2   & \approx 0.693147 \\
    \ln 10  & \approx 2.30259 \\
    \ln 0.5 & \approx -0.693147 \\
  \end{align*}

  \section{Differentiating Logarithms} % (fold)
  
  \begin{align*}
    y                  & = \ln x \\
    x                  & = e^y \\
    \\
    1                  & = e^y \frac{dy}{dx} \\
    \frac{dy}{dx}      & = \frac{1}{e^y} \\
                       & = \frac{1}{x} \\
    \\
    \frac{d}{dx} \ln x & = \frac{1}{x} \\
  \end{align*}

  When you have a function you can't solve for $y$ or if you solve for $y$ and you get something you
  can't differentiate, you still may be able to find the derivative by:

  \begin{itemize*}
    \item differentiate both sides with respect to x
    \item solve for $\frac{dy}{dx}$.
    \item if you have an explicit formula for $y$, plug it in to the result
  \end{itemize*}

  \begin{align*}
    y                        & = \log_{10} x \\
    x                        & = 10^y \\
                             & = \left( e^{\ln 10} \right)^y \\
                             & = e^{\ln 10 \cdot y} \\
    \\
    1                        & = \ln 10 \cdot e^y \frac{dy}{dx} \\
    \frac{dy}{dx}            & = \frac{1}{\ln 10 \cdot e^y} \\
                             & = \frac{1}{\ln 10 \cdot x} \\
    \\
    \frac{d}{dx} \log_{10} x & = \frac{1}{\ln 10 \cdot x} \\
    \\
    \frac{d}{dx} \log_a x    & = \frac{1}{\ln a \cdot x} \\
  \end{align*}

  \section{Differentiating Circles} % (fold)
  
  Ellipses are not functions since they don't pass the vertical line test.

  \begin{align*}
    x^2 + y^2             & = 4 \\
    2x + 2y \frac{dy}{dx} & = 0 \\
    \frac{dy}{dx}         & = - \frac{x}{y} \\
  \end{align*}

  Draw a circle with various tangent lines and slopes.

  The top half of a circle is a function:
  \begin{align*}
    y             & = \left( 4 - x^2 \right)^{1/2} \\
    \frac{dy}{dx} & = \frac{1}{2} \left( 4 - x^2 \right)^{-1/2} \cdot 2x \\
                  & = - \frac{x}{\sqrt{4 - x^2}} \\
  \end{align*}

  This is the same thing you get when you plug the expression for $y$ into the first result:
  \begin{align*}
    y             & = \sqrt{ 4 - x^2 } \\
    \frac{dy}{dx} & = - \frac{x}{y} \\
                  & = - \frac{x}{ \sqrt{ 4 - x^2 } } \\
  \end{align*}
  
  \newpage

  \section{Examples} % (fold)
  
  \begin{enumerate}
    \item 
      \begin{align*}
        4x^3 + 2xy - y^3 & = 3 \\
        \frac{dy}{dx}    & = \frac{2 \left(6x^2 + y\right)}{3 y^2 - 2x } \\
      \end{align*}

      You can also find $\frac{dx}{dy}$, which is the reciprocal of $\frac{dy}{dx}$:
      \begin{align*}
        4x^3 + 2xy - y^3 & = 3 \\
        \frac{dx}{dy}    & = \frac{3 y^2 - 2x}{2 \left( 6x^2 + y \right)} \\
      \end{align*}

    \item 
      \begin{align*}
        xy - x \cos y & = 0 \\
        y'            & = \frac{\cos y - y}{x (\sin y + 1)} \\
      \end{align*}
      
    \item 
      \begin{align*}
        e^{xy} & = x + y \\
        y'     & = \frac{1 - y e^{x y}}{x e^{x y} - 1} \\
      \end{align*}

    \item 
      \begin{align*}
        \sin(x + y) & = \sin x + \sin y \\
        y'          & = \frac{\cos (x) - \cos (x + y)}{\cos (x + y) - \cos (y)} \\
      \end{align*}

    \item 
      \begin{align*}
        4x^2 + 3 y^2 + 4 & = 8x + 12 y \\
        y'               & = - \frac{4 (x - 1)}{3 (y - 2)} \\
      \end{align*}

    \item 
      \begin{align*}
        x^3 - y^3         & = 4 \\
        \frac{dy}{dx}     & = \frac{x^2}{y^2} \\
        \frac{d^2y}{dx^2} & = \frac{2 x (y-x)}{y^3} \\
      \end{align*}
      
    \item Do a few equation of tangent lines from the book.
  \end{enumerate}

  \section{Inverse Trigonometric Functions} % (fold)
  
  Review inverse trigonometric functions and their domains.

  \begin{align*}
    y             & = \arcsin x \\
    x             & = \sin y \\
    \\
    1             & = \cos y \frac{dy}{dx} \\
    \frac{dy}{dx} & = \frac{1}{\cos y} \\
                  & = \frac{1}{\sqrt{1 - \sin^2 y}} \\
                  & = \frac{1}{\sqrt{1 - x^2}} \\
  \end{align*}

  Works because $\cos x \geq 0$ when $-\frac{\pi}{2} \leq x \leq \frac{\pi}{2}$.

\end{document}

