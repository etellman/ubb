\documentclass[letterpaper, landscape]{exam}
\usepackage{2in1, lscape} 
\printanswers

\usepackage{units} 
\usepackage[fleqn]{amsmath}
\usepackage{float}
\usepackage{mdwlist}
\usepackage{booktabs}
\usepackage{caption}
\usepackage{fullpage}
\usepackage{enumerate}
\usepackage{graphicx}
\usepackage[justification=justified]{caption}

\setcounter{tocdepth}{1}
\everymath{\displaystyle}

\author{}
\date{\today}
\title{Calculus I \\ Week Twelve \\ The Chain Rule}

\begin{document}

  \maketitle
  \tableofcontents

  \newpage 

  \section{Example/Motivation} % (fold)

  Draw function boxes with widget machine and item machine which produces items out of widgets. With
  faster widget production we can get more items or with more items per widget we can get more
  items.
  
  \begin{align*}
    i(t)          & = i(w(t)) \\
    \frac{di}{dt} & = \frac{di}{dw} \cdot \frac{dw}{dt} \\
  \end{align*}

  Doubling the rate at which widgets come out doubles the rate of item production. Or making twice as
  many items out of the same number of widgets doubles the rate of item production. The overall rate
  is the product of the ``items per widget'' rate and the ``widgets per time.''
  
  In general with a composition of functions, the overall rate is the product of the two individual
  rates.

  Another example:
  \begin{align*}
    f(x)  & = \left( x^2 \right)^3 \\
    f'(x) & = 6x^5 \\
  \end{align*}

  Alternate:
  \begin{align*}
    u(x)          & = x^2 \\
    f(u)          & = u^3 \\
    \\
    \frac{du}{dx} & = 2x \\
    \frac{df}{du} & = 3u^2 \\
    \\
    \frac{df}{dx} & = 3 \cdot \left( x^2 \right)^2 \cdot 2x \\
                  & = 6x^5 \\
  \end{align*}


  \begin{align*}
    f(x) & = \sqrt{2x^2 - 7} \\
    g(x) & = \left( 2x^2 - 7 \right)^{10} \\
  \end{align*}

  \section{Rule} % (fold)
  
  \begin{align*}
    f(x) & = u(v(x)) \\
    y    & = u \circ v \\
  \end{align*}

  For our example:
  \begin{align*}
    v(x) & = 2x^2 - 7 \\
    u(v) & = \sqrt{v}
  \end{align*}

  If all the $\Delta$s are non-zero, then:
  \begin{align*}
    f(x) &= u(v(x)) \\
    \\
    \frac{\Delta u}{\Delta x} & = \frac{\Delta u}{\Delta x} \cdot \frac{\Delta v}{\Delta v} \\
                              & = \frac{\Delta u}{\Delta v} \cdot \frac{\Delta v}{\Delta x} \\
  \end{align*}

  If $\frac{du}{dv}$ and $\frac{dv}{dx}$ are both defined, then:
  \begin{align*}
    \lim_{\Delta x \to 0} \Delta v &= 0 \\
    \lim_{\Delta v \to 0} \Delta u &= 0 \\
  \end{align*}

  Take the limit:
  \begin{align*}
    \lim_{\Delta x \to 0} \frac{\Delta u}{\Delta x} 
                  & = \lim_{\Delta x \to 0} \left[ \frac{\Delta u}{\Delta v} \cdot \frac{\Delta v}{\Delta x} \right] \\
                  & = \lim_{\Delta v \to 0} \frac{\Delta u}{\Delta v} \cdot \lim_{\Delta x \to 0} \frac{\Delta v}{\Delta x} \\
    \frac{du}{dx} & = \frac{du}{dv} \cdot \frac{dv}{dx} \\
  \end{align*}

  \section{Examples}

  \section{Alternate Proof of Quotient Rule} % (fold)
  
  \begin{align*}
    f(x)  & = \frac{u(x)}{v(x)} \\
          & = u(x) \cdot [ v(x) ]^{-1} \\
          & = uv^{-1} \\
    \\
    f'(x) & = u \cdot (- v^{-2} v') + v^{-1} u' \\
          & = - \frac{uv'}{v^2} + \frac{u'}{v} \\
          & = \frac{u'v}{v^2} - \frac{uv'}{v^2} \\
          & = \frac{u'v - uv'}{v^2} \\
  \end{align*}

\end{document}

