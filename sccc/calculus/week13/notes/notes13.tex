\documentclass[letterpaper, landscape]{exam}
\usepackage{2in1, lscape} 
\printanswers

\usepackage{units} 
\usepackage[fleqn]{amsmath}
\usepackage{float}
\usepackage{mdwlist}
\usepackage{booktabs}
\usepackage{caption}
\usepackage{fullpage}
\usepackage{enumerate}
\usepackage{graphicx}
\usepackage[justification=justified]{caption}

\setcounter{tocdepth}{1}
\everymath{\displaystyle}

\author{}
\date{\today}
\title{Calculus I \\ Week Twelve \\ The Chain Rule}

\begin{document}

  \maketitle
  \tableofcontents

  \section{Example/Motivation} % (fold)
  
  \begin{align*}
    y &= \sqrt{x^2 + 1} \\
    y &= \left( x^2 + 1 \right)^{10} \\
  \end{align*}

  \section{Rule} % (fold)
  
  \begin{align*}
    f(x) & = u(v(x)) \\
    y    & = u \circ v \\
  \end{align*}

  For our example:
  \begin{align*}
    u(x) & = x^2 + 1 \\
    y(u) & = \sqrt{u}
  \end{align*}

  If all the $\Delta$s are non-zero, then:
  \begin{align*}
    f(x) &= y(u(x)) \\
    \\
    \frac{\Delta y}{\Delta x} & = \frac{\Delta y}{\Delta x} \cdot \frac{\Delta u}{\Delta u} \\
                              & = \frac{\Delta y}{\Delta u} \cdot \frac{\Delta u}{\Delta x} \\
  \end{align*}

  If $\frac{dy}{du}$ and $\frac{du}{dx}$ are both defined, then:
  \begin{align*}
    \lim_{\Delta x \to 0} \Delta u &= 0 \\
    \lim_{\Delta u \to 0} \Delta y &= 0 \\
  \end{align*}

  Take the limit:
  \begin{align*}
    \lim_{\Delta x \to 0} \frac{\Delta y}{\Delta x} 
                  & = \lim_{\Delta x \to 0} \left[ \frac{\Delta y}{\Delta u} \cdot \frac{\Delta u}{\Delta x} \right] \\
                  & = \lim_{\Delta u \to 0} \frac{\Delta y}{\Delta u} \cdot \lim_{\Delta x \to 0} \frac{\Delta u}{\Delta x} \\
    \frac{dy}{dx} & = \frac{dy}{du} \cdot \frac{du}{dx} \\
  \end{align*}


  \section{Alternate Proof of Quotient Rule} % (fold)
  
  \begin{align*}
    f(x)  & = \frac{u(x)}{v(x)} \\
          & = u(x) \cdot [ v(x) ]^{-1} \\
          & = uv^{-1} \\
    \\
    f'(x) & = u \cdot (- v^{-2} v') + v^{-1} u' \\
          & = - \frac{uv'}{v^2} + \frac{u'}{v} \\
          & = \frac{u'v}{v^2} - \frac{uv'}{v^2} \\
          & = \frac{u'v - uv'}{v^2} \\
  \end{align*}

\end{document}

