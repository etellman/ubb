\documentclass[letterpaper, landscape]{exam}
\usepackage{2in1, lscape} 
\printanswers

\usepackage{units} 
\usepackage[fleqn]{amsmath}
\usepackage{float}
\usepackage{mdwlist}
\usepackage{booktabs}
\usepackage{caption}
\usepackage{fullpage}
\usepackage{enumerate}
\usepackage{graphicx}
\usepackage[justification=justified]{caption}

\setcounter{tocdepth}{1}
\everymath{\displaystyle}

\author{}
\date{\today}
\title{Calculus I \\ Week Twelve \\ The Chain Rule}

\begin{document}

  \maketitle
  \tableofcontents

  \newpage 

  \section{Example/Motivation} % (fold)

  \subsection{Car Manufacturing} % (fold)
  
  Draw function boxes with steel supply (dollars in, steel out) and car manufacturer (steel in, cars
  out).

  \begin{itemize}
    \item first function is amount of steel a given amount of money will provide
    \item second function is number of cars a given amount of steel will provide
    \item draw graph of piecewise linear cost function for steel--breaks are when cost goes down
      because the manufacturer is buying a large quantity
    \item draw graph of linear function for steel vs. cars. More steel produces more cars
    \item talk about slope is incremental amount you have to spend to get a little more steel, or
      incremental amount of steel you need to use to make a little more car
    \item You can save money on cars by either finding a cheaper supply of steel or finding a way to
      make cars with less steel per car.
  \end{itemize}

  \subsection{Online Retailer} % (fold)
  
  Draw function boxes with time in, searches out and searches in orders out.

  \begin{itemize}
    \item first function is total number of searches since 2000 (or whatever)
    \item second function is number of orders a search will produce
    \item draw graphs 
    \item slope of searches graph is searches/minute
    \item slope of orders graph is orders/search
    \item You can sell more stuff by getting more searches (more searches per minute) or by making
      each search more productive (cheaper stuff, better selection, better search algorithm, etc.)
  \end{itemize}



  \begin{align*}
    f(x) & = \sqrt{2x^2 - 7} \\
    g(x) & = \left( 2x^2 - 7 \right)^{10} \\
  \end{align*}

  \section{Rule} % (fold)
  
  \begin{align*}
    f(x) & = u(v(x)) \\
    y    & = u \circ v \\
  \end{align*}

  For our example:
  \begin{align*}
    v(x) & = 2x^2 - 7 \\
    u(v) & = \sqrt{v}
  \end{align*}

  If all the $\Delta$s are non-zero, then:
  \begin{align*}
    f(x) &= u(v(x)) \\
    \\
    \frac{\Delta u}{\Delta x} & = \frac{\Delta u}{\Delta x} \cdot \frac{\Delta v}{\Delta v} \\
                              & = \frac{\Delta u}{\Delta v} \cdot \frac{\Delta v}{\Delta x} \\
  \end{align*}

  If $\frac{du}{dv}$ and $\frac{dv}{dx}$ are both defined, then:
  \begin{align*}
    \lim_{\Delta x \to 0} \Delta v &= 0 \\
    \lim_{\Delta v \to 0} \Delta u &= 0 \\
  \end{align*}

  Take the limit:
  \begin{align*}
    \lim_{\Delta x \to 0} \frac{\Delta u}{\Delta x} 
                  & = \lim_{\Delta x \to 0} \left[ \frac{\Delta u}{\Delta v} \cdot \frac{\Delta v}{\Delta x} \right] \\
                  & = \lim_{\Delta v \to 0} \frac{\Delta u}{\Delta v} \cdot \lim_{\Delta x \to 0} \frac{\Delta v}{\Delta x} \\
    \frac{du}{dx} & = \frac{du}{dv} \cdot \frac{dv}{dx} \\
  \end{align*}

  \section{Examples}

  Another example:
  \begin{align*}
    f(x)  & = \left( x^2 \right)^3 \\
    f'(x) & = 6x^5 \\
  \end{align*}

  Alternate:
  \begin{align*}
    u(x)          & = x^2 \\
    f(u)          & = u^3 \\
    \\
    \frac{du}{dx} & = 2x \\
    \frac{df}{du} & = 3u^2 \\
    \\
    \frac{df}{dx} & = 3 \cdot \left( x^2 \right)^2 \cdot 2x \\
                  & = 6x^5 \\
  \end{align*}

  \section{Alternate Proof of Quotient Rule} % (fold)
  
  \begin{align*}
    f(x)  & = \frac{u(x)}{v(x)} \\
          & = u(x) \cdot [ v(x) ]^{-1} \\
          & = uv^{-1} \\
    \\
    f'(x) & = u \cdot (- v^{-2} v') + v^{-1} u' \\
          & = - \frac{uv'}{v^2} + \frac{u'}{v} \\
          & = \frac{u'v}{v^2} - \frac{uv'}{v^2} \\
          & = \frac{u'v - uv'}{v^2} \\
  \end{align*}

\end{document}

