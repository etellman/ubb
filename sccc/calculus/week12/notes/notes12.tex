\documentclass[letterpaper, landscape]{exam}
\usepackage{2in1, lscape} 
\printanswers

\usepackage{units} 
\usepackage[fleqn]{amsmath}
\usepackage{float}
\usepackage{mdwlist}
\usepackage{booktabs}
\usepackage{caption}
\usepackage{fullpage}
\usepackage{enumerate}
\usepackage{graphicx}
\usepackage[justification=justified]{caption}

\setcounter{tocdepth}{1}
\everymath{\displaystyle}

\author{}
\date{\today}
\title{Calculus I \\ Week Twelve \\ The Chain Rule}

\begin{document}

  \maketitle
  \tableofcontents

  \newpage 

  \section{Substitution} % (fold)
  
  \begin{align*}
    f(x)                & = \frac{\sin 4x}{6x} \\
    \\
    u                   & = 4x \\
    x                   & = \frac{u}{4} \\
    6x                  & = \frac{3u}{2} \\
    \\
    f(u)                & = \frac{\sin u}{3u/2} \\
                        & = \frac{2}{3} \cdot \frac{\sin u}{u} \\
    \\
    \lim_{u \to 0} f(u) & = \lim_{u \to 0} \frac{2}{3} \cdot \frac{\sin u}{u} \\
                        & = \frac{2}{3} \\
  \end{align*}

  \newpage

  \section{Chain Rule} % (fold)
  
  \subsection{Example/Motivation} % (fold)

  \subsubsection{Car Manufacturing} % (fold)
  
  Draw function boxes with steel supply (dollars in, steel out) and car
  manufacturer (steel in, cars out).

  \begin{itemize}
    \item first function is amount of steel a given amount of money will provide
    \item second function is number of cars a given amount of steel will provide
    \item draw graph of piecewise linear cost function for steel--breaks are
      when cost goes down
      because the manufacturer is buying a large quantity
    \item draw graph of linear function for steel vs. cars. More steel produces more cars
    \item talk about slope is incremental amount you have to spend to get a
      little more steel, or incremental amount of steel you need to use to make
      a little more car
    \item You can save money on cars by either finding a cheaper supply of
      steel or finding a way to make cars with less steel per car.
  \end{itemize}

  \subsubsection{Online Retailer} % (fold)
  
  Draw function boxes with time in, visitors out and visitors in orders out.

  \begin{itemize}
    \item first function is total number of visits in some amount of time
    \item second function is number of orders a that many visits produces
    \item draw graphs 
    \item slope of searches graph is visits/minute
    \item slope of orders graph is orders/visits
    \item You can sell more stuff by getting more visits (more visits per
      minute) or by making each visit more productive (cheaper stuff, better
      selection, better search algorithm, etc.)
  \end{itemize}

  Other examples:
  \begin{align*}
    f(x) & = \sqrt{2x^2 - 7} \\
    g(x) & = \left( 2x^2 - 7 \right)^{10} \\
  \end{align*}

  \subsection{Rule} % (fold)
  
  \begin{align*}
    f(x) & = u(v(x)) \\
    y    & = u \circ v \\
  \end{align*}

  For our example:
  \begin{align*}
    v(x) & = 2x^2 - 7 \\
    u(v) & = \sqrt{v}
  \end{align*}

  If all the $\Delta$s are non-zero, then:
  \begin{align*}
    f(x) &= u(v(x)) \\
    \\
    \frac{\Delta u}{\Delta x} & = \frac{\Delta u}{\Delta x} \cdot \frac{\Delta v}{\Delta v} \\
                              & = \frac{\Delta u}{\Delta v} \cdot \frac{\Delta v}{\Delta x} \\
  \end{align*}

  If $\frac{du}{dv}$ and $\frac{dv}{dx}$ are both defined, then:
  \begin{align*}
    \lim_{\Delta x \to 0} \Delta v &= 0 \\
    \lim_{\Delta v \to 0} \Delta u &= 0 \\
  \end{align*}

  Take the limit:
  \begin{align*}
    \lim_{\Delta x \to 0} \frac{\Delta u}{\Delta x} 
                  & = \lim_{\Delta x \to 0} \left[ \frac{\Delta u}{\Delta v} \cdot \frac{\Delta v}{\Delta x} \right] \\
                  & = \lim_{\Delta v \to 0} \frac{\Delta u}{\Delta v} \cdot \lim_{\Delta x \to 0} \frac{\Delta v}{\Delta x} \\
    \frac{du}{dx} & = \frac{du}{dv} \cdot \frac{dv}{dx} \\
  \end{align*}

  \subsection{Examples}

  Another example:
  \begin{align*}
    f(x)  & = \left( x^2 \right)^3 \\
    f'(x) & = 6x^5 \\
  \end{align*}

  Alternate:
  \begin{align*}
    u(x)          & = x^2 \\
    f(u)          & = u^3 \\
    \\
    \frac{du}{dx} & = 2x \\
    \frac{df}{du} & = 3u^2 \\
    \\
    \frac{df}{dx} & = 3 \cdot \left( x^2 \right)^2 \cdot 2x \\
                  & = 6x^5 \\
  \end{align*}

  \subsection{Alternate Proof of Quotient Rule} % (fold)
  
  \begin{align*}
    f(x)  & = \frac{u(x)}{v(x)} \\
          & = u(x) \cdot [ v(x) ]^{-1} \\
          & = uv^{-1} \\
    \\
    f'(x) & = u \cdot (- v^{-2} v') + v^{-1} u' \\
          & = - \frac{uv'}{v^2} + \frac{u'}{v} \\
          & = \frac{u'v}{v^2} - \frac{uv'}{v^2} \\
          & = \frac{u'v - uv'}{v^2} \\
  \end{align*}

  \subsection{More Examples} % (fold)
  
  \begin{enumerate}

    \item 
      \[
        f(x) = \sqrt{x^2 + 12x - 3}
      \]

    \item 
      \[
        f(x) = \left( x^2 + 2x \right)^4
      \]

    \item 
      \[
        f(x) = \sin \left( x^2 + 2x + 3 \right)
      \]

    \item 
      \[
        f(x) = \cos \left( x^2 - 1 \right)^2
      \]

    \item 
      \[
        f(x) = \cos^2 \left( x^2 - 1 \right)
      \]

  \end{enumerate}

  \newpage

  \section{$e^{xi}$} % (fold)
  
  Powers of $i$:
  \begin{align*}
    i^0 & = 1 \\
    i^1 & = i \\
    i^2 & = -1 \\
    i^3 & = -i \\
    i^4 & = 1 \\
    \vdots
  \end{align*}

  \[
    e^i = 1 + i + \frac{i^2}{2!} + \frac{i^3}{3!} + \frac{i^4}{4!} + \dots
  \]

  Powers of $e^ix$ between 0 and $\pi$, incrementing by $0.2$:
  \begin{align*}
    1. + 0. i \\
    0.980067 + 0.198669 i \\
    0.921061 + 0.389418 i \\
    0.825336 + 0.564642 i \\
    0.696707 + 0.717356 i \\
    0.540302 + 0.841471 i \\
    0.362358 + 0.932039 i \\
    0.169967 + 0.98545 i \\
    -0.0291995 + 0.999574 i \\
    -0.227202 + 0.973848 i \\
    -0.416147 + 0.909297 i \\
    -0.588501 + 0.808496 i \\
    -0.737394 + 0.675463 i \\
    -0.856889 + 0.515501 i \\
    -0.942222 + 0.334988 i \\
    -0.989992 + 0.14112 i \\
  \end{align*}

  \begin{align*}
    e^{ix}        & = \cos x + i \sin x \\
    e^{\pi i}     & = -1 \\
    e^{\pi i} + 1 & = 0 \\
    \\
    \cos x & = 1 - \frac{x^2}{2!} + \frac{x^4}{4!} - \frac{x^6}{6!} \dots \\
    \sin x & = 1 - \frac{x^3}{3!} + \frac{x^5}{5!} - \frac{x^7}{7!} \dots \\
  \end{align*}

\end{document}

