\documentclass[fleqn, onecolumn]{article}

\usepackage{fullpage}
\usepackage{graphicx}
\usepackage{float}
\usepackage{amsmath}
\usepackage{amssymb}
\usepackage{polynom}
\usepackage{caption}
\usepackage{mdwlist}
\usepackage{parskip}
\usepackage{booktabs}

\newcommand{\degree}{\ensuremath{^\circ}} 

\everymath{\displaystyle}
\setlength{\mathindent}{1 cm}

\title{Calculus I}
\author{University Beyond Bars}
\date{\today}

\begin{document}

  \maketitle

  \section{Administrative}
  \subsection{Textbook}
  The course will cover chapters 1-4 of {\em Calculus with Early
  Transcendentals}, by James Stewart.  

  \subsection{Prerequisites}
  You should be familiar with the topics from Seattle Central's math 141-142 series
  which includes:
  \begin{itemize*}
    \item definition of functions
    \item polynomial and rational functions
    \item exponentials and logarithms
    \item trigonometry
  \end{itemize*}

  We'll review these topics, but it will be easier if you've seen them before.

  \subsection{Homework and Exams}

  You should expect to spend three or four hours each week doing homework.  Feel
  free to work together with other students on the homework.

  Chapters 2-4 from the textbook will be followed by an in-class test.  There
  will also be a final exam which covers the entire course.

  There won't be a test for Chapter 1, since this chapter is mostly review and
  we wont be covering all of it.

  \subsection{Calculators}
  Calculators won't be required to do the homework. If you have a graphing
  calculator, you can use it to check your work and do some of the optional
  calculator exercises from the text.

  \subsection{Credit}
  Unfortunately we don't have a way to get college credit for this course.
  We're trying to work something out with Seattle Central Community College,
  but can't promise anything.

  \section{Course Overview}
  One of the two parts of Calculus is about calculating the rates at which
  things change. This is the part of calculus covered by this course.

  For example, if you have some equation that tells you where a ball will be $t$
  seconds after it left the ground, you can use calculus to find out:
  \begin{itemize*}
    \item how fast is it going $t$ seconds after it left the ground?
    \item what is its acceleration $t$ seconds after it left the ground?
  \end{itemize*}

  We'll start out be describing the general idea for finding a formula for speed
  when you have a simple formula for position. The remaining chapters cover
  techniques for doing the same thing with more and more complicated equations.

  We'll also talk about other practical applications. You can, for example, use
  calculus to figure out the optimum number of widgets to make in your widget
  factory in order to maximize your profit, and we'll talk about how to do this.

\end{document}

