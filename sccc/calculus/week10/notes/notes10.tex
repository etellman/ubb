\documentclass[letterpaper, landscape]{exam}
\usepackage{2in1, lscape} 
\printanswers

\usepackage{units} 
\usepackage[fleqn]{amsmath}
\usepackage{float}
\usepackage{mdwlist}
\usepackage{booktabs}
\usepackage{caption}
\usepackage{fullpage}
\usepackage{enumerate}
\usepackage{graphicx}
\usepackage[justification=justified]{caption}

\setcounter{tocdepth}{1}
\everymath{\displaystyle}

\title{Calculus I \\ Week Ten}
\author{}
\date{\today}

\begin{document}

  \maketitle
  \tableofcontents

  \section{Product Rule} % (fold)
  \label{sec:pr}
  
  \subsection{Proof} % (fold)
  \label{sub:pr.proof}
  
  Show that intuitive approach is wrong:
  \begin{align*}
    f(x)                       & = x^2 \\
    g(x)                       & = x^3 \\
    f'(x) g'(x)                & = 2x \cdot 3x^2 \\
                               & = 6x^3 \\
    \\
    \frac{d}{dx} [ f(x) g(x) ] & = \frac{d}{dx} x^5 \\
                               & = 5x^4 \\
  \end{align*}

  Draw picture of rectangle with $\Delta x$, $\Delta y$ and $\Delta A$. The area
  of the new horizontal slice depends on the change in the x direction and the
  current height. The area of the new vertical slice depends on the change in
  the y direction and the current x length.

  \begin{align*}
    y(x)                      & = u(x) v(x) \\
    \\
    y_0                       & = u_0 v_0 \\
    y_0 + \Delta y            & = (u_0 + \Delta u) (v_0 + \Delta v) \\
    y_0 + \Delta y            & = u_0 v_0 + u_0 \Delta v + v_0 \Delta u + \Delta u \Delta v \\
    \Delta y                  & = u_0 \Delta v + v_0 \Delta u + \Delta u \Delta v \\
    \frac{\Delta y}{\Delta x} & = u_0 \frac{\Delta v}{\Delta x} + v_0 \frac{\Delta u}{\Delta x} + \frac{\Delta u \Delta v}{\Delta x} \\
    \lim_{\Delta x \to 0} \frac{\Delta y}{\Delta x} & = \lim_{\Delta x \to 0} u_0 \frac{\Delta v}{\Delta x} 
      + v_0 \frac{\Delta u}{\Delta x} + \frac{\Delta u \Delta v}{\Delta x} \\
    \frac{dy}{dx} & = u_0 \frac{dv}{dx} + v_0 \frac{du}{dx} + \lim_{\Delta x \to 0} \Delta v \frac{du}{dx} \\
                  & = u_0 \frac{dv}{dx} + v_0 \frac{du}{dx} \\
  \end{align*}

  \subsection{Examples} % (fold)
  
  \begin{enumerate}
    \item $f(x) = \sqrt{x} e^x$

    \item $f(x) = (x + 1)(x^2 + 3x)$ (also show expanding and then taking the derivative)

    \item see exercises
  \end{enumerate}

  \section{Quotient Rule} % (fold)
  \label{sec:qr}
  
  \subsection{Proofs} % (fold)
  
  \subsection{Proof One} % (fold)
  
  \begin{align*}
    y(x)                      & = \frac{u(x)}{v(x)} \\
    y_0                       & = \frac{u_0}{v_0} \\
    \\
    y_0 + \Delta y            & = \frac{u_0 + \Delta u}{v_0 + \Delta v} \\
    \Delta y                  & = \frac{u_0 + \Delta u}{v_0 + \Delta v} - \frac{u_0}{v_0}\\
    \Delta y                  & = \frac{v_0 (u_0 + \Delta u) - u_0 (v_0 + \Delta v)}{v_0 (v_0 + \Delta v)}\\
                              & = \frac{v_0 \Delta u - u_0 \Delta v}{v_0 (v_0 + \Delta v)}\\
    \frac{\Delta y}{\Delta x} & = \frac{v_0 \frac{\Delta u}{\Delta x} - u_0 \frac{\Delta v}{\Delta x}}
                                                                         {v_0 (v_0 + \Delta v)} \\
    \lim_{\Delta x \to 0} \frac{\Delta y}{\Delta x} & = \lim_{\Delta x \to 0} \frac{v_0 \frac{\Delta u}{\Delta x} - u_0 \frac{\Delta v}{\Delta x}}
                                                                                     {v_0 (v_0 + \Delta v)}\\
    \frac{dy}{dx} & = \frac{v_0 \frac{du}{dx} - u_0 \frac{dv}{dx}}{v_0^2} \\
    y'            & = \frac{vu' - uv'}{v^2} \\
  \end{align*}

  \subsection{Proof Two} % (fold)
  
  use the Product Rule:

  \begin{align*}
    y(x) & = \frac{u(x)}{v(x)} \\
         & = u v^{-1} \\
    \\
    y'   & = u \cdot \left( -v^{-2} v' \right) + v^{-1} u' \\
         & = \frac{u'}{v} - \frac{uv'}{v^2} \\
         & = \frac{u'v}{v^2} - \frac{uv'}{v^2} \\
    y'   & = \frac{vu' - uv'}{v^2} \\
  \end{align*}

  \subsection{Examples} % (fold)
  
  \begin{enumerate}
    \item 
      \[
        \frac{x^3}{(x^2 + 1} 
      \]

    \item 
      \[
        \frac{2x^2 + 3x - 1}{x^4 - 3}
      \]

    \item see exercises

  \end{enumerate}
\end{document}

