% no answer key
% \documentclass[letterpaper]{exam}

% answer key
\documentclass[letterpaper, landscape]{exam}
\usepackage{2in1, lscape} 
\printanswers

\usepackage{units} 
\usepackage{xfrac} 
\usepackage[fleqn]{amsmath}
\usepackage{float}
\usepackage{mdwlist}
\usepackage{booktabs}
\usepackage{cancel}
\usepackage{polynom}
\usepackage{caption}
\usepackage{fullpage}
\usepackage{comment}
\usepackage{enumerate}
\usepackage{graphicx}
\usepackage{commath}

\usepackage{mathtools} 

\newcommand{\dg}{\ensuremath{^\circ}} 
\newcommand{\sgn}{\operatorname{sgn}}
\newcommand{\dx}{\dif x}
\newcommand{\dy}{\dif y}

\everymath{\displaystyle}
\title{Calculus I \\ Homework Sixteen \\ Section 3.10}
\author{}
\date{\today}

\begin{document}

  \maketitle

  \section{Homework}
    \begin{itemize*}
      \item read Section 3.10
      \item exercises: TO DO
    \end{itemize*}

  \ifprintanswers

  \section{Solutions}

  \begin{description}
    \item[1] 
      \[
        L(x) = -10x - 6
      \]

    \item[2] 
      \[
        L(x) = x - 1
      \]

    \item[3] 
      \[
        L(x) = -x + \frac{\pi}{2} 
      \]

    \item[4] 
      \[
        L(x) = \frac{3}{8} x + 2
      \]

    \item[11]
      \begin{enumerate}[(a)]
        \item 
          \[
            \dy = 2x \del{ \sin 2x + x \cos 2x } \dx
          \]

        \item
          \[
            \dy = \frac{t}{1 + t^2} \dx
          \]

      \end{enumerate}

    \item[12]
      \begin{enumerate}[(a)]
        \item 
          \[
            \dy = \frac{1}{\del{2s + 1}^2} \dif s
          \]

        \item
          \[
            \dy = - e^{-u} \del{\sin u + \cos u} \dif s
          \]

      \end{enumerate}

    \item[15]
      \begin{align*}
        \dy & = \eval{ \frac{e^{x/10}}{10} \dx }_{x = 0, \dx = 0.1} \\
            & = \boxed{ 0.01 } \\
      \end{align*}

    \item[16]
      \begin{align*}
        \dy & = \eval{ -\frac{1}{(x+1)^2} \dx }_{x = 1, \dx = -0.01} \\
            & = \boxed{ 0.0025 } \\
      \end{align*}

    \item[23]
      \begin{align*}
        y       & = x^5 \\
        \dy     & =  \eval{ 5x^4 \dx }_{x = 2, \dx = 0.001} \\
                & = 0.08 \\
        \\
        2.001^5 & \approx 2^5 + 0.08 \\
                & = \boxed{ 32.08 } \\
      \end{align*}

    \item[24]
      \begin{align*}
        y          & = e^x \\
        \dy        & =  \eval{ e^x \dx }_{x = 0, \dx = -0.015 } \\
                   & = -0.015 \\
        \\
        e^{-0.015} & \approx e^0 - 0.015 \\
                   & = \boxed{ 0.985 } \\
      \end{align*}

    \item[25]
      \begin{align*}
        y          & = x^{2/3} \\
        \dy        & =  \eval{ \frac{2}{3} x^{-1/3} \dx }_{x = 2, \dx = 0.06 } \\
                   & = 0.02 \\
        \\
        8.06^{2/3} & \approx 8^{2/3} + 0.02 \\
                   & = \boxed{ 4.02 } \\
      \end{align*}

    \item[26]
      \begin{align*}
        y              & = \frac{1}{x} \\
        \dy            & =  \eval{ - \frac{1}{x^2} \dx }_{x = 1000, \dx = 2 } \\
                       & = - \frac{1}{500,000} \\
        \\
        \frac{1}{1002} & \approx \frac{1}{1000} - \frac{1}{500,000} \\
                       & = \boxed{ \frac{499}{500,000} }
      \end{align*}

  \end{description}

  \else
    \vspace{10 cm}
    \begin{quote}
      \begin{em}
        TO DO
      \end{em}
    \end{quote}
    \hspace{2 cm} --TO DO
  \fi

\end{document}

