% no answer key
% \documentclass[letterpaper]{exam}

% answer key
\documentclass[letterpaper, landscape]{exam}
\usepackage{2in1, lscape} 
\printanswers

\usepackage{units} 
\usepackage{xfrac} 
\usepackage[fleqn]{amsmath}
\usepackage{float}
\usepackage{mdwlist}
\usepackage{booktabs}
\usepackage{cancel}
\usepackage{polynom}
\usepackage{caption}
\usepackage{fullpage}
\usepackage{comment}
\usepackage{enumerate}
\usepackage{graphicx}
\usepackage{commath}

\usepackage{mathtools} 

\newcommand{\dg}{\ensuremath{^\circ}} 
\newcommand{\sgn}{\operatorname{sgn}}
\newcommand{\dx}{\dif x}
\newcommand{\dy}{\dif y}

\everymath{\displaystyle}
\title{Calculus I \\ Homework Sixteen \\ Section 3.10}
\author{}
\date{\today}

\begin{document}

  \maketitle

  \section{Homework}
    \begin{itemize*}
      \item read Section 3.10
      \item exercises: 1-4, 11-12, 15-16, 23-26
    \end{itemize*}

  \ifprintanswers

  \section{Solutions}

  \begin{description}
    \item[1] 
      \[
        L(x) = -10x - 6
      \]

    \item[2] 
      \[
        L(x) = x - 1
      \]

    \item[3] 
      \[
        L(x) = -x + \frac{\pi}{2} 
      \]

    \item[4] 
      \[
        L(x) = \frac{3}{8} x + 2
      \]

    \item[11]
      \begin{enumerate}[(a)]
        \item 
          \[
            \dy = 2x \del{ \sin 2x + x \cos 2x } \dx
          \]

        \item
          \[
            \dy = \frac{t}{1 + t^2} \dx
          \]

      \end{enumerate}

    \item[12]
      \begin{enumerate}[(a)]
        \item 
          \[
            \dy = \frac{1}{\del{2s + 1}^2} \dif s
          \]

        \item
          \[
            \dy = - e^{-u} \del{\sin u + \cos u} \dif s
          \]

      \end{enumerate}

    \item[15]
      \begin{align*}
        \dy & = \eval{ \frac{e^{x/10}}{10} \dx }_{x = 0, \dx = 0.1} \\
            & = \boxed{ 0.01 } \\
      \end{align*}

    \item[16]
      \begin{align*}
        \dy & = \eval{ -\frac{1}{(x+1)^2} \dx }_{x = 1, \dx = -0.01} \\
            & = \boxed{ 0.0025 } \\
      \end{align*}

    \item[23]
      \begin{align*}
        y       & = x^5 \\
        \dy     & =  \eval{ 5x^4 \dx }_{x = 2, \dx = 0.001} \\
                & = 0.08 \\
        \\
        2.001^5 & \approx 2^5 + 0.08 \\
                & = \boxed{ 32.08 } \\
      \end{align*}

    \item[24]
      \begin{align*}
        y          & = e^x \\
        \dy        & =  \eval{ e^x \dx }_{x = 0, \dx = -0.015 } \\
                   & = -0.015 \\
        \\
        e^{-0.015} & \approx e^0 - 0.015 \\
                   & = \boxed{ 0.985 } \\
      \end{align*}

    \item[25]
      \begin{align*}
        y          & = x^{2/3} \\
        \dy        & =  \eval{ \frac{2}{3} x^{-1/3} \dx }_{x = 2, \dx = 0.06 } \\
                   & = 0.02 \\
        \\
        8.06^{2/3} & \approx 8^{2/3} + 0.02 \\
                   & = \boxed{ 4.02 } \\
      \end{align*}

    \item[26]
      \begin{align*}
        y              & = \frac{1}{x} \\
        \dy            & =  \eval{ - \frac{1}{x^2} \dx }_{x = 1000, \dx = 2 } \\
                       & = - \frac{1}{500,000} \\
        \\
        \frac{1}{1002} & \approx \frac{1}{1000} - \frac{1}{500,000} \\
                       & = \boxed{ \frac{499}{500,000} }
     \end{align*}

    \newpage

    \item[33]
      \begin{enumerate}[(a)]
        \item 
          Find the maximum error for the area:
          \begin{align*}
            A(x)   & = 6x^2 \\
            \dif A & = \eval{ 12x \dx }_{x = 30, \dx = 0.1} \\
                   & = \unit[36]{cm^2} \\
          \end{align*}

          The relative error is:
          \[
            \frac{\dif A}{A} = \frac{36}{5400} \approx 0.0067 \\
          \]

          The percentage error is 0.67\%

        \item 
          Find the maximum error for the volume:
          \begin{align*}
            V(x)   & = x^3 \\
            \dif V & = \eval{ 3x^2 \dx }_{x = 30, \dx = 0.1} \\
                   & = \unit[270]{cm^3} \\
          \end{align*}

          The relative error is:
          \[
            \frac{\dif V}{V} = \frac{270}{27,000} = 0.01 \\
          \]

          The percentage error is 1\%

      \end{enumerate}

    \newpage

    \item[36]
      The amount of paint needed is the difference in volume when the radius grows from 
      $\unit[50]{m}$ to $\unit[50.05]{m}$.

      \begin{align*}
        V(r)   & = \frac{2}{3} \pi r^3 \\
        \dif V & = \eval{ 2 \pi r^2 \dif r }_{r = 5,000, \dif r = 0.05 } \\
               & \approx \unit[7.85 \times 10^6]{cm^3} \\
               & = \boxed{ \unit[7.85]{m^3} } \\
      \end{align*}

  \end{description}

  \else
    \vspace{9 cm}
    \begin{quote}
      \begin{em}
        Scholars, who pride themselves on speaking their minds, often engage in a form of
        self-censorship which is called ``realism.'' To be ``realistic'' in dealing with a problem
        is to work only among the alternatives which the most powerful in society put forth. It is
        as if we are all confined to a, b, c, or d in the multiple choice test, when we know there
        is another possible answer. American society, although it has more freedom of expression
        than most societies in the world, thus sets limits beyond which respectable people are not
        supposed to think or speak.
      \end{em}
    \end{quote}
    \hspace{2 cm} --Howard Zinn
  \fi

\end{document}

