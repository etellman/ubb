\documentclass[letterpaper, landscape]{exam}
\usepackage{2in1, lscape} 
\printanswers

\usepackage{units} 
\usepackage[fleqn]{amsmath}
\usepackage{float}
\usepackage{mdwlist}
\usepackage{booktabs}
\usepackage{caption}
\usepackage{fullpage}
\usepackage{enumerate}
\usepackage{graphicx}
\usepackage[justification=justified]{caption}

\setcounter{tocdepth}{1}
\everymath{\displaystyle}

\author{}
\date{\today}
\title{Calculus I \\ Week Fourteen}

\begin{document}

  \maketitle
  \tableofcontents

  \section{Section 3.7}

  \begin{description}
    \item[5-6] See problems 5 and 6:
      \begin{itemize*}
        \item draw position graph and show when speeding up, slowing down
        \item draw velocity graph and show when speeding up, slowing down
      \end{itemize*}

    \item[problem 8]
      \begin{enumerate}[(a)]
        \item 
          \begin{align*}
            v(t) & = 6t + 5 \\
            v(2) & = \boxed{ \unit[17]{m} } \\
          \end{align*}

        \item
          \begin{align*}
            6t + 5 & = 35 \\
            t      & = \boxed{ \unit[5]{s} } \\
          \end{align*}
      \end{enumerate}

    \item[similar to 10]
      A ball thrown directly up at $\unit[144]{ft/s}$ has a height in feet of:
      \[
        s(t) = 144t - 16t^2
      \]

      \begin{enumerate}[(a)]
        \item find formulas for velocity and acceleration
          \begin{solution}
            \begin{align*}
              v(t) &= 144 - 32 t \\
              a(t) &= -32 \\
            \end{align*}
          \end{solution}

        \item What is the maximum height reached?
          \begin{solution}
            $s(\unit[4.5]{s}) = \unit[324]{ft}$
          \end{solution}

        \item How long does the flight last?
          \begin{solution}
            9 seconds--twice the maximum or set original equation equal to zero and solve
          \end{solution}

      \end{enumerate}

    \item[12]
      \begin{enumerate}[(a)]
        \item 

          \begin{solution}
            $V'(3) = \unit[27]{mm^3/s}$
          \end{solution}

        \item
          \begin{solution}
            \begin{align*}
              V'(x) & = 3x^2 \\
              A(x)  & = 6x^2 \\
            \end{align*}

            Draw picture of square and show for area of a square, adding $dx$ to each sides 
            results in:
            \begin{align*}
              dA            & = 2x dx \\
              \frac{dA}{dx} & = 2x \\
            \end{align*}

            Draw picture of cube and show for volume of a cube, adding $dx$ to each sides 
            results in:
            \begin{align*}
              dV            & = 3 \left( x^2 dx \right) \\
              dV            & = 3 x^2 dx \\
              \frac{dV}{dx} & = 3x^2 \\
            \end{align*}

          \end{solution}

      \end{enumerate}

    \item[13]
      \begin{enumerate}[(a)]
        \item 
          \begin{enumerate}[(i)]
            \item 2 to 3: $15.71$
            \item 2 to 2.5: $14.14$
            \item 2 to 2.1: $12.88$
          \end{enumerate}
        \item 
          \begin{align*}
            A(r)  & = \pi r^2 \\
            A'(r) & = 2 \pi r \\
            A'(2) & = 4 \pi \\
          \end{align*}

        \item Adding $\Delta r$ to the radius adds an almost rectangular strip with area:
          \[
            \Delta A \approx Delta r \cdot 2 \pi r \\
          \]
      \end{enumerate}

    \item[20]
      \begin{enumerate}[(a)]
        \item $F'(r) = -\frac{2 G m M}{r^3}$

          The minus sign indicates that the gravity decreases as $r$ increases.

        \item Since we are dealing with the one pair of objects, $GmM$ is a constant.
          \begin{align*}
            k         & = G m M \\
            \\
            -2        & = - \frac{2 k}{20000^3} \\
            k         & = 8 \times 10^{12} \\
            \\
            F'(r)     & = - \frac{1.6 \times 10^{13}}{r^3} \\
            F'(10000) & = \unit[-16]{N/km} \\
          \end{align*}

      \end{enumerate}

  \end{description}

  \section{Section 3.9} % (fold)

  \begin{description}

    \item[1]
      \begin{align*}
        V(x)          & = x^3 \\
        \frac{dV}{dt} & = \boxed{ 3x^2 \frac{dx}{dt} } \\
      \end{align*}

    \item[2]
      \begin{enumerate}[(a)]
        \item 
          \begin{align*}
            A(r)          & = \pi r^2 \\
            \frac{dA}{dt} & = \boxed{ 2 \pi r \frac{dr}{dt} } \\
          \end{align*}

        \item 
          \[
            A'(\unit[30]{m}) = 60 \pi \approx \unit[188.5]{m^2/s}
          \]
      \end{enumerate}

    \item[13]
      Draw picture with:
      \begin{itemize*}
        \item x distance from pole
        \item y length of shadow
      \end{itemize*}

      \begin{align*}
        \frac{15}{6}  & = \frac{x + y}{y} \\
        \frac{dy}{dt} & = \frac{2}{3} \frac{dx}{dt} \\
                      & = \unit[\frac{10}{3}]{ft/s} \\
      \end{align*}

      Rate of shadow is the sum of growth of shadow and rate of man, or $\unit[\frac{25}{3}]{ft/s}$.

      The shadow moves at a constant rate--the distance from the pole doesn't matter.

    \item[15]
      \begin{align*}
        s             & = \sqrt{x^2 + y^2} \\
        \frac{ds}{dt} & = \frac{x x' + y y'}{\sqrt{x^2 + y^2}} \\
      \end{align*}

      This equation works for any function of $x$ and $y$ where the objects are moving at right angles.

      notes:
      \begin{itemize*}
        \item when one $x = x' = 0$, $s' = y'$
        \item when $x' = 0$ and $x \neq 0$, $\lim_{y \to \infty} s' = y'$
        \item separation doesn't depend on time because at any point, the triangle is similar to
          every other point.
      \end{itemize*}

      At the time we're interested in:
      \begin{align*}
        x             & = 50 \\
        x'            & = 25 \\
        y             & = 120 \\
        y'            & = 60 \\
        \\
        \frac{ds}{dt} & = 65 \\
      \end{align*}

      In this case, it's easy to get a simple expression for $s$:
      \begin{align*}
        s(t)          & = \sqrt{(25t^2) + (60t^2)} \\
                      & = 65t \\
        \\
        \frac{ds}{dt} & = 65 \\
      \end{align*}

      If cars aren't starting from the origin:
      \begin{align*}
        x &= 100 + 25t \\
        y &= 50 + 60t \\
      \end{align*}

      Initially:
      \begin{align*}
        x             & = 100 \\
        y             & = 50 \\
        \frac{ds}{dt} & \approx \unit[49.2]{mph}
      \end{align*}

      After one hour:
      \begin{align*}
        x             & = 125 \\
        y             & = 110 \\
        \frac{ds}{dt} & \approx \unit[58.4]{mph}
      \end{align*}

      After two hours:
      \begin{align*}
        x             & = 150 \\
        y             & = 170 \\
        \frac{ds}{dt} & \approx \unit[61.5]{mph}
      \end{align*}

      After 50 hours:
      \begin{align*}
        x             & = 1350 \\
        y             & = 3050 \\
        \frac{ds}{dt} & \approx \unit[64.98]{mph}
      \end{align*}

      Over time the initial conditions affect the outcome less and less and the result approaches 
      65 mph.

    \item[20]
      If $x$ is the distance between the boat and the dock and $r$ is the length of the rope:
      \begin{align*}
        r^2           & = x^2 + 1 \\
        \frac{dx}{dt} & = \frac{r}{x} \cdot \frac{dr}{dt} \\
      \end{align*}

      When the boat is 8 feet from the dock:
      \[
        \frac{dx}{dt} = \boxed{ \unit[\frac{\sqrt{65}}{8}]{ft/s} }
      \]

    \item[23]
      Because of the shape of the cone:
      \begin{align*}
        r             & = \frac{h}{3} \\
        \\
        V             & = \frac{1}{3} \pi r^2 h \\
                      & = \frac{1}{27} \pi h^3 \\
        \\
        \frac{dV}{dt} & = \frac{1}{9} \pi h^2 \frac{dh}{dt} \\
                      & = \frac{1}{9} \pi \cdot 200^2 \cdot 20 \\
                      & \approx \unit[279,253]{cm^3/min}
      \end{align*}

      The rate is:
      \begin{align*}
        r_{total} & = r_{in} - r_{out} \\
        r_{in}    & = r_{total} + r_{out} \\
                  & = \boxed{ \unit[289,253]{cm^3/min} } \\
      \end{align*}
  \end{description}

\end{document}

