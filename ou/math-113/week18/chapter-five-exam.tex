\documentclass[fleqn,addpoints]{exam}
\usepackage{amsmath}
\usepackage{graphicx}
\usepackage{cancel}
\usepackage{polynom}

\printanswers

\ifprintanswers
\usepackage{2in1, lscape}
\fi

\title{Math 113 Chapter Five Exam}
\author{}
\date{\today}

% \oddsidemargin 0in
% \topmargin -0.5in
% \textwidth 6.5in

% \extrawidth{-1 in}
% \setlength{\mathindent}{0in}

\begin{document}

\maketitle

\ifprintanswers
\else
\vspace{0.2in}
\makebox[\textwidth]{Name:\enspace\hrulefill}
\vspace{0.2in}

\begin{center}
\gradetable[h][pages]
\end{center}

\fi

\section{Simplifying  Expressions}

For problems \ref{simplify:first}-\ref{simplify:last}, simplify each expression so there are no zero or negative exponents.

\begin{questions}

\question[3] \( 3^{-4} \cdot 3^6 \)
\label{simplify:first}
\begin{solution}[2 cm]
\[
  3^{-4} \cdot 3^6 = 3^2 = 9
\]
\end{solution}

\question[3] \( \left( - \dfrac{1}{2} \right)^{-3} \)
\begin{solution}[2 cm]
\[
  \left( - \dfrac{1}{2} \right)^{-3} = \frac{-1}{2^{-3}} = -8
\]
\end{solution}


\question[5] \( (3^{-1} - 5^{-1} )^{-1} \)
\begin{solution}[3 cm]
$-2$ was a common answer for this one.  But this is not correct because $(x^a + y^b)^c \neq x^{ac} + y^{bc}$.  You can
only apply the outer exponent to everything on the inside when everything on the inside is multiplied together.

Here's the correct way:
\[
  (3^{-1} - 5^{-1} )^{-1} = (\frac{1}{3} - \frac{1}{5})^{-1} = \left( \frac{5}{15} - \frac{3}{15} \right )^{-1} 
  = \left( \frac{2}{15} \right)^{-1} = \frac{15}{2}
\]
\end{solution}

\question[5] \( \left( \dfrac{y^3}{x^2} \right)^{-2} \)
\begin{solution}[3 cm]
\[
  \left( \frac{y^3}{x^2} \right)^{-2} = \frac{y^{-6}}{x^{-4}} = \frac{x^4}{y^6}
\]
\end{solution}

\question[7] \( \left( \dfrac{8xy^3}{-4x^4y^2} \right)^{-3} \)
\label{simplify:last}
\begin{solution}[3 cm]
\begin{align*}
  \left( \frac{8xy^3}{-4x^4y^2} \right)^{-3} &= \left( \frac{-2xy^3}{x^4y^2} \right)^{-3} \\
  &= (-2x^{-3}y)^{-3} \\
  &= (-2)^{-3}x^9y^{-3} \\
  &= - \frac{x^9}{8y^3} \\
\end{align*}
\end{solution}

\section{Simplest Radical Form}

For problems \ref{srf:first}-\ref{srf:last}, convert each expression to simplest radical form.

\question[5] \( \sqrt{48} \)
\label{srf:first}
\begin{solution}[3 cm]
\[
  \sqrt{48} = \sqrt{2^4 \cdot 3} = 4\sqrt{3}
\]
\end{solution}

\question[5] \( \dfrac{1}{5} \sqrt{75} \)
\begin{solution}[3 cm]
\[
  \dfrac{1}{5} \sqrt{75} = \frac{\sqrt{5^2 \cdot 3}}{5} = \frac{5 \sqrt{3}}{5} = \sqrt{3}
\]
\end{solution}

\question[5] \( \dfrac{2 \sqrt{3}}{3 \sqrt{2}} \)
\begin{solution}[3 cm]
\[
  \frac{2 \sqrt{3}}{3 \sqrt{2}} \left( \frac{\sqrt{2}}{\sqrt{2}} \right) = \frac{2 \sqrt{6}}{2 \cdot 3} = \frac{\sqrt{6}}{3}
\]
\end{solution}

\question[5] \( \sqrt{\dfrac{22}{24}} \)
\begin{solution}[3 cm]
\[
  \sqrt{\frac{22}{24}} = \sqrt{\frac{11}{12}} = \frac{\sqrt{11}}{\sqrt{2^2 \cdot 3}} 
  = \frac{\sqrt{11}}{2 \sqrt{3}} \left( \frac{\sqrt{3}}{\sqrt{3}} \right)
  = \frac{\sqrt{33}}{6}
\]
\end{solution}

% \question[5] \( \dfrac{7}{\sqrt[3]{7}} \)
% \begin{solution}[3 cm]
% \[
% \]
% \end{solution}

\question[7] \( \dfrac{\sqrt{12a^2b^2}}{\sqrt{6a^3b}} \)
\begin{solution}[3 cm]
\[
  \frac{\sqrt{12a^2b^2}}{\sqrt{6a^3b}} = \sqrt{\frac{6a^2b^2}{a^3b^3}} 
  = \frac{\sqrt{2b}}{\sqrt{a}} \left( \frac{\sqrt{a}}{\sqrt{a}} \right)
  = \frac{\sqrt{2ab}}{a}
\]
\end{solution}

\question[7] \( \dfrac{\sqrt[3]{2y}}{\sqrt[3]{24x^4}} \)
\label{srf:last}
\begin{solution}[5 cm]
\begin{align*}
  \frac{\sqrt[3]{2y}}{\sqrt[3]{24x^4}} &= \frac{\sqrt[3]{y}}{x \sqrt[3]{12x}} \\
  &= \frac{\sqrt[3]{y}}{x \sqrt[3]{2^2 \cdot 3 \cdot x}} 
      \left( \frac{\sqrt[3]{2 \cdot 3^2 \cdot x^2}}{\sqrt[3]{2 \cdot 3^2 \cdot x^2}} \right) \\
  &= \frac{\sqrt[3]{18x^2y}}{6x^2}
\end{align*}

\end{solution}

\section{Adding and Subtracting Radical Expressions}

For problems \ref{add:first}-\ref{add:last}, perform each addition and/or subtraction and express the result in simplest
radical form.

\question[5] \( 2\sqrt{8x} - \sqrt{18x} + 4\sqrt{2x} \)
\label{add:first}
\begin{solution}[5 cm]
\[
  2\sqrt{8x} - \sqrt{18x} + 4\sqrt{2x}
  = 4 \sqrt{2x} - 3\sqrt{2x} + 4\sqrt{2x}
  = 5 \sqrt{2x}
\]
\end{solution}

\question[5] \( \dfrac{2 \sqrt{90}}{3} - \dfrac{3 \sqrt{40}}{2} \)
\begin{solution}[5 cm]
\[
  \frac{2 \sqrt{90}}{3} - \frac{3 \sqrt{40}}{2} 
  = \frac{6 \sqrt{10}}{3} - \frac{6 \sqrt{10}}{2} 
  = 2 \sqrt{10} - 3 \sqrt{10}
  = - \sqrt{10}
\]
\end{solution}

\question[5] \( 2 \sqrt[3]{24} + 5 \sqrt[3]{3} - 3\sqrt[3]{81} \)
\label{add:last}
\begin{solution}[5 cm]
\[
  2 \sqrt[3]{24} + 5 \sqrt[3]{3} - 3\sqrt[3]{81}
  = 4\sqrt[3]{3} + 5\sqrt[3]{3} - 9 \sqrt[3]{3}
  = 0
\]
\end{solution}

% \question[5] \( 2 \sqrt{2ab} - 3 \sqrt{72 ab} + 2 \sqrt{8ab} \)
% \begin{solution}[5 cm]
% \[
% \]
% \end{solution}


\section{Multiplying Radical Expressions}

For problems \ref{multiply:first}-\ref{multiply:last}, perform each multiplication or division, simplify the result as
much as possible, and express the answer in simplest radical form.

\question[3] \( \sqrt{3x} ( \sqrt{2x} - \sqrt{7}) \)
\label{multiply:first}
\begin{solution}[5 cm]
\[
  \sqrt{3x} ( \sqrt{2x} - \sqrt{7}) = x\sqrt{6} - \sqrt{21x}
\]
\end{solution}

\question[7] \( (3\sqrt{x} - 4 \sqrt{y})(3 \sqrt{x} + 4\sqrt{y}) \)
\begin{solution}[5 cm]
\begin{align*}
  (3\sqrt{x} - 4 \sqrt{y})(3 \sqrt{x} + 4\sqrt{y}) &= 9x + 12\sqrt{xy} - 12\sqrt{xy} - 16 y \\
  &= 9x-16y
\end{align*}


\end{solution}

\question[7] \( (\sqrt{8} + 5 \sqrt{10})(2\sqrt{8} - \sqrt{10}) \)
\label{multiply:last}
\begin{solution}[5 cm]
\begin{align*}
  (\sqrt{8} + 5 \sqrt{10})(2\sqrt{8} - \sqrt{10}) &= 16 - \sqrt{8 \cdot 10} + 10 \sqrt{8 \cdot 10} - 50 \\
  &= -34 + 9 \sqrt{2^4 \cdot 5} \\
  &= -34 + 36 \sqrt{5} \\
\end{align*}
\end{solution}


\section{Rationalizing the Denominator}

For problems \ref{rationalize:first}-\ref{rationalize:last}, rationalize each denominator and simplify the result as
much as possible.

\question[5] \( \dfrac{\sqrt{5}}{x \sqrt{5} + \sqrt{2}} \)
\label{rationalize:first}
\begin{solution}[3 cm]
\[
  \frac{\sqrt{5}}{x \sqrt{5} + \sqrt{2}} \left( \frac{x \sqrt{5} - \sqrt{2}}{x \sqrt{5} - \sqrt{2}} \right)
  = \frac{5x-\sqrt{10}}{5x^2-2}
\]
\end{solution}

\question[5] \( \dfrac{\sqrt{x} + 1}{\sqrt{x} - 3} \)
\label{rationalize:last}
\begin{solution}[3 cm]
\[
  \frac{\sqrt{x} + 1}{\sqrt{x} - 3} \left( \frac{\sqrt{x} + 3}{\sqrt{x} + 3} \right)
  = \frac{x + 4\sqrt{x} + 3}{x-9}
\]
\end{solution}

\section{Equations with Radicals}

For problems \ref{equation:first}-\ref{equation:last}, solve each equation.  Don't forget to check the answer to make
sure it actually works in the original equation.

\question[5] \( \sqrt{3x-1}  - \sqrt{x+5} = 0\)
\label{equation:first}
\begin{solution}[5 cm]
\begin{align*}
  \sqrt{3x-1}  - \sqrt{x+5} &= 0 \\
  \sqrt{3x-1} &= \sqrt{x+5} \\
  3x-1 &= x+5 \\
  2x &= 6 \\
  x &= 3 \\
\end{align*}

check: \( \sqrt{3 \cdot 3 -1}  - \sqrt{3+5} = 0\)

\end{solution}

% \question[5] \( \sqrt{x^2 - 8} = x+4\)
% \begin{solution}[5 cm]
% \end{solution}

% \question[5] \( \sqrt{x+1} = 2 - \sqrt{x}\)
% \begin{solution}[5 cm]
% \end{solution}

\question[7] \( \sqrt{7x+2} = x + 2 \)
\begin{solution}[5 cm]
\begin{align*}
  \sqrt{7x+2} &= x + 2 \\
  7x+2 &= (x + 2)^2 \\
  7x+2 &= x^2+4x+4 \\
  x^2-3x+2 &= 0 \\
  (x-2)(x-1) &= 0 \\
\end{align*}

$x = \{1, 2\}$

check: 
\begin{itemize}
  \item \( \sqrt{7+2} = 1 + 2 \)
  \item \( \sqrt{14+2} = 2 + 2 \)
\end{itemize}
\end{solution}

% \question[5] \( \sqrt{2x-3} = x - 3 \)
% \begin{solution}[5 cm]
% \end{solution}

\question[10] \( \sqrt{2x+4} - \sqrt{x+3} = 1 \)
\label{equation:last}
\begin{solution}[7 cm]
\begin{align*}
  \sqrt{2x+4} - \sqrt{x+3} &= 1 \\
  \sqrt{2x+4}  &= 1 + \sqrt{x+3} \\
  2x+4  &= 1 + 2\sqrt{x+3} + x + 3\\
  x  &= 2\sqrt{x+3} \\
  x^2  &= 4(x+3) \\
  x^2  &= 4x+12 \\
  x^2 - 4x - 12 &= 0 \\
  (x - 6)(x+2) &= 0 \\
\end{align*}

$x = \{-2, 6\}$

check:
\begin{itemize}
  \item $\sqrt{2(-2)+4} - \sqrt{(-2)+3} = -1 \neq 1$
  \item $\sqrt{2(6)+4} - \sqrt{6+3} = 1$
\end{itemize}

So $6$ is the only solution.

\end{solution}

\section{Fractional Exponents}

For problems \ref{fe:first}-\ref{fe:last}, evaluate each expression.

\question[5] \( \left( \dfrac{16}{25} \right)^{- \frac{1}{2}} \)
\label{fe:first}
\begin{solution}[2 cm]
\[
  \left( \dfrac{16}{25} \right)^{- \frac{1}{2}} = \left( \dfrac{25}{16} \right)^{\frac{1}{2}} = \sqrt{\frac{25}{16}} = \frac{5}{4}
\]
\end{solution}

\question[5] \( -25^{\frac{3}{2}} \)
\label{fe:last}
\begin{solution}[2 cm]
\[
  -25^{\frac{3}{2}} = - (\sqrt{25})^3 = -5^3 = -125
\]
\end{solution}

For problems \ref{rf:first}-\ref{rf:last}, write in simplest radical form.

\question[3] \( (3x^2y^3)^{\frac{1}{2}} \)
\label{rf:first}
\begin{solution}[2 cm]
\[
  (3x^2y^3)^{\frac{1}{2}} = \sqrt{3x^2y^3} = xy\sqrt{3y}
\]
\end{solution}

\question[3] \( (2x+y)^{\frac{2}{5}} \)
\label{rf:last}
\begin{solution}[2 cm]
\[
  (2x+y)^{\frac{2}{5}} = \sqrt[5]{(2x+y)^2}
\]
\end{solution}

For problems \ref{pre:first}-\ref{pre:last}, write each expression using positive rational exponents.

\question[3] \( \sqrt[3]{x^2y^6} \)
\label{pre:first}
\begin{solution}[3 cm]
\[
  \sqrt[3]{x^2y^6} = x^{\frac{2}{3}} y^{\frac{6}{3}} = x^{\frac{2}{3}} y^2
\]
\end{solution}

\question[3] \( \sqrt{2xy^3} \)
\label{pre:last}
\begin{solution}[3 cm]
\[
  \sqrt{2xy^3} = 2^{\frac{1}{2}} x^{\frac{1}{2}} y^{\frac{3}{2}}
\]
\end{solution}

\ifprintanswers
\else
\pagebreak
\fi

For problems \ref{resimp:first}-\ref{resimp:last}, simplify each expression and express the result using positive
rational exponents.

\question[5] \( (3x^{\frac{1}{4}})(5x^{\frac{1}{3}}) \)
\label{resimp:first}
\begin{solution}[3 cm]
\[
  (3x^{\frac{1}{4}})(5x^{\frac{1}{3}}) = 15 x^{\frac{1}{3} + \frac{1}{4}} = 15 x^{\frac{4}{12} + \frac{3}{12}} = 15 x^{\frac{7}{12}}
\]
\end{solution}

\question[5] \( (9x^6y^8)^{\frac{1}{2}} \)
\begin{solution}[3 cm]
\[
  (9x^6y^8)^{\frac{1}{2}} = 9^{\frac{1}{2}} x^{\frac{6}{2}} y^{\frac{8}{2}} = 3x^3y^4
\]
\end{solution}

\question[5] \( \dfrac{15x^{\frac{3}{4}} y^{\frac{1}{2}}} {5x^{\frac{1}{2}} y^{- \frac{1}{2}}} \)
\begin{solution}[3 cm]
\[
  \frac{15x^{\frac{3}{4}} y^{\frac{1}{2}}} {5x^{\frac{1}{2}} y^{- \frac{1}{2}}}
  = 3 x^{\frac{3}{4} - \frac{1}{2}} y^{\frac{1}{2} + \frac{1}{2}} 
  = 3 x^{\frac{1}{4}} y
\]
\end{solution}

\question[5] \( \left( \dfrac{2x^{\frac{1}{3}}}{xy^{\frac{2}{3}}} \right)^{-3} \)
\label{resimp:last}
\begin{solution}[3 cm]
\[
  \left( \dfrac{2x^{\frac{1}{3}}}{xy^{\frac{2}{3}}} \right)^{-3}
  = \frac{2^{-3} x^{-1}}{x^{-3}y^{-2}} = \frac{x^2y^2}{8}
\]
\end{solution}

\ifprintanswers
\else
\pagebreak
\fi

\section{Scientific Notation}

For problems \ref{tosci:first}-\ref{tosci:last}, write each expression in scientific notation.

\question[3] \( 44,000,000 \)
\label{tosci:first}
\begin{solution}[1 cm]
\( 44,000,000 = (4.4)(10^7)\)
\end{solution}

\question[3] \( 0.00075 \)
\label{tosci:last}
\begin{solution}[1 cm]
\( 0.00075 = (7.5)(10^{-4}) \)
\end{solution}

For problems \ref{fromsci:first}-\ref{fromsci:last}, write each expression in decimal notation.

\question[3] \( (3.4)(10^7) \)
\label{fromsci:first}
\begin{solution}[1 cm]
  \( (3.4)(10^7) = 34,000,000 \)
\end{solution}

\question[3] \( (1.25)(10^{-5}) \)
\label{fromsci:last}
\begin{solution}[1 cm]
\( (1.25)(10^{-5}) = 0.0000125 \)
\end{solution}

For problems \ref{calcsci:first}-\ref{calcsci:last}, use scientific notation to calculate the result.

\question[5] \( \dfrac{42,000}{0.00006} \)
\label{calcsci:first}
\begin{solution}[4 cm]
\[
  \frac{42,000}{0.00006} = \frac{(42)(10^3)}{(6)(10^{-5})} = (7)(10^8) = 700,000,000
\]
\end{solution}

\question[5] \( (0.00008)(3,000,000) \)
\label{calcsci:last}
\begin{solution}[4 cm]
\[
  (0.00008)(3,000,000) = (8)(10^{-5}) (3)(10^6) = (24)(10^1) = 240
\]
\end{solution}

\pagebreak

\noaddpoints

\section{Extra Credit}

For problems \ref{extra:first}-\ref{extra:last}, solve each equation.

\question[7]
\label{extra:first} \( \sqrt{ \sqrt{x+25} + \sqrt{x}} = 5 \) 
\begin{solution}[9 cm]
\begin{align*}
  \sqrt{ \sqrt{x+25} + \sqrt{x}} &= 5 \\
  \sqrt{x+25} + \sqrt{x} &= 25 \\
  \sqrt{x+25}  &= 25 - \sqrt{x} \\
  x + 25  &= 25^2 - 50\sqrt{x} + x \\
  25  &= 25^2 - 50\sqrt{x} \\
  1  &= 25 - 2\sqrt{x} \\
  -24  &= -2\sqrt{x} \\
  \sqrt{x} &= 12 \\
  x &= 144 \\
\end{align*}

check: 
\begin{align*}
  \sqrt{ \sqrt{144+25} + \sqrt{144}} &= \sqrt{ \sqrt{169} + \sqrt{144}} \\
  &= \sqrt{13 + 12} \\
  &= \sqrt{25} \\
  &= 5 \\
\end{align*}

\end{solution}

\question[7]
\label{extra:last} \( (3x-1)^{\frac{2}{3}} = (5x^2 - x)^{\frac{1}{3}} \) 
\begin{solution}[7 cm]
\begin{align*}
  (3x-1)^{\frac{2}{3}} = (5x^2 - x)^{\frac{1}{3}} \\
  ((3x-1)^{\frac{2}{3}})^3 = ((5x^2 - x)^{\frac{1}{3}})^3 \\
  (3x-1)^2 = 5x^2 - x \\
  9x^2-6x+1 = 5x^2 - x \\
  4x^2-5x+1 = 0 \\
  (4x-1)(x-1) = 0 \\
\end{align*}

$x = \{ \dfrac{1}{4}, 1 \}$

check:
\begin{itemize}
  \item \( (3-1)^{\frac{2}{3}} = 2^{\frac{2}{3}} = 4^{\frac{1}{3}} = (5 - 1)^{\frac{1}{3}} \)
  \item \( \displaystyle \left( \frac{3}{4}-1 \right)^{\frac{2}{3}} = \left(- \frac{1}{4} \right)^{\frac{2}{3}} 
    = \left( \frac{1}{16} \right)^{\frac{1}{3}} = \left(\frac{5}{16} - \frac{1}{4} \right)^{\frac{1}{3}}\)
\end{itemize}

\end{solution}

\end{questions}
\end{document}


