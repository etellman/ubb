\documentclass[fleqn,addpoints]{exam}
\usepackage{amsmath}
\usepackage{graphicx}
\usepackage{cancel}
\usepackage{polynom}

\printanswers

\ifprintanswers
\usepackage{2in1, lscape}
\fi

\title{Math 113 Chapter Six Exam}
\author{}
\date{\today}

% \oddsidemargin 0in
% \topmargin -0.5in
% \textwidth 6.5in

% \extrawidth{-1 in}
% \setlength{\mathindent}{0in}

\begin{document}

\maketitle

\ifprintanswers
\else
\vspace{0.2in}
\makebox[\textwidth]{Name:\enspace\hrulefill}
\vspace{0.2in}

\begin{center}
\gradetable[h][pages]
\end{center}

\fi

\section{Complex Numbers}

For problems \ref{complex-arithmetic:first}-\ref{complex-arithmetic:last} add, subtract, or multiply as indicated, 

\begin{questions}

\question[3] \( (2 + 5i) + (3 - 2i) \)
\label{complex-arithmetic:first}
\begin{solution}[3 cm]
\[
  (2 + 5i) + (3 - 2i) = (2+3) + (5-2)i = 5+3i
\]
\end{solution}

\question[3] \( (10 -7i) - (10 - 2i) \)
\begin{solution}[3 cm]
\begin{align*}
  (10 -7i) - (10 - 2i) &= (10-10) + (-7 - (-2))i \\
  &= (-7+2i) \\
  &= -5i \\
\end{align*}

\end{solution}

\question[5] \( (5 + 6i) (3 - 3i) \)
\begin{solution}[4 cm]
\begin{align*}
  (5 + 6i) (3 - 3i) &= 15 + 18i - 15i - 18i^2 \\
  & = 15 + 3i + 18 \\
  & = 33+3i
\end{align*}

\end{solution}

\question[5] \( (-2 - i) (-2 + i) \)
\label{complex-arithmetic:last}
\begin{solution}[4 cm]
\[
  (-2 - i) (-2 + i) = 4 - i^2 = 4+1 = 5
\]
\end{solution}

For problems \ref{complex-simplify:first}-\ref{complex-simplify:last} write in terms if $i$, perform the indicated
operations and simplify.

\question[3] \( \sqrt{-45} \)
\label{complex-simplify:first}
\begin{solution}[3 cm]
\[
  \sqrt{-45} = i \sqrt{9 \cdot 5} = 3i \sqrt{5}
\]
\end{solution}

\question[3] \( -3 \sqrt{-16} \)
\begin{solution}[3 cm]
\[
  -3 \sqrt{-16} = -3i \sqrt{16} = -12i
\]
\end{solution}

\question[5] \( \sqrt{-9} \sqrt{-12} \)
\begin{solution}[4 cm]
\[
  \sqrt{-9} \sqrt{-12} = (i \sqrt{9})(i \sqrt{12}) = i^2 \cdot 3 \cdot 2\sqrt{3} = -6 \sqrt{3}
\]
\end{solution}

\question[5] \( \dfrac{\sqrt{-50}} {\sqrt{75}} \)
\label{complex-simplify:last}
\begin{solution}[4 cm]
If you simplify early on, as in the solution below, you can avoid doing calculations like $50*75$.

\begin{align*}
  \frac{\sqrt{-50}} {\sqrt{75}} &= \frac{i \sqrt{50}} {\sqrt{75}}  \\
  &= i \sqrt{\frac{50}{75}} \\
  &= i \sqrt{ \frac{\cancel{25} \cdot 2}{\cancel{25} \cdot 3}} \\
  &= \frac{i \sqrt{2}}{\sqrt{3}} \\
  &= \frac{i \sqrt{6}}{3} \\
\end{align*}

\end{solution}

\ifprintanswers
\else
\pagebreak
\fi

For problems \ref{complex-divide:first}-\ref{complex-divide:last} find the quotient and express the answer in the
standard form of a complex number.

\question[5] \( \dfrac{3}{5i} \)
\label{complex-divide:first}
\begin{solution}[4 cm]
Very few people did the simplest thing, which is:
\[
  \frac{3}{5i} = \frac{3}{5i} \left( \frac{i}{i} \right) = \frac{3i}{5i^2} = - \frac{3i}{5}
\]

Most people multiplied the numerator and denominator by $5i$, which also gives the correct solution but is a little more work.

\end{solution}

\question[7] \( \dfrac{-2 + 3i}{2 - 2i} \)
\label{complex-divide:last}
\begin{solution}[5 cm]
For this problem, you need to multiply the numerator and denominator by the complex conjugate (the same number with the
sign of the imaginary part reversed):
\begin{align*}
  \frac{-2 + 3i}{2 - 2i} &= \frac{-2 + 3i}{2 - 2i} \left( \frac{2+2i}{2+2i} \right) \\
  &= \frac{-4 + 6i - 4i + 6i^2}{4-4i^2} \\
  &= \frac{-4 + 2i - 6}{4+4} \\
  &= \frac{-10 + 2i}{8} \\
  &= \frac{2(-5 + i)}{8} \\
  &= \frac{-5 + i}{4} \\
  &= -\frac{5}{4} + \frac{i}{4} \\
\end{align*}



\end{solution}

\section{Completing the Square}
For problems \ref{square:first}-\ref{square:last}, solve each problem by completing the square.

\question[10] \( x^2 - 12x + 3 = 0 \)
\label{square:first}
\begin{solution}[6 cm]
\begin{align*}
  x^2 - 12x + 3 &= 0 \\
  x^2 - 12x &= -3 \\
  x^2 - 12x + 36 &= -3 + 36 \\
  (x-6)^2 &= 33 \\
  x-6 &= \pm \sqrt{33} \\
  x &= 6 \pm \sqrt{33} \\
\end{align*}

check:
\begin{itemize}
  \item \( (6 + \sqrt{33}) + (6 - \sqrt{33}) = 12 \)
  \item \( (6 + \sqrt{33})(6 - \sqrt{33}) = 36 - 33 = 3 \)
\end{itemize}

\end{solution}

\question[10] \( 2x^2 + 3x + 5 = 0 \)
\label{square:last}
\begin{solution}[6 cm]
When you get rid of the $2$ in $2x^2$, make sure you divide all the factors by $2$, not just the
$5$ on the right side of the equation.

\begin{align*}
  2x^2 + 3x + 5 &= 0 \\
  2x^2 + 3x &= -5 \\
  x^2 + \frac{3x}{2}  &= - \frac{5}{2} \\
  x^2 + \frac{3x}{2} + \frac{9}{16} &= - \frac{5}{2} + \frac{9}{16} \\
  \left(x + \frac{3}{4} \right)^2 &= - \frac{40}{16} + \frac{9}{16} \\
  \left(x + \frac{3}{4} \right)^2 &= - \frac{31}{16} \\
  x + \frac{3}{4} &= \pm \sqrt{-\frac{31}{16}} \\
  x  &= -\frac{3}{4} \pm \frac{i \sqrt{31}}{4} \\
  x &= \frac{-3 \pm i \sqrt{31}}{4} \\
\end{align*}
\end{solution}

% \question[10] \( 3x^2 - 7x - 11 = 0 \)
% \label{square:last}
% \begin{solution}[6 cm]
% \[
% \]
% \end{solution}

\section{Quadratic Formula}
For problems \ref{quadratic-formula:first}-\ref{quadratic-formula:last}, solve each problem using the quadratic formula.

\question[10] \( x^2 + 8x - 2 = 0\)
\label{quadratic-formula:first}
\begin{solution}[6 cm]
\begin{align*}
  x &= \frac{-8 \pm \sqrt{64 - 4(-2)}}{2} \\
  &= \frac{-8 \pm \sqrt{72}}{2} \\
  &= \frac{-8 \pm \sqrt{36 \cdot 2}}{2} \\
  &= \frac{-8 \pm 6 \sqrt{2}}{2} \\
  &= \frac{2(-4 \pm 3 \sqrt{2})}{2} \\
  &= -4 \pm 3 \sqrt{2} \\  
\end{align*}
\end{solution}

\question[10] \( -x^2 + 5x - 3 = 0 \)
\begin{solution}[6 cm]
Some people started out by converting the equation to $x^2-5x+3=0$ which is also fine and may make it less likely to
make a mistake with the signs.

\begin{align*}
  x &= \frac{-5 \pm \sqrt{25 - 4(-1)(-3)}}{-2} \\
    &= \frac{-5 \pm \sqrt{25 - 12}}{-2} \\
    &= \frac{-5 \pm \sqrt{13}}{-2} \\
    &= \frac{5 \pm \sqrt{13}}{2} \\
\end{align*}

\end{solution}

\question[10] \( 4x^2 - 9x = -7 \)
\label{quadratic-formula:last}
\begin{solution}[6 cm]
\begin{align*}
  x &= \frac{9 \pm \sqrt{81 - 4 \cdot 4 \cdot 7}}{8} \\
   &= \frac{9 \pm \sqrt{81 - 112}}{8} \\
   &= \frac{9 \pm \sqrt{-31}}{8} \\
   &= \frac{9 \pm i \sqrt{31}}{8} \\
\end{align*}
\end{solution}

% \section{Quadratic Equations}
% For problems \ref{quadratic:first}-\ref{quadratic:last}, solve each problem using any method.  Or, as Malcolm X might
% say, ``By any means necessary.''

% \question[7] \( (x+2)^2 = -5 \)
% \label{quadratic:first}
% \begin{solution}[5 cm]
% \[
% \]
% \end{solution}

% \question[7] \( 3x^2 + 5x + 2 = 0 \)
% \label{quadratic:last}
% \begin{solution}[5 cm]
% \[
% \]
% \end{solution}

\section{Nonlinear Inequalities}

For problems \ref{nonlinear:first}-\ref{nonlinear:last}, solve each inequality.

\question[10] \( 2x^2 + 5x - 3 \geq 0 \)
\label{nonlinear:first}
\begin{solution}[6 cm]
First solve for $x$:
\begin{align*}
  2x^2 + 5x - 3 &= 0 \\
  (2x-1)(x+3) &= 0 \\
  x &= \left \{ -3, \frac{1}{2} \right \}
\end{align*}

Then use a sample point from each region to find the solution:
\[
 x = (-\infty, -3] \cup \left[ \frac{1}{2}, \infty \right )
\]
\end{solution}

\question[10] \( \dfrac{2x+1}{x-5} < 0 \)
\label{nonlinear:last}
\begin{solution}[6 cm]
The interesting points are: $x = - \dfrac{1}{2}$ and $x = 5$.
\[
  x = \left( - \frac{1}{2}, 5 \right)- \frac{1}{2} < x < 5
\]
\end{solution}


\section{Word Problems}

\question[10] 
The length of a rectangle is 1 inch less than twice its width.  find the sides of the rectangle if its area is 5 inches.
\begin{solution}[9 cm]
From the problem statement:
\begin{itemize}
  \item The length is 1 inch less than twice the width: $L = 2W - 1$
  \item The area is 5 square inches: $LW=5$
\end{itemize}

We can plug the value for $L$ into the area equation and solve for $W$:
\begin{align*}
  W(2W-1) &= 5 \\
  2W^2-W &= 5 \\
  2W^2-W = 5 &= 0 \\
\end{align*}

Using the quadratic formula:
\begin{align*}
  W &= \frac{1 \pm \sqrt{1 - 5(-5)(2)}}{2} \\
  &= \frac{1 \pm \sqrt{41}}{4} \\
\end{align*}

The only positive choice for $W$ is $W = \dfrac{1 + \sqrt{41}}{4}$.  We can use this to get $L$
\[
  L = 2W-1 = 2 \left( \frac{1 + \sqrt{41}}{4} \right) - 1 = \frac{1 + \sqrt{41}}{2} - \frac{2}{2} = \frac{-1 + \sqrt{41}}{2}
\]

To check the area:
\begin{align*}
  LW &= \left( \frac{1 + \sqrt{41}}{4} \right) \left( \frac{-1 + \sqrt{41}}{2} \right) \\
  &= \frac{-1 + 41}{8} \\
  &= 5 \\
\end{align*}

\end{solution}

\question[10] 
Working together, 2 people can complete a task in 5 hours.  Working alone, one person takes 2 hours longer than the
other.  How long would it take each person to do the task alone?

\begin{solution}[9 cm]
From the problem statement:
\begin{itemize}
  \item 2 people can complete a task in 5 hours: $t_{1+2} = 5$
  \item one person takes two hours longer than the other: $t_2 = t_1 + 2$
\end{itemize}

The rate for two people working together is the sum of their individual rates: 

$r_{1+2} = r_1 + r_2 = \dfrac{1}{t_1} + \frac{1}{t_2} = \dfrac{1}{t_1} + \frac{1}{t_1+2}$.

In 5 hours, the complete 1 job.  This gives us an equation to solve:
\begin{align*}
  5 \left( \dfrac{1}{t_1} + \frac{1}{t_1+2} \right) &= 1 \\
  \frac{5}{t_1} + \frac{5}{t_1+2} = 1 \\
  \frac{5t_1 + 10 + 5t_1}{t_1(t_1+2)} = 1 \\
  t_1^2 - 8t_1 - 10 &= 0 \\
\end{align*}

Using the quadratic formula:
\begin{align*}
  t_1 &= \frac{8 \pm \sqrt{64+40}}{2} \\
  &= 4 \pm \sqrt{26}
\end{align*}

Since $4 + \sqrt{26}$ is the only positive choice, this is the
time for one person.  The other person takes 2 hours longer, or $6 + \sqrt{26}$.

check:
\begin{align*}
  5\left( \frac{1}{4+\sqrt{26}} + \frac{1}{6 + \sqrt{26}} \right) 
  &= 5 \left( \frac{6 + \sqrt{26} + 4 + \sqrt{26}}{24 + 10 \sqrt{26} + 26} \right) \\
  &= 5 \left( \frac{10 + 2 \sqrt{26}}{50 + 10 \sqrt{26}} \right) \\
  &= 5 \left( \frac{2 \cancel{(5 + 2 \sqrt{26})} }{10 \cancel{(5 + 2 \sqrt{26})}} \right) \\
  &= 5 \left( \frac{2}{10} \right) \\
  &= 1 \\
\end{align*}

\end{solution}

\pagebreak
\noaddpoints

\section{Extra Credit}
\question[10] 
Show how to use the ``complete the square'' technique to derive the quadratic equation.  In other words, show how to get
from 
\[ ax^2+bx+c=0 \] 
to 
\[ x = \dfrac{-b \pm \sqrt{b^2-4ac}}{2a} \]

\begin{solution}[6 cm]
\begin{align*}
  ax^2+bx+c &= 0 \\
  ax^2+bx &= -c \\
  x^2+\frac{bx}{a} &= -\frac{c}{a} \\
  x^2+\frac{bx}{a} + \frac{b^2}{4a^2} &=  \frac{b^2}{4a^2} -\frac{c}{a} \\
  \left( x + \frac{b}{2a} \right)^2 &= \frac{b^2}{4a^2} - \frac{4ac}{4a^2}  \\
  \left( x + \frac{b}{2a} \right)^2 &= \frac{b^2 - 4ac}{4a^2} \\
  x + \frac{b}{2a} &= \pm \sqrt{\frac{b^2 - 4ac}{4a^2}} \\
  x + \frac{b}{2a} &= \frac{\pm \sqrt{b^2 - 4ac}}{2a} \\
  x &= -\frac{b}{2a} \pm \frac{\sqrt{b^2 - 4ac}}{2a} \\
  x &= \frac{-b \pm \sqrt{b^2 - 4ac}}{2a} \\
\end{align*}
\end{solution}

\end{questions}
\end{document}


