\documentclass{article}
\usepackage{mdwlist}
\usepackage{fullpage}
\everymath{\displaystyle}

\title{Math 113 Syllabus}
\date{\today}

\begin{document}

  \maketitle

  \section{Introduction}
  This course will cover chapters 2-7 of {\em Intermediate Algebra}, by Kaufmann and Schwitters.  We will briefly review
  chapter 1, but you should already be familiar with most of the material in that chapter.

  Algebra is about taking numbers and facts about them and using them to find other numbers which you are interested
  in.  For example, you can use algebra to solve these problems:

  \begin{itemize}
    \item If Alan Iverson made 10 baskets and 8 free throws and scored 31 points, how many of his baskets were three pointers?
    \item If the Citibank executives' bonuses were 1\% of the Citibank federal bailout, there are 300,000,000 people in the
      United States, and everyone in the country contributed \$2 to a Citibank executive's bonus, what was the total federal
      bailout for Citibank?
    \item If Carlos Martinez hits a baseball at a 45 degree angle which travels 300 feet before bouncing off of Jose
      Canseco's head and over the fence, how fast was it traveling when it left the bat?
  \end{itemize}

  \section{Homework and Tests}

  You should expect to spend one or two hours each week doing homework.

  Math is like learning piano, basketball, or bicycle mechanics.  Watching someone else do it or reading about it in a
  book is helpful.  But you can't actually learn how to do it yourself unless you've practiced on your own.

  We won't spend any time doing homework in class.  But we will discuss some of the homework problems from the
  previous week, and answer any questions which came up.

  Each chapter will be followed by an in-class test to make sure that everyone is keeping up.

  \section{Attendance}

  I'll try to make the classes interesting and useful, so that that attending class helps you to understand the
  material and do the homework.

  The DOC and UBB frown on students who sign up for class but don't attend regularly.  If you find you are unable to
  attend, let me know know and the TAs and I will try to help you to catch up on whatever you missed.

  \section{Course Overview}

  Here's what we will cover in this course.  Don't worry if some of the terms are unfamiliar or the equations look
  complicated.  Mathematicians like to use imposing terms for simple concepts, as it helps their job security.  

  \subsection{Chapter One--Review}
  Chapter one of the textbook includes material that you should already be mostly familiar with.  

  Subjects include:
  \begin{itemize*}
    \item{sets}
    \item{integers, rational numbers, irrational numbers, and real numbers}
    \item{evaluating numerical expressions}
    \item{absolute value}
  \end{itemize*}

  \subsection{Chapter Two--First Degree Equations and Inequalities}

  First degree equations are equations where the variable has an exponent of one.  We'll talk about how to solve equations
  and inequalities like

  \begin{itemize}
    \item \(x + 12 = 17 \)
    \item \( \frac{x}{2} + \frac{x}{3} = 12 \)
    \item \(|2x - 1| < 5\)
  \end{itemize}

  \subsection{Chapter Three--Polynomials and Factoring}

  A polynomial is an equation which includes more than one variable or one variable with different exponents.  Examples of
  polynomials are:
  \begin{itemize}
    \item \(x^2 + 2x -5 = 0\)
    \item \(x^2 - y^2 = 10 \)
  \end{itemize}

  We'll talk about adding, subtracting, and multiplying polynomials.

  We'll also talk about factoring polynomials.  When you factor a polynomial, you take a large and frightening
  polynomial and turn it into several simpler polynomials.  The solutions for the simpler polynomials are easy to find
  using the techniques from {\em Chapter 2} and the solutions to the simpler polynomials are also the solutions to the
  original polynomial.

  \subsection{Chapter Four--Rational Expressions}

  Rational expressions are equations which include a ratio of two numbers.  Examples of rational expressions are:

  \begin{itemize}
    \item \( \frac{x - 2}{x + 3} \)
    \item \( \frac{x - 2}{3} + \frac{x + 1}{4} = \frac{1}{6} \)
  \end{itemize}

  The topics in this chapter include:

  \begin{itemize*}
    \item Simplifying rational expressions
    \item Multiplying, dividing, adding, and subtracting rational expressions.
    \item Rational expressions with polynomials
    \item Fractional equations
  \end{itemize*}

  \subsection{Chapter Five--Exponents and Radicals}

  Exponents and radicals are two different notations for the same idea.  This chapter discusses both notations.  We'll
  discuss how to work with and simplify expressions like:

  \begin{itemize}
    \item \( x^3y^2 \)
    \item \(  \left( x^{1/3} y^{1/2} \right)^2 \)
  \end{itemize}

  \subsection{Chapter Six--More Quadratic Equations and Inequalities}

  Quadratic equations are equations that include \( x^2 \).  A quadratic equation always has two solutions, but one or
  both of the solutions may be a complex number, so we'll talk about complex numbers.  Then we'll talk about more
  techniques for solving quadratic equations including ``completing the square'' and the quadratic formula.

  \subsection{Chapter Seven--Linear Equations and Inequalities in Two Variables}

  In this chapter we'll go back to talking about linear equations, but we'll talk about equations with two variables.  The
  topics include:

  \begin{itemize*}
    \item linear equations and inequalities in two variables
    \item graphing linear equations
    \item distance and slope
    \item determining the equation of a line
  \end{itemize*}

\end{document}

