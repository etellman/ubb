\documentclass{article}

\oddsidemargin 0in
\topmargin -0.5in
\textwidth 6.5in

\title{Math 113 Notes}
\date{August 29, 2010}

\begin{document}

\maketitle

\section{Overview}

This document describes some things that worked well and some things that worked less well in Math 113, 2010 edition.  I
hope it helps you avoid some of the mistakes I made and learn from my experience.

This was my first time teaching a math class and my first time teaching at the prison, so I am far from an expert.  Your
actual mileage may vary.

\section{Placement Exam}
On the first day of class I gave a placement exam.  It covered the things that you were expected to know before the
course started.  

I let the students decide whether they wanted to continue.  I didn't kick anyone out if they didn't do well on the
test.  Some of the students who scored around 20 percent stuck with the course while others decided to take math 103
instead.  Everyone that stuck around did OK and worked hard to get extra help and catch up.  

Overall, I thought the placement exam worked out pretty well because it let everyone know where they stood at the start
of the class.  It also sets expectations so they know, for example, not to expect to learn how to add fractions in this
class.

\section{Pacing}
I think I started out going a little too slowly and spent too much time in chapters one and two.  The pace I eventually
ended up with was covering 2-3 sections each week.  This gives about one chapter per month, which seemed about right for
most people.

\section{Daily Course Structure}
I found it was best to cover new material at the start of the day, deferring questions until later.

After the first few weeks, I started writing an outline for the day on the board at the start of the day.  Then when
something was done, I checked it off so everyone could see where we were on the agenda.

The schedule was generally something like this:
\begin{enumerate}
  \item talk about new material
  \item five minute break
  \item homework questions (or any other questions)
  \item practice problems
\end{enumerate}

Towards the end of the semester, I found that I could often get student volunteers to go to the board and answer the
homework questions.

Many students have to leave around 8:00.  This is the second to last time to move around during the day and the only
opportunity for doing some things like phoning home.  So I tried to always cover everything new before 8:00.  After 8:00
was a good time for answering questions, working on practice problems, etc.
 
I also tried answering homework questions at the
start of class.  But this didn't work out well because homework questions can go on for a while and may only be
interesting to a subset of the students.

I had a few students who were big question askers.  I had a little bit of a hard time getting a balance between getting
their questions answered and not boring everyone else.  If I thought a question was only going to be helpful to a few
students, I would sometimes tell the student that we would talk about it privately at the end of the course.

\section{Extra Credit}

I included an extra credit problem on every homework and exam.  The homework extra credit problems didn't 
necessarily require algebra.  I often used some problem from one of Martin Gardner's (or similar) puzzle books.

Depending on the problem, about a third to a half of the class would attempt it.  I thought this worked out OK, because
it gave the students something a little outside of the normal coursework to think about, if they had the time and inclination

For the exams, I tried to always make the extra credit problems related to the material covered in that exam.  But the
extra credit problems were something you hadn't seen in the homework and required a little extra insight to figure out.

\section{Practice Problems}
About half way through the course, my TA, Rico suggested working practice problems in class at the end of each class.  The
practice problems were a few representatives of what you would see on the homework.  We tried it out and found it seemed
to be very helpful.

Not everyone sticks around for the practice problems, but the students that did seem to find it beneficial.  Often you
can watch someone else do problems on the board and think you understand.  But when you try to do the same thing
yourself, you find that you get stuck.  Working a few problems in class with assistance sets you on the right track for
doing the homework.  

There was always a lot of cooperation in the practice problems.  Some students worked ahead a bit and generally had a
good grasp of the material.  They would usually also stick around for the practice problems to help the other students.

\section{Homework} 
I gave homework every week, which seemed to work out well.  The first few homeworks were probably lighter than the pace
I eventually ended up with.  I started out by selecting easy-medium problems from the book but ended up using the
medium-hard ones, which seemed to work out better and probably prepares people better for the final.

After the first few weeks, I handed out typed solutions to everything.  The first few weeks I tried xeroxed hand-written
solutions.  But my hand writing is not the best and xeroxing everything was a hassle.  Typing everything is also a bit
time consuming, but future instructors can re-use my solutions.

\section{Final Exam}
I let everyone who wanted to sign up for the OU final.  This turned
out to be almost everyone in left in the class at the end of the semester.

I didn't do a good job getting the exam forms filled out on time.  With one thing and another, they didn't end up getting
filled out until nearly the last week of class.  This leaves a big gap between when the course is over and when the
forms arrive, which is a bummer since the students have to try to keep the material fresh, but there's nothing new to
talk about in class.  Anyway, I would recommend getting the forms filled out and sent to OU a month before class ends.

Another of Rico's ideas that we did and seemed to work well was to include some cumulative review problems on each of the last
month's homeworks.  This helped to get everyone focused on the course as a whole, so they would be in the right frame of
mind for the final.

\end{document}


