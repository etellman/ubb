\documentclass[fleqn, onecolumn]{article}

\usepackage{fullpage}
\usepackage{graphicx}
\usepackage{float}
\usepackage{amsmath}
\usepackage{amssymb}
\usepackage{polynom}
\usepackage{caption}
\usepackage{mdwlist}

\newcommand{\degree}{\ensuremath{^\circ}} 

\everymath{\displaystyle}
\setlength{\mathindent}{1 cm}

\title{Math 263a Syllabus}
\date{January 11, 2012}

\begin{document}

\maketitle

\section{Introduction}
This course will cover chapters 2-4 of {\em Calculus}, by Varberg and Purcell.

\section{Homework and Exams}

You should expect to spend two or three hours each week doing homework.

Math is like learning piano, basketball, or bicycle mechanics.  Watching someone else do it or reading about it in a
book is helpful.  But you can't actually learn how to do it yourself unless you've practiced on your own.

Each chapter will be followed by an in-class test.  The primary reason to have in-class tests is so that you can
practice for the OU exam which is the way you receive credit at the end of the course.  Taking tests is like any other
skill--you get better at it with practice.

\section{Course Overview}

Here's what we will cover in this course. 

\subsection{Chapter Two--Functions and Limits}
If you took Math 115, some of this material will be review.

Subjects include:
\begin{itemize*}
\item functions
\item trigonometric functions
\item limits
\item continuity
\end{itemize*}

\subsection{Chapter Three--The Derivative}

Derivatives are one of the two big subjects in calculus and this chapter covers them in great detail.  Derivatives let
you find an expression for how fast something is changing if you already have an expression for how much of something
you have.  

For instance, if you know that your position is changing according to:
\[
  d(t) = \frac{1}{2} at^2
\]

You can use your knowledge of derivatives to find that your speed is:
\[
  s(t) = at
\]


In this chapter, you learn how to find derivatives of many kinds of functions.

\subsection{Chapter Four--Applications of the Derivative}

Derivatives are useful in many different applications.  This chapter describes some of them, including
\begin{itemize*}
  \item finding minimum and maximum values of functions
  \item economics
  \item graphing
\end{itemize*}

\end{document}

