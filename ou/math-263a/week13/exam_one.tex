
\documentclass[fleqn,addpoints]{exam}
\usepackage{amsmath}
\usepackage{graphicx}
\usepackage{float}
\usepackage{caption}
\usepackage{polynom}
\usepackage{mdwlist}
\usepackage{cancel}

\usepackage{unitsdef} 
\newunit{\inch}{in}
\newunit{\mile}{mile}
\newunit{\mph}{mph}
\newunit{\foot}{ft}
\newunit{\knot}{knot}
\newunit{\gallon}{gallon}

\printanswers
\bracketedpoints
\everymath{\displaystyle}

%% \ifprintanswers
%% \usepackage{2in1, lscape}
%% \fi

\title{Math 263A Exam One \\ Chapters 2 and 3}
\date{April 18, 2012}

\author{}

\begin{document}

\maketitle  

\ifprintanswers
\else
\vspace{0.2in}
\makebox[\textwidth]{Name:\enspace\hrulefill}
\vspace{0.2in}

\begin{center}
\gradetable[h][pages]
\bonusgradetable[h][pages]
\end{center}

% \vspace{1 cm}
%% \section{Instructions}

%% Calculators aren't allowed on the final, so they also aren't allowed on any other tests.  If the answer involves a
%% logarithm or exponential, just leave the answer as $\dfrac{\ln 7}{2}$ (or whatever it is).

\fi

\begin{questions}

\uplevel{\section{Limits and Continuity}}
 
\uplevel{For questions \ref{limit:first} and \ref{limit:last}, evaluate the limits.}

\vspace{1 cm}

%% \question[5]
%% \[
%%   \lim_{x \to 1} \frac{x^2 - 1}{x - 1} 
%% \]
%% \begin{solution}[2 cm]

%% \begin{align*}
%%   \lim_{x \to 1} \frac{x^2 - 1}{x - 1} &= \lim_{x \to 1} \frac{(x + 1) \cancel{(x - 1)}}{\cancel{(x - 1)}} \\
%%   &= \lim_{x \to 1} (x + 1) \\
%%   &= 2 \\
%% \end{align*}

%% \end{solution}

\question[5]
\label{limit:first} 
\[
  \lim_{x \to -2} \frac{x^2 + x - 2}{x^2 + 5x + 6}
\]

\begin{solution}[1 cm]

\begin{align*}
  \lim_{x \to -2} \frac{x^2 + x  - 2}{x^2 + 5x + 6} &= \lim_{x \to -2} \frac{(x - 1) \cancel{(x + 2)}}{(x + 3) \cancel{(x + 2)}} \\
  &= \lim_{x \to -2} \frac{x - 1}{x + 3} \\
  &= -3 \\
\end{align*}

\end{solution}


\pagebreak

\question[7]
\label{limit:last}
\[
  \lim_{x \to 0} \left( \frac{1}{x} \right) \left( \frac{1}{\sqrt{x + 1}} - 1 \right)
\]
\begin{solution}[5 cm]
The problem in this equation is the $x$ in the denominator.  If you rationalize the numerator, you are left with an $x$
by itself in the numerator which cancels the $x$ in the denominator.

\begin{align*}
  \lim_{x \to 0} \left( \frac{1}{x} \right) \left( \frac{1}{\sqrt{x + 1}} - 1 \right) 
      &= \lim_{x \to 0} \left( \frac{1}{x} \right) \left( \frac{1 - \sqrt{x + 1}}{\sqrt{x + 1}} \right) \\
      &= \lim_{x \to 0} \left( \frac{1 - \sqrt{x + 1}}{x \sqrt{x + 1}} \right) \left( \frac{1 + \sqrt{x + 1}}{1 + \sqrt{x + 1}} \right) \\
      &= \lim_{x \to 0} \frac{1 - (x + 1)}{x (\sqrt{x + 1})(1 + \sqrt{x+1})} \\
      &= \lim_{x \to 0} \frac{- \cancel{x}}{\cancel{x} (\sqrt{x + 1})(1 + \sqrt{x+1})} \\
      &= \lim_{x \to 0} \frac{-1}{(\sqrt{x + 1})(1 + \sqrt{x+1})} \\
      &= - \frac{1}{2} \\
\end{align*}
\end{solution}

\question[7]
How should you define $f(16)$ to make $f$ continuous at 16?
\[
  f(x) = \frac{x - 16}{\sqrt{x} - 4}
\]
\begin{solution}[4 cm]
For a function to be continuous at 16:
\[
  f(16) = \lim_{x \to 16} f(x)
\]

\begin{align*}
  \lim_{x \to 16} \frac{x - 16}{\sqrt{x} - 4} 
    &= \lim_{x \to 16} \frac{\cancel{(\sqrt{x} - 4)}(\sqrt{x} + 4)}{\cancel{\sqrt{x} - 4}} \\
    &= \lim_{x \to 16} \sqrt{x} + 4 \\
    &= 8 \\
\end{align*}

So if you define $f(16) = 8$, the function will be continuous at 16.

\end{solution}

%% \uplevel{For problems \ref{continuity_range:first} and \ref{continuity_range:last}, find the range of numbers where
%%   $f(x)$ is continuous.}

%% \question[5]
%% \label{continuity_range:first}
%% \[
%%   f(x) = \frac{3x - 5}{2x^2 - x -3}
%% \]
%% \begin{solution}[4 cm]
%% \end{solution}

%% \question[5]
%% Over what range of values is $f$ continuous?
%% \[
%%   f(x) = \frac{\sqrt{9-x}}{\sqrt{x - 6}}
%% \]
%% \begin{solution}[4 cm]
%% \end{solution}

\uplevel{\section{Derivatives}}

\uplevel{For problems \ref{derivative:first} to \ref{derivative:last}, find $f'(x)$.}

\question[3]
\label{derivative:first}
\[
  f(x) = 2x^3 - 3x^2 + 7x - \pi^3
\]
\begin{solution}[2 cm]
\[
  f'(x) = 6x^2 - 6x + 7
\]

$\pi^3$ is a constant so its derivative is 0.

\end{solution}

\ifprintanswers
\else
\pagebreak
\fi

\question[5]
\[
  f(x) = \frac{1}{2x^4 - 7x^2}
\]
\begin{solution}[4 cm]
\[
  f'(x) = \frac{-(8x^3 - 14x)}{(2x^4 - 7x^2)^2} = \frac{14x - 8x^3}{(2x^4 - 7x^2)^2} \\
\]

\end{solution}

\question[7]
\[
  f(x) = \sqrt{x} \cos x
\]
\begin{solution}[5 cm]
\begin{align*}
  f(x) &= x^{1/2} \cdot \cos x
  \\
  f'(x) &= x^{1/2} \cdot (-\sin x) + \cos x \cdot \frac{1}{2} x^{-1/2} \\
        &= \frac{\cos x}{2 \sqrt{x}} - \sqrt{x} \cdot \sin x \\
\end{align*}

\end{solution}

\ifprintanswers
\pagebreak
\fi

\question[10]
\[
  f(x) = \frac{2x^2 - 5x + 7}{x^3 + 4}
\]
\begin{solution}[7 cm]
\begin{align*}
  f'(x) &= \frac{(x^3 + 4) \cdot D_x(2x^2 - 5x + 7) - (2x^2 - 5x + 7) \cdot D_x(x^3 + 4)}{(x^3 + 4)^2} \\
        &= \frac{(x^3 + 4)(4x - 5) - (2x^2 - 5x + 7)(3x^2)}{(x^3 + 4)^2} \\
        &= \frac{4x^4 - 5x^3 + 16x - 20 - (6x^4 - 15x^3 + 21x^2)}{(x^3 + 4)^2} \\
        &= \frac{-2x^4 + 10x^3 - 21x^2 + 16x - 20}{(x^3 + 4)^2} \\
\end{align*}
\end{solution}

\ifprintanswers
\else
\pagebreak
\fi

\question[10]
\[
  f(x) = \left( \frac{2x+ 1}{3x^2 - 2} \right)^4
\]
\begin{solution}[8 cm]
\begin{align*}
  f'(x) &= 4 \left( \frac{2x+ 1}{3x^2 - 2} \right)^3 \cdot D_x \left( \frac{2x+ 1}{3x^2 - 2} \right) \\
        &= 4 \left( \frac{2x+ 1}{3x^2 - 2} \right)^3 
             \left( \frac{(3x^2 - 2) \cdot Dx(2x+ 1) - (2x + 1) \cdot D_x(3x^2 - 2)}{(3x^2 - 2)^2} \right) \\
        &= 4 \left( \frac{2x+ 1}{3x^2 - 2} \right)^3 \left( \frac{(3x^2 - 2)(2) - (2x + 1)(6x)}{(3x^2 - 2)^2} \right) \\
        &= 4 \left( \frac{2x+ 1}{3x^2 - 2} \right)^3 \left( \frac{6x^2 - 4 - (12x^2 + 6x)}{(3x^2 - 2)^2} \right) \\
        &= 4 \left( \frac{2x+ 1}{3x^2 - 2} \right)^3 \left( \frac{-6x^2 + 6x - 4}{(3x^2 - 2)^2} \right) \\
        &= \frac{4(2x+ 1)^3(-6x^2 + 6x - 4)}{(3x^2 - 2)^5}  \\
\end{align*}
\end{solution}

\ifprintanswers
\pagebreak
\fi

\question[10]
\[
  f(x) = \sin^2 \left( \pi x^3 \right)
\]
\begin{solution}[6 cm]
\begin{align*}
  f'(x) &= 2(\sin \left( \pi x^3 \right)) \cdot D_x \sin \left( \pi x^3 \right) \\
        &= 2 \left( \sin \left( \pi x^3 \right) \right) \cdot \cos \left( \pi x^3 \right) \cdot D_x \left( \pi x^3 \right) \\
        &= 2(\sin \left( \pi x^3 \right)) \cdot \cos \left( \pi x^3 \right) \cdot \left( 3 \pi x^2 \right) \\
        &= 6 \pi x^2 \sin \left( \pi x^3 \right) \cos \left( \pi x^3 \right) \\
\end{align*}
\end{solution}

\ifprintanswers
\else
\pagebreak
\fi

\question[10]
\label{derivative:last}
\[
  f(x) = \frac{\sin x^2}{\cos^2 x}
\]
\begin{solution}[8 cm]
\begin{align*}
  f'(x) &= \frac{\cos^2 x \cdot D_x \left( \sin x^2 \right) - \sin x^2 \cdot D_x \left( \cos^2 x \right)}{ \cos^4 x} \\
        &= \frac{\cos^2 x \cdot \cos x^2 \cdot D_x(x^2) - \sin x^2 ( 2 \cos x) \cdot D_x(\cos x)}{ \cos^4 x} \\
        &= \frac{2 x \cos^2 x \cos x^2 - 2 \sin x^2 \cos x(-\sin x)}{ \cos^4 x} \\
        &= \frac{2 x \cos^2 x \cos x^2 + 2 \sin x^2 \cos x \sin x}{ \cos^4 x} \\
        &= \frac{2 x \cos x \cos x^2 + 2 \sin x \sin x^2 }{ \cos^3 x} \\
\end{align*}

\end{solution}

\ifprintanswers
\pagebreak
\fi


\question $y = x^4 + 3x^3 - 7x + 12$

\begin{parts}

\part[3]
Find $\frac{dy}{dx}$
\begin{solution}[3 cm]
\[
  \frac{dy}{dx} = 4x^3 + 9x^2 - 7
\]
\end{solution}

\part[3]
Find $\frac{d^2y}{dx^2}$
\begin{solution}[3 cm]
\[
  \frac{d^2y}{dx^2} = 12x^2 + 18x
\]
\end{solution}

\end{parts}

\ifprintanswers
\else
\pagebreak
\fi

\question[10] Use implicit differentiation to find $\frac{dy}{dx}$.
\[
  xy^2 + 3x^2y = 7
\]
\begin{solution}[5 cm]
\begin{align*}
  xy^2 + 3x^2y &= 7 \\
  D_x(xy^2 + 3x^2y) &= D_x(7) \\
  x \cdot D_x(y^2) + y^2 + 3x^2 \frac{dy}{dx} + y \cdot 6x &= 0 \\
  % x \left( 2 y \frac{dy}{dx} \right) + y^2 + 3x^2 \frac{dy}{dx} + 6xy &= 0 \\
  2xy \frac{dy}{dx} + y^2 + 3x^2 \frac{dy}{dx} + 6xy &= 0 \\
  2xy \frac{dy}{dx} + 3x^2 \frac{dy}{dx} &= -6xy - y^2 \\
  \frac{dy}{dx} &= \frac{-6xy - y^2}{2xy + 3x^2} \\
\end{align*}
\end{solution}

\pagebreak

\uplevel{\section{Applications}}

\question 
Newton is in the park playing fetch with his dogs Eff and Eff Prime.  Each time he throws the ball, he watches its
graceful flight as the two dogs bound enthusiastically in pursuit.  As the ball is sailing through the air, he
notices that its height seems to be approximately given by the expression:
\[
  h = 64t - 16t^2 
\]
where the time is in seconds and the distance in feet.

\vspace{1 cm}

\begin{parts}
\part[2]
How fast is the ball going after 1 second?
\begin{solution}[4 cm]
The velocity is the derivative of the height:
\[
  h'(t) = 64 - 32t
\]
After 1 second:
\[
  h'(1) = 64 - 32 = 32 \foot / \second
\]
\end{solution}

\part[2]
What is the ball's acceleration?
\begin{solution}[3 cm]
\[
  h''(t) = -32 \foot / \second^2
\]
The acceleration is negative because the ball is slowing down because of gravity.

\end{solution}

\part[2]
When does the ball reach the peak of its flight?
\begin{solution}[4 cm]
At the peak of the flight, the velocity is zero.
\begin{align*}
  64 - 32t &= 0 \\
  32t &= 64 \\
  t &= 2 \second \\
\end{align*}
\end{solution}

\part[2]
What is the maximum height reached by the ball?
\begin{solution}[2 cm]
\[
  h(2) = 64 \cdot 2 - 16 \cdot 4 = 64 \foot
\]
\end{solution}

\end{parts}

\pagebreak

\question[10] Newton and Leibniz are having a heated discussion about calculus.  They agree on mundane facts like
$\sqrt{289} = 17$ and $257/17 \approx 15.12$. But they don't see eye to eye on the best notation to use for derivatives.

Eventually, Newton stomps off in disgust, heading east at 4 mph muttering angrily about stubborn Germans.  Leibniz sits
pensively for an hour and then hops on his bike and starts riding north at a meditative 15 mph.

At what rate is the distance between them changing 1 hour after Leibniz starts biking?

\begin{solution}[11 cm]
Before you start, you can do a quick common sense check to see what range the value should be in.  If the two mathemeticians
traveled in opposite directions, the rate would be $15 + 4 = 19 \mph$.  If the two mathemeticians
traveled in the same direction, the rate would be $15 - 4 = 11 \mph$.  So the answer must be somewhere between
11 and 19 mile/hour.

The distance between the two mathemeticians is the hypotenuse of a right triangle, so if $x$ is Newton's position, $y$
is Leibniz's position and $r$ the distance between them:
\[
  r^2 = x^2 + y^2 \\
\]

The rate of change of this distance over time is:
\begin{align*}
  2r \cdot \frac{dr}{dt} &= 2x \cdot \frac{dx}{dt} + 2y \cdot \frac{dy}{dt} \\
  \frac{dr}{dt} &= \frac{x}{r} \cdot \frac{dx}{dt} + \frac{y}{r} \cdot \frac{dy}{dt} \\
\end{align*}

This is the same formula you get for any problem like this where two objects are traveling perpendicular to each other
and you want the rate the distance between them is changing.  The only things that differ from problem to problem are
the rates and distances you plug into this formula.

At the time we're interested in:
\begin{itemize*}
\item Newton has been walking for 2 hours at 4 mph and is 8 miles from the origin.
\item Leibniz has been biking for 1 hour at 15 mph and is 15 miles from the origin.
\item The distance between them is: $\sqrt{8^2 + 15^2} = \sqrt{289} = 17 \mile$.
\end{itemize*}
%% Let the time be zero when Leibniz starts biking.  At this time, Newton has been walking for an hour so, he's 4 miles
%% away from the starting point.  If he walks along the x-axis, his position is given by:
%% \[
%%   x = 4 + 4t
%% \]
%% Leibniz starts biking up the y-axis at time 0, so his position is:
%% \[
%%   y = 15 t
%% \]

%% The values 1 hour after Leibniz starts biking are:
%% \begin{align*}
%%   x &= 8 \mile \\
%%   y &= 15 \mile \\
%%   r &= \sqrt{8^2 + 15^2} = \sqrt{289} = 17
%% \end{align*}

So the desired rate is:
\begin{align*}
  \frac{dr}{dt} &= \frac{8}{17} \cdot 4 + \frac{15}{17} \cdot 15 \\
                &= \frac{257}{17} \\
                &\approx 15.12 \\
\end{align*}

\end{solution}

\ifprintanswers
\pagebreak
\fi

\question[7]
A spherical balloon is being inflated with gas.  Use differentials to estimate how much the volume changes when the
radius grows from 3 ft to 3.02 ft.

You can express your answer as some number times $\pi$ ($1.234 \pi$, for example).

The volume of a sphere is: $V = \frac{4}{3} \pi r^3$

\begin{solution}[3 cm]
\begin{align*}
  V  &= \frac{4}{3} \pi r^3 \\ 
  dV &= 4 \pi r^2 dr \\
  \\
  dV &= 4 \pi \cdot 3^2 \cdot 0.02 \\
     &= 0.72 \pi \foot^3 \\
\end{align*}
\end{solution}

\ifprintanswers
\else
\pagebreak
\fi

\uplevel{\section{Extra Credit}}
\bonusquestion[10] 
A light is at the top of a 15 foot pole.  A 5 foot tall boy walks away from the pole at 4 ft/s.  At what rate is the tip
of his shadow moving?

\begin{solution}[9 cm] 
I drew a picture of the situation and called $x$ the distance of the boy from the pole and $L$ the length of the
shadow.  With this labeling, there are two similar triangles and:
\begin{align*}
  \frac{5}{L} &= \frac{15}{x + L} \\
  L &= \frac{x}{2} \\
\end{align*}
We know the rate $x$ is changing, since it is the boy's walking speed.  So we can differentiate to find the rate the
length of the shadow is changing:
\[
  \frac{dL}{dt} = \frac{1}{2} \frac{dx}{dt} = \frac{1}{2} \cdot 4 = 2 \foot / \second
\]

The shadow is both growing and moving because the boy is walking.  So the rate the tip moves is:
\[
  4 \mph + 2 \mph = 6 \foot / \second
\]
\end{solution}

\ifprintanswers
\pagebreak
\fi

\bonusquestion[10] 
Use only the product rule (not the power rule or the chain rule) to show that:
\[
  D_x f^3(x) = 3 f^2(x) \cdot f'(x)
\]

\begin{solution}[8 cm] 
You have to apply the product rule twice and do a little algebra to get the desired result.
\begin{align*}
  D_x(f^3(x)) &= D_x(f^2(x)f(x)) \\
  &= f^2(x) f'(x) + f(x) \cdot D_x(f^2(x)) \\
  &= f^2(x) f'(x) + f(x) \cdot D_x(f(x) \cdot f(x)) \\
  &= f^2(x) f'(x) + f(x) \cdot ( f(x) f'(x) + f(x) f'(x) ) \\
  &= f^2(x) f'(x) + f(x) \cdot \left( 2f(x) f'(x) \right) \\
  &= f^2(x) f'(x) + 2f^2(x) f'(x) \\
  &= 3 f^2(x) f'(x) \\
\end{align*}
\end{solution}

%% \bonusquestion[7] 
%% A cable is submerged in the ocean.  Because of corrosion, the surface area of the cable decreases at a rate of
%% $\frac{dA}{dt}$.  The corrosion of the ends is negligible, so the length remains constant.

%% Find the rate at which the radius of the cable is decreasing in terms of the length ($L$) and radius
%% ($r$) of the cable.

%% \begin{solution}[5 cm] 
%% \begin{align*}
%%   A &= \pi r^2 L \\
%%   \frac{dA}{dt} &= 2 \pi r L \frac{dr}{dt} \\
%%   \frac{dr}{dt} &= \frac{1}{2 \pi r L} \frac{dA}{dt}
%% \end{align*}

%% \end{solution}

%% \bonusquestion[10] 
%% When two resistors ($R_1$ and $R_2$) are connected in parallel, the total resistance of the circuit ($R$) is given by
%% \[
%%   \frac{1}{R} = \frac{1}{R_1} + \frac{1}{R_2}
%% \]

%% Find an expression for the rate the total resistance is changing in terms of the rates the other two resistances are
%% changing and the values of the other two resistances.

%% \begin{solution}[5 cm] 
%% \end{solution}

\end{questions}

\end{document}
