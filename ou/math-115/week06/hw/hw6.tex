
\documentclass[fleqn,addpoints]{exam}

\usepackage{graphicx}
\usepackage{xfrac}
\usepackage{float}
\usepackage{amsmath}
\usepackage{cancel}
\usepackage{polynom}
\usepackage{caption}

\printanswers

\ifprintanswers \usepackage{2in1, lscape} \fi

\title{Math 115 Homework 6}
\date{November 9, 2010}

\begin{document}

\maketitle
 
\section{From the Book}

\vspace{0.2 cm} 

Read section 1.10 and 1.11.
 
\begin{itemize}
  \item p. 81: 2-7, 11
  \item p. 89: 7-15, 17, 20, 23
\end{itemize}

\section{Page 81}
\begin{itemize}
\item[2]
$f(x) = x^2$; $g(x) = x+1$
\begin{align*}
  (f+g)(x) &= x^2 + x + 1 \\
  (f-g)(x) &= x^2 - x - 1 \\
  (f \cdot g)(x) &= x^2(x+1) = x^3 + x^2 \\
  (f/g)(x) &= \dfrac{x^2}{x+1} \\
\end{align*}

The domain for $f/g$ is $x \neq -1$.  The domain for the remaining functions is $(-\infty, \infty)$.

\item[3]
$f(x) = \dfrac{1}{x}$; $g(x) = \sqrt{x-1}$

\begin{align*}
  (f+g)(x) &= \dfrac{1}{x} + \sqrt{x-1} \\
  (f-g)(x) &= \dfrac{1}{x} - \sqrt{x-1} \\
  (f \cdot g)(x) &= \dfrac{\sqrt{x-1}}{x} \\
  (f/g)(x) &= \dfrac{1}{x\sqrt{x-1}} \\
\end{align*}

The domain for $f/g$ is $(1, \infty)$.  The domain for the remaining functions is $[1, \infty)$.

\item[4]
$f(x) = \dfrac{1}{x-1}$; $g(x) = \dfrac{1}{x+1}$
\begin{align*}
  (f+g)(x) &= \dfrac{1}{x-1} + \dfrac{1}{x+1} = \dfrac{x+1+x-1}{(x+1)(x-1)} = \dfrac{2x}{x^2-1} \\
  (f-g)(x) &= \dfrac{1}{x-1} - \dfrac{1}{x+1} = \dfrac{x+1 - x + 1}{(x+1)(x-1)} = \dfrac{2}{x^2-1} \\
  (f \cdot g)(x) &= \dfrac{1}{x-1} \cdot \dfrac{1}{x+1} = \dfrac{1}{x^2-1} \\
  (f/g)(x) &= \dfrac{1}{x-1} \cdot \dfrac{x+1}{1} = \dfrac{x+1}{x-1} \\
\end{align*}

The domain for all the functions is $x \neq \pm 1$.

\item[5]
$f(x) = \sqrt{x+2}$; $g(x) = \sqrt{2-x}$
\begin{align*}
  (f+g)(x) &= \sqrt{x+2} + \sqrt{2-x} \\
  (f-g)(x) &= \sqrt{x+2} - \sqrt{2-x} \\
  (f \cdot g)(x) &= \sqrt{x+2} \cdot \sqrt{2-x} = \sqrt{4-x^2} \\
  (f/g)(x) &= \dfrac{\sqrt{x+2}}{\sqrt{2-x}} = \dfrac{\sqrt{x+2}}{\sqrt{2-x}} \left( \frac{\sqrt{2-x}}{\sqrt{2-x}} \right) \\
           &= \dfrac{\sqrt{x+2}}{\sqrt{2-x}} \left( \frac{\sqrt{2-x}}{\sqrt{2-x}} \right) = \frac{\sqrt{4-x^2}}{2-x}  \\
\end{align*}

The domain for $f/g$ is $[-2, 2)$.  The domain for the remaining functions is $[-2, 2]$.

\item[6]
$f(x) = \dfrac{1}{x}$; $g(x) = \dfrac{x}{x-2}$
\begin{align*}
  (f+g)(x) &= \dfrac{1}{x} + \dfrac{x}{x-2} = \dfrac{x^2 + x-2}{x(x-2)} \\
  (f-g)(x) &= \dfrac{1}{x} - \dfrac{x}{x-2} = \dfrac{-x^2 + x - 2}{x(x-2)} \\
  (f \cdot g)(x) &= \dfrac{1}{ \cancel{x}} \cdot \dfrac{ \cancel{x}}{x-2} = \dfrac{1}{x-2} \\
  (f/g)(x) &= \dfrac{\sfrac{1}{x}}{\sfrac{x}{(x-2)}} = \dfrac{x-2}{x^2} \\
\end{align*}

The domain for all functions is $\{x | x \neq 0 \text{ and } x \neq 2\}$
\item[7]
\[
  f(x) = 
  \begin{cases}
    -1      & \text{if } x < 0 \\
     1      & \text{if } x \geq 0
  \end{cases}
\]
\[
  g(x) = 
  \begin{cases}
    1      & \text{if } x < 0 \\
    0      & \text{if } x \geq 0
  \end{cases}
\]

\[
    (f+g)(x) = 
    \begin{cases}
      0      & \text{if } x < 0 \\
      1      & \text{if } x \geq 0
    \end{cases}
\]
\[
    (f-g)(x) = 
    \begin{cases}
      -2      & \text{if } x < 0 \\
       1      & \text{if } x \geq 0
    \end{cases}
\]
\[
    (fg)(x) = 
    \begin{cases}
      -1      & \text{if } x < 0 \\
       0      & \text{if } x < 0
    \end{cases}
\]
\[  (f/g)(x) = -1$, $x<0 \]

\item[11]
2 is in the domain of $g(x)$ and not in the domain of $f(x)$.

\end{itemize}

\section{Page 89}
\begin{itemize}

\item[7]
$f(x) = 2x+1$; $g(x) = 3x-1$
\begin{align*}
  (f \circ g)(x) &= 2(3x-1) + 1 = 6x-1 \\
  (g \circ f)(x) &= 3(2x+1)-1 = 6x+2 \\
  (f \circ f)(x) &= 2(2x+1) + 1 = 4x+3 \\
  (g \circ g)(x) &= 3(3x-1) - 1 = 9x-4 \\
\end{align*}

The domain for all functions is $(-\infty, \infty)$.

\item[8]
$f(x) = x^2+1$; $g(x) = x-1$
\begin{align*}
  (f \circ g)(x) &= (x-1)^2 + 1 = x^2-2x+2 \\
  (g \circ f)(x) &= x^2+1-1 = x^2 \\
  (f \circ f)(x) &= (x^2+1)^2+1 = x^4 + 2x^2+2 \\
  (g \circ g)(x) &= (x-1)-1 = x-2 \\
\end{align*}

The domain for all functions is $(-\infty, \infty)$.

\item[9]
$f(x) = \dfrac{1}{x}$; $g(x) = x^2+2x$
\begin{align*}
  (f \circ g)(x) &= \frac{1}{x^2+2x} = \frac{1}{x(x+2)}  \text{; $x \neq 0$ and $x \neq -2$} \\
  (g \circ f)(x) &= \frac{1}{x^2} + \frac{2}{x} \text{; $x \neq 0$} \\
  (f \circ f)(x) &= \frac{1}{\sfrac{1}{x}} = x \text{; $x \neq 0$}\\
  (g \circ g)(x) &= (x^2+2x)^2 + 2(x^2+2x) = x^4 + 4x^3 + 6x^2 + 4x\\
\end{align*}

\item[10]
$f(x) = \sqrt{x-1}$; $g(x) = x^2-3$
\begin{align*}
  (f \circ g)(x) &= \sqrt{x^2-3-1} = \sqrt{x^2-4} \text{; $x = (-\infty, -2] \cup [2, \infty)$} \\
  (g \circ f)(x) &= (\sqrt{x-1})^2 - 3 = x-4 \text{; $x \geq 1$} \\
  (f \circ f)(x) &= \sqrt{\sqrt{x-1} - 1} \text{; $x \geq 2$} \\
  (g \circ g)(x) &= (x^2-3)^2 - 3 = x^4-6x^2+6 \\
\end{align*}

\item[11]
\begin{align*}
  h(x) &= (2-3x^2)^4 \\
  f(x) &= x^4 \\
  g(x) &= 2-3x^2 \\
\end{align*}

\item[12]
\begin{align*}
  h(x) &= \sqrt{x-2} \\
  f(x) &= \sqrt{x} \\
  g(x) &= x-2 \\
\end{align*}

\item[13]
\begin{align*}
  h(x) &=  \frac{1}{x+2}\\
  f(x) &=  \frac{1}{x}\\
  g(x) &= x+2 \\
\end{align*}

\item[14]
\begin{align*}
  h(x) &=  |x^2+x+1|\\
  f(x) &=  |x|\\
  g(x) &= x^2+x+1\\
\end{align*}

\item[15]
\begin{parts}
  \part $(f \circ g)(1) = f(2) = 0$
  \part $(g \circ f)(-1) = g(-2) = -1$
  \part $(g \circ f)(0) = g(0) = 1$
  \part $(f \circ g)(-2) = f(-1) = -1$
\end{parts}

\item[17]

Whenever $f$ and $g$ are inverses of each other, $f \circ g = g \circ f = x$.  Here are a few functions that work.  The
back of the book has some different examples.
\begin{itemize}
  \item $f(x) = x+1$; $g(x) = x-1$
  \item $f(x) = x^3$; $g(x) = \sqrt[3]{x}$
\end{itemize}

\pagebreak

\item[20]
$(f \circ f) = 4x+3$

\begin{align*}
  4x+ 3 &= m(mx+b)+b \\
  4x+ 3 &= m^2x+mb+b \\
\end{align*}

From this we can see that the slope must be $m^2$ and the y-intercept must be $mb+b$:
\begin{align*}
  m^2 &= 4 \\
  m &= 2 \\
  \\
  mb+b &= 2 \\
  2b+b &= 3 \\
  b &= 1 \\
\end{align*}

So the function is: $f(x) = 2x+1$.

check: $f \circ f = f(2x+1) = 2(2x+1) + 1 = 4x+3$


\item[23]
From the front of the book, the function for the volume of a sphere is $V(r) = \dfrac{4}{3} \pi r^3$.  So the function
for the volume in terms of time is:
\[
  V(t) = \dfrac{4}{3} \pi (3 + 0.01t)^3
\]

\end{itemize}

\section{Other Problems}

The recent election led to some interesting math problems.  The fate of several of the ballot measures depended on
people not understanding the math behind them.  Here are a few problems which may help you decide how you should vote if
you have your own economic interests in mind.

Your tax rate is the amount of tax you pay divided by your income.

\begin{questions}

\question
Proposition 1098 proposed a 4\% income tax on income over \$200,000/year.  

For people with this income, there are two functions they need to think about:
\begin{align*}
  f(x) &= x - 200,000 \\
  g(x) &= 0.04x \\
\end{align*}

\begin{parts}
\part
Which of $f \circ g$ or $g \circ f$ is the correct function for calculating the tax?

\begin{solution}[1cm]

$f(x)$ is the function for determining how much of your income is taxable and $g(x)$ is the function for determining how
much tax you need to pay, given your taxable income.  so the correct function for calculating your tax given your income
is: \[(g \circ f)(x) = 0.04(x - 200,000) \]

\end{solution}

\part
How much tax would someone making \$300,000 pay?
\begin{solution}[1cm]
\[
  (g \circ f)(300,000) = 0.04(300,000 - 200,000) = 4,000
\]
\end{solution}

\part
What is the tax rate for someone making \$300,000?

\begin{solution}
The tax rate is the tax divided by the income, or: \[\frac{4,000}{300,000} = 0.013 \]

The tax rate for this income is about 1.3\%.
\end{solution}

\end{parts}

\pagebreak

\question
The current WA system uses only a sales tax of around 10\%.  I made up a few numbers about this to make an interesting
problem.  I'm not sure how accurate the numbers are, but the general idea is correct.

Everyone needs to buy some stuff to survive.  Everyone needs clothes, transportation, etc.  If you have a big income,
you tend to buy more stuff and more expensive stuff.  However, you probably don't spend all of your money on buying
taxable stuff in WA.  You spend most of it on your vacation home in Aspen, your European vacation, and your kid's
tuition at Harvard.  

If we suppose that everyone needs at least \$5,000 worth of stuff, a guess at the amount of money a person spends on
taxable stuff might be:

\[
  T(x) = 5000 + 0.1x
\]

$x$ is your income.  Everyone buys \$5,000 worth of stuff, and then everyone spends 10\% of their income on other
taxable stuff they want or need.  Everyone spends the rest of their income on non-taxable stuff like rent or
Harvard tuition.

So if you make \$10,000, for example, you spend \$6,000 on taxable stuff.  If you make \$300,000, you spend \$35,000 on
taxable stuff.

With a 10\% sales tax, the amount of tax you pay when you buy something taxable is:
\[
  S(x) = .1x
\]
$x$ is the price of the item.

\begin{parts}
\part 
Use the functions:
\begin{align*}
  T(x) &= 5000 + 0.1x \\
  S(x) &= .1x
\end{align*}

to find a composite function for the amount of sales tax you pay, as a function of income.

\begin{solution}[1cm]
\[
  f(x) = (S \circ T)(x) = .1(5,000 + .1x) = 500 + .01x
\]

\end{solution}

\part
Use your function from part a to find the amount of sales tax someone with an income of \$300,000 would pay.
\[
  f(300,000) = 500 + .01(300,000) = 3,500
\]

\part
Use your function from part a to find the amount of sales tax someone with an income of \$10,000 would pay.
\[
  f(10,000) = = 500 + .01(10,000) = 600
\]

\part
What is the tax rate for the person with a \$300,000 income?

\begin{solution}
\[
  \frac{3,500}{300,000} = .012
\]

or about 1.2\%

\end{solution}

\part
What is the tax rate for the person with a \$10,000 income?
\begin{solution}[1cm]
\[
  \frac{600}{10,000} = .06
\]

or about 6\%

Many people who make more than \$200,000/year were naturally concerned about Proposition 1098.  If it had passed, they
might have ended up with a tax rate of around 2\%, which is frighteningly close to the tax rate of around 6\% which
people with smaller incomes pay now (according to the calculations from this problem).

The proposed income tax was a \emph{progressive} tax, where higher incomes receive higher tax rates.  The current sales
tax is a \emph{regressive} tax, where lower incomes receive higher tax rates.

\end{solution}



\end{parts}

\section{Extra Credit}

\question
The graphs of the functions:
\begin{align*}
  f(x) &= m_1x + b_1 \\
  g(x) &= m_2x + b_2 \\
\end{align*}

are lines with slopes $m_1$ and $m_2$, respectively.  Is the graph of $f \circ g$ a line?  If so, what is its slope?

\begin{solution}[1cm]
\[
  (f \circ g)(x) = m_1(m_2x + b_2) + b_1 = m_1m_2x + (m_1b_2 + b_1)
\]

This function is a line with slope $m_1m_2$ and y-intercept $(m_1b_2 + b_1)$.

\end{solution}

\end{questions}

\ifprintanswers


\fi

\ifprintanswers
\else
\vspace{3 cm}

\begin{ttfamily}
\begin{verse}
l(a

le \\
af \\
fa 

ll 

s) \\
one \\
l 

iness
\end{verse}

\vspace{.1 cm}
\hspace{1 cm} --e.e. cummings

\end{ttfamily}

\fi

\end{document}

