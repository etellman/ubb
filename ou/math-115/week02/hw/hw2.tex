
\documentclass[fleqn,addpoints]{exam}

\usepackage{graphicx}
\usepackage{float}
\usepackage{amsmath}
\usepackage{cancel}
\usepackage{polynom}

\printanswers

\ifprintanswers \usepackage{2in1, lscape} \fi

\title{Math 115 Homework 2}
\date{October 5, 2010}

\begin{document}

\maketitle

\ifprintanswers
\else
\section{Key Points}
\subsection{Symmetry}

There are three kinds of symmetry:
\begin{description}                    
  \item[y-axis] whenever $(x, y)$ is on the graph, $(-x, y)$ is also on the graph.  To check for y-axis symmetry, plug
    $-x$ into the equation and see if you end up with the same equation you started out with.  

    An example of an equation with y-axis symmetry is $y=x^2$.

  \item[x-axis] whenever $(x, y)$ is on the graph, $(x, -y)$ is also on the graph.  To check for x-axis symmetry, plug
    $-y$ into the equation and see if you end up with the same equation you started out with.

    An example of an equation with x-axis symmetry is $x=y^2$.

  \item[origin] whenever $(x, y)$ is on the graph, $(-x, -y)$ is also on the graph.  To check for origin symmetry, plug
    $(-x, -y)$ into the equation and see if you end up with the same equation you started out with.

    An example of an equation with origin symmetry is $y=x^3$.
\end{description}

\subsection{Intercepts}
Intercepts are points where the graph of an equation crosses one of the axes.
\begin{description}
  \item[x-intercept] the x-intercepts happen when $y$ is 0, so to find the x-intercepts, set $y$ to 0 and solve for x.
  \item[y-intercept] the y-intercepts happen when $x$ is 0, so to find the y-intercepts, set $x$ to 0 and solve for y.
\end{description}

\subsection{Functions}

\begin{itemize}

\item A function converts values from its {\em domain} to values in its {\em range}.

% \item functional notation:
% \begin{description}
%   \item[f, g, etc.] the name of the function
%   \item[$f(x)$, $g(x)$, etc] the value of the function at $x$
%   \item[x] the independent variable
% \end{description}

\item Each number in the domain goes with a single number in the range.  A number in the range may go with more than one
  number in the domain.  

  For example, the domain of $f(x) = x^2$ is any real number.  Both $-2$ and $2$ are in the
  domain, $4$ is in the range, and $f(-2) = f(2) = 4$.

\item You can check if a graph is a graph of a function by drawing vertical lines in the graph.  If you can find a place
  where a vertical line intersects the graph in more than one point, then the graph is not a graph of a function.

\item a dotted line on a graph of a function usually marks points which are not in the domain or range of the function

\item {\em even} functions have y-axis symmetry and {\em odd} functions have origin symmetry

\end{itemize}

\pagebreak

\fi

\section{Homework}

\subsection{From the Book}

Read pages 17-22 and 31-43.
 
\begin{itemize}
  \item p. 22: 7-9, 12-13, 15, 27-28
  \item pp. 43: 1, 3-9, 12, 15-17, 19, 22-23, 26-27, 35-36, 40-43
\end{itemize}

\ifprintanswers
\section{Page 22}
\begin{itemize}

\item[7]
\begin{align*}
  x + y &= 1 \\
  y &= -x + 1 \\
\end{align*}

no symmetry

\item[8]
\begin{align*}
  2x - y &= 2 \\
  -y &= -2x + 2 \\
  y &= 2x - 2 \\
\end{align*}

no symmetry
\item[9]
\[
  y = x^2 - 1 
\]

y-axis symmetry

\item[12]
\[
  3y = x^2 \\
\]
y-axis symmetry

\item[13]
\[
  x = y^2
\]

x-axis symmetry

\item[15]
\[
  y = x^3 + 1 
\]

no symmetry

% \item[20]
% \[
%   y = \sqrt{x-1}
% \]

% no symmetry

% \item[22]
% \[
%   (x-1)^2 + y^2 = 1
% \]

% x-axis symmetry

\item[27]
\[
  y = |x| - 1
\]

y-axis symmetry

\item[28]
\[
  y = 2 - |x|
\]

y-axis symmetry

\end{itemize}

\section{Pages 43-46}
\begin{itemize}

\item[1]
\begin{itemize}
\item $f(2) = 4 \cdot 2^2 + 5 = 21$
\item $f(\sqrt{3}) = 4(\sqrt{3})^2 + 5 = 4 \cdot 3 + 5 = 17$
\item $f(2 + \sqrt{3}) = 4(2 + \sqrt{3})^2 + 5 = 33 + 16 \sqrt{3}$
\item $f(2) + f(\sqrt{3}) = 21 + 17 = 38$
\item $f(2x) = 4(2x)^2 + 5 = 16x^2 + 5$
\item $f(1-x) = 4(1-x)^2 + 5 = 4(1-2x+x^2) + 5 = 4x^2-8x+9$
\end{itemize}

\item[3-8]
numbers 3, 5, and 8 are functions

\item[9]
domain and range: $(-\infty, 0) \cup (0, \infty)$

\item[12]
domain: $(-\infty, 2)$
range: $(0, \infty)$

\item[15]
\[
  f(x) = \frac{x^4-x^2}{x^2-1} = \frac{x^2\cancel{(x^2-1)}}{\cancel{x^2-1}} = x^2
\]

In the original equation, $x$ can't be $-1$, since this would make the denominator zero.  So the domain is $(-\infty, -1) \cup (-1, 1) \cup (1, \infty)$.

The range is: $[0, 1) \cup (1, \infty)$.

\item[16]
\[
  f(x) = \frac{1}{\sqrt{x-3}}
\]

The domain is the values where $x-3>0$ or $(3, \infty)$

The range is: $(0, \infty)$

\item[17]
\[
  f(x) = \sqrt{x(x-2)}
\]

The domain is the values where $x(x-2) \geq 0$ or $(-\infty, 0] \cup [2, \infty)$

The range is $[0, \infty]$

\item[19]
The domain is $(-\infty, \infty)$ and the range is $\{-1, 1\}$.

\item[22]
This is a function of $x$.  Each value for $x$ gives exactly one value for $y$.

\item[23]
This does not describe $y$ as a function of $x$.  When $x=4$, for example, $y$ might be either $-1$ or $1$, so a single
value of $x$ corresponds to more than one value for $y$. 

\item[26]
\begin{align*}
  f(x) &= x^2 + 2x + 3 \\
  f(-x) &= (-x)^2 + 2(-x) + 3  = x^2 - 2x + 3\\
  -f(x) &= -(x^2 + 2x + 3) = -x^2 - 2x - 3 \\
  f\left( \frac{1}{x} \right) &= \left( \frac{1}{x} \right)^2 + 2 \left( \frac{1}{x} \right) + 3 
      = \frac{1}{x^2} + \frac{2}{x} + 3 \\
  \frac{1}{f(x)} &= \frac{1}{x^2 + 2x + 3} \\
  f(\sqrt{x}) &= (\sqrt{x})^2 + 2\sqrt{x} + 3 = x + 2 \sqrt{x} + 3 \\
  \sqrt{f(x)} &= \sqrt{x^2 + 2x + 3} \\
\end{align*}

\item[27]
\begin{align*}
  f(x)  &= \dfrac{1}{x} \\
  f(-x) &= - \dfrac{1}{x} \\
  -f(x) &= - \dfrac{1}{x} \\
  f\left( \frac{1}{x} \right) &= \frac{1}{1/x} = x \\
  \dfrac{1}{f(x)} &= \frac{1}{1/x} = x \\
  f(\sqrt{x}) &= \dfrac{1}{\sqrt{x}} = \frac{\sqrt{x}}{x} \\
  \sqrt{f(x)} &= \sqrt{\dfrac{1}{x}} = \frac{\sqrt{x}}{x} \\
\end{align*}

\item[35]
  $f(x) = x^2$

\[
f(x+h) = (x + h)^2  = x^2 + 2xh + h^2 \\
\]

\begin{align*}
  \frac{f(x+h) - f(x)}{h} &= \frac{x^2 + 2xh + h^2 - x^2}{h} \\
  &= \frac{\cancel{h}(2x + h)}{\cancel{h}} \\
  &= 2x + h \\
\end{align*}

\item[36]
  $f(x) = 4x^2 + 3x + 1$

\begin{align*}
  f(x+h) &= 4(x + h)^2 + 3(x+h) + 1 \\
         &= 4(x^2 + 2xh + h^2) + 3x + 3h + 1 \\
         &= 4x^2 + 8xh + 4h^2 + 3x + 3h + 1 \\
         &= 4x^2 + 8xh + 4h^2 + 3x + 3h + 1 \\
\end{align*}

\begin{align*}
  \frac{f(x+h) - f(x)}{h} &= \frac{4x^2 + 8xh + 4h^2 + 3x + 3h + 1 - (4x^2 + 3x + 1)}{h} \\
  &= \frac{4x^2 + 8xh + 4h^2 + 3x + 3h + 1 - 4x^2 - 3x - 1}{h} \\
  &= \frac{8xh + 4h^2 + 3h)}{h} \\
  &= \frac{\cancel{h} (8x + 4h + 3)}{\cancel{h}} \\
  &= 8x + 4h + 3 \\
\end{align*}

\item[40]
even

\item[41]
neither odd nor even

\item[42]
odd

\item[43]
\begin{itemize}
\item[a] even
\item[b] neither even nor odd
\item[c] odd
\item[d] neither even nor odd
\item[e] neither even nor odd
\item[f] neither even nor odd
\item[g] even
\item[h] odd
\end{itemize}

% \item[28]
% \begin{align*}
%   f(x) &= \sqrt{x} \\
%   f(-x) &= \sqrt{-x} \\
%   -f(x) &= - \sqrt{x} \\
%   f \left( \frac{1}{x} \right) &= \sqrt{\dfrac{1}{x}} = \frac{\sqrt{x}}{x} \\
%   \dfrac{1}{f(x)} &= \dfrac{1}{\sqrt{x}} = \frac{\sqrt{x}}{x} \\
%   f(\sqrt{x}) &= \sqrt[4]{x} \\
%   \sqrt{f(x)} &= \sqrt[4]{x} \\
% \end{align*}

\end{itemize}

% \pagebreak

\fi

\subsection{Extra Credit}
\begin{questions}

\question

There are twenty coins sitting on the table, ten are currently heads and tens are currently tails. You are sitting at
the table with a blindfold and gloves on. You are able to feel where the coins are, but are unable to see or feel if
they heads or tails. You must create two sets of coins. Each set must have the same number of heads and tails as the
other group. You can only move or flip the coins, you are unable to determine their current state. How do you create two
even groups of coins with the same number of heads and tails in each group?

\begin{solution}

First divide the coins arbitrarily into two groups of 10.  If the first group has $x$ heads, the numbers of heads and
tails in the two groups will be:

\begin{tabular}{|c||c|c||c|}
\hline
         & group one & group two & total   \\
\hline
   heads &    $x$      &    $10-x$   & 10      \\
   tails &    $10-x$   &    $x$      & 10      \\
\hline
   total &    10     &    10     & 10      \\
\hline
\end{tabular}

For example, if the first group has 7 heads and 3 tails, the other group will have 3 heads and 7 tails.  

So all you need to do to get the same number of heads and tails in each group is to flip over all the coins in the
second group.  Then the second group also will have $x$ heads and $10-x$ tails.

\end{solution}

\end{questions}

\ifprintanswers
\else
\vspace{5 in}

% {\em The more you can increase fear of drugs and crime, welfare mothers, immigrants and aliens, the more you control all
%  the people. }

{\em Any dictator would admire the uniformity and obedience of the U.S. media.}

\vspace{.1 cm}
\hspace{1 cm} --Noam Chomsky

\fi

\end{document}

