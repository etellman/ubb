% \documentclass[fleqn,addpoints,landscape]{exam}
\documentclass[fleqn,addpoints]{exam}
\usepackage{amsmath}
\usepackage{graphics}
\usepackage{mdwlist}

\title{Math 113 Homework Six}
\author{}
\date{\today}

% \printanswers

\ifprintanswers
\usepackage{2in1, lscape}
\fi

\begin{document}

\maketitle

\section{From the Book}

Read pages 129-143.

The formulas for areas and volumes of various shapes are on the front cover of the text book.  You'll need a few of them
for the word problems.

\begin{itemize*}
  \item pp. 135-136: 35-44, 55-59, 73-78, 85-87, 90, 95
  \item pp 142-143: 5, 6, 15-19, 30-34, 45, 46, 53-56, 57, 59, 65-69, 74, 75
\end{itemize*}

\section{Additional Problems}

\begin{questions}

\question 

Because of the Pythagorean Theorem, the length of the diagonal of a rectangle with length $L$ and width $W$
is \( \sqrt{L^2 + W^2} \).  What is the length of the diagonal (from one corner to the opposite corner) of a
three-dimensional box with length $L$, width $W$, and height $H$?

\begin{solution}
If we let $D$ be the length of the diagonal across the floor, we can compute its length using this equation: \(D^2 = L^2 + W^2 \).  

You can draw a right triangle with this diagonal as the base and the height of the box as the height of the triangle.
The hypotenuse of this triangle is also the diagonal of the box.  If we let $x$ be the length of the diagonal across
the two corners:

\begin{align*}
  x^2 &= D^2 + H^2 \\
      &= L^2 + W^2 + H^2 \\
  x   &= \sqrt{L^2 + W^2 + H^2}
\end{align*}

\end{solution}

\question
We talked in class about a few ways to prove the Pythagorean Theorem.  

The following figure was used by the Indian mathematician, Bhaskara (born 1114 CE) for another proof.  His proof was a
little short on detail, as the entire proof was the figure and the phrase ``See!''.  Fill in the additional details by
writing an equation which proves the Pythagorean Theorem using this figure. 

The area of a triangle is: \( A = \frac{1}{2}bh \) where $b$ is the length of the base and $h$ is the height.

\ifprintanswers
\else
\begin{figure*}[h]
  \centering
  \includegraphics*{proof.eps}  
  \caption{Pythagorean Theorem Proof}
  \label{fig:proof}
\end{figure*}
\fi

\begin{solution}
The area of the big square is equal to the sum of the areas of the four triangles and the area of the small square.

If we let $c$ be the hypotenuse, $a$ be the shorter side and $b$ be the longer side:

\begin{itemize*}
  \item The area of each triangle is $ \dfrac{1}{2} ab $. 
  \item The area of the small square is $ (b - a)^2 $. 
\end{itemize*}

Putting it all together:
\begin{align*}
  c^2 &= 4 \left( \frac{1}{2} ab \right) + (b - a)^2 \\
      &= 2ab + b^2 - 2ab + a^2 \\
      &= a^2 + b^2 \\
\end{align*}

\end{solution}

\ifprintanswers
\else
\pagebreak
\fi
\section{Extra Credit}

\question

A fisherman wearing a large straw hat was fishing from a rowboat in a river that flowed at a speed of three miles an
hour.  His boat drifted down the river at the same rate.

``I think I'll row upstream a few miles,'' he said to himself.  ``The fish don't seem to be biting here.''

Just as he started to row, the wind blew off his hat and it fell into the water beside the boat.  But the fisherman did
not notice his hat was gone until he had rowed upstream and was five miles away from his hat.  Then he realized what
must have happened, so he immediately started rowing back downstream again until he came to his floating hat.

In still water, the fisherman's rowing speed is always five miles an hour.  When he rowed upstream and back, he rowed at
this same constant speed.  But of course this would not be his speed relative to the {\em shore} of the river.  For
instance, when he rowed upstream at five miles an hour, the river would be carrying him downstream at three miles an
hour, so he would be passing objects on the shore at only two miles an hour.  And when he rowed downstream, his rowing
speed and the speed of the river would combine to make his speed eight miles an hour with respect to the shore.

If the fisherman lost his hat at 2:00 in the afternoon, what time was it when he recovered it?

\begin{solution}

The speed of the river doesn't really matter in this problem.  When the fisherman is rowing, he is always moving at 5
mph relative to his hat.

\begin{itemize}
  \item when going upstream, the fisherman travels at 2 mph upstream while the hat travels at 3 mph downstream, for a
    total difference of 5 mph.
  \item when going downstream, the fisherman travels at 8 mph downstream while the hat travels at 3 mph downstream, for a
    total difference of 5 mph.
\end{itemize}

So it takes the fisherman one hour to get five miles away from his hat and one hour to row back to his hat and he
recovers his hat at 4:00, two hours after he lost it.

A slightly different way to look at it is to just ignore the speed of the river and do all the calculations
relative to the river instead of relative to the shore.  With this approach, the hat is always stationary (relative to
the river) and the fisherman is always moving at 5 mph (relative to the river).  So the fisherman rows at 5 mph for an
hour to get away from the hat and then at 5 mph for an hour to get back.

\end{solution}

\end{questions}

\ifprintanswers

\section{Solutions}

\subsection{Pages 135-136}

\begin{description}

\item[35]
\( 8x^4 + 12x^3 - 24x^2 = 4x^2(2x^2 + 3x - 6) \)

\item[36]
\( 6x^5 - 18x^3 + 24x = 6x(x^4 - 3x^2 + 4) \)

\item[37]
\( 5x + 7x^2 + 9x^4 = x(5 + 7x + 9x^3) \)

\item[38]
\( 9x^2 - 17x^4 + 21x^5 = x^2(9 - 17x^2 + 21x^3) \)

\item[39]
\( 15x^2y^3 + 20xy^2 + 35x^3y^4 = 5xy^2(3xy + 4 + 7x^2y^2) \)

\item[40]
\( 8x^5y^3 - 6x^4y^5 + 12x^2y^3 = 2x^2y^3(4x^3 - 3x^2y^2 + 6) \)

\item[41]
\( x(y + 2) + 3(y + 2) = (x + 3)(y + 2) \)

\item[42]
\( x(y - 1) + 5(y - 1) = (x + 5)(y - 1) \)

\item[43]
\( 3x(2a + b) - 2y(2a + b) = (3x - 2y)(2a + b) \)

\item[44]
\( 5x(y - b) + y(a - b) = (5x + y)(a - b) \)

\item[55]
\( ax^2 - x^2 + 2a - 2 = x^2(a - 1) + 2(a - 1) = (x^2 + 2)(a - 1) \)

\item[56]
\( ax^2 - 2x^2 + 3a - 6 = x^2(a - 2) + 3(a - 2) = (x^2 + 3)(a - 2) \)

\item[57]
\begin{eqnarray*}
   && 2ac + 3bd + 2bc + 3ad \\
  &=& 2c(a + b) + 3d(a + b) \\
  &=& (2c + 3d)(a + b) \\
\end{eqnarray*}

\item[58]
\begin{eqnarray*}
  && 2bx + cy + cx + 2by \\
  &=& 2bx + cx + cy + 2by \\
  &=& x(2b + c) + y(2b + c) \\
  &=& (x + y)(2b + c) \\
\end{eqnarray*}

\item[59]
\begin{eqnarray*}
  &&  ax - by + bx - ay \\
  &=& ax + bx -ay - by \\
  &=& x(a + b) - y(a + b) \\
  &=& (x - y)(a + b) \\
\end{eqnarray*}

\item[73]
\begin{eqnarray*}
  3x^2 + 7x &=& 0 \\
  x(3x + 7) &=& 0 \\ 
\end{eqnarray*}
\begin{eqnarray*}
  x = 0 &or& 3x + 7 = 0 \\  
  x = 0 &or& 3x = -7 \\  
  x = 0 &or& x = -7/3 \\  
\end{eqnarray*}

\item[74]
\begin{align*}
  -4x^2 + 9x &= 0 \\
  x(-4x + 9) &= 0 \\
\end{align*}
\begin{eqnarray*}
  x = 0 &or& -4x + 9 = 0 \\
  x = 0 &or& -4x = -9 \\
  x = 0 &or& x = 9/4 \\
\end{eqnarray*}

\item[75]
\begin{align*}
  4x^2 &= 5x \\
  4x^2 - 5x &= 0 \\
  x(4x - 5) &= 0 \\
\end{align*}

\begin{eqnarray*}
  x = 0 &or& 4x - 5 = 0 \\
  x = 0 &or& 4x = 5 \\
  x = 0 &or& x = 5/4 \\
\end{eqnarray*}

\item[76]
\begin{align*}
3x &= 11x^2 \\
11x^2 - 3x &= 0 \\
x(11x - 3) &= 0 \\
\end{align*}

\begin{eqnarray*}
x = 0 &or& 11x - 3 = 0 \\
x = 0 &or& 11x = 3 \\
x = 0 &or& x = 3/11 \\
\end{eqnarray*}

\item[77]
\begin{align*}
  x - 4x^2 &= 0 \\
  x(1 - 4x) &= 0 \\
\end{align*}

\begin{eqnarray*}
  x = 0 &or& 1 - 4x = 0 \\
  x = 0 &or& -4x = -1 \\
  x = 0 &or& x = 1/4 \\
\end{eqnarray*}

\item[78]
\begin{align*}
  x - 6x^2 &= 0 \\
  x(1 - 6x) &= 0 \\
\end{align*}

\begin{eqnarray*}
  x = 0 &or& 1 - 6x = 0 \\
  x = 0 &or&  -6x = -1 \\
  x = 0 &or&  x = 1/6 \\
\end{eqnarray*}

\item[85]
\begin{align*}
  y^2 - ay + 2by - 2ab &= 0 \\
  y(y - a) + 2b(y - a) &= 0 \\
  (y + 2b)(y - a) &= 0 \\
\end{align*}

\begin{eqnarray*}
  y + 2b = 0 &or& y - a = 0 \\
  y = -2b &or& y = a \\
\end{eqnarray*}

\item[86]
\begin{align*}
  x^2 + ax + bx + ab &= 0 \\
  x(x + a) + b(x + a) &= 0 \\
  (x + b)(x + a) &= 0 \\
\end{align*}

\begin{eqnarray*}
  x + b = 0 &or& x + a = 0 \\
  x = -b &or& x = -a \\
\end{eqnarray*}

\item[87]
\begin{align*}
  x^2 &= 7x \\
  x^2 - 7x &= 0 \\
  x(x - 7) &= 0 \\
\end{align*}

\begin{eqnarray*}
  x = 0 &or& x - 7 = 0 \\
  x = 0 &or& x = 7 \\
\end{eqnarray*}

\item[90]
\begin{align*}
  2 \pi r &= \pi r^2 \\
  2r &= r^2 \\
  0 &= r^2 - 2r \\
  r^2 - 2r &= 0 \\
  r(r - 2) &= 0 \\
\end{align*}

In this case, $r = 0$ doesn't make sense, since you can't have a circle with radius zero.  So the answer is 
$r = 2$.

\item[95]
\begin{align*}
  \frac{4}{3} \pi r^3 &= 2 \cdot 4 \pi r^2 \\
  \frac{4}{3} r^3 &= 8r^2 \\
  4r^3 &= 24r^2 \\
  4r^3 - 24r^2 &= 0 \\
  r^2(4r - 24) &= 0 \\
\end{align*}

In this case, $r = 0$ doesn't make sense, since you can't have a circle with radius zero.  So the answer is:
\begin{align*}
  4r - 24 &= 0 \\
  4(r - 6) &= 0 \\
  r - 6 &= 0 \\
  r &= 6 \\
\end{align*}

Checking, the volume is 288 and the area is 144, so the answer is correct.

\subsection{Pages 142-143}

\item[5]
\( 9x^2 - 25y^2 = (3x)^2 - (5y)^2 = (3x + 5y)(3x - 5y) \)

\item[6]
\( x^2 - 64y^2 = x^2 - (8y)^2 = (x + 8y)(x - 8y) \)

\item[15]
\begin{eqnarray*}
  &&  4x^2 - (y + 1)^2 \\
  &=& (2x)^2 - (y + 1)^2 \\
  &=& (2x + y + 1)(2x - (y + 1)) \\ 
  &=& (2x + y + 1)(2x - y - 1) \\ 
\end{eqnarray*}

\item[16]
\begin{eqnarray*}
  &&  x^2 - (y - 5)^2\\
  &=& x^2 - (y - 5)^2 \\
  &=& (x + y - 5)(x - (y - 5)) \\ 
  &=& (x + y - 5)(x - y + 5) \\ 
\end{eqnarray*}

\item[17]
\begin{eqnarray*}
  &&  9a^2 - (2b + 3)^2 \\
  &=& (3a)2 - (2b + 3)^2 \\
  &=& (3a + 2b + 3)(3a - (2b + 3)) \\ 
  &=& (3a + 2b + 3)(3a - 2b - 3) \\ 
\end{eqnarray*}

\item[18]
\begin{eqnarray*}
  &&  16s^2 - (3t + 1)^2 \\
  &=& (4s)^2 - (3t + 1)^2 \\
  &=& (4s + 3t + 1)(4s - (3t + 1) \\ 
  &=& (4s + 3t + 1)(4s - 3t - 1) \\ 
\end{eqnarray*}

\item[19]
\begin{eqnarray*}
  &&  (x + 2)^2 - (x + 7)^2 \\
  &=& (x + 2 + x + 7)(x + 2 - (x + 7)) \\
  &=& (2x + 9)(x + 2 - x - 7) \\ 
  &=& (2x + 9)(-5) \\ 
  &=& -5(2x + 9) \\ 
\end{eqnarray*}


\item[30]
\begin{eqnarray*}
  && x^4 - 16 \\
  &=& (x^2)^2 - 4^2 \\
  &=& (x^2 + 4)(x^2 - 4) \\
  &=& (x^2 + 4)(x + 2)(x - 2) \\
\end{eqnarray*}

\item[31]
\begin{eqnarray*}
  && n^4 - 81 \\
  &=& (n^2)^2 - 9^2 \\
  &=& (n^2 + 9)(n^2 - 9) \\
  &=& (n^2 + 9)(n + 3)(n - 3) \\
\end{eqnarray*}

\item[32]
$4x^2 + 9$ can't be factored using integers.

\item[33]
\( 3x^3 + 27x = 3x(x^2 + 9) \)

\item[34]
\( 20x^3 + 45x = 5x(4x^2 + 9) \)

\item[45]
\begin{eqnarray*}
  &&  a^3 - 64 \\
  &=& a^3 - 4^3 \\
  &=& (a - 4)(a^2 + 4a + 16) \\ 
\end{eqnarray*}

\item[46]
\begin{eqnarray*}
  &&  a^3 - 27 \\
  &=& a^3 - 3^3 \\
  &=& (a - 3)(a^2 + 3a + 9) \\ 
\end{eqnarray*}

\item[53]
\begin{eqnarray*}
  &&  x^3y^3 - 1 \\
  &=& (xy)^3 - 1^3 \\
  &=& (xy - 1)((xy)^2 + 1 \cdot xy + 1^2) \\ 
  &=& (xy - 1)(x^2y^2 + xy + 1) \\ 
\end{eqnarray*}

\item[54]
\begin{eqnarray*}
  &&  125x^3 + 27y^3 \\
  &=& (5x)^3 + (3y)^3 \\
  &=& (5x + 3y)((5x)^2 - (5x)(3y) + (3y)^2) \\ 
  &=& (5x + 3y)(25x^2 - 15xy + 9y^2) \\ 
\end{eqnarray*}

\item[55]
Using the sum of two cubes:
\begin{eqnarray*}
  &&  x^6 - y^6 \\
  &=& (x^2)^3 - (y^2)^3 \\
  &=& (x^2 - y^2)((x^2)^2 + x^2y^2 + (y^2)^2) \\ 
  &=& (x + y)(x - y)(x^4 + x^2y^2 + y^4) \\ 
\end{eqnarray*}

Here is a different factorization which starts out by using the sum of two squares:
\begin{eqnarray*}
  &&  x^6 - y^6 \\
  &=& (x^3)^2 - (y^3)^2 \\
  &=& (x^3 + y^3)(x^3 - y^3)\\ 
  &=& (x + y)(x^2 - xy + y^2)(x - y)(x^2 + xy + y^2) \\ 
  &=& (x + y)(x - y)(x^2 - xy + y^2)(x^2 + xy + y^2) \\ 
\end{eqnarray*}

\item[56]
For this one, the only approach is the sum of two cubes:
\begin{eqnarray*}
  &&  x^6 + y^6 \\
  &=& (x^2)^3 + (y^2)^3 \\
  &=& (x^2 + y^2)((x^2)^2 - x^2y^2 + (y^2)^2) \\ 
  &=& (x^2 + y^2)(x^4 - x^2y^2 + y^4) \\ 
\end{eqnarray*}

\item[57]
\begin{eqnarray*}
  x^2 - 25 &=& 0 \\
  (x + 5)(x - 5) &=& 0 \\
\\
  x + 5 = 0 & or & x - 5 = 0 \\
  x = -5 & or & x = 5 \\
\end{eqnarray*}

\item[59]
\begin{eqnarray*}
  9^2 - 49 &=& 0 \\
  (3x + 7)(3x - 7) &=& 0 \\
\\
  3x + 7 = 0 & or & 3x - 7 = 0 \\
  x = 7/3 & or & x = -7/3 \\
\end{eqnarray*}

\item[65]
\begin{eqnarray*}
  20 - 5x^2 &=& 0 \\
  5(4 - x^2) &=& 0 \\
  5(2 - x)(2 + x) &=& 0 \\
\\
  2 - x = 0 & or & 2 + x = 0 \\
  x = 2 & or & x = -2 \\
\end{eqnarray*}

\item[66]
\begin{eqnarray*}
  54 - 6x^2 &=& 0 \\
  6(9 - x^2) &=& 0 \\
  9 - x^2 &=& 0 \\
  (3 + x)(3 - x) &=& 0 \\
\\
  3 + x = 0 & or & 3 - x = 0 \\
  x = -3 & or & x = 3 \\
\end{eqnarray*}

\item[67]
\begin{eqnarray*}
  x^4 - 81 &=& 0 \\
  ((x^2)^2 - 9^2) &=& 0 \\
  (x^2 + 9)(x^2 - 9) &=& 0 \\
  (x^2 + 9)(x + 3)(x - 3) &=& 0 \\
\\
  x + 3 = 0 & or & x - 3 = 0 \\
  x = -3 & or & x = 3 \\
\end{eqnarray*}

$x^2 + 9 = 0$ doesn't have a real number solution, since no real number squared is $-9$.

\item[68]
\begin{eqnarray*}
  x^5 - x &=& 0 \\
  x((x^2)^2 - 1^2) &=& 0 \\
  x(x^2 + 1)(x^2 - 1) &=& 0 \\
  x(x^2 + 1)(x + 1)(x - 1) &=& 0 \\
\end{eqnarray*}

\begin{itemize}
  \item $x = 0$
  \item $x^2 + 1 = 0$ has no real number solutions
  \item $x + 1 = 0; x = -1$
  \item $x - 1 = 0; x = 1$
\end{itemize}

So the solution set is: $\{0, -1, 1\}$.  Looking back at the original equation, all of these work, since you either 
get $0 - 0 = 0$, $-1 + 1 = 0$, or $1 - 1 = 0$.

\item[69]
\begin{eqnarray*}
  6x^3 + 24x &=& 0 \\
  6x(x^2 + 4) &=& 0 \\
  x(x^2 + 4) &=& 0 \\
\end{eqnarray*}

$x^2 + 4$ has no real number solutions, so the only solution is $x = 0$.

\item[74]
\begin{itemize}
  \item $s$ is the length of a side in the small square
  \item $5s$ is the length of a side in the large square
  \item $s^2$ is the area of the small square
  \item $(5s)^2 = 25s^2$ is the area of the large square
\end{itemize}

So the equation is:
\begin{align*}
  s^2 + 25s^2 &= 26  \\
  26s^2 &= 26  \\
  s^2   &= 1 \\
  s^2 - 1  &= 0 \\
  (s + 1)(s - 1) = 0 \\
\end{align*}

The solution set for this equation is: $\{-1, 1\}$.  But you can't have a negative length.  So the only
solution is $s = 1$.

% \item[74]
% \begin{itemize}
%   \item $r$ is the radius of the smaller circle
%   \item $2r$ is the radius of the larger circle
%   \item $\pi r^2$ is the area of the smaller circle
%   \item $\pi (2r)^2 = 4 \pi r^2$ is the area of the larger circle
% \end{itemize}

% So the equation is:
% \begin{eqnarray*}
%   \pi r^2 + 4 \pi r^2 &=& 80 \pi \\
%   r^2 + 4r^2 = 80 \\
%   5r^2 = 80 \\
%   r^2 = 16 \\
%   r^2 - 16 = 0 \\
%   (r + 4)(r - 4) = 0 \\
% \end{eqnarray*}

% The solution set for this equation is: $\{-4, 4\}$.  But you can't have a circle with a negative radius.  So the only
% solution is $r = 4$.

\item[75] 
\begin{itemize}
  \item $w$ is the width
  \item $2w$ is the length
  \item $w \cdot 2w = 2w^2$ is the area
\end{itemize}

\begin{align*}
  2w^2 = 50 \\
  w^2 = 25 \\
  w^2 - 25 = 0 \\
  (w + 5)(w - 5) = 0 \\
\end{align*}
\end{description}

The solution set for this equation is: $\{-5, 5\}$.  But you can't have a negative width.  So the only
solution is $w = 5$.

\else
\vspace{5 in}

{\em If you must be bothered by criticism, be bothered by the criticism of an enlightened man.}

\vspace{.2 cm}
\hspace{1 cm} --Dogen

\fi

\end{document}
