% no answer key
\documentclass[letterpaper]{exam}

% answer key
% \documentclass[letterpaper, landscape]{exam}
% \usepackage{2in1, lscape} 
% \printanswers{}

\usepackage{units} 
\usepackage{xfrac} 
\usepackage[fleqn]{amsmath}
\usepackage{commath}
\usepackage{cancel}
\usepackage{float}
\usepackage{mdwlist}
\usepackage{booktabs}
\usepackage{cancel}
\usepackage{polynom}
\usepackage{caption}
\usepackage{fullpage}
\usepackage{comment}
\usepackage{enumerate}
\usepackage{graphicx}
\usepackage{mathtools} 
\usepackage{parskip} 

\newcommand{\degree}{\ensuremath{^\circ}} 
\everymath{\displaystyle}

\title{Algebra Homework 13 \\ Sections 4.3 and 4.6}
\author{}
\date{\today}

\begin{document}

  \maketitle

  \section{Homework}

  \begin{itemize*}
    \item Section 4.3: 20--24, 40--44, 60--64
    \item Section 4.6: 10--14, 20--24, 30--34, 40--44, 46, 51
  \end{itemize*}

  \section{Extra Credit}

  A lady gave the postage stamp clerk a one dollar bill and said, ``Please give
  me some two-cent stamps, ten times as many one-cent stamps, and the balance
  in fives.''  How can the clerk fulfill this puzzling request?

  \begin{solution}

    If:
    \begin{itemize*}
      \item $x$ is the number of two-cent stamps purchased
      \item $2x$ is the amount spent on two-cent stamps
      \item $10x$ the number of one-cent stamps purchased, and, since each
        stamp costs one-cent, the amount spent on one-cent stamps.
      \item $100 - 10x - 2x$ is the amount spent on five-cent stamps
    \end{itemize*}

    She spent 100 cents on stamps, so this equation must be true (all amounts
    are in cents):

    \[
      (10x) + (2x) + (100 - 10x - 2x) = 100 
    \]

    This equation simplifies to: $100 = 100$ which is true for any value of $x$.  

    The problem is that the equation doesn't include the additional constraint
    for this problem which is: ``you can only buy a positive integer number of
    stamps.''  You can't buy half a stamp, and you can't buy $-2$ stamps, for
    example.

    We have to think about the problem a little more.  She spends whatever
    money is left over after the one and two cent stamps on five-cent stamps.
    This means that she must have a multiple of 5 left over after she pays for
    the one and two cent stamps.  And this means that the amount she spends on
    one and two cent stamps must be a multiple of 5.  

    If we stick with $x$ as the number of two-cent stamps purchased, $x + 10x$
    must be a multiple of five less than 100.  Letting $x$ be 5 works, and the
    next possibility is $x=10$ which is too big.  So she must have purchased 5
    two-cent stamps.  She purchased ten times as many one-cent stamps, so she
    bought 50 one-cent stamps.  She has 40 cents left over, which she spends on
    8 5-cent stamps.  
    
    The final totals are:
    \begin{itemize*}
      \item 5 two cent stamps
      \item 50 one-cent stamps
      \item 8 five-cent stamps
    \end{itemize*}

  \end{solution}

  \ifprintanswers{}
    \section{Section 4.3} % (fold)
    
    \begin{description}
      \item[20] $\frac{9x + 14}{20}$

      \item[21] $\frac{12a + 1}{12}$

      \item[22] $\frac{16a - 19}{24}$

      \item[23] $\frac{n + 14}{18}$

      \item[24] $\frac{5n-5}{36}$

      \item[41] $\frac{11x - 10}{6x^2}$

      \item[42] $\frac{28 - 57x}{12x^2}$

      \item[50] $\frac{3x^2 - 2x + 8}{x(x-4)}$

      \item[51] $\frac{a^2 - a - 8}{a(a+4)}$

      \item[52] $\frac{a^2 + 1}{a(a+1)}$

      \item[53] $\frac{-41n - 55}{(4n + 5)(3n + 5)}$

      \item[54] $\frac{-10n + 30}{(n-6)(2n+3)}$

    \end{description}

  \fi
  \ifprintanswers{}
  \else
    \vspace{9 cm}
    \begin{quote}
      \begin{em}
        Imagination is more important than knowledge.
      \end{em}
    \end{quote}
    \hspace{2 cm}--Albert Einstein
  \fi

\end{document}

