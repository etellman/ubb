% no answer key
\documentclass[letterpaper]{article}

\usepackage{units} 
\usepackage{xfrac} 
\usepackage[fleqn]{amsmath}
\usepackage{commath}
\usepackage{cancel}
\usepackage{float}
\usepackage{mdwlist}
\usepackage{booktabs}
\usepackage{cancel}
\usepackage{polynom}
\usepackage{caption}
\usepackage{fullpage}
\usepackage{comment}
\usepackage{enumerate}
\usepackage{graphicx}
\usepackage{mathtools} 

\newcommand{\degree}{\ensuremath{^\circ}} 
\everymath{\displaystyle}

\title{Algebra Notes \\ Sections 4.3 and 4.6} 
\author{}

\date{\today}

\begin{document}

  \maketitle
  
  \section{Addition} % (fold)

  \begin{itemize*}
    \item find common denominator
    \item add numerators
    \item factor/simplify
  \end{itemize*}

  \section{Equations} % (fold)

  \begin{itemize*}
    \item note values which would make the denominator zero
    \item multiply by LCM of denominators
    \item solve
    \item solution is everything other than values noted at step one
  \end{itemize*}

  Cross-multiplication is shortcut for equations like:
  \[
    \frac{a}{b} = \frac{c}{d}
  \]
  
\end{document}
