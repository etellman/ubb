\documentclass[fleqn,addpoints]{exam}
\usepackage{amsmath}
\usepackage{graphicx}
\usepackage{cancel}
\usepackage{polynom}

\printanswers

\ifprintanswers
\usepackage{2in1, lscape}
\fi

\title{Math 113 Chapter Four Exam}
\author{}
\date{\today}

% \oddsidemargin 0in
% \topmargin -0.5in
% \textwidth 6.5in


% \extrawidth{-1 in}
% \setlength{\mathindent}{0in}

\begin{document}

\maketitle

\ifprintanswers
\else
\vspace{0.2in}
\makebox[\textwidth]{Name:\enspace\hrulefill}
\vspace{0.2in}

\begin{center}
\gradetable[h][pages]
\end{center}

\fi

\section{Questions}

\subsection{Simplifying Rational Expressions}

For problems \ref{simplify:first}-\ref{simplify:last}, simplify each expression.

\begin{questions}

\question[5] \( \displaystyle \frac{42x^4y^3}{35x^3y^5} \)
\label{simplify:first}
\begin{solution}[2 cm]
\[ 
\frac{42x^4y^3}{35x^3y^5} = \frac{6x}{5y^2}
\]
\end{solution}

\question[5] \( \displaystyle \frac{x^2-9}{x^2+3x} \)
\begin{solution}[2 cm]
\[
\frac{x^2-9}{x^2+3x}  = \frac{\cancel{(x+3)}(x-3)}{x \cancel{(x+3)}} = \frac{x-3}{x}
\]
\end{solution}

\question[5] \( \displaystyle \frac{2x^2+x-28}{2x^2-13x+21} \)
\begin{solution}[3 cm]
\[
  \frac{2x^2+x-28}{2x^2-13x+21} = \frac{\cancel{(2x-7)}(x+4)}{\cancel{(2x-7)}(x+3)} = \frac{x+4}{x-3}
\]
\end{solution}

\question[7] \( \displaystyle \frac{3x^3y - 2x^2y - 16xy}{x^2y^2 + 6xy^2 + 8y^2} \)
\label{simplify:last}
\begin{solution}[6 cm]
\[
  \frac{3x^3y - 2x^2y - 16xy}{x^2y^2 + 6xy^2 + 8y^2} 
  = \frac{x(3x-8) \cancel{(x+2)}} {y(x+4) \cancel{x+2)}} 
  = \frac{x(3x-8)}{y(x+4)} 
\]

\end{solution}

\subsection{Multiplying and Dividing Rational Expressions}

For problems \ref{multiply:first}-\ref{multiply:last}, perform each multiplication or division and simplify the result
as much as possible.

\question[7] \( \displaystyle \frac{20a^2b}{7a^3} \cdot \frac{3a^3b^2}{12a^2b^2} \)
\label{multiply:first}
\begin{solution}[3 cm]
\[
  \frac{20a^2b}{7a^3} \cdot \frac{3a^3b^2}{12a^2b^2}  = \frac{5b}{7}
\]
\end{solution}

\question[7] \( \displaystyle \frac{2xy}{x^2-4x+4} \div \frac{16y}{x^2-4}\)
\begin{solution}[4 cm]
\[
  \frac{2xy}{(x-2) \cancel{(x-2)}} \cdot \frac{(x+2) \cancel{(x-2)}}{16y} = \frac{x(x+2)}{8(x-2)}
\]
\end{solution}

\question[7] \( \displaystyle \frac{9x^2y}{10x} \cdot \frac{35xy}{12y^2} \div \frac{3x^3}{4xy^2} \)
\begin{solution}[3 cm]
\[
  \frac{9x^2y}{10x} \cdot \frac{35xy}{12y^2} \div \frac{3x^3}{4xy^2} 
  = \frac{9x^2y}{10x} \cdot \frac{35xy}{12y^2} \cdot \frac{4xy^2}{3x^3}
  = \frac{7y^2}{2}
\]
\end{solution}

\question[10] \( \displaystyle \frac{x^2-6x-7}{2x^2-18x+28} \cdot \frac{-4x^2 + 2x + 12}{2x^2+5x+3} \)
\label{multiply:last}
\begin{solution}[7 cm]
\begin{align*}
  & \frac{x^2-6x-7}{2x^2-18x+28} \cdot \frac{-4x^2 + 2x + 12}{2x^2+5x+3} \\
  &= \frac{(x-7) \cancel{(x+1)} (-2)(2x^2-x-6)}{2(x^2-9x+14)(2x+3) \cancel{(x+1)}} \\
  &= \frac{- \cancel{(2)} \cancel{(x-7)} \cancel{(2x+3)} \cancel{(x-2)} } {\cancel{(2)}\cancel{(x-7)} \cancel{(x-2)} \cancel{(2x+3)}} \\
  &= -1 \\
\end{align*}

\end{solution}

\subsection{Adding and Subtracting Rational Expressions}

For problems \ref{add:first}-\ref{add:last}, perform each addiiton or subtraction and simplify the result
as much as possible.

\question[5] \( \displaystyle \frac{2}{x+2} - \frac{3}{x-2} \)
\label{add:first}
\begin{solution}[5 cm]
\begin{align*}
  \frac{2}{x+2} - \frac{3}{x-2} &= \frac{2(x-2) - 3(x+2)}{(x+2)(x-2)} \\
  &= \frac{2x-4-3x-6}{(x+2)(x-2)} \\
  &= \frac{-x-10}{(x+2)(x-2)} \\
\end{align*}
\end{solution}

\question[5] \( \displaystyle \frac{x}{x^2-9} + \frac{3}{x-3} \)
\begin{solution}[5 cm]
\begin{align*}
  \frac{x}{x^2-9} + \frac{3}{x-3} &= \frac{x}{(x+3)(x-3)} + \frac{3}{x-3} \\
  &= \frac{x}{(x+3)(x-3)} + \frac{3(x+3)}{(x+3)(x-3)} \\
  &= \frac{x + 3x + 9}{(x+3)(x-3)} \\
  &= \frac{4x + 9}{(x+3)(x-3)} \\
\end{align*}

\end{solution}

\question[10] \( \displaystyle \frac{3x}{x^2+6x+8} + \frac{2}{x+2} - \frac{4x-1}{x+4} \)
\label{add:last}
\begin{solution}[5 cm]
\begin{align*}
  & \frac{3x}{x^2+6x+8} + \frac{2}{x+2} - \frac{4x-1}{x+4} \\
  &= \frac{3x}{(x+2)(x+4)} + \frac{2}{x+2} - \frac{4x-1}{x+4} \\
  &= \frac{3x}{(x+2)(x+4)} + \frac{2(x+4)}{(x+2)(x+4)} - \frac{(x+2)(4x-1)}{(x+2)(x+4)} \\
  &= \frac{3x+2x+8-(4x^2+7x-2)}{(x+2)(x+4)} \\
  &= \frac{5x+8-4x^2-7x+2}{(x+2)(x+4)} \\
  &= \frac{-4x^2 -2x + 10}{(x+2)(x+4)} \\
\end{align*}

\end{solution}

\subsection{Complex Fractions}

For problems \ref{complex:first}-\ref{complex:last}, simplify each complex fraction as much as possible.

\question[5] \( \displaystyle \frac{\cfrac{1}{x} - \cfrac{2}{y}} {\cfrac{3}{y} + \cfrac{7}{xy}} \)
\label{complex:first}
\begin{solution}[4 cm]
\[
  \frac{\cfrac{1}{x} - \cfrac{2}{y}} {\cfrac{3}{y} + \cfrac{7}{xy}} 
  = \left( \frac{xy}{xy} \right) \left( \frac{\cfrac{1}{x} - \cfrac{2}{y}} {\cfrac{3}{y} + \cfrac{7}{xy}} \right)
  = \frac{y-2x}{3x+7}
\]
\end{solution}

\question[7] \( \displaystyle \frac{\cfrac{2}{x-2} - \cfrac{3}{x+2}} { \cfrac{5}{x^2-4} + \cfrac{9}{x-2} }\)
\begin{solution}[5 cm]
\begin{align*}
  & \frac{\cfrac{2}{x-2} - \cfrac{3}{x+2}} { \cfrac{5}{x^2-4} + \cfrac{9}{x-2} } \\
  &= \frac{\cfrac{2}{x-2} - \cfrac{3}{x+2}} { \cfrac{5}{(x+2)(x-2)} + \cfrac{9}{x-2} } \\
  &= \left( \frac{(x+2)(x-2)}{(x+2)(x-2)} \right) \left( \frac{\cfrac{2}{x-2} - \cfrac{3}{x+2}} { \cfrac{5}{(x+2)(x-2)} + \cfrac{9}{x-2} } \right) \\
  &= \frac{2(x+2) - 3(x-2)}{5 + 9(x+2)} \\
  &= \frac{2x+4 - 3x + 6}{5 + 9x+ 18} \\
  &= \frac{-x+10}{9x+ 23} \\
\end{align*}

\end{solution}

\question[7] \( \displaystyle \frac{4x}{x - \cfrac{1}{x}} - 1 \)
\label{complex:last}
\begin{solution}[5 cm]
\begin{align*}
  & \frac{4x}{x - \cfrac{1}{x}} - 1 \\
  &= \left( \frac{x}{x} \right) \left( \frac{4x}{x - \cfrac{1}{x}} \right) - 1 \\  
  &=  \frac{4x^2}{x^2 - 1} - 1 \\
  &=  \frac{4x^2}{x^2 - 1} - \frac{x^2-1}{x^2-1} \\
  &=  \frac{4x^2 - (x^2 - 1)}{x^2 - 1} \\
  &=  \frac{3x^2 + 1}{x^2 - 1} \\
\end{align*}

\end{solution}

\subsection{Long Division}

For problems \ref{long-div:first}-\ref{long-div:last}, perform each long division.

\question[7] \( (2x^2-7x-72) \div (x-8) \)
\label{long-div:first}
\begin{solution}[7 cm]
\[ \polylongdiv{2x^2-7x-72}{x-8} \]
\[ 2x+9 \]
\end{solution}

\question[10] \( (x^3-10) \div (x+4) \)
\label{long-div:last}
\begin{solution}[5 cm]
\[ \polylongdiv{x^3-10}{x+4} \]
\[ x^2-4x+16 - \frac{74}{x+4}\]
\end{solution}

\subsection{Equations with Rational Expressions}

For problems \ref{equation:first}-\ref{equation:last}, solve each equation.

\question[7] \( \displaystyle \frac{4}{2x-3} = \frac{8}{x+3} \)
\label{equation:first}
\begin{solution}[6 cm]
\begin{align*}
  \frac{4}{2x-3} &= \frac{8}{x+3} \\
  4(x+3) &= 8(2x-3) \\
  4x + 12 &= 16x - 24 \\
  12x &= 36 \\
  x &= 3 \\
\end{align*}
\end{solution}

\question[7] \( \displaystyle \frac{x}{x+3} + 1 = \frac{4}{x+3}\)
\begin{solution}[6 cm]
\begin{align*}
  \frac{x}{x+3} + 1 &= \frac{4}{x+3} \\
  (x + 3) \left( \frac{x}{x+3} + 1 \right) &= (x+3) \left( \frac{4}{x+3} \right) \\
  x + x + 3 &= 4 \\
  2x + 3 &= 4 \\
  2x  &= 1 \\
  x  &= \frac{1}{2} \\
\end{align*}
\end{solution}

\question[10] \( \displaystyle \frac{3}{x-5} + \frac{4}{x+7} = \frac{2x+11}{x^2+2x-35} \)
\label{equation:last}
\begin{solution}[5 cm]
\begin{align*}
  \frac{3}{x-5} + \frac{4}{x+7} &= \frac{2x+11}{x^2+2x-35} \\
  \frac{3}{x-5} + \frac{4}{x+7} &= \frac{2x+11}{(x-5)(x+7)} \\
  3(x+7) + 4(x-5) &= 2x+11 \\
  3x+21 + 4x-20 &= 2x+11 \\
  7x+1 &= 2x+11 \\
  5x &= 10 \\
  x &= 2 \\
\end{align*}
\end{solution}

\subsection{Word Problems}

For problems \ref{word:first}-\ref{word:last}, use an equation to solve each problem.

\question[7]
\label{word:first}

In a recent Washington case, the court ruled that the Washington law which says that felons can't vote unfairly
discriminiates against African-Americans because African-Americans are disproportionalty represented in the Washington
prison system.

One of the arguments the plaintiffs made was that the prison population doesn't match the population of the state.  If
the the ratio of African-Americans to non-African-Americans in the state is 1 to 9, and the prison population did match
the state population, how many African-American prisoners would you find in a prison with 800 prisoners?

\begin{solution}[7 cm]

If we let $x$ be the expected number of African-American prisoners, there are $800-x$ other prisoners:

\begin{align*}
  \frac{1}{9} &= \frac{x}{800-x} \\
  800-x &= 9x \\
  10x &= 800 \\
  x &= 80 \\
\end{align*}

Another approach is to use the ratio of African-Americans to population instead of African-Americans to
non-African-Americans.  If for every African-American there are 9 non-African-Americans, 1 out of every 10 people in the
state is African-American.  So we can use this ratio:

\begin{align*}
  \frac{1}{10} &= \frac{x}{800} \\
  800 &= 10x \\
  x &= 80 \\
\end{align*}

A common mistake was to say that: \( \displaystyle \frac{1}{9} = \frac{x}{800} \) which is not correct since the
denominator of the first fraction represents the number of non-African-Americans and the denomonator of the second
fraction represents the total number (of both categories) of prisoners.

\end{solution}

\question[10]
\label{word:last}
One day a swallow went in search of food.  He flew for 20 miles and found a very tasty looking coconut which he
picked up and carried 20 miles back to his nest.  Naturally, carrying a coconut is no easy task for a swallow, so he
travelled twice as fast without the coconut as with the coconut.  If the entire trip (both ways) took 3 hours, how fast
did the swallow fly while carrying the coconut?

\begin{solution}[5 cm]

Here's what we know about the trip:
\begin{itemize}
  \item $d_{out} = d_{in} = 20$
  \item $r_{out} = 2 r_{in}$
  \item $\displaystyle t_{out} = \frac{d_{out}}{r_{out}} = \frac{20}{2r_{in}}$
  \item $\displaystyle t_{in} = \frac{d_{in}}{r_{in}} = \frac{20}{r_{in}}$
  \item $\displaystyle t_{total} = t_{out} + t_{in} = \frac{20}{2r_{in}} + \frac{20}{r_{in}} = 3$
\end{itemize}

The last equation just has one variable, so we can solve it for $r_in$:
\begin{align*}
  \frac{20}{2r_{in}} + \frac{20}{r_{in}} &= 3  \\
  \frac{10}{r_{in}} + \frac{20}{r_{in}} &= 3  \\
  10 + 20 &= 3 r_{in}  \\
  30 &= 3 r_{in}  \\
  r_{in} &= 10  \\
\end{align*}

He went 10 mph carrying the coconut and 20 mph without the coconut.  We can check this, since going 20 miles at 10 mph
would take 20/10 = 2 hours and going 20 miles at 20 mph would take 20/20 = 1 hour, accounting for all three hours.


\end{solution}

\noaddpoints

\ifprintanswers
\else
\pagebreak
\fi

\subsection{Extra Credit}

\question[10]

When a trip is made by car, the car will of course travel different speeds at different times.  If the total distance is
divided by the total driving time, the result is called the {\em average speed} for that trip.

Alice planned to drive the 30 miles from Seattle to Tacoma and back again.  She wanted to average 30 mph for the 60 mile
round trip.  

On the way to Tacoma, she was stuck in traffic and only managed to average 20 mph.  What must her average speed be on
the return trip in order to raise her average for the round trip to 30 mph?

\begin{solution}[5 cm]

The first thing to do is to figure out how long the trip needs to take for the average speed to be 30 mph.  Since the
round trip is 60 miles, the desired time for the trip will be $t=d/r = 60/30 = 2$ hours.

The next thing to do is to figure out how much of this time was spent on the outgoing trip.  She drove 30 miles at 20
mph, which took $30/20 = 3/2$ hours.

Since she took $3/2$ hours on the way to Tacoma and she wants to take 2 hours for the whole trip, she has half an hour to get
back.  To travel 30 miles in half an hour requires a rate of: $\displaystyle r = \frac{d}{t} = \frac{30}{1/2} = 60$.

So she needs to drive 60 mph on the return trip to get her average speed up to 30 mph.

\end{solution}


\end{questions}

\end{document}


