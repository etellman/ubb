% no answer key
\documentclass[letterpaper]{exam}

% answer key
% \documentclass[letterpaper, landscape]{exam}
% \usepackage{2in1, lscape} 
% \printanswers{}

\usepackage{units} 
\usepackage{xfrac} 
\usepackage[fleqn]{amsmath}
\usepackage{commath}
\usepackage{cancel}
\usepackage{float}
\usepackage{mdwlist}
\usepackage{booktabs}
\usepackage{cancel}
\usepackage{polynom}
\usepackage{caption}
\usepackage{fullpage}
\usepackage{comment}
\usepackage{enumerate}
\usepackage{graphicx}
\usepackage{mathtools} 
\usepackage{parskip} 

\newcommand{\degree}{\ensuremath{^\circ}} 
\everymath{\displaystyle}

\title{Algebra Homework 10 \\ Section 3.7}
\author{}
\date{\today}

\begin{document}

  \maketitle

  \section{Homework}

  Section 3.7: 1--5, 11--15, 21--25, 31--35, 41--45, 51--52, 55, 59, 65

  \section{Extra Credit}

  An airplane flies in a straight line from airport A to airport B, then back
  in a straight line from B to A.  It travels with a constant engine speed and
  there is no wind.  Will its travel time for the same round trip be greater,
  less, or the same if, throughout both flights, at the same engine speed, a
  constant wind blows from A to B\@?  Explain.

  \begin{solution}
    The flight with the wind blowing takes longer.  There are several ways to
    look at it.

    The plane is slowed down from A to B and is sped up from B to A.  The
    distance is the same in both cases, but the trip from A to B takes longer,
    since the plane is going slower.  So the average speed for the round trip
    is longer because the plane spends a longer time going slower than it
    spends going faster.

    One idea for problems like this is to plug in some extreme numbers and see
    what happens.  This will sometimes give a hint towards the solution.
    Suppose the distance between the cities is 100 miles, the plane travels 100
    mph without wind, and the wind is blowing at 99 mph.  Without wind, the
    round trip takes 2 hours.  With the wind, the plane only goes 1 mph from A
    to B, so the trip from A to B takes 100 hours.  On the way back, the plane
    goes 199 mph, so the return trip takes slightly more than 1/2 hour.  So for
    this example, the total time for the trip with the wind would be about
    100.5 hours.

    More generally, let $r$ be the rate without the wind, $w$ be the wind
    speed, and $d$ be the distance between the two cities.  

    Using $t = \frac{d}{r}$:
    \begin{itemize*}
    \item Without wind, the trip takes: $t_{no wind} = \frac{2d}{r}$
    \item With wind, the trip takes: $t_{wind} = \frac{d}{r - w} + \frac{d}{r + w}$
    \end{itemize*}

    As $w$ gets close to $r$, the first term gets close to $\infty$ and the
    second term gets close to $\frac{d}{2r}$.  Of course, $w$ can never be
    greater than or equal to $r$ or the plane would never be able to get from A
    to B.

  \end{solution}

  \ifprintanswers{}
    \section{Section 3.7} % (fold)
    
    \begin{description}
      \item[11] $n = \cbr{ -12, -13 }$.

      \item[12] $n = \cbr{ 8, 16 }$.

      \item[13] $t = \cbr{ -5, \frac{1}{3} }$.

      \item[14] $t = \cbr{ - \frac{5}{4}, 6 }$.

      \item[14] $t = \cbr{ - \frac{5}{4}, 6 }$.

      \item[15] $x = \cbr{ - \frac{2}{3}, - \frac{7}{2} }$.

      \item[51] $x = \cbr{ -1, \frac{5}{3} }$.

      \item[52] $x = \cbr{-6, 4}$.

      \item[55]
        \begin{align*}
          x(x + 1)       & = 72 \\
          x^2 + x - 72   & = 0 \\
          (x + 9)(x - 8) & = 0 \\
        \end{align*}

        The solution set is: $\{-9, 8\}$.  Since the question doesn't say the
        integers have to be positive, there are two pairs of integers that work:
        $\{-9, -8\}$ and $\{8, 9\}$.

    \end{description}

  \fi
  \ifprintanswers{}
  \else
    \vspace{9 cm}
    \begin{quote}
      \begin{em}
        I'm for truth, no matter who tells it. I'm for justice, no matter who it's for or against.
      \end{em}
    \end{quote}
    \hspace{2 cm}--Malcolm X
  \fi

\end{document}

