\documentclass[letterpaper, landscape]{exam}
\usepackage{2in1, lscape} 
\printanswers{}

\usepackage{units} 
\usepackage{xfrac} 
\usepackage[fleqn]{amsmath}
\usepackage{commath}
\usepackage{float}
\usepackage{mdwlist}
\usepackage{booktabs}
\usepackage{cancel}
\usepackage{polynom}
\usepackage{caption}
\usepackage{fullpage}
\usepackage{comment}
\usepackage{enumerate}
\usepackage{graphicx}
\usepackage{parskip}
\usepackage{mathtools} 

\everymath{\displaystyle}

\title{Algebra (Math Prep II) Syllabus}
\date{\today}

\begin{document}

    \maketitle

    \section{Introduction}
    This course will cover chapters 2--7 of {\em Intermediate Algebra}, by Kaufmann and Schwitters.  

    Algebra is about taking numbers and facts about them and using them to find other numbers.  For
    example, you can use algebra to solve these problems:

    \begin{itemize}
        \item If Alan Iverson made 10 baskets and 8 free throws and scored 31 points, how many of
            his baskets were three pointers?

        \item If the Citibank executives' bonuses were 1\% of the Citibank federal bailout, there
            are 300,000,000 people in the United States, and everyone in the country contributed \$2
            to a Citibank executive's bonus, what was the total federal bailout for Citibank?

        \item If Carlos Martinez hits a baseball at a 45 degree angle which travels 300 feet before
            bouncing off of Jose Canseco's head and over the fence, how fast was it traveling when
            it left the bat?

    \end{itemize}

    \newpage

    \section{Homework and Tests}

    You should expect to spend one or two hours each week doing homework.

    Math is like learning piano, basketball, or bicycle mechanics.  Watching someone else do it or
    reading about it in a book is helpful.  But you can't actually learn how to do it yourself
    unless you've practiced on your own.

    Each chapter will be followed by an in-class test to make sure that everyone is keeping up.

    \section{Course Overview}

    Here's what we will cover in this course.  Don't worry if some of the terms are unfamiliar or
    the equations look complicated.  Mathematicians like to use imposing terms for simple concepts,
    as it helps their job security.  

    \subsection{Chapter Two---First Degree Equations and Inequalities}

    First degree equations are equations where the variable has an exponent of one.  We'll talk
    about how to solve equations and inequalities like

    \begin{itemize}
        \item $x + 12 = 17 $
        \item $\frac{x}{2} + \frac{x}{3} = 12$
        \item $\abs{2x - 1} < 5$
    \end{itemize}

    \newpage

    \subsection{Chapter Three---Polynomials and Factoring}

    A polynomial is an equation which includes more than one variable or one variable with different
    exponents.  
    
    Examples of polynomials are:
    \begin{itemize}
        \item $x^2 + 2x -5 = 0$
        \item $x^2 - y^2 = 10 $
    \end{itemize}

    We'll talk about adding, subtracting, and multiplying polynomials.

    We'll also talk about factoring polynomials.  When you factor a polynomial, you take a large and
    complicated polynomial and turn it into several simpler polynomials.  

    \subsection{Chapter Four---Rational Expressions}

    Rational expressions are equations which include a ratio of two numbers.  Examples of rational
    expressions are:

    \begin{itemize}
        \item $\frac{x - 2}{x + 3}$
        \item $\frac{x - 2}{3} + \frac{x + 1}{4} = \frac{1}{6}$
    \end{itemize}

    \subsection{Chapter Five---Exponents and Radicals}

    Exponents and radicals are two different notations for the same idea.  This chapter discusses
    both notations.  We'll discuss how to work with and simplify expressions like:

    \begin{itemize}
        \item $x^3y^2$
        \item $\del{ x^{1/3} y^{1/2} }^2$
    \end{itemize}

    \subsection{Chapter Six---Quadratic Equations and Inequalities}

    Quadratic equations are equations that include $x^2$.  A quadratic equation always has two
    solutions, but one or both of the solutions may be a complex number, so we'll talk about complex
    numbers.  Then we'll talk about more techniques for solving quadratic equations.

    \subsection{Chapter Seven---Linear Equations and Inequalities in Two Variables}

    In this chapter we'll go back to talking about linear equations, but we'll talk about equations
    with two variables.  
    
    The topics include:
    \begin{itemize*}
        \item linear equations and inequalities in two variables
        \item graphing linear equations
        \item distance and slope
        \item determining the equation of a line
    \end{itemize*}

\end{document}

