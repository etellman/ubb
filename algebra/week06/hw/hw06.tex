% no answer key
\documentclass[letterpaper]{exam}

% answer key
% \documentclass[letterpaper, landscape]{exam}
% \usepackage{2in1, lscape} 
% \printanswers{}

\usepackage{units} 
\usepackage{xfrac} 
\usepackage[fleqn]{amsmath}
\usepackage{commath}
\usepackage{cancel}
\usepackage{float}
\usepackage{mdwlist}
\usepackage{booktabs}
\usepackage{cancel}
\usepackage{polynom}
\usepackage{caption}
\usepackage{fullpage}
\usepackage{comment}
\usepackage{enumerate}
\usepackage{graphicx}
\usepackage{mathtools} 
\usepackage{parskip} 

\newcommand{\degree}{\ensuremath{^\circ}} 
\everymath{\displaystyle}

\title{Algebra \\ Homework Six \\ Section 2.7 }
\author{}
\date{\today}


\begin{document}

  \maketitle

  \section{Calendar}

  \begin{itemize*}
    \item March 11: Chapter 2 Review
    \item March 18: Chapter 2 Exam
  \end{itemize*}

  \section{Section 2.7}
  1--20, 25--29, 35--39, 45--50, 61--63

  \ifprintanswers{}

    \section{Section 2.7 Solutions} 

   \begin{description}

      \item[1] $ ( -5, 5 ) $

      \item[2] $ ( -1, 1 ) $

      \item[3] $ [ -2, 2 ] $

      \item[4] $ [ -4, 4 ] $

      \item[5] $ ( -\infty, -2 ) \cup ( 2, \infty ) $

      \item[6] $ ( -\infty, -3 ) \cup ( 3, \infty ) $

      \item[7] $ ( -1, 3 ) $

      \item[8] $ ( -2, 6 ) $

      \item[9] $ [ -6, 2 ] $

      \item[10] $ [ -2, 0 ] $

      \item[11] $ ( -\infty, -3 ) \cup ( -1, \infty ) $

      \item[12] $ ( -\infty, -4 ) \cup ( 2, \infty ) $

      \item[13] $ \intoc{ -\infty, 1 } \cup \intoc{ 5, \infty } $

      \item[14] $ \intoc{ -\infty, 1 } \cup \intoc{ 3, \infty } $

      \item[15] $ \cbr{ -7, 9 } $

      \item[16] $ \cbr{ -11, 7 } $

      \item[17] $ \intoc{ -\infty, -4 } \cup \intoc{ 8, \infty } $

      \item[18] $ \intoc{ -\infty, -6 } \cup \intoc{ 12, \infty } $

      \item[19] $ \intoo{ -8, 2 } $

      \item[20] $ \intoo{ -9, 7 } $

      \item[25] $ \intoc{ -\infty, - \frac{7}{2} } \cup \intoc{ \frac{5}{2}, \infty } $

      \item[26] $ \intoc{ -\infty, - \frac{8}{5} } \cup \intoc{ \frac{12}{5}, \infty } $

      \item[27] $ \cbr{ -5, \frac{7}{3} } $

      \item[28] $ \cbr{ - \frac{7}{5}, \frac{21}{5} } $

      \item[29] $ \cbr{ -1, 5 } $

      \item[35] $ \intcc{ -5, \frac{7}{5} } $

      \item[36] $ \intoc{ -\infty, - \frac{16}{7} } \cup \intoc{ 4, \infty } $

      \item[37] $ \cbr{ \frac{1}{12}, \frac{17}{12} } $

      \item[38] $ \cbr{ - \frac{11}{10}, \frac{1}{10} } $

      \item[39] $ \intcc{ -3, 10 } $

      \item[45] $ \cbr{ 0, 3 } $

      \item[46] $ \cbr{ -3, \frac{11}{3} } $

      \item[47] $ \intoc{ -\infty, -14 } \cup \intoc{ 0, \infty } $

      \item[48] $ \intoc{ -\infty, -4 } \cup \intoc{ 8, \infty } $

      \item[49] $ \intcc{ -2, 3 } $

      \item[50] $ \intcc{ - \frac{5}{2}, 1 } $

      \item[61] The smallest value for the absolute value of anything is 0, so
        this problem is equivalent to:

        \begin{align*}
          2x + 5 & = 0 \\
          x      & = - \frac{5}{2} \\
        \end{align*}

      \item[62] This is the same as problem 61. You just have to solve $x - 2 = 0$.

      \item[63] You can remove the absolute value symbol and get an equivalent equation:
        \begin{align*}
          2x - 3 & = 0 \\
          x      & = \frac{3}{2} \\
        \end{align*}

    \end{description}
  \fi

  \ifprintanswers{}
  \else
    \vspace{7 cm}

    \begin{verse}
      seeker of truth

      follow no path \\
      all paths lead where

      truth is here 
    \end{verse}
    \hspace{1 in} e.e.\ cummings

  \fi

\end{document}


