% no answer key
% \documentclass[letterpaper]{exam}

% answer key
\documentclass[letterpaper, landscape]{exam}
\usepackage{2in1, lscape} 
\printanswers{}

\usepackage{units} 
\usepackage{xfrac} 
\usepackage[fleqn]{amsmath}
\usepackage{commath}
\usepackage{cancel}
\usepackage{float}
\usepackage{mdwlist}
\usepackage{booktabs}
\usepackage{cancel}
\usepackage{polynom}
\usepackage{caption}
\usepackage{fullpage}
\usepackage{comment}
\usepackage{enumerate}
\usepackage{graphicx}
\usepackage{mathtools} 
\usepackage{parskip} 

\newcommand{\degree}{\ensuremath{^\circ}} 
\everymath{\displaystyle}

\title{Algebra Homework 10 \\ Section 3.5 and 3.6}
\author{}
\date{\today}

\begin{document}

  \maketitle

  \section{Homework}

  \begin{itemize*}
    \item Section 3.5: 1--5, 11--15, 31--40, 45--49, 61--65
    \item Section 3.6: 1--5, 11--15, 21--25, 31--35, 41--45, 51--55, 70--79
  \end{itemize*}

  \section{Extra Credit}

  A fisherman wearing a large straw hat was fishing from a rowboat in a river
  that flowed at a speed of three miles an hour.  His boat drifted down the
  river at the same rate.

  ``I think I'll row upstream a few miles'', he said to himself.  ``The fish
  don't seem to be biting here.''

  Just as he started to row, the wind blew off his hat and it fell into the
  water beside the boat.  But the fisherman did not notice his hat was gone
  until he had rowed upstream and was five miles away from his hat.  Then he
  realized what must have happened, so he immediately started rowing back
  downstream again until he came to his floating hat.

  In still water, the fisherman's rowing speed is always five miles an hour.
  When he rowed upstream and back, he rowed at this same constant speed.  But
  of course this would not be his speed relative to the {\em shore\/} of the
  river.  For instance, when he rowed upstream at five miles an hour, the river
  would be carrying him downstream at three miles an hour, so he would be
  passing objects on the shore at only two miles an hour.  And when he rowed
  downstream, his rowing speed and the speed of the river would combine to make
  his speed eight miles an hour with respect to the shore.

  If the fisherman lost his hat at 2:00 in the afternoon, what time was it when
  he recovered it?

  (Martin Gardner)

  \begin{solution}
    The speed of the river doesn't matter in this problem.  When the
    fisherman is rowing, he is always moving at 5 mph relative to his hat.

    \begin{itemize} 
      \item when going upstream, the fisherman travels at 2 mph upstream while
        the hat travels at 3 mph downstream, for a total difference of 5 mph.  

      \item when going downstream, the fisherman travels at 8 mph downstream
        while the hat travels at 3 mph downstream, for a total difference of 5
        mph.  
    \end{itemize}

    It takes the fisherman one hour to get five miles away from his hat and
    one hour to row back to his hat and he recovers his hat at 4:00, two hours
    after he lost it.

    A slightly different way to look at it is to just ignore the speed of the
    river and do all the calculations relative to the river instead of relative
    to the shore.  With this approach, the hat is always stationary (relative to
    the river) and the fisherman is always moving at 5 mph (relative to the
    river).  So the fisherman rows at 5 mph for an hour to get away from the hat
    and then at 5 mph for an hour to get back.

  \end{solution}

  \ifprintanswers{}
    \section{Section 3.5} % (fold)
    
    \begin{description}
      \item[1] $(x + 1)(x - 1)$

      \item[2] $(x + 3)(x - 3)$

      \item[3] $(4x + 5)(4x - 5)$

      \item[4] $(2x + 7)(2x - 7)$

      \item[5] $\del{3x + 5y} \del{3x - 5y}$

      \item[11] $\del{1 + 12n} \del{1 - 12n}$

      \item[12] $\del{5 + 7n} \del{5 - 7n}$

      \item[13] $\del{x + 2 + y} \del{x + 2 - y}$

      \item[14] $\del{3x + 5 + y} \del{3x + 5 - y}$

      \item[15] 
        \begin{align*}
          4x^2 - \del{y + 1}^2 & = (2x + (y + 1))(2x - (y + 1)) \\
                               & = \boxed{ (2x + y + 1) \del{2x - y - 1} } \\
        \end{align*}

      \item[31] $\del{n^2 + 9} (n + 3)(n - 3)$

      \item[32] can't be factored.

      \item[33] $3x(x^2 + 9)$

      \item[34] $5x(4x^2 + 9)$

      \item[35] 
        \begin{align*}
          4x^3y - 64xy^3 & = 4xy \del{ x^2 - 16y^2 } \\
                         & = \boxed{ 4xy \del{x + 4y} \del{x - 4y} } \\
        \end{align*}

      \item[36] 
        \begin{align*}
          12x^3 - 27xy^2 & = 3x \del{ 4x^2 - 9y^2 } \\
                         & = \boxed{ 3x \del{2x + 3y} \del{2x - 3y} } \\
        \end{align*}

      \item[37] 
        \begin{align*}
          6x - 6x^3 & = 6x(1 - x^2) \\
                    & = \boxed{ 6x \del{1 + x} \del{1 - x} } \\
        \end{align*}

      \item[38] $\del{1 + 4x^2} (1 - 2x)(1 + 2x)$

      \item[39] $\del{1 + x^2y^2} (1 - xy)(1 + xy)$

      \item[40] 
        \begin{align*}
          20x - 5x^3 & = 5x \del{ 4 - x^2 } \\
                     & = \boxed{ 5x \del{2 + x} \del{2 - x} } \\
        \end{align*}

      \item[45] $\del{a - 4} \del{a^2 + 4a + 16}$

      \item[46] $\del{a - 3} \del{a^2 + 3a + 9}$

      \item[47] $\del{x + 1} \del{x^2 - x + 1}$

      \item[48] $\del{x + 2} \del{x^2 - 2x + 4}$

      \item[49] $\del{3x + 4y} \del{9x^2 - 12xy + 16y^2}$

      \item[61] $x = \cbr{ -2, 2 }$
        
      \item[62] $x = \cbr{ -6, 6 }$

      \item[63] $x = \cbr{ 0, -1, 1 }$

      \item[64] $x = \cbr{ 0, -4, 4 }$

      \item[65] $x = \cbr{ 0, -2, 2 }$

    \end{description}

    \section{Section 3.6} % (fold)
    
    \begin{description}
      \item[1] $(x + 4)(x + 5)$

      \item[2] $(x + 8)(x + 3)$

      \item[3] $(x - 4)(x - 7)$

      \item[4] $(x - 6)(x - 2)$

      \item[5] $(a + 9)(a - 4)$

      \item[11] not factorable

      \item[12] $(5 - x)(x + 7)$

      \item[13] $(6 - x)(1 + x)$

      \item[14] not factorable

      \item[15] $(x + 3y)(x + 12y)$

      \item[21] $(4x - 3)(3x + 2)$

      \item[22] $(4x - 3)(5x + 1)$

      \item[23] $(4a - 9)(a + 3)$

      \item[24] $(2a - 1)(6a + 5)$

      \item[25] $(3n + 5)(n - 4)$

      \item[31] $(4x - 5)(2x + 9)$

      \item[32] $(3x + 11)(2x - 3)$

      \item[33] $(1 - 6x)(x + 6)$

      \item[34] $(2 -5x)(3x + 2)$

      \item[35] $(4y - 1)(5y + 9)$

      \item[41] $(x + 10)(x + 15)$

      \item[42] $(x + 9)(x + 12)$

      \item[43] $(n - 16)(n - 20)$

      \item[44] $(n - 14)(n - 12)$

      \item[45] $(t - 12)(t + 15)$

      \item[51] $(x^2 - 8)(x + 1)(x - 1)$

      \item[52] $(x^2 + 3)(x + 2)(x - 2)$

      \item[53] $(3n + 1)(3n - 1)(2n^2 + 3)$

      \item[54] $(2n + 3)(2n - 3)(n^2 + 3)$

      \item[55] $(x + 1)(x - 1)(x + 4)(x - 4)$

      \item[70] $(5x - 3)(x + 9)$

      \item[71] $(x + y - 7)(x - y + 7)$

      \item[72] $2n(n^2 + 3n + 5)$

      \item[73] $(1 - 2x)(1 + 2x)(4x^2 + 1)$

      \item[74] $\del{ 3a - 5 }^2$

      \item[75] $(4n + 9)(n + 4)$

      \item[76] $x(x + 3)(x - 3)$

      \item[77] $n(n + 7)(n - 7)$

      \item[78] $4 \del{x^2 + 4}$

      \item[79] $(x - 8)(x + 1)$

    \end{description}

  \fi
  \ifprintanswers{}
  \else
    \vspace{2 cm}
    \begin{quote}
      \begin{em}
        When it shall be said in any country in the world, my poor are happy;
        neither ignorance nor distress is to be found among them; my jails are
        empty of prisoners, my streets of beggars; the aged are not in want,
        the taxes are not oppressive; the rational world is my friend, because
        I am a friend of its happiness: When these things can be said, then may
        the country boast of its constitution and its government. 
      \end{em}
    \end{quote}
    \hspace{2 cm}--Thomas Paine
  \fi

\end{document}

