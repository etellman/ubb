
\documentclass[fleqn,addpoints]{exam}

\usepackage{amsmath}
\usepackage{graphics}
\usepackage{cancel}
\usepackage{polynom}
\usepackage{2in1, lscape}

\title{Math 113 Chapter Six Study Guide}
\author{}
\date{June 23, 2010}

\begin{document}

\maketitle

\section{Section 6.1--Complex Numbers}

\subsection{Notes}

\begin{itemize}
\item
Always convert square roots of negative numbers to imaginary numbers before doing anything else.  

For example:
\[
  \sqrt{-4} \sqrt{-12} = i\sqrt{4} \cdot i\sqrt{12} = 2i^2 \sqrt{4 \cdot 3} = 4i^2 \sqrt{3} = -4 \sqrt{3}
\]

If you start out by multiplying $-4$ and $-12$, you'll end up with a positive answer instead of a negative one.

\item
Use the standard FOIL rules when multiplying two complex numbers.
\[
  (2 + 5i)(4 + 3i) = 8 + 6i + 20i + 15i^2 = 8 + 26i - 15 = -7 + 26i
\]

\item
  To convert a fraction with an imaginary number in the denominator to standard form, multiply top and bottom by the
  {\em complex conjugate} (the same number with the sign of the imaginary part reversed) of the denominator.

\begin{align*}
  \frac{2+7i}{3-2i} &= \frac{2+7i}{3-2i} \left( \frac{3+2i}{3+2i} \right) \\
  &= \frac{6 + 4i + 21i + 14i^2}{9 - 4i^2} \\
  &= \frac{6 + 25i - 14}{9 + 4} \\
  &= \frac{-8 + 25i}{13} \\
  &= - \frac{8}{13} + \frac{25}{13}i \\
\end{align*}

\end{itemize}

\subsection{Examples}
\begin{itemize}
  \item $(5+2i) - (1 - 3i)$
  \item $(\dfrac{2}{3}-\dfrac{1}{2}i) + (\dfrac{3}{5} + \dfrac{1}{3} i)$
  \item $\sqrt{-40}$
  \item $\sqrt{-4} \sqrt{-9}$
  \item $(7+3i) (5 - 4i)$
  \item $\dfrac{-3+i}{10-2i}$
\end{itemize}

\pagebreak

\section{Section 6.2-6.5--Quadratic Equations}

\subsection{Notes}
\begin{itemize}
\item
You can solve equations where the left hand side is something squared by taking the square root of both sides.  Remember
to include both the positive and negative square roots in the solution.  For example:
\begin{align*}
  (x+2)^2 = 9 \\
  x+2 = \pm \sqrt{9} \\
  x+2 = \pm 3 \\
  x = -2 \pm 3 \\
  x = \{-5, 1\} \\
\end{align*}

Make sure the entire left side is a square.  For example, with an equation like: \((x+2)^2  + 4 = 17 \), you would
have to move the $4$ over to the right side before taking the square root.

\item
You can convert any equation to a form which is solvable using the above approach by ``completing the square.''  The steps are:
\begin{enumerate}
\item write the equation as: $x^2 + bx = -c$
\item add $\dfrac{b^2}{4}$ (or, equivalently, $\left( \dfrac{b}{2} \right)^2$) to both sides
\item factor the left side as: $\left(x + \dfrac{b}{2} \right)^2$
\item solve by taking the square root of both sides
\end{enumerate}

\item You can solve any equation using the quadratic formula.  The steps are:
\begin{enumerate}
\item write the equation as: $ax^2 + bx + c = 0$
\item make a note of the values for $a$, $b$, and $c$
\item plug $a$, $b$, and $c$ into the quadratic formula, which is:
\[
  x = \frac{-b \pm \sqrt{b^2 - 4ac}}{2a}
\]
\end{enumerate}

\item
You can check the solutions to any quadratic equation by adding and multiplying them together.  
\begin{itemize}
  \item the sum of the solutions should equal $- \dfrac{b}{a}$ 
  \item the product of the solutions should equal $\dfrac{c}{a}$.
\end{itemize}

\item
You can tell how many solutions to expect using the {\em discriminant}.  The discriminant is $b^2 - 4ac$.  The
possibilities are:

\begin{description}
  \item[negative] two non-real solutions
  \item[zero] one real solution
  \item[positive] two real solutions
\end{description}

\item
Usually there is more than one way to solve a particular quadratic equation.  Unless the instructions say to use a
particular method, you should use whatever approach is easiest for you for that equation.

\item
There are some helpful suggestions for solving word problems on page 313 of the textbook.

\end{itemize}

\subsection{Examples}
\begin{itemize}
  \item $x^2 = -9$
  \item $(x+2)^2 = 8$
  \item $3(x+7)^2 -4 = 17$
  \item solve $x^2+4x-2=0$ by completing the square
  \item solve $x^2-7x+1=0$ using the quadratic formula
  \item solve $x^2-6x+8=0$ using whatever approach seems most appropriate
  \item word problems like the ones on pages 318-320
\end{itemize}

\pagebreak

\section{Section 6.6--Inequalities}

\subsection{Notes}
To solve quadratic inequalities, you need to check all the points where the sign of the result might change.  This
happens whenever the value of the equation is undefined or zero.  

The procedure is:

\begin{enumerate}
\item find the points where the equation is zero or undefined.  These points divide the domain into different
  regions where the sign of the result might change
\item try a value from each region and see what the sign of the result is
\item write the answer as a solution set
\end{enumerate}

\subsection{Examples}
\begin{itemize}
  \item $(x+2)(x-1) \geq 0$
  \item $(x+1)(x-1)(x-3) < 0$
  \item $\dfrac{-x+2}{x-1} \leq 0$
\end{itemize}

\end{document}