
\documentclass[fleqn,addpoints]{exam}

\usepackage{amsmath}
\usepackage{mdwlist}
\usepackage{graphicx}
\usepackage{cancel}
% \usepackage{polynom}
\usepackage{float}

\printanswers

\ifprintanswers
\usepackage{2in1, lscape}
\fi

\title{Math 113 Homework 21}
\author{}
\date{\today}

\begin{document}

\maketitle

\section{From the Book}

Read section 10.3.

\begin{itemize*}
  \item pp. 516-517: 5-11, 27
\end{itemize*}

\ifprintanswers
\section{Pages 516-517}
\begin{description}
\item[5]
\begin{align*}
  2x - y + z &= 0 \\
  3x - 2y + 4z &= 11 \\
  5x + y - 6z &= -32 \\
\\
  2x - y + z &= 0 \\
  3x - 2y + 4z &= 11 \\
  7x  - 5z &= -32 \\
\\
  2x - y + z &= 0 \\
  -x + 2z &= 11 \\
  7x  - 5z &= -32 \\
\\
  2x - y + z &= 0 \\
  -x + 2z &= 11 \\
  9z &= 45 \\ 
\\
  2x - y + z &= 0 \\
  -x + 2z &= 11 \\
  z &= 5 \\ 
\end{align*}

\framebox{$(-1, 3, 5)$}

\item[6]
\begin{align*}
  x-2y+3z &= 7 \\
  2x+y+5z &= 17 \\
  3x-4y-2z &= 1 \\
\\
  x-2y+3z &= 7 \\
  5y - z &= 3 \\
  2y - 11z &= -20 \\
\\
  x-2y+3z &= 7 \\
  -10y + 2z &= 6 \\
  10y - 55z &= -100 \\
\\
  x-2y+3z &= 7 \\
  5y - z &= 3 \\
   -53z &= -106 \\
\\
\\
  x-2y+3z &= 7 \\
  5y - z &= 3 \\
   z &= 2 \\
\\
\end{align*}

\framebox{(3, 1, 2)}

\item[7]
\begin{align*}
  4x-y+z &= 5 \\
  3x+y+2z &= 4 \\
  x-2y-z &= 1 \\
\\
  4x-y+z &= 5 \\
  7x + 3z &= 9 \\
  -7x - 3z &= -9 \\
\\
  4x-y+z &= 5 \\
  7x + 3z &= 9 \\
  0 &= 0 \\
\\
\end{align*}


\framebox{infinitely many solutions}

\item[8]
\begin{align*}
  2x - y + 3z &= -14 \\
  4x + 2y - z &= 12 \\
  6x - 3y + 4z &= -22 \\
\\
  2x - y + 3z &= -14 \\
  4y - z &= 40 \\
  -5z &= 20 \\
\\
  2x - y + 3z &= -14 \\
  4y - z &= 40 \\
  z &= -4 \\
\\
\end{align*}

\framebox{$ \left(\dfrac{1}{2}, 3, -4 \right) $}

\item[9]
\begin{align*}
  x-y+2z &= 4 \\
  2x-2y+4z &= 7 \\
  3x - 3y + 4z &= 1 \\
\\
  x - y + 2z &= 4 \\
  0 &= -1 \\
\end{align*}

\framebox{no solution}

\item[10]
\begin{align*}
  x + y - z &= 2 \\
  3x - 4y + 2z &= 5 \\
  2x + 2y - 2z &= 7 \\
\\
  x + y - z &= 2 \\
  -7y + 5z &= -1 \\
  0 &= -3 \\
\end{align*}

\framebox{no solution}

\item[11]
\begin{align*}
  x-2y+z &= -4 \\
  2x+4y-3z &= -1 \\
  -3x-6y+7z &= 4 \\
\\
  x-2y+z &= -4 \\
  8y-5z &= 7 \\
  -12y+10z &= -8 \\
\\
  x-2y+z &= -4 \\
  8y-5z &= 7 \\
  -6y+5z &= -4 \\
\\
  x-2y+z &= -4 \\
  8y-5z &= 7 \\
  2y &= 3 \\
\\
  x-2y+z &= -4 \\
  8y-5z &= 7 \\
  y &= \frac{3}{2} \\
\\
\end{align*}

\framebox{ $\left(-2, \dfrac{3}{2}, 1 \right)$ }

\pagebreak

\item[27]
Let $x$, $y$, and $z$ be the small, medium, and large angles, respectively.

The equations are:
\begin{align*}
  x + y + z &= 180 \\
  z &= 2x \\
  x + z &= 2y \\
\end{align*}

Solve
\begin{align*}
  x + y + z &= 180 \\
  -2x + z &= 0 \\
  x -2y + z &= 0 \\
\\
  x + y + z &= 180 \\
  -2x + z &= 0 \\
  3x + 3z &= 360 \\
\\
  x + y + z &= 180 \\
  -2x + z &= 0 \\
  x + z &= 120 \\
\\
  x + y + z &= 180 \\
  -2x + z &= 0 \\
  3z &= 240 \\
\\
  x + y + z &= 180 \\
  -2x + z &= 0 \\
  z &= 80 \\
\\
\end{align*}

So the angles are 40, 60, and 80 degrees.

\end{description}

\section{Cumulative Review Problems}

\begin{questions}

\question 
Evaluate: $(2^{-2} + 3^{-1})^{-2}$

\begin{solution}
\[
  (2^{-2} + 3^{-1})^{-2} = \left( \frac{1}{4} + \frac{1}{3} \right)^{-2} = \left( \frac{7}{12} \right)^{-2} = \frac{144}{49}
\]
\end{solution}

\question 
Evaluate: $(2^{-2} \cdot 3^{-1})^{-2}$
\begin{solution}
\[
  (2^{-2} \cdot 3^{-1})^{-2} = 2^4 \cdot 3^2 = 144
\]
\end{solution}

\question 
Solve for $x$: $x^2-11x+28 = 0$
\begin{solution}
\begin{align*}
  x^2-11x+28 &= 0 \\
  (x-7)(x-4) &= 0 
\end{align*}

$x = \{4, 7\}$

\end{solution}

\pagebreak

\question 
Solve for $x$: $4x^2+4x+1 = 0$
\begin{solution}

\begin{align*}
  4x^2+4x+1 &= 0 \\
  (2x+1)(2x+1) &= 0 \\
\end{align*}

$x = - \dfrac{1}{2}$

\end{solution}

\question 
Solve for $x$: $x^2+6x-3 = 0$
\begin{solution}

Using the quadratic formlula:
\[
  x = \frac{-6 \pm \sqrt{36 - 4(-3)}}{2} = -3 \pm 2\sqrt{3}
\]
\end{solution}

% \question 
% Solve for $x$: $x^2+6x+4 = 0$
% \begin{solution}
% \end{solution}

\question 
Solve for $x$: $-4x^2-6x+3 = 0$
\begin{solution}

Using the quadratic formula:
\[
  x = \frac{-6 \pm \sqrt{36 - 4 (4) (-3)}}{8} = \frac{-3 \pm \sqrt{21}}{4}
\]

\end{solution}

\pagebreak

\question 
Solve for $x$: $2x(x-18) + 3(x-18) = 0$
\begin{solution}
\begin{align*}
  2x(x-18) + 3(x-18) &= 0 \\
  (2x+3)(x-18) &= 0
\end{align*}

$x = \left\{ -\dfrac{3}{2}, 18 \right\}$

\end{solution}

\question 
Solve for $x$: $7x(x+2) + 5 = 3x(x+1)$
\begin{solution}

\begin{align*}
  7x(x+2) + 5 &= 3x(x+1) \\
  7x^2 + 14x + 5 &= 3x^2 + 3x \\
  4x^2 + 11x + 5 &= 0 
\end{align*}

using the quadratic formula:
\[
  x = \frac{-11 \pm \sqrt{121 - 4 \cdot 4 \cdot 5}}{8} = \frac{-11 \pm \sqrt{41}}{8}
\]

\end{solution}

\question 
Solve the inequality: $x^2+4x > 0$
\begin{solution}
The interesting values for $x$ are $0$ and $-4$.  With test points, we find the solution: $(-\infty, 4) \cup (0, \infty)$
\end{solution}

\question 
Solve the inequality: $x^2-3x-10 \geq 0$
\begin{solution}
The interesting values for $x$ are $5$ and $-2$.  With test points, we find the solution: $(-\infty, -2] \cup ([5, \infty)$
\end{solution}

\question 
Solve the inequality: $\dfrac{x+2}{4x+6} \leq 0$
\begin{solution}
The interesting values for $x$ are $-2$ and $-\dfrac{3}{2}$.  With test points, we find the solution: $\left[ -2, -\dfrac{3}{2} \right]$
\end{solution}


\end{questions}

\fi

\ifprintanswers
\else
\vspace{10 cm}

\begin{em}
I were asked to answer the following question: What is slavery? and I should answer in one word, It is murder!, my
meaning would be understood at once. No extended argument would be required \ldots Why, then, to this other question:
What is property? may I not likewise answer, It is robbery!, without the certainty of being misunderstood; the second
proposition being no other than a transformation of the first? 
\end{em}

\vspace{0.1 in}
\hspace{0.5 in} -- Pierre-Joseph Proudhon 

\begin{em}
\end{em}

\vspace{0.1 in}
\hspace{0.5 in} 

\fi

\end{document}

