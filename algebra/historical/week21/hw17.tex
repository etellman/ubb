
\documentclass[fleqn,addpoints]{exam}

\usepackage{amsmath}
\usepackage{graphics}
\usepackage{cancel}
\usepackage{polynom}
\usepackage{mdwlist}

\printanswers

\ifprintanswers
\usepackage{2in1, lscape}
\fi

\title{Math 113 Homework 17}
\author{}
\date{\today}

\begin{document}

\maketitle

%% \ifprintanswers
%% \else
%% \section{Chapter 6 Exam}

%% The {\em Chapter 6} test will be next week.  

%% For this exam, you may use a single page (both sides) of notes. You can write anything you think might be helpful when
%% doing the exam, as long as it fits on one page.

%% \vspace{.1 in}
%% To prepare for the exam, you should:
%% \begin{itemize*}
%%   \item take a glance at the study guide
%%   \item do some of the problems in the review problem set
%%   \item prepare a single page of notes (optional)
%% \end{itemize*}

%% \fi

\section{From the Book}

\begin{itemize*}
  \item pp. 318-320: 1, 3, 10, 23, 27, 33-34, 46, 49, 52-53, 59, 68 
  \item pp. 326-327: 1-5, 11-12, 19-20, 25-29, 40, 41
\end{itemize*}

% 145 points possible

\ifprintanswers

\section{Pages 318-320}
\begin{description}

\item[1]
\begin{align*}
  x^2-4x-6 &= 0 \\
  x^2-4x &= 6 \\
  x^2-4x + 4 &= 10 \\
  (x-2)^2 &= 10 \\
  x-2 &= \pm \sqrt{10} \\
  x &= 2 \pm \sqrt{10} \\
\end{align*}

\item[3]
\begin{align*}
  3x^2+23x-36 &= 0 \\
  (3x-4)(x+9) &= 0 \\
 x &= \left \{ -9, \frac{4}{3} \right \} \\
\end{align*}

\item[10]
\( 2x^2+x-28 = 0 \)

\begin{align*}
  x &= \frac{-1 \pm \sqrt{1 - 4(2)(-28)}}{4} \\
    &= \frac{-1 \pm \sqrt{169}}{4} \\
    &= \frac{-1 \pm 15}{4} \\
    &= \left \{ -4, \frac{7}{2} \right \} \\
\end{align*}

\item[23]
\begin{align*}
  \frac{3}{x} + \frac{7}{x-1} &= 1 \\
  \frac{3x-3+7x}{x(x-1)} &= 1 \\
  \frac{10x-3}{x^2-x} &= 1 \\
  10x-3 &= x^2-x \\
  x^2 - 11x+3 &= 0 \\
\end{align*}

\begin{align*}
  x &= \frac{11 \pm \sqrt{121-12}}{2} \\
    &= \frac{11 \pm \sqrt{109}}{2} \\
\end{align*}

\item[27]
\begin{align*}
  \frac{3}{x-1} - \frac{2}{x} &= \frac{5}{2} \\
  \frac{3x - 2(x-1)}{x^2-x} &= \frac{5}{2} \\
  3x^2-2x+2 &= \frac{5x^2 - 5x}{2} \\
  2x+4 &= 5x^2-5x \\
  5x^2-7x-4 &= 0 \\
\end{align*}

\begin{align*}
  x &= \frac{7 \pm \sqrt{49 - 4(5)(-4)}}{10} \\
  &= \frac{7 \pm \sqrt{129}}{10} \\
\end{align*}

\item[33]
\begin{align*}
  x^4 - 18x^2 + 72 &= 0 \\
  (x^2-6)(x^2-12) &= 0 \\
  x &= \{ \pm \sqrt{6}, \pm 2\sqrt{3} \\
\end{align*}

\item[34]
\begin{align*}
  x^4 - 21x^2+54 &= 0 \\
  (x^2-18)(x^2-3) &= 0 \\
  x &= \{ \pm 3\sqrt{2}, \pm \sqrt{3} \} \\
\end{align*}

\item[46] 
If we let $x$ and $y$ be the two numbers:
\begin{itemize*}
  \item $x+y = 6$
  \item $xy = 7$
\end{itemize*}

Solving the first equation for $y$ gives us $y = 6-x$.  We can then plug the value for $y$ into the second equation,
giving us an equation to solve:
\begin{align*}
  x(6-x) &= 7 \\
  6x-x^2 &= 7 \\
  x^2-6x &= -7 \\
  x^2-6x +9 &= 2 \\
  (x-3)^2 &= 2 \\
  x-3 &= \pm \sqrt{2} \\
  x = 3 \pm \sqrt{2}
\end{align*}

We can go back and find $y$. 
\begin{itemize*}
  \item For $x = 3 + \sqrt{2}$: \( y = 6 - x = 6 - (3+\sqrt{2}) = 3 - \sqrt{2} \)
  \item For $x = 3 - \sqrt{2}$: \( y = 6 - x = 6 - (3-\sqrt{2}) = 3 + \sqrt{2} \) 
\end{itemize*}

So the two numbers are: $3 + \sqrt{2}$ and $3 - \sqrt{2}$.

check:
\begin{itemize*}
  \item $(3 + \sqrt{2}) + (3 - \sqrt{2}) = 3 + 3 = 6$
  \item $(3 + \sqrt{2})(3 - \sqrt{2}) = 9-2 = 7$
\end{itemize*}

\item[49]
The sides of the triangle are:
\begin{itemize*}
  \item $a$
  \item $b=21-a$
  \item $c=15$
\end{itemize*}

Using $a^2+b^2=c^2$
\begin{align*}
  a^2 + (21-a)^2 &= 15^2 \\
  a^2 + 441 -42a + a^2 &= 225 \\
  2a^2 -42a + 216 &= 0 \\
  a^2 -21a + 108 &= 0 \\
  (a-12)(a-9) &= 0 \\
\end{align*}
$a = \{ 9, 12 \}$

So the three sides are: 
\begin{itemize*}
  \item $a=9$
  \item $b=12$
  \item $c=15$
\end{itemize*}

check: $9^2 + 12^2 = 81 + 144 = 225 = 15^2$

\item[52]

If we let $x$ be the width of the frame, the dimensions of the frame and picture together are:
\begin{itemize*}
  \item $W = 5 + 2x$
  \item $L = 7 + 2x$
  \item $area = LW = (5+2x)(7+2x)$
\end{itemize*}

We can solve the area equation to find the width of the frame:
\begin{align*}
  (2x+5)(2x+7) &= 80 \\
  4x^2 + 24x + 35 &= 80 \\
  4x^2 + 24x &= 45 \\
  x^2 + 6x &= \frac{45}{4} \\
  x^2 + 6x +9 &= 9 + \frac{45}{4} \\
  (x+3)^2 &=  \frac{36+45}{4} \\
  x+3 &=  \pm \sqrt{\frac{81}{4}} \\
  x &= -3 \pm \frac{9}{2} \\
  x &= \frac{-6 \pm 9}{2} \\
  x &= \left \{ - \dfrac{3}{2}, \dfrac{3}{2} \right \}
\end{align*}

Since a negative length doesn't make sense, $x = \dfrac{3}{2}$.

check: \( \left( 5 + 2\left( \dfrac{3}{2} \right) \right) \left(7 + 2\left( \dfrac{3}{2} \right) \right) = (5+3)(7+3) = (8)(10) = 80 \)

\item[53]
Here's what we know:
\begin{itemize*}
  \item The perimeter is 44 inches and there are 4 sides, so: $2L + 2W = 44$
  \item The area is 112 square inches: $LW = 112$
\end{itemize*}

Solving the perimeter equation for $L$ gives: $L = 22-W$.  We can plug the value for $L$ into the area equation and
solve that to get $W$:

\begin{align*}
  W(22-W) &= 112 \\
  22W -W^2 &= 112 \\
  W^2 - 22W &= -112 \\
  W^2 - 22W + 121 &= -112 + 121 \\
  (W-11)^2 &= 9 \\
  W-11 &= \pm 3 \\
  W &= 11 \pm 3 \\
  W &= \{8, 14\} \\ 
\end{align*}

If $W = 8$ then $L = 22 - 8 = 14$ and if $W=14$ then $L = 22-14 = 8$ so the two dimensions are 8 and 14.

check:
\begin{itemize*}
  \item $2 \cdot 8 + 2 \cdot 14 = 16 + 28 = 44$
  \item $8 \cdot 14 = 112$
\end{itemize*}

\item[59]
Here's what we know:
\begin{itemize*}
  \item Terry takes two hours longer to complete one job: \\ $t_{tom} = t_{terry} - 2$
  \item They completed 1 job by working together for 3 hours and Terry working by himself for one hour: \\ $3 \cdot r_{both} + 1 \cdot r_{terry} = 1$
\end{itemize*}

The rate when they both work is: 
\begin{align*}
  r_{both} &= r_{tom} + r_{terry} \\
  &= \frac{1}{t_{tom}} + \frac{1}{t_{terry}} \\
  &= \frac{1}{t_{terry} - 2} + \frac{1}{t_{terry}} 
\end{align*}

We can use this for the rate in the second equation and solve for $t_{terry}$  Equations with $t_{terry}$ are a bit hard to read, so we'll let $x = t_{terry}$ :
\begin{align*}
  3r_{both} + t_{terry} &= 1 \\
  3 \left( \frac{1}{x - 2} + \frac{1}{x} \right ) + \frac{1}{x} &= 1 \\
  3 \left( \frac{x + x - 2}{x(x - 2)} \right ) + \frac{1}{x} &= 1 \\
  3 \left( \frac{2x - 2}{x(x - 2)} \right ) + \frac{1}{x} &= 1 \\
  \frac{6x - 6}{x(x - 2)} + \frac{x}{x(x-2)} &= 1 \\
  7x - 8 &= x(x - 2) \\
  7x - 8 &= x^2 - 2x \\
  x^2 - 9x + 8 &= 0 \\
  (x-1)(x-8) &= 0 \\
  x &= \{ 1, 8 \} \\
\end{align*}

Since $t_{tom} = t_{terry} - 2$, $t_{terry} = 8$ and $t_{tom} = 6$.

\item[68]

The three sides of the right triangle are:
\begin{itemize*}
  \item $16$
  \item $a$
  \item $a+4$
\end{itemize*}

From the Pythagorean theorem: $a^2 + 16^2 = (a+4)^2$

\begin{align*}
  a^2 + 16^2 = (a+4)^2 \\
  a^2 + 256 &= a^2 + 8a + 16 \\
  8a + 16 &= 256 \\
  a + 2 &= 32 \\
  a = 30 \\
\end{align*}

So the sides are 16, 30, and 34 yards.

check: $16^2 + 30^2 = 256 + 900 = 1,156 = 34^2$

\end{description}

\section{Pages 326-327}
\begin{description}

\item[1]
\begin{align*}
  (x+2)(x-1) &> 0 \\
  x &= (-\infty, -2) \cup (1, \infty) \\
\end{align*}

\item[2]
\begin{align*}
  (x-2)(x+3) &> 0 \\
  x &= (-\infty, -3) \cup (2, \infty) \\
\end{align*}

\item[3]
\begin{align*}
  (x+1)(x+4) &< 0 \\
  x &= (-4, -1) \\
\end{align*}

\item[4]
\begin{align*}
  (x-3)(x-1) &< 0 \\
  x &= (1, 3) \\
\end{align*}

\item[5]
\begin{align*}
  (2x-1)(3x+7) & \geq 0 \\
  x &= \left(-\infty, -\frac{7}{3} \right] \cup \left[ \frac{1}{2}, \infty \right) \\
\end{align*}

\item[12]
\begin{align*}
  x(x+3)(x-3) & \leq 0 \\
  x = (-\infty, -3] & \cup [0, 3] \\
\end{align*}

\item[19]
\begin{align*}
  \frac{-x+2}{x-1} & \leq 0 \\
  x &= (-\infty, 1) \cup [2, \infty) \\
\end{align*}

\item[20]
\begin{align*}
  \frac{3-x}{x+4} & \leq 0 \\
  x &= (-\infty, -4) \cup [3, \infty) \\
\end{align*}

\item[25]
\begin{align*}
  3x^2+13x-10 & \leq 0 \\
  (3x-2)(x+5) & \leq 0 \\
  x &= \left[ -5, \frac{2}{3} \right] \\
\end{align*}

\item[26]
\begin{align*}
  4x^2-x-14 & \leq 0 \\
  x & = \left[- \frac{7}{4}, 2 \right]
\end{align*}

\item[27]
\begin{align*}
  8x^2+22x+5 & \geq 0 \\
  x = \left( -\infty, - \frac{5}{2} \right] \cup \left[-\frac{1}{4}, \infty \right) \\
\end{align*}

\item[28]
\begin{align*}
  12x^2-20x+3 & \geq 0 \\
  x &= \left( -\infty, \frac{1}{6} \right] \cup \left[ \frac{3}{2}, \infty \right) \\
\end{align*}

\item[29]
\begin{align*}
  x(5x-36) &> 32 \\
  5x^2-36x-32 &> 0 \\
  x = \left( -\infty, -\frac{4}{5} \right) \cup (8, \infty) \\
\end{align*}

\item[40]
\begin{align*}
  \frac{x+2}{x+4} & \leq 3 \\
  \frac{x+2}{x+4} - 3 & \leq 0 \\
  \frac{-2x-10}{x+4} & \leq 0 \\
  \frac{-x-5}{x+4} & \leq 0 \\
  x &= (-\infty, -5] \cup (-4, \infty) \\
\end{align*}

\item[41]
\begin{align*}
  \frac{x+2}{x-3} &> -2 \\
  \frac{x+2}{x-3} + 2 &> 0 \\
  \frac{3x-4}{x-3} &> 0 \\
  x &= \left( -\infty, \frac{4}{3} \right) \cup (3, \infty) \\
\end{align*}


\end{description}

\fi

\ifprintanswers
\else
\vspace{2.5 in}

\begin{em}
Why should we be in such desperate haste to succeed, and in such desperate enterprises? If a man does not keep pace with
his companions, perhaps it is because he hears a different drummer. 

%   As for the Pyramids, there is nothing to wonder at in them so much as the fact that so many men could be found
%   degraded enough to spend their lives constructing a tomb for some ambitious booby, whom it would have been wiser and
%   manlier to have drowned in the Nile, and then given his body to the dogs.
\end{em}

\vspace{0.1 in}
\hspace{0.5 in} --Henry David Thoreau

\fi

\end{document}

