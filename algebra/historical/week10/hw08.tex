% \documentclass[fleqn,addpoints,landscape]{exam}
\documentclass[fleqn,addpoints]{exam}
\usepackage{amsmath}
\usepackage{graphics}
\usepackage{mdwlist}

\title{Math 113 Homework Eight}
\author{}
\date{\today}

% \printanswers

\usepackage{cancel}

\ifprintanswers
\usepackage{2in1, lscape}
\fi

\begin{document}

\maketitle

\section{From the Book}

Read pages 166-184

\begin{itemize*}
  \item pp. 171-172: 20-24, 40-44, 61, 63
  \item pp. 177-178: 17-21, 30-34, 47
  \item pp. 185-186: 20-24, 41, 42, 50-54
\end{itemize*}

\section{Extra Credit}

A lady gave the postage stamp clerk a one dollar bill and said, ``Please give me some two-cent stamps, ten times as many
one-cent stamps, and the balance in fives.''  How can the clerk fulfill this puzzling request?

\begin{solution}

If:
\begin{itemize*}
  \item $x$ is the number of two-cent stamps purchased
  \item $2x$ is the amount spent on two-cent stamps
  \item $10x$ the number of one-cent stamps purchased, and, since each stamp costs one-cent, the amount spent on one-cent stamps.
  \item $100 - 10x - 2x$ is the amount spent on five-cent stamps
\end{itemize*}

She spent 100 cents on stamps, so this equation must be true (all amounts are in cents):

\( (x) + (2x) + (100 - 10x - 2x) = 100 \)

This equation simplifies to: $ 100 = 100 $ which is true for any value of $x$.  

The problem is that the equation doesn't include the additional constraint for this problem which is: ``you can only
buy a positive integer number of stamps.''  You can't buy half a stamp, and you can't buy $-2$ stamps, for example.

So we have to think about the problem a little more.  She spends whatever money is left over after the one and
two cent stamps on five-cent stamps.  This means that she must have a multiple of 5 left over after she pays for the one
and two cent stamps.  And this means that the amount she spends on one and two cent stamps must be a multiple of 5.  

If we stick with $x$ as the number of two-cent stamps purchased, $x + 10x$ must be a multiple of five less than 100.
Letting $x$ be 5 works, and the next possibility is $x=10$ which is too big.  So she must have purchased 5 two-cent
stamps.  She purchased ten times as many one-cent stamps, so she bought 50 one-cent stamps.  She has 40 cents left over,
which she spends on 8 5-cent stamps.  So the final totals are:

\begin{itemize*}
  \item 5 two cent stamps
  \item 50 one-cent stamps
  \item 8 five-cent stamps
\end{itemize*}

\( 2 \cdot 5 + 50 + 8 \cdot 5 = 100 \), so the whole dollar is accounted for.

\end{solution}

\ifprintanswers

\subsection{Pages 171-172}
\begin{description}

% \item[15]
% \[ \frac{54c^2d}{-78cd^2} = \frac{-9c}{13d} \]

% \item[16]
% \[ \frac{60x^3z}{-64xyz^2} = \frac{-15x^2}{16yz} \]

% \item[17]
% \[ \frac{-40x^3y}{-24xy^4} = \frac{-5x^2}{3y} \]

% \item[18]
% \[ \frac{-30x^2y^2z^2}{-35xz^3} = \frac{6xy^2}{7z} \]

% \item[19]
% \[\frac{x^2 - 4}{x^2 + 2x} = \frac{\cancel{(x + 2)}(x - 2)}{x \cancel{(x + 2)}} = \frac{x - 2}{x} \]

\item[20]
\[\frac{xy + y^2}{x^2 - y^2} = \frac{y \cancel{(x + y)}}{\cancel{(x + y)}(x - y)} = \frac{y}{x - y} \]

\item[21]
\[\frac{18x + 12}{12x - 6} = \frac{\cancel{6}(3x + 2)}{ \cancel{6} (2x - 1) } = \frac{3x + 2}{2x - 1} \]

\item[22]
\[\frac{20x + 50}{15x - 30} = \frac{\cancel{5} \cdot 2 (2x + 5) }{\cancel{5} \cdot 3 (x - 2)} = \frac{2(2x + 5)}{3(x - 2)} \]

\item[23]
\[\frac{a^2 + 7a + 10}{a^2 - 7a - 18} = \frac{\cancel{ (a + 2)} (a + 5)}{ \cancel{(a + 2)} (a - 9)} = \frac{a + 5}{a - 9} \]

\item[24]
\[\frac{a^2 + 4a - 32}{3a^2 + 26a + 16} = \frac{\cancel{(a + 8)} (a - 4) }{ \cancel{(a + 8)} (3a + 2)} 
= \frac{a - 4}{3a + 2} \]

\item[40]
\[\frac{5n^2 + 18n - 8}{3n^2 + 13n + 4} = \frac{ (5n - 2)\cancel{(n + 4)} }{ (3n + 1) \cancel{(n + 4)} } 
= \frac{5n - 2}{3n + 1} \]

\item[41]
\[\frac{4x^2y + 8xy^2 - 12y^3}{18x^3y - 12x^2y^2 - 6xy^3} 
= \frac{2 \cdot \cancel{2} \cdot \cancel{y}(x + 3y)\cancel{(x - y)}} {3 \cdot \cancel{2} \cdot x\cancel{y} (3x + y) \cancel{(x - y}} 
= \frac{2(x + 3y)}{3x(3x + y)} \]

\item[42]
\begin{align*}
  \frac{3 + x - 2x^2}{2 + x - x^2} &= \left(\frac{-1}{-1} \right) \left( \frac{3 + x - 2x^2}{2 + x - x^2} \right) \\
  &= \frac{2x^2 - x - 3}{x^2 - x - 2} \\
  &= \frac{(2x - 3) \cancel{(x + 1)}} { (x - 2) \cancel{(x + 1)}  } \\
  &= \frac{2x - 3} {x - 2} \\
\end{align*}

\item[43]
\[ \frac{3n^2 + 16n - 12}{7n^2 + 44n + 12} = \frac{(3n - 2) \cancel{(n + 6)}} { (7n + 2) \cancel{(n + 6)} } 
= \frac{3n - 2} {7n + 2} \]

\item[44]
\[\frac{x^4 - 2x^2 - 15}{2x^4 + 9x^2 + 9} = \frac{\cancel{(x^2 + 3)} (x^2 - 5)  } { (2x^2 + 3) \cancel{(x^2 + 3)} } 
= \frac{x^2 - 5}{2x^2 + 3} \]

\item[61]
\begin{align*}
  \frac{n^2 - 49}{7 - n} &= \frac{n^2 - 49}{(-1)(n - 7)} \\
  &= - \frac{n^2 - 49}{n - 7} \\
  &= - \frac{(n + 7)\cancel{(n - 7)}}{\cancel{(n - 7)}} \\
  &= - (n + 7) \\
  &= -n - 7 \\
\end{align*}

\item[63]
\begin{align*}
  \frac{2y - 2xy}{x^2y - y} &= \frac{2y(1  - x)}{y(x^2 - 1)}\\
  &= \frac{2 \cancel{y} (-1) \cancel{(x - 1)}}{\cancel{y} \cancel{(x - 1)}(x+1)} \\
  &= -\frac{2}{x+1} \\
\end{align*}

\end{description}

\subsection{Pages 177-178}

\begin{description}

\item[17]
\[ \frac{5xy}{8y^2} \cdot \frac{18x^2y}{15} 
= \frac{5 \cdot 18 \cdot x^3y^2}{ 8 \cdot 15 \cdot y^2} = \frac{3x^3}{4} \]

\item[18]
\[ \frac{4x^2}{5y^2} \cdot \frac{15xy}{24x^2y^2} = \frac{4 \cdot 15 \cdot x^3y}{5 \cdot 24 \cdot x^2y^4} = \frac{x}{2y^3} \]

\item[19]
\begin{align*}
  \frac{5x^4}{12x^2y^3} \div \frac{9}{5xy} &= \frac{5x^4}{12x^2y^3} \cdot \frac{5xy}{9} \\
  &= \frac{5 \cdot 5 \cdot x^3y}{12 \cdot 9 \cdot x^2y^3} \\
  &= \frac{25x^3}{108y^2}
\end{align*}

\item[20]
\begin{align*}
  \frac{7x^2y}{9xy^3} \div \frac{3x^4}{2x^2y^2} &= \frac{7x^2y}{9xy^3} \cdot \frac{2x^2y^2}{3x^4} \\
  &= \frac{7 \cdot 2 \cdot x^4y^3}{9 \cdot 3 \cdot x^5y^3} \\
  &= \frac{14}{27x} \\
\end{align*}

\item[21]
\begin{align*}
  \frac{9a^2c}{12bc^2} \div \frac{21ab}{14c^3} &= \frac{9a^2c}{12bc^2} \cdot \frac{14c^3}{21ab} \\
  &= \frac{9 \cdot 14 \cdot a^2c^4}{12 \cdot 21 \cdot ab^2c^2} \\
  &= \frac{ac^2}{2b^2} \\
\end{align*}

\item[30]
\begin{eqnarray*}
  &   & \frac{6n^2 + 11n - 10}{3n^2 + 19n - 14} \cdot \frac{2n^2 + 6n - 56}{2n^2 - 3n - 20} \\ 
  & = & \frac{ \cancel{(2n + 5)} \cancel{(3n - 2)} (2n - 8) \cancel{(n + 7)} } {\cancel{(3n - 2)} \cancel{(n + 7)} \cancel{(2n + 5)} (n - 4) } \\
  & = & \frac{2 \cancel{(n - 4)}}{ \cancel{(n - 4)} } \\
  & = & 2 \\  
\end{eqnarray*}

\item[31]
\begin{align*}
  \frac{9y^2}{x^2 + 12x + 36} \div \frac{12y}{x^2 + 6x} &= \frac{9y^2}{(x + 6)\cancel{(x + 6)}} \cdot \frac{x \cancel{(x + 6)}}{12y} \\
  &= \frac{3xy}{4(x + 6)}
\end{align*}

\item[32]
\begin{align*}
  \frac{7xy}{x^2 - 4x + 4} \div \frac{14y}{x^2 - 4} &= 
            \frac{7xy}{\cancel{(x - 2)}(x - 2)} \cdot \frac{(x + 2)\cancel{(x - 2)}}{14y} \\
  &= \frac{x(x + 2)}{2(x - 2)} \\
\end{align*}

\item[33]
\begin{eqnarray*}
  &   & \frac{x^2 - 4xy + 4y^2}{7xy^2} \div \frac{4x^2 - 3xy - 10y^2}{20x^2y + 25xy^2} \\
  & = & \frac{(x - 2y)\cancel{(x-2y)}}{7 \cancel{x} y^2} \cdot \frac{5 \cancel{x} y \cancel{(4x + 5y)} }{ \cancel{(4x + 5y)} \cancel{(x - 2y)}} \\
  & = & \frac{5(x - 2y)}{7y}
\end{eqnarray*}

\item[34]
\begin{eqnarray*}
  &   & \frac{x^2 + 5xy - 6y^2}{xy^2 - y^3} \cdot \frac{2x^2 + 15xy + 18y^2}{xy + 4y^2} \\
  & = & \frac{(x + 6y) \cancel{(x - y)} }{ y^2 \cancel{(x - y)}} \cdot \frac{(2x + 3y)(x + 6y)}{y(x + 4y)} \\
  & = & \frac{(x + 6y)^2(2x + 3y)}{y^3(x + 4y)} \\
\end{eqnarray*}

\item[47]
\begin{eqnarray*}
  &   & \frac{x^2 - x}{4y} \cdot \frac{10y^2x}{2x - 2} \div \frac{3x^2 + 3x}{15x^2y^2} \\
  & = & \frac{\cancel{x} \cancel{(x - 1)} }{4y} \cdot 
  \frac{5 \cdot \cancel{2} \cdot y^2}{ \cancel{2} \cancel{(x - 1)}} \cdot 
  \frac{15x^2y^2} {3 \cancel{x}(x + 1)} \\
  & = &\frac{5y^2 \cdot \cancel{3} \cdot 5 \cdot x^2y^2}{\cancel{3} \cdot 4 \cdot y(x + 1)} \\
  & = & \frac{25x^3y^3}{4(x + 1)} \\
\end{eqnarray*}

\end{description}
\subsection{Pages 185-186}

\begin{description}
\item[20]
\[ \frac{x - 2}{4} + \frac{x + 6}{5} = \frac{5(x - 2) + 4(x + 6)}{20} = \frac{9x + 14}{20} \]

\item[21]
\[ \frac{2a - 1}{4} + \frac{3a + 2}{6} = \frac{6a - 3}{12} + \frac{6a + 4}{12} = \frac{12a + 1}{12} \]

\item[22]
\[ \frac{a - 4}{6} + \frac{4a - 1}{8} = \frac{4(a - 4)}{24} + \frac{3(4a - 1)}{24} = \frac{16a - 19}{24} \]

\item[23]
\[ \frac{n+2}{6} - \frac{n-4}{9} = \frac{3(n+2)}{18} - \frac{2(n-4)}{18} = \frac{n + 14}{18}\]

\item[24]
\[ \frac{2n+1}{9} - \frac{n+3}{12} = \frac{4(2n+1)}{36} - \frac{3(n+3)}{36} = \frac{5n-5}{36}\]

\item[41]
\[ \frac{3}{x} - \frac{5}{3x^2} - \frac{7}{6x} = \frac{18x - 10 -7x}{6x^2} = \frac{11x - 10}{6x^2} \]

\item[42]
\[ \frac{7}{3x^2} - \frac{9}{4x} - \frac{5}{2x} = \frac{28 - 27x - 30x}{12x^2} = \frac{28 - 57x}{12x^2} \]

\item[50]
\[ \frac{3x}{x-4} - \frac{2}{x} = \frac{3x^2 - 2(x-4)}{x(x-4)} = \frac{3x^2 - 2x + 8}{x(x-4)} \]

\item[51]
\[ \frac{a-2}{a} - \frac{3}{a+4} = \frac{(a-2)(a+4) - 3a}{a(a+4)} = \frac{a^2 + 2a - 8 - 3a}{a(a+4)} 
= \frac{a^2 - a - 8}{a(a+4)} \]

\item[52]
\[ \frac{a+1}{a} - \frac{2}{a+1} = \frac{(a+1)^2 - 2a}{a(a+1)} = \frac{a^2 + 2a + 1 - 2a}{a(a+1)} 
= \frac{a^2 + 1}{a(a+1)} \]

\item[53]
\begin{align*}
  \frac{-3}{4n+5} - \frac{8}{3n+5} &= \frac{-3(3n+5) - 8(4n+5)}{(4n+5)(3n+5)} \\
  &= \frac{-9n - 15 - 32n - 40}{(4n + 5)(3n+5)} \\
  &= \frac{-41n - 55}{(4n + 5)(3n + 5)}
\end{align*}

\item[54]
\begin{align*}
  \frac{-2}{n-6} - \frac{6}{2n+3} &= \frac{-2(2n+3) - 6(n-6)}{(n-6)(2n+3)} \\
  &= \frac{-4n - 6 - 6n + 36}{(n-6)(2n+3)} \\
  &= \frac{-10n + 30}{(n-6)(2n+3)}
\end{align*}

\end{description}
\fi

\ifprintanswers
\else
\vspace{9 cm}

{\em Instead of having "answers" on a math test, they should just call them ``impressions,'' and if you got a different
  "impression," so what, can't we all be brothers?}

\vspace{0.1 in}
\hspace{0.5 in} --Jack Handey

% {\em Do not worry about your difficulties in Mathematics. I can assure you mine are still greater.}
% {\em Imagination is more important than knowledge.}

% \hspace{0.5 in} --Albert Einstein

\fi

\end{document}
