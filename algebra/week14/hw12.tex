% no answer key
% \documentclass[letterpaper]{exam}

% answer key
\documentclass[letterpaper, landscape]{exam}
\usepackage{2in1, lscape} 
\printanswers{}

\usepackage{units} 
\usepackage{xfrac} 
\usepackage[fleqn]{amsmath}
\usepackage{commath}
\usepackage{cancel}
\usepackage{float}
\usepackage{mdwlist}
\usepackage{booktabs}
\usepackage{cancel}
\usepackage{polynom}
\usepackage{caption}
\usepackage{fullpage}
\usepackage{comment}
\usepackage{enumerate}
\usepackage{graphicx}
\usepackage{mathtools} 
\usepackage{parskip} 

\newcommand{\degree}{\ensuremath{^\circ}} 
\everymath{\displaystyle}

\title{Algebra Homework 12 \\ Section 4.2}
\author{}
\date{\today}

\begin{document}

  \maketitle

  \section{Homework}

  Section 4.2: 15--19, 25--29, 35--39, 45--49

  \section{Extra Credit}

  A 100 pound bag of potatoes is 99\% water.  After sitting out in the sun for
  a while, some of the water evaporates, so the potatoes are only 98\% water.
  How much does the bag now weigh?

  \begin{solution}
    Since the bag starts out 99\% water, it initially contains 99 pounds of water and 1 pound of other stuff.  

    After some of the water evaporates, the other stuff remains, so there is still one pound of other stuff.  We also know
    that the bag is now 2\% other stuff, since it is now 98\% water.  If we let $x$ be the weight of the bag after the
    evaporation, we know that:

    \( .02x = 1 \)

    Solving for $x$:
    \begin{align*}
      .02x           & = 1 \\
      \frac{2x}{100} & = 1 \\
      x              & = 50 \\
    \end{align*}

    The bag now weighs \fbox{ 50 pounds }

  \end{solution}

  \ifprintanswers{}
    \section{Section 4.2} % (fold)
    
    \begin{description}
      \item[15] $\frac{2a^3}{3b}$

      \item[16] $\frac{15b}{a^2}$

      \item[17] $\frac{3x^3}{4}$

      \item[18] $\frac{x}{2y^3}$

      \item[19] $\frac{25 x^3}{108 y^2}$

      \item[25] $\frac{3 \del{x^2 + 4}}{5y(x + 8)}$

      \item[26] $5y$

      \item[27] $\frac{5 (a+3)}{a (a-2)}$

      \item[28] $\frac{a \del{a^2 + 3}}{2(2a - 1)}$

      \item[29] $\frac{3}{2}$

      \item[35] $\frac{n + 5}{3 - n}$

      \item[36] $\frac{6 - 5n}{n + 1}$

      \item[37] $\frac{x^2 + 1}{x^2 - 10}$

      \item[38] $\frac{2x^2 + 3}{2x^2 + 1}$

      \item[39] $\frac{6x + 5}{3x + 4}$

      \item[45] $\frac{n + 3}{n(n - 2)}$

      \item[46] $\frac{(x - 2)(y + c)}{2(y - 2c)(3x - 1)}$

      \item[47] $\frac{25 x^3 y^3}{4(x + 1)}$

      \item[48] $\frac{6 x^6 y}{7}$

      \item[49] $\frac{2 (a - 2b)}{a(3a - 2b)}$

    \end{description}

  \fi
  \ifprintanswers{}
  \else
    \vspace{8 cm}
    \begin{quote}
      \begin{em}
        That some desperate wretches should be willing to steal and enslave men by
        violence and murder for gain, is rather lamentable than strange. But that many
        civilized, nay, Christianized people should approve, and be concerned in the
        savage practice, is surprising; and still persist, though it has been so often
        proved contrary to the light of nature, to every principle of Justice and
        Humanity, and even good policy, by a succession of eminent men, and several
        late publications.
      \end{em}
    \end{quote}
    \hspace{2 cm}--Thomas Paine
  \fi

\end{document}

