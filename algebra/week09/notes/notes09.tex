% no answer key
% \documentclass[letterpaper]{exam}

% answer key
\documentclass[letterpaper, landscape]{exam}
\usepackage{2in1, lscape} 
\printanswers{}

\usepackage{units} 
\usepackage{xfrac} 
\usepackage[fleqn]{amsmath}
\usepackage{commath}
\usepackage{cancel}
\usepackage{float}
\usepackage{mdwlist}
\usepackage{booktabs}
\usepackage{cancel}
\usepackage{polynom}
\usepackage{caption}
\usepackage{fullpage}
\usepackage{comment}
\usepackage{enumerate}
\usepackage{graphicx}
\usepackage{mathtools} 

\newcommand{\degree}{\ensuremath{^\circ}} 
\everymath{\displaystyle}

\title{Algebra Notes \\ Section 2.7 }
\author{}

\date{\today}

\begin{document}

  \maketitle

  \section{Equations With Absolute Value}

  \subsection{Notes} % (fold)
  
  \begin{itemize*}
    \item show absolute value as distance on number line
    \item show absolute value as $x = a$ or $x = -a$
  \end{itemize*}

  cases for $\abs{ z } = a$, where $z$ is some expression:
  \begin{itemize*}
    \item $a > 0$: two solutions
    \item $a = 0$: one solutions
    \item $a < 0$: no solutions
  \end{itemize*}

  procedure:
  \begin{enumerate*}
    \item Isolate the expression to turn it into $\abs{ ax + b } = c$
    \item Solve two equations: $ax + b = c$ and $ax + b = -c$
    \item Answer is the two solutions
  \end{enumerate*}

  \subsection{Examples} % (fold)
  
  \begin{enumerate}

    \item $\abs{ x } = 4$

    \item $\abs{ 2x + 1 } = 3$

    \item $\abs{ -x - 3 } = 7$

    \item $\abs{ \frac{x}{2} - 1 } = 5$

    \item $2 \abs{ 3x - 7 } = 42$

    \item $\abs{ 5x - 6 } - 8 = 9$

    \item $4 \abs{ 3x + 8 } - 18 = 14$

    \item $\frac{1}{3} \abs{ 3x - 3 } + 13 = 25$


  \end{enumerate}

  \section{Less Than}

  \subsection{Notes} % (fold)
  
  For a non-negative value of $a$:
  \begin{align*}
    |x|    & < a \leftrightarrow -a < x < a \\
    |x|    & < a \leftrightarrow -a < x \text{ and } x < a \\
    |f(x)| & < a \leftrightarrow -a < f(x) \text{ and } f(x) < a \\
  \end{align*}

  \begin{itemize*}
    \item show number line---distance from origin must be less than $a$.
    \item talk about how equation doesn't make sense for $a < 0$
    \item if $a = 0$ only one solution ($\leq$) or no solutions ($<$)
    \item if $a < 0$ no solutions
  \end{itemize*}

  procedure:
  \begin{enumerate*}
    \item Isolate the expression to turn it into $\abs{ ax + b } < c$
    \item Solve the equations: $ax + b = c$ and $ax + b = -c$
    \item Answer is the range between the two solutions
  \end{enumerate*}

  \subsection{Examples} % (fold)
  
  \begin{enumerate}
    \item $\abs{ x } \leq 1$

    \item $\abs{ 3x - 1 } < 2$

    \item $4 + 3 \abs{ 4x - 2 } < 22$

    \item $\abs{ \frac{x}{2} + \frac{x}{3} } < 5$

  \end{enumerate}

  \subsection{Greater Than}
  
  For a non-negative value of $a$:
  \[
    |x| > a \leftrightarrow x < -a \text{ or } x > a \\
  \]

  \begin{itemize*}
    \item show number line---distance from origin must be greater than $a$.
    \item $x$ must be one of
      \begin{itemize*}
        \item 0, if $a = 0$
        \item a positive number greater than $a$
        \item a negative number less than $a$
      \end{itemize*}
    \item if $a = 0$, solution is a all numbers ($\geq$) or one number ($>$).
    \item if $a < 0$, there is no solution
  \end{itemize*}

  The same idea works if $x$ is some more complicated expression.

  procedure:
  \begin{enumerate*}
    \item Isolate the expression to turn it into $\abs{ ax + b } < c$
    \item Solve the equations: $ax + b = c$ or $ax + b = -c$
    \item Answer is everything but the range between the two solutions
  \end{enumerate*}

  \subsection{Examples} % (fold)
  
  \begin{enumerate}
    \item $\abs{ x } > 3$

    \item $\abs{ 2x + 1 } \geq 4$

    \item $\abs{ \frac{2x}{9} - \frac{x}{6} } > \frac{1}{2}$

    \item $-7 + 3 \abs{ 5x - 10 } > 38$

    \item $12 + 3 \abs{ x - 6 } \geq 48$

    \item $-3 + 2 \abs{ x + 4 } \geq 7$

    \item $5 \abs{ 18 - 6x } > 30$

  \end{enumerate}

\end{document}
