% no answer key
% \documentclass[letterpaper]{exam}

% answer key
\documentclass[letterpaper, landscape]{exam}
\usepackage{2in1, lscape} 
\printanswers{}

\usepackage{units} 
\usepackage{xfrac} 
\usepackage[fleqn]{amsmath}
\usepackage{commath}
\usepackage{cancel}
\usepackage{float}
\usepackage{mdwlist}
\usepackage{booktabs}
\usepackage{cancel}
\usepackage{polynom}
\usepackage{caption}
\usepackage{fullpage}
\usepackage{comment}
\usepackage{enumerate}
\usepackage{graphicx}
\usepackage{mathtools} 
\usepackage{parskip} 

\newcommand{\degree}{\ensuremath{^\circ}} 
\everymath{\displaystyle}

\title{Algebra \\ Homework Six \\ Section 2.7 }
\author{}
\date{\today}


\begin{document}

    \maketitle

    \ifprintanswers{}
    \else
        \section{Calendar}

        \begin{itemize*}
            \item March 18: Chapter 2 Review
            \item March 25: Chapter 2 Exam
        \end{itemize*}
    \fi

    \section{Homework}

    \begin{itemize*}
        \item Section 3.1: 50--55, 70--74
        \item Section 3.2: 5--10, 19, 21--25, 50--55, 65--70, 81--84, 92
    \end{itemize*}

    \section{Extra Credit}

    A silver prospector was unable to pay his March rent in advance.  He owned a bar of pure silver, 31 inches long, so he
    made the following arrangement with his landlady.  He would cut the bar, he said, into smaller pieces.  On the first day
    of March he would give the lady an inch of the bar, and on each succeeding day he would add another inch to the amount
    of silver.  She would keep this silver as security.  At the end of the month, when the prospector expected to be able to
    pay his rent in full, she would return the pieces to him.

    March has 31 days, so one way to cut the bar would be to cut it into 31 sections, each an inch long.  But since it
    required considerable labor to cut the bar, the prospector wished to carry out his arrangement with the fewest possible
    number of pieces.  For example, he might give the lady an inch on the first day, another inch the second day, then on
    the third day he could take back the two pieces and give her a solid 3-inch section.

    Assuming the portions of the bar are traded back and forth in this fashion, what is the smallest number of pieces into
    which the prospector needs to cut his silver bar?

    \begin{solution}

        The most efficient approach is that each time you run out of pieces
        you cut off a new piece exactly as long as you need to meet that
        day's rent.  

        \begin{itemize}
            \item On the first day, you have zero pieces, so you cut off a piece of size 1.  
            \item On the second day, you need a piece of size 2, so you cut off a piece of size 2.
            \item On the third day, you give the landlady both pieces.
            \item On the fourth day, you need another piece, so you cut off a piece of size 4.
            \item On days 5--7, you use the pieces you already have.
            \item On the eighth day, you need another piece, so you cut off a piece of size 8.
            \item On days 9--15, you use the pieces you already have.
            \item On the sixteenth day, you need another piece.  Fortunately, the rest of the bar is size 16, so you use that.
            \item On days 17--31, you use the pieces you already have.
        \end{itemize}

        You need to cut the bar into pieces of size 1, 2, 4, 8, and 16 and four cuts are required.

    \end{solution}
    \ifprintanswers{}

        \section{Section 3.1} 

        \begin{description}

            % \item[11] $10x - 3$

            % \item[12] $14x + 3$

            % \item[13] $-11t + 5$

            % \item[14] $-10t + 8$

            % \item[15] $-x^2 + 2x - 2$

            \item[16] $-x^2 + x - 6$

            \item[17] $17a^2b^2 - 5ab$

            \item[18] $-5a^2b^2 - 7ab$

            \item[19] $-9x + 7$

            \item[20] $-2x^2 -2x - 5$

            \item[26] $-2x^2 - 2x - 2$

            \item[27] $-6a^2 + 12a + 14$

            \item[28] $6a^2 + 12a - 14$

            \item[29] $3x^3 + x^2 + 13x - 11$

            \item[30] $3x^3 - x^2 - x - 6$

            \item[36] $-5x^2 -3x + 16$

            \item[37] $-3x^3 + 5x^2 - 2x + 9$

            \item[38] $-x^3 + 5x^2 - 5$

            \item[39] $5x^2 - 4x + 11$

            \item[40] $-x^3 - 2x^2 + 7x - 2$

            \item[50] $3x + 7$

            \item[51] $-9x^2 -12x + 4$

            \item[52] $-10x^2 + 2x + 14$

            \item[53] $10x^2 + 13x - 18$ 

            \item[54] $-3x^2 - 2x + 10$

            \item[55] $-n^2 -4n -4$

            \item[61] $-7n^2 + n + 6$

            \item[62] $-11n^2 - n + 4$

            \item[63] $t^2 - 4t + 8$

            \item[64] $-4n^2 + 3n - 1$

            \item[65] $4n^2 - n - 12$

            \item[66] $6$

            \item[67] $-4x - 2y$

            \item[68] $7xy - 2x + 3y$

        \end{description}

        \section{Section 3.2}

        \begin{description}
            \item[5] $4a^3b^4$

            \item[6] $24a^3b^5$

            \item[7] $-3x^3y^2z^6$

            \item[8] $2x^3y^5z^3$

            \item[9] $-30xy^4$

            \item[10] $-28x^5y$

            \item[21] $30x^6$

            \item[22] $12x^6$

            \item[23] $-18x^9$

            \item[24] $-84x^6$

            \item[25] $-3x^6y^6$

            \item[50] $16a^8b^{16}$

            \item[51] $-16a^4b^4$

            \item[52] $-81a^4b^4$

            \item[53] $-x^6y^{12}z^{18}$

            \item[54] $-x^8y^{16}z^{24}$

            \item[55] $-125a^6b^6c^3$

            \item[65] $-18xyz^4$
            
            \item[66] $-32x^2y^4z^5$

            \item[67] $-a^2b^3c^2$

            \item[68] $-a^2b$

            \item[69] $9$

            \item[70] $-8y$

            \item[81] $a^{5n+1}$

            \item[82] $x^{3n}$

            \item[83] $-10x^{2n}$

            \item[84] $-12x^{3n}$ 

            \item[92]
                The sides come in pairs: 
                \begin{itemize}
                  \item $ 2x^2 $
                  \item $ 5x $
                  \item $ 10x $
                \end{itemize}

                So the total area is: $ 2(2x^2) + 2(5x) + 2(10x) = 4x^2 + 30x $

                To get the volume, multiply the three side lengths together: 
                \[
                    x \cdot 2x \cdot 5 = 10x^2 
                \]
        \end{description}

    \fi
    \ifprintanswers{}
    \else
        \vspace{8 cm}
        \begin{quote}
            \begin{em}
                TO DO
            \end{em}
        \end{quote}
        \hspace{1 cm} --TO DO
    \fi

\end{document}

