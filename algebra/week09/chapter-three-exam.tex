\documentclass[fleqn,addpoints]{exam}
\usepackage{amsmath}
\usepackage{mdwlist}
\usepackage{graphicx}

\everymath{\displaystyle}

\title{Math 113 Chapter Three Exam}
\author{}
\date{\today}

% \oddsidemargin 0in
% \topmargin -0.5in
% \textwidth 6.5in

\printanswers

\ifprintanswers
\usepackage[landscape, bottom=.5in]{geometry}
\usepackage{2in1, lscape}
\fi

% \extrawidth{-1 in}
% \setlength{\mathindent}{0in}

\begin{document}



\maketitle

\ifprintanswers
\else
\vspace{0.2in}
\makebox[\textwidth]{Name:\enspace\hrulefill}
\vspace{0.2in}

\begin{center}
\gradetable[h][pages]
\end{center}

\fi

\ifprintanswers
\section{General Comments}

\subsection{Review}
Everyone should take the time to review the material from early in the chapter.  The material from the early sections
may be easier than the material from the later sections, but you still can forget it if you don't brush up a bit.

\subsection{Check Your Answers}
Most of the problems from this chapter are extremely easy to check.  If you invest five minutes in
finding a solution, invest 30 seconds more in making sure you have the correct solution.  

For example, suppose you had the following solution to this problem: 

\( x^2 - 2x - 24 = (x - 6)(x - 4) \)

You can check by multiplying the result back out: 

$x^2 - 6x - 4x + 24 = x^2 - 10x + 24$.

You will immediately realize you got a sign wrong, and go back and fix the answer.

The problems with numbers as solutions are even easier to check. For example, suppose you had the following solution to
this problem:
\begin{align*}  
  x^2 + x - 2 &= 0 \\
  (x - 2)(x + 1)  &= 0 \\
\end{align*}
With the solution set: $\{2, -1\}$.
  
If you spend about 10 seconds plugging the answers into the original equation: 
\begin{align*}  
  2^2 + 2 - 1 &= 5 \\ 
  (-1)^2 - 1 - 2 &= -2 \\
\end{align*}
You will realize that something is wrong and go back and fix it.

When I take the test, I make as many of these kinds of mistakes as anyone else.  But I check my answers, so I fix
most of them before I make the answer key.  You should do the same thing.


\subsection{Strategy}

Many people missed the ``sum of two cubes'' and/or ``difference of two squares'' questions.  One thing that might have
helped is to remember the chapter table of contents on page 108 of the text.  Section 3.5 is titled {\em Difference of
  Two Squares and Sum or Difference of Two Cubes}.  Since there is a whole section devoted to this subject, you can be
pretty sure there will be some problems on the test for it.

So when you are working on the test and you get to a problem you can't figure out how to do, one thing you can do is
mentally go back over what we covered in the chapter.  The thought process might be something like:
\begin{itemize*}
  \item ``adding polynomials \ldots yep''
  \item ``multiplying/dividing polynomials \ldots check''
  \item ``factoring polynomials \ldots yep''
  \item ``difference/sum of squares/cubes \ldots nope''
\end{itemize*}

Then you might wonder if the problem you are stuck on might be the missing ``difference/sum of squares/cubes'' question
and look for cubes and squares in the equation.

Another strategy note is that questions often have a specific purpose, so you can sometimes learn something by comparing
a question to its neighbors.  The three questions about powers each had a purpose.  One question had the negative sign
outside the parentheses with an even power, one question had the negative sign inside the parentheses with an even
power, and one question had the negative sign inside the parentheses with an odd power.  If you got a negative result
for both of the even power questions, you might look back at the questions and notice the difference in the placement of
the parentheses.  This should lead you to think about why they are different, which might help you remember how they
should be treated differently.

\fi

\section{Questions}

\subsection{Sums and Differences of Polynomials}

For problems \ref{simplify:first}-\ref{simplify:last}, simplify each polynomial.

\begin{questions}

% \question[5]
% \[ (x + 2) - (2x + 1) + (-x + 7) \]
% \begin{solution}[3 cm]
% \end{solution}

\question[5]
\label{simplify:first}
\[ (3x^2 + 7x - 12) - (x^2 + 1) + (-2x^2 - 6x)\]
\begin{solution}[4 cm]

\begin{align*}
  & (3x^2 + 7x - 12) - (x^2 + 1) + (-2x^2 - 6x) \\
  &= 3x^2 + 7x - 12 - x^2 - 1 - 2x^2 - 6x \\
  &= x - 13 \\
\end{align*}

\end{solution}

\question[5]
\[ [2x^2 + (x - 1) + 2] - [x^2 - (5x^2 - 2x + 3)] \]
\label{simplify:last}
\begin{solution}[4 cm]

\begin{align*}
  & 2x^2 + (x - 1) + 2] - [x^2 - (5x^2 - 2x + 3)] \\
  &= 2x^2 + x + 1 - [x^2 - 5x + 2x - 3 ] \\
  &= 2x^2 + x + 1 - [-4x^2 + 2x - 3 ] \\
  &= 2x^2 + x + 1 + 4x^2 - 2x + 3 \\
  &= 6x^2 - x + 4 \\
\end{align*}

\end{solution}

\subsection{Powers}

For problems \ref{power:first}-\ref{power:last}, raise to the indicated power.

\question[3]
\label{power:first}
\[ -(2a^2b^5)^4 \]
\begin{solution}[3 cm]
\( -16a^8b^{20} \)
\end{solution}

\question[3]
\[ (-2ab^2)^3 \]
\begin{solution}[3 cm]
\( -8a^3b^6 \)
\end{solution}

\question[3]
\label{power:last}
\[ (-3x^3y^2z)^2 \]
\begin{solution}[3 cm]
\( 9x^6y^4z^2 \)
\end{solution}

% \question[5]
% \[ (-5ab^2c)^3 \]
% \begin{solution}[2 cm]
% \end{solution}

% \question[5]
% \label{power:last}
% \[ (2x - 3a)^2 \]
% \begin{solution}[4 cm]
% \end{solution}

\subsection{Quotients}

For problems \ref{quotient:first}-\ref{quotient:last}, find each quotient.

\question[3]
\label{quotient:first}
\[ \frac{9a^3b^2}{3ab} \]
\begin{solution}[3 cm]
\( 3a^2b \)
\end{solution}

\question[4]
\label{quotient:last}
\[ \frac{16x^4y^{10}}{-4xy^7} \]
\begin{solution}[3 cm]
$ -4x^3y^3 $
\end{solution}

% \question[5]
% \[ \frac{-10ab^2c^5}{-5bc^3} \]
% \label{quotient:last}
% \begin{solution}[2 cm]
% \end{solution}

\subsection{Products}

For problems \ref{product:first}-\ref{product:last}, find each product.

% \question[5]
% \label{product:first}
% \[ (3x^2yz^2)(5xy^2x) \]
% \begin{solution}[2 cm]
% \end{solution}

\question[5]
\label{product:first}
\[ (-3x)(-2xy^2)(4x^3y) \]
\begin{solution}[3 cm]

\begin{align*}
  (-3x)(-2xy^2)(4x^3y) &= (6x^2y^2)(4x^3y) \\
  &= 24x^5y^3 \\
\end{align*}

\end{solution}

\question[5]
\[ 5xy(2x^2y + x^3y^2) \]
\begin{solution}[4 cm]

\( 10x^3y^2 + 5x^4y^3 \)

\end{solution}

% \question[5]
% \[ (5x + 7)(8x - 2) \]
% \begin{solution}[4 cm]
% \end{solution}

% \question[5]
% \[ (4x - 1)(2x^2 + x - 5) \]
% \begin{solution}[4 cm]
% \end{solution}

\question[5]
\label{product:last}
\[ (2x^2 + 3x - 1)(7x^2 - 2x + 4) \]
\begin{solution}[4 cm]

\begin{align*}
  & (2x^2 + 3x - 1)(7x^2 - 2x + 4) \\
  &= 2x^2(7x^2 - 2x + 4) + 3x(7x^2 - 2x + 4) - (7x^2 - 2x + 4) \\
  &= 14x^4 - 4x^3 + 8x^2 + 21x^3 - 6x^2 + 12x - 7x^2 + 2x - 4 \\
  &= 14x^4 + 17x^3 + 2x^2 + 14x - 7x^2 - 4 \\
  &= 14x^4 + 17x^3 - 5x^2 + 14x - 4 \\
\end{align*}

\end{solution}

\subsection{Factoring}

For problems \ref{factor:first}-\ref{factor:last}, factor each equation completely.

\question[5]
\label{factor:first}
\[ 18a^2b + 27ab^3 \]
\begin{solution}[4 cm]

\( 9ab(2a + 3b^2) \)

\end{solution}

\question[5]
\[ 3x(a + 2b) - 2y(a + 2b) \]
\begin{solution}[4 cm]

$(3x - 2y)(a + 2b) $
\end{solution}

\question[5]
\[ 2a^2 + 3ab - 8ab - 12b^2 \]
\begin{solution}[4 cm]

\begin{align*}
  2a^2 + 3ab - 8ab - 12b^2 &= 2a^2 - 8ab + 3ab - 12b^2 \\
  &= 2a(a - 4b) + 3b(a - 4b) \\
  &= (2a + 3b)(a - 4b) \\
\end{align*}

\end{solution}

% \pagebreak

\question[5]
\[ 4x^2 - 9 \]
\begin{solution}[4 cm]

\begin{align*}
  4x^2 - 9 &= (2x)^2 + 3^2 \\
  &= (2x + 3)(2x - 3) \\
\end{align*}

\end{solution}

\question[5]
\[ 27x^3y^3 + 8 \]
\begin{solution}[4 cm]

\begin{align*}
  27x^3y^3 + 8 &= (3xy)^3 + 2^3 \\
  &= (3xy + 2)(9x^2y^2 - 6xy + 4)
\end{align*}

\end{solution}

\question[5]
\[ x^2 + 11x + 18 \]
\begin{solution}[4 cm]
\( (x + 9)(x + 2) \)
\end{solution}

\question[5]
\[ x^2 - 2x - 24 \]
\begin{solution}[4 cm]
$ (x + 4)(x - 6) $
\end{solution}

\question[5]
\label{factor:last}
\[ 12x^2 - 23x - 24 \]
\begin{solution}[4 cm]

\begin{align*}
  12x^2 - 23x - 24 &= 12x^2 - 32x + 9x - 24 \\
  &= 4x(3x - 8) + 3(3x - 8) \\
  &= (4x + 3)(3x - 8) \\
\end{align*}

\end{solution}

\subsection{Equation Solving}

For problems \ref{solve:first}-\ref{solve:last}, solve each equation for $x$.

\question[5]
\label{solve:first}
\[ x^2 + x - 2 = 0 \]
\begin{solution}[4 cm]

\begin{align*}
  x^2 + x - 2 &= 0 \\
  (x + 2)(x - 1) &= 0 \\
\end{align*}

$\{-2, 1\}$

\end{solution}

\question[5]
\[ 6x^2 - x - 15 = 0 \]
\begin{solution}[4 cm]

\begin{eqnarray*}
  6x^2 - x - 15 &=& 0 \\
  6x^2 - 10x + 9x - 15 &=& 0 \\
  2x(3x - 5) + 3(3x - 5) &=& 0 \\
  (2x + 3)(3x - 5) &=& 0 \\
  \\
  2x + 3 = 0 &or& 3x - 5 = 0 \\
  2x = -3 &or& 3x  = 5 \\
  x = -3/2 &or& x  = 5/3 \\
\end{eqnarray*}

$ \{-3/2, 5/3 \}$

\end{solution}

% \pagebreak

\question[5]
\[ x^4 - 10x^2 + 9 = 0 \]
\begin{solution}[5 cm]

\begin{align*}
  x^4 - 10x^2 + 9 &= 0 \\
  (x^2 - 1)(x^2 - 9) &= 0 \\
  (x + 1)(x - 1)(x + 3)(x - 3) &= 0 \\
\end{align*}

$ \{-1, 1, -3, 3 \} $

\end{solution}

\question[5]
\label{solve:last}
\[ (x + 2)^2 - 64 = 0 \]
\begin{solution}[5 cm]

The easiest way to do this one is to use the ``difference of two squares'' pattern.


\begin{align*}
  (x + 2)^2 - 64 &= 0 \\
  (x + 2)^2 - 8^2 &= 0 \\
  (x + 2 + 8)(x + 2 - 8) &= 0 \\
  (x + 10)(x - 6) &= 0 \\
\end{align*}

$ \{-10, 6\} $

\end{solution}

\question[5]
\[ (x - 6)(x - 2) = -3 \]
\begin{solution}[5 cm]

\begin{align*}
  (x - 6)(x - 2) &= -3 \\
  x^2 - 8x + 12  &= -3 \\
  x^2 - 8x + 15  &= 0 \\
  (x - 3)(x - 5) &= 0 \\
\end{align*}

$ \{3, 5\} $

\end{solution}


\ifprintanswers
\else
\pagebreak
\fi

\subsection{Word Problems}

For problems \ref{word:first}-\ref{word:last}, use an equation to solve each problem.

\question[7]
\label{word:first}
The sum of the area of two squares is 45.  The side of one square is twice the side of the other square.  Find the
dimensions of both squares.
\begin{solution}[7 cm]

The sides of the two squares are $x$ and $2x$.

\begin{align*}
  x^2 + (2x)^2 &= 45 \\
  5x^2 &= 45 \\
  x^2 &= 9 \\
  x^2 - 9 &= 0 \\
  (x + 3)(x - 3) &= 0 \\
\end{align*}

The solution set to this equation is $\{-3, 3\}$.  But $-3$ doesn't make sense for a length, so $3$ is the solution.
The side of the other square is $6$.  

We can check the answer: $3^2 + 6^2 = 9 + 36 = 45$.

\end{solution}

\question[7]
\label{word:last}
Find two consecutive positive odd integers where the sum of their squares is 74.
\begin{solution}[7 cm]

Since the integers are both odd, the second integer is two more than the first integer.  We can let $x$ be the first
integer and $x + 2$ be the second integer.

\begin{align*}
  x^2 + (x + 2)^2 &= 74 \\
  x^2 + (x^2 + 4x + 4) &= 74 \\
  2x^2 + 4x + 4  &= 74 \\
  2x^2 + 4x - 70 &= 0 \\
  \frac{1}{2}(2x^2 + 4x - 70) &= \frac{1}{2}(0) \\
  x^2 + 2x - 35 &= 0 \\
  (x + 7)(x - 5) &= 0 \\
\end{align*}

The solution set is: $\{-7, 5\}$.  The question asks for positive integers, so the two positive integers are 5 and 7.
It's easy to verify that these are the two correct numbers since: $5^2 + 7^2 = 25 + 49 = 74$.
\end{solution}

\pagebreak

\subsection{Extra Credit}

\noaddpoints

\ifprintanswers
\else
\question[5]
\begin{figure*}[h]
  \centering
  \includegraphics*{extra-credit.eps}  
  \caption{Extra Credit}
  \label{fig:extra-credit}
\end{figure*}
\fi

A metal sheet has the shape of a two-foot square with semicircles on opposite sides.  If a disk with a diameter of two
feet is removed from the center as shown, what is the area of the remaining metal?

The area of a circle is given by: \(A = \pi r^2 \).

\begin{solution}[5 cm]

This question doesn't require any calculations if you look at the figure closely.  

The two semicircles on the ends exactly match the missing center of the square.  If you were to chop them off and put
them in the middle, the square would be its original area of 4 square feet.

You can also do some calculations to arrive at the same result:
\begin{itemize*}
  \item the original area of the square was 4 square feet
  \item the area of the circle removed from the square is $\pi \cdot 1^2 = \pi$ square feet
  \item the area of one semicircle on the end is is $\frac{\pi \cdot 1^2}{2} = \frac{\pi}{2}$ square feet
\end{itemize*}

Putting it all together, the total area is: 
\[ 4 - \pi + 2 \cdot \frac{\pi}{2} = 4 - \pi + \pi = 4 \]

It's easier (and more accurate) to leave $\pi$ in the equation rather than turning it into $3.14$ and doing calculations
with $3.14$.  All the $\pi$s eventually go away, so you end up not needing to do any calculations at all.

\end{solution}


\end{questions}

\end{document}


