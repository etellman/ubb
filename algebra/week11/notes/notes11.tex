% no answer key
% \documentclass[letterpaper]{exam}

% answer key
\documentclass[letterpaper, landscape]{exam}
\usepackage{2in1, lscape} 
\printanswers{}

\usepackage{units} 
\usepackage{xfrac} 
\usepackage[fleqn]{amsmath}
\usepackage{commath}
\usepackage{cancel}
\usepackage{float}
\usepackage{mdwlist}
\usepackage{booktabs}
\usepackage{cancel}
\usepackage{polynom}
\usepackage{caption}
\usepackage{fullpage}
\usepackage{comment}
\usepackage{enumerate}
\usepackage{graphicx}
\usepackage{mathtools} 

\newcommand{\degree}{\ensuremath{^\circ}} 
\everymath{\displaystyle}

\title{Algebra Notes \\ Section 3.3 }
\author{}

\date{\today}

\begin{document}

  \maketitle

  \section{Monomial/Polynomial}

  \begin{align*}
    2(3 + 4) & = 2 \cdot 7 \\
             & = 14 \\
    2(3 + 4) & = 2 \cdot 3 + 2 \cdot 4 \\
             & = 6 + 8 \\
             & = 14 \\
             \\
    a(b + c) & = ab + ac \\
  \end{align*}

  examples:
  \begin{align*}
    3x(x + 2)               & = 3x^2 + 6x \\
    2x^2y(3xy + 2y^2 + x^3) & = 6x^3y^2 + 4x^2y^3 + 2x^5y \\
    \ldots \\
  \end{align*}

  \section{Two Binomials}

  \begin{align*}
    (2 + 3)(4 + 5) & = 5 \cdot 9 \\
                   & = 45 \\
    (2 + 3)(4 + 5) & = (2 + 3) \cdot 4 + (2 + 3) 5 \\
                   & = 4(2 + 3) + 5(2 + 3) \\
                   & = 8 + 12 + 10 + 15 \\
                   & = 45 \\
                   \\
    (a + b)(c + d) & = ac + ad + bc + bd \\
  \end{align*}

  examples:
  \begin{itemize*}
    \item $(x + 1)(x + 2)$
    \item $(x - 1)(x - 2)$
    \item $(2x + 3)(x - 2)$
    \item $\del{x + 5}^2$
    \item $(x + 5)(x - 5)$
    \item $(x + 5)(x - 5)$
    \item $(x^a + 1)(3x^a + 2)$
    \item $(x^{2n} + 3)(x^{2n} - 1)$
    \item etc.
  \end{itemize*}

  \newpage

  \section{Polynomials}
  Same idea generalizes to multiplying any two polynomials.

  examples:
  \begin{itemize*}
    \item $(x + 1)(2x^2 + 4x - 2)$
    \item $(2x - 3)(x^2 - x + 4)$
    \item $(x + 1)(x - 2)(x + 7)$
    \item $(2x^2 - 3x - 4)(x^2 + x + 1)$
    \item $\del{x + 2}^3$
    \item Word problems 85--87
    \item etc.
  \end{itemize*}

\end{document}
