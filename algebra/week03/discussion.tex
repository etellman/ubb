\documentclass[fleqn]{article}
\usepackage{amsmath}

\title{Math 113--Week Two Discussion}
\author{}
\date{February 3, 2010}

\oddsidemargin 0in
\topmargin -0.5in
\textwidth 6.5in
\textheight 9in

\setlength{\mathindent}{1in}

\begin{document}

\maketitle

\section{Homework}

\begin{itemize}
\item 38 - positive integers greater than $-7$

\item repeating decimals are rational.  $3/11$ example

\item problem 3: L in terms of T means:

\(L = T^2\) not \(T = \sqrt{L}\)

Herb noticed interesting relationship.

\vspace{3 in}

\item show your work

\item extra credit was intended to imply that you should also describe the procedure, not just provide the number.

\item is there a paper shortage?

\item write out all the steps--don't try to do two steps at once in your head

\item always check your work.  you are only half done when you get the answer.

\item next week more and harder homework.  almost everyone is getting nearly 100\%

\item Rico's system for problems with ``two consecutive odd'' or ``two consecutive even''

\end{itemize}

\section{Schedule}

Finish chpater 2 next week.  Test following week.

\section{Linear Equations}

\begin{itemize}

\item Why are linear equations called ``linear?''

\item Equation with no solution: \( 2x + 3 = 2(2x + 2) - 2(x + 1) \)

\item Equation with every number as solution: \( 2x + 3 = 2(2x + 2) - 2(x + 1/2) \)

\end{itemize}

\section{More Equations With Fractions}

\begin{itemize}

\item \(2/3(2t + 1) - 1/2(3t - 2) = 2 \)

\( t = -2 \)

\item \( 3x - 1 + 2/7(7x - 2) = -11/7 \)

\(x = 0 \)

\item

\[\frac{2a - 3}{6} + \frac{3a - 2}{4} + \frac{5a + 6}{12} = 4 \]

\( a = 3 \)

\item common mistake is to mess up sign after clearing fraction:

\[ 3 - \frac{x + 2}{2} = 4 \]

\item page 59 problem 41

\(1/2 n = 2/3 n - 3 \)

\item page 59 problem 51

\(m = 2x \)

\[x + 12 = 5/8(2x + 12) \]

\(x = 18, m = 36\)

\end{itemize}

\section{ Equations With Decimals}

\begin{itemize}
\item
just like fractions--multiply by number big enough to get rid of decimals.  or just leave the decimals

\item (13 on p 66)
\(0.08(x + 200) = 0.07x + 20\)
\vspace{2 in}


\item (22 on p 66)
\(.10t + 0.12(t + 1000) = 560 \)
\vspace{2 in}


\item (problem 29)

Judy bought a coat at a 20\% discount sale for \$72.  What was the original price of the coat?

\vspace{2 in}


\item (problem 44)

A total of \$4,000 was invested, part of it at 8\% interest and the remainder at 9\%.  If the total yearly interest
amounted to \$350, how much was invested at each rate?

\vspace{2 in}
\end{itemize}

\section{Formulas}

\begin{itemize}
\item a formula seems to be an equation which a specific practical purpose like calculating the area of a square,
  interest, etc.

\item sometimes you need to solve the formula for one of the other variables, depending on what values you know.

\item p. 77, number 3
Solve \(i = Prt\) for $t$, given that P=\$400, r = 11\%, and i \ \$132.

\vspace{1 in}

\item p. 79, number 59

30\% solution of alcohol and 70\% solution of alcohol.  How many quarts of each to get 20 quarts of 40\%?

\item
voting case

WA is 4\% black

three-strikes population is 40\% black

prison population is 20\% black

other minority groups are over-represented

\vspace{3 in}

\item
solve for x:
\(x/a + y/b = 1 \)

\vspace{1 in}

\item
solve for x:
\(y = mx + b \)


\end{itemize}

\section{Puzzles}

\begin{itemize}

\item
chess board and dominoes

\item 
monk and mountain

\end{itemize}

\end{document}


