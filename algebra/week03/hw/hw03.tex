% no answer key
\documentclass[letterpaper]{exam}

% answer key
% \documentclass[letterpaper, landscape]{exam}
% \usepackage{2in1, lscape} 
% \printanswers{}

\usepackage{units} 
\usepackage{xfrac} 
\usepackage[fleqn]{amsmath}
\usepackage{commath}
\usepackage{cancel}
\usepackage{float}
\usepackage{mdwlist}
\usepackage{booktabs}
\usepackage{cancel}
\usepackage{polynom}
\usepackage{caption}
\usepackage{fullpage}
\usepackage{comment}
\usepackage{enumerate}
\usepackage{graphicx}
\usepackage{mathtools} 
\usepackage{parskip} 

\newcommand{\degree}{\ensuremath{^\circ}} 
\everymath{\displaystyle}

\title{Algebra \\ Homework Three \\ Sections 2.3 and 2.4}
\author{}
\date{\today}

\begin{document}
  \maketitle

  \section{Sections 2.3 and 2.4}
  \begin{itemize*}
    \item Section 2.3: 6--10, 15--20, 27--28, 30, 33, 43
    \item Section 2.4: 1, 5, 24, 28, 32, 35, 36, 52, 54, 57
  \end{itemize*}

  \section{Problems}

  \begin{questions}
    % \question{}
      % It would be a shame to let Super Bowl week go by without a prediction.  Here's my prediction.

      % At half time the Colts will have scored as many field goals as the Saints have scored touchdowns.  In the second
      % half, the Colts will score three touchdowns while holding the Saints to 9 points.  The Colts will win the game by
      % one point by making a two point conversion after the final touchdown as time expires.  Of course, since this is the
      % Super Bowl, nobody misses extra points or gets safeties, and the only two point conversion will be the one I
      % mentioned.

      % If my prediction comes true, what will the final score be?

      % \begin{solution}
      %   Let $x$ be the number of Colt field goals in the first half.

      %   The total Colt score for the game will be: $3x + 22$.

      %   The total Saint score for the game will be: $7x + 9$.

      %   The Colts win the game by a point:
      %   \begin{align*}
      %     7x + 9 + 1 & = 3x + 22 \\
      %     x          & = 3 \\
      %   \end{align*}

      %   The scoreboard might look something like:

      %   \begin{tabular}[H]{cccccc}
      %     \toprule
      %              & 1  & 2 & 3  & 4 & Total \\
      %     \midrule
      %       Saints & 14 & 7 & 6  & 3 & 30 \\
      %       Colts  & 0  & 9 & 14 & 8 & 31 \\
      %     \bottomrule
      %   \end{tabular}

      % \end{solution}

    \question{}
      A judge in Pennsylvania was having trouble paying his bills with his meager judicial salary.  When a prison
      moved into town, he saw a golden business opportunity.  He worked out a deal with the prison where, for a small
      fee, he would sentence juveniles to lengthy prison terms.  If a particular offense was usually punished with a
      fine, he would instead issue the maximum punishment so the defendant would enjoy some time in the new prison.

      It was a terrific arrangement for everyone.  Even with the rising prices of yachts, the judge was able to make
      ends meet.  The prison had plenty of residents.  And since they charged the state for each prisoner, the prison
      owners were thrilled. The modest investment in bribes paid off handsomely.

      \begin{parts}
        \part{}
          The judge wanted to make \$1,300,000 from the scheme.  He decided a reasonable plan would be to charge an
          initial down payment for his participation, and then a fee for each sentence handed out.  If the initial down
          payment was \$200,000, and he charged a bribe of \$2,000 per case, how many cases would he need to process to
          reach his goal?

          \begin{solution}
            Letting $x$ be the number of cases:
            \begin{align*}
              200,000 + 2,000x & = 1,300,000 \\
              x                & = \boxed{ 550 } \\
            \end{align*}

            check; 
            
            If he got bribes for 550 cases, and a \$200,000 down payment, he would make 
            \[
              \$200,000 + 550 \cdot \$2,000 = \$1,300,000
            \]
          \end{solution}

        \part{}
          Judging is hard work, even when you've already decided the outcome in advance, and the judge decided to retire
          a bit early.  He still wanted to make \$1,300,000 from the scheme, but he wanted to retire after only taking
          in \$1,000,000, investing his money to make up the difference.  

          He found a bank that paid 5\% simple interest ($i = Prt$) per year. Assuming he can manage to survive without
          spending any of the principal or interest, how long will he have to wait for his \$1,000,000 to turn into
          \$1,300,000?

          \begin{solution}
            Solving the equation for $t$ gives: $t = \frac{i}{P r}$.  
            
            He needs to make \$300,000 in interest to get to
            \$1.3M.  We can plug in the numbers for $i$, $P$, and $r$, to get:

              \begin{align*}
                t & = \frac{300,000}{1,000,000 \cdot .05}   \\
                t & = \frac{300,000}{50,000}   \\
                t & = \boxed{ \unit[6]{yr} } \\
              \end{align*}

              After six years of retirement, he will have the full \$1,300,000.

          \end{solution}

      \end{parts}

    % \question{}
    %   The equation for a Celsius temperature in terms of a Fahrenheit temperature is:
    %   \[
    %     C = \frac{5}{9} ( F - 32 ) 
    %   \]

    %   \begin{parts}
    %     \part{} What is the equation for a Fahrenheit temperature in terms of a Celsius temperature?
    %       \begin{solution}
    %         \begin{align*}
    %           C             & = \frac{5}{9} (F - 32) \\
    %           \frac{9}{5} C & = F - 32 \\
    %           F             & = \frac{9}{5} C + 32 \\
    %         \end{align*}
    %       \end{solution}

    %     \part{}
    %       There is one temperature where the temperature in Celsius is the same as the temperature in Fahrenheit.  In
    %       other words: $x$ degrees Fahrenheit = $x$ degrees Celsius.

    %       What is this temperature?

    %   \begin{solution}
    %     You can use either form of the equation.

    %     \begin{align*}
    %       x             & = \frac{5}{9} (x - 32) \\
    %       \frac{9}{5} x & = x - 32 \\
    %       \frac{4}{5} x & = -32 \\
    %       x             & = -40 \\
    %     \end{align*}
    %   \end{solution}

    %   \end{parts}

    \question{}
      Two police cars depart from the same point, traveling in opposite directions towards different donut shops. One
      car travels 6 mph faster than the other car.  Find the speed of each car if they are 176 miles apart at the end of
      55 minutes.

      \begin{solution}
        Let $x$ be the speed of the slower car.

        \begin{align*}
          \frac{55}{60} x + \frac{55}{60}(x + 6) & = 176 \\
          \frac{11}{12} (x + x + 6)              & = 176 \\
          2x + 6                                 & = \frac{12}{11} \cdot 176 \\
          x                                      & = 93 \\
        \end{align*}

        The slower car is traveling at \fbox{ 93 mph } and the faster car is traveling at \fbox{ 99 mph }.

      \end{solution}

    \ifprintanswers{}
    \else
      \newpage
    \fi

    \question{}
      Last week I was chatting with one of the UBB English professors.  She asked me if I'd heard the poem by Robert
      Browning which begins: {\em ``A rose-red city, half as old as time\ldots ''\/}

      ``Wait!'' I interrupted,  ``That would make a good algebra problem.''  I thought about it for a minute or two and
      came up with this poem:

      \begin{verbatim}
        A rose-red city, half as old as Time.
        One billion years ago, the city's age
        Was just two-fifths of what Time's age will be
        A billion years from now.  Can you compute
        How old the crimson city is today?
      \end{verbatim}

      She gave me a funny look and quickly changed the subject.  How old is the rose-red city?

      (problem actually by Martin Gardner)

      \begin{solution}
        If $T$ is the age of time:
        \begin{align*}
          \frac{T}{2} - 1 & = \frac{2}{5} (T + 1) \\
          T               & = 14 \\
        \end{align*}

        Time is 14 billion years old.  The city is half as old as time, so it is 7 billion years old.

        Astronomers estimate the universe actually is about 13.7 billion years old. The earth seems to only be
        about 4.5 billion years old, so the age of the city may be slightly exaggerated.

      \end{solution}
  \end{questions}

  \section{Extra Credit}

  A commuter is in the habit of arriving at his suburban station each evening exactly at 5:00.  His wife always meets
  the train and drives him home.  One day he takes an earlier train, arriving at the station at 4:00.  The weather is
  pleasant, so instead of telephoning home he starts walking along the route always taken by his wife.  They meet
  somewhere along the way.  He gets into the car and they drive home, arriving at their house ten minutes earlier than
  usual.  

  Assuming that the wife always drives at a constant speed, and that on this occasion she left just in time to meet
  the 5:00 train, how long did the husband walk before he was picked up?

  (Martin Gardner)

  \begin{solution}
    The wife took 10 minutes off her usual trip.  She must have picked her husband up 5 minutes early, saving 5
    minutes on the way to the station and 5 minutes on the way back.  She planned to pick him up at 5:00, so she must
    have instead picked him up at 4:55.  He started walking at 4:00, so he walked for 55 minutes.
  \end{solution}

  \ifprintanswers{}
      
    \section{Section 2.3}

    \begin{description}

      \item[6]
        \begin{align*}
          n - 0.5n & = 12 \\
          10n - 5n & = 120 \\
          n        & = \boxed{ 24 } \\
        \end{align*}

      \item[7]
        \begin{align*}
          s     & = 9 + 0.25 s \\
          100 s & = 900 + 25 s \\
          c     & = \boxed{ 12 } \\
        \end{align*}

      \item[8]
        \begin{align*}
          s   & = 15 + 0.4s \\
          10s & = 150 + 4s \\
          s   & = \boxed{ 25 } \\
        \end{align*}

      \item[9]
        \begin{align*}
          s    & = 3.3 + 0.45s \\
          100s & = 330 + 45 s \\
          s    & = \boxed{ 6 } \\
        \end{align*}

      \item[10]
        \begin{align*}
          s   & = 2.1 + 0.6s \\
          10s & = 21 + 6s \\
          s   & = \boxed{ 5.25 } \\
        \end{align*}

      \item[15]
        \begin{align*}
          0.12t - 2.1 & = 0.07t - 0.2 \\
          12t - 210   & = 7t - 20 \\
          t           & = \boxed{ 38 } \\
        \end{align*}

      \item[16]
        \begin{align*}
          0.13t - 3.4      & = 0.08t - 0.4 \\
          13t - 340        & = 8t - 40 \\
          t                & = \boxed{ 60 } \\
        \end{align*}

      \item[17]
        \begin{align*}
          0.92 + 0.9(x - 0.3)      & = 2x - 5.95 \\
          92 + 90(x - 0.3)         & = 200x - 595 \\
          x                        & = \boxed{ 6 }
        \end{align*}

      \item[18]
        \begin{align*}
          0.3(2n - 5) & = 11 - 0.65n \\
          30(2n - 5)  & = 1100 - 65n \\
          n           & = \boxed{ 10 } \\
        \end{align*}

      \item[19]
        \begin{align*}
          0.1d + 0.11 (d + 1500) & = 795 \\
          10d + 11 (d + 1500)    & = 795 \\
          d                      & = \boxed{ 3000 } \\
        \end{align*}

      \item[20]
        \begin{align*}
          0.8x + 0.9 (850 - x) & = 715 \\
          8x + 9(850 - x)      & = 7150 \\
          x                    & = \boxed{ 500 } \\
        \end{align*}

      \item[27]
        \begin{align*}
          0.1 (x - 0.1) - 0.4(x + 2) & = -5.31 \\
          10(x - 0.1) - 40(x + 2)    & = -531 \\
          10x - 1 - 40x - 80         & = -531 \\
          x                          & = \boxed{ 15 } \\
        \end{align*}

      \item[28]
        \begin{align*}
          0.2(x + 0.2) + 0.5(x - 0.4) & = 5.44 \\
          20(x + 0.2) + 50(x - 0.4)   & = 544 \\
          20x + 4 + 50x - 20          & = 544 \\
          x                           & = \boxed{ 8 } \\
        \end{align*}

      \item[30]
        \begin{align*}
          24 & = x - 0.25x \\
          24 & = .75x \\
          x  & = \boxed{ \$32 } \\
        \end{align*}

      \item[33] 
        If $x$ is the sale price:
        \begin{align*}
          x & = 30 + 0.6 \cdot 30 \\
          x & = \boxed{ \$48 } \\
        \end{align*}

      \item[43] Let $x$ be the amount invested at 10\%.
        \begin{align*}
          0.1x + .11(x + 1,500) & = 795 \\
          10x + 11(x + 1,500)   & = 79,500 \\
          x                    & = \boxed{ 3,000 } \\
        \end{align*}

      \end{description}
    \section{Section 2.4} % (fold)
    \begin{description}
      
      \item[1] Using $i = Prt$, $300(.08)(5) = \boxed{ 120 }$

      \item[5] Using \(r = \frac{i}{Pt}\), \(\frac{90}{600 \cdot 2.5} = \boxed{ 0.06 } \).

      \item[24]
        \begin{align*}
          A  & = \frac{h}{2} (b_1 + b_2) \\
          2A & = h(b_1 + b_2) \\
          h  & = \frac{2A}{b_1 + b_2} \\
        \end{align*}

      \item[28]
      \begin{align*}
        \frac{x}{a} + \frac{y}{b}           & = 1 \\
        a \del{ \frac{x}{a} + \frac{y}{b} } & = a  \\
        x + \frac{ya}{b}                    & = a  \\
        x                                   & = \boxed{ a - \frac{ya}{b} } \\
      \end{align*}

      An alternate form of the solution is: 
      \[
        \boxed{ x = \frac{ab - ay}{b} } 
      \]

      \item[32]
        \begin{align*}
          x(a-b)   & = m(x-c) \\
          xa-xb    & = mx-mc \\
          xa-xb-mx & = -mc \\
          x(a-b-m) & = -mc \\
          x        & = \frac{-mc}{a-b-m} \\
        \end{align*}

      An alternate form of the solution is: 
      \[
        \boxed{ x = \frac{mc}{m - a + b} } 
      \]

      \item[35]
      \begin{align*}
        \frac{1}{3}x + a & = \frac{1}{2}b \\
        \frac{1}{3}x     & = \frac{1}{2}b - a \\
        3(\frac{1}{3}x)  & = 2(\frac{1}{2}b - a) \\
        x                & = \boxed{ \frac{3}{2}b - 3a } \\
      \end{align*}

      An alternate form of the solution is: 
      \[
        \boxed{ x = \frac{3b - 6a}{2} } 
      \]

      \item[36]
      \begin{align*}
        \frac{2}{3}x - \frac{1}{4}a    & = b \\
        \frac{2}{3}x                   & = b + \frac{1}{4}a \\
        2x                             & = 3b + \frac{3}{4}a \\
        x                              & = \boxed{ \frac{3}{2} b + \frac{3}{8} a } \\
      \end{align*}

      An alternate form of the solution is: 
      \[
        \boxed{ x = \frac{12b + 3a}{8} }
      \]

      \item[52] We can use $i = Prt$ with $r = 0.1$ and $i = 2P$.  We need three times as much money, and we already have
        the principal so we need the interest to supply twice the principal.  I think you could also interpret this
        problem as meaning you want the interest to be three times the principal.

        \begin{align*}
          t & = \frac{2P}{P \cdot 0.1} \\
          t & = \frac{2}{0.1} \\
          t & = \boxed{ 20 } \\
        \end{align*}

      \item[54] The two cyclists are approaching each other at the sum of their two speeds, or \(18 + 14 = 32\) miles per
        hour.  The distance is 112 miles.  Using $t = \frac{d}{r}$, 
        
        \[
          t = \frac{112}{32} = \boxed{ \unit[3.5]{h} }
        \]

        check:
        
        In 3.5 hours the faster cyclist travels 63 miles and the slower cyclist travels 49 miles.  
        $63 + 49 = 112$

      \item[57] Let $x$ be the time at 20 mph and $4.5 - x$ (the rest of the time) be the time he spent at 12 mph.  Since he
        traveled 70 miles altogether:

        \begin{align*}
          20x + 12(4.5 - x) & = 70 \\
          x                 & = \boxed{ 2 } \\
        \end{align*}

      He spent 2 hours at 20 mph and 2.5 hours at 12 mph.  

      The question asks how far he traveled at each speed, not how long he spent at each speed.  We need to plug these
      numbers into $d = rt$, which gives 
      
      \fbox{ 40 miles at 20 mph and 30 miles at 12 mph }.  

  \end{description}
  \fi

  \ifprintanswers{}
  \else
    \vspace{8 cm}
    \begin{quote}
      \begin{em}
        Society in every state is a blessing, but government, even in its best state, is but a
        necessary evil; in its worst state an intolerable one. 
      \end{em}
    \end{quote}
    \hspace{1 cm} --Thomas Paine
  \fi

\end{document}
