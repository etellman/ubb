% no answer key
\documentclass[letterpaper]{exam}

% answer key
% \documentclass[letterpaper, landscape]{exam}
% \usepackage{2in1, lscape} 
% \printanswers{}

\usepackage{units} 
\usepackage{xfrac} 
\usepackage[fleqn]{amsmath}
\usepackage{commath}
\usepackage{cancel}
\usepackage{float}
\usepackage{mdwlist}
\usepackage{booktabs}
\usepackage{cancel}
\usepackage{polynom}
\usepackage{caption}
\usepackage{fullpage}
\usepackage{comment}
\usepackage{enumerate}
\usepackage{graphicx}
\usepackage{mathtools} 
\usepackage{parskip} 

\newcommand{\degree}{\ensuremath{^\circ}} 
\everymath{\displaystyle}

\addpoints{}

\title{Algebra \\ Chapter Two Exam}
\author{}
\date{\today}

\begin{document}

  \maketitle

  % \begin{center}
  %   \gradetable[h][pages]
  %   \bonusgradetable[h][pages]
  % \end{center}

  \section{Instructions} % (fold)

  \begin{itemize*}
    \item Show your work.
    \item Check your answers.
    \item Write the solutions to inequalities in interval form. For instance, if the answer is 
      $x \geq 2$, the solution should be written as $\intco{ 2, \infty }$.
  \end{itemize*}

  \section{Questions} % (fold)

  \begin{questions}
    
    % \question[5]
    %   \[ 
    %     2 (6 + x) = 9 - x
    %   \]

    \question[5]
      \[ 
        17 - 3(5 - 2x) = 8 + 4x
      \]

    \question[5]
      \[
        3(x - 2) - 6 = 4x - (3x - 1)
      \]

    \question[10]
      \[ 
        \frac{7}{3} (x + 2) = \frac{3}{4} (x + 3)
      \]

    \question[10]\label{e:first}
      \[ 
        \frac{4x - 8}{3} = \frac{4}{5} (x - 10)
      \]

    % \question[7]
    %   \[ 
    %     \frac{3}{4} (x + 6) = \frac{5}{12} \del{ x - \frac{2}{5} }
    %   \]

    % \question[7]
    %   \[ 
    %     \frac{2}{3} (x + 9) = \frac{1}{10} (x - 14)
    %   \]

    % \question[10]
    %   \[
    %     \frac{4}{3} (x - 2) - \frac{1}{2} = 2 \del{ \frac{3}{4} x - 1 }
    %   \]

    \question[10]
      \[
        \frac{3}{5} - \frac{4}{5} (x + 1) = 2 - \frac{3}{10} (3x - 4)
      \]

    \question[10] 
    \[
      0.02 (x - 0.5) - 0.04 (x + 0.25) = 0.6
    \]

    \question[5]
      \[
        2x + 1 < 3x
      \]

      \begin{solution}
        \begin{align*}
          2x + 1 & < 3x \\
          1      & < x \\
          x      & > 1 \\
        \end{align*}
        In interval form: $\intoo{ 1, \infty }$
    \end{solution}

    \question[10]
      \[ 
        \frac{x - 2}{3} - \frac{x + 1}{4} \leq \frac{3}{2} 
      \]

      \begin{solution}
        \begin{align*}
          \frac{x - 2}{3} - \frac{x + 1}{4}            & \leq \frac{3}{2} \\
          12 \del{ \frac{x - 2}{3} - \frac{x + 1}{4} } & \leq 12 \del{ \frac{3}{2} } \\
          4(x - 2) - 3(x + 1)                          & \leq 18 \\
          4x - 8 - 3x - 3                              & \leq 18 \\
          x - 11                                       & \leq 18 \\
          x                                            & \leq 29 \\
        \end{align*}
        In interval form: $\intoc{ -\infty, 29 }$
      \end{solution}

    \question[10]
      \[ 
         \frac{3}{2} - \frac{1}{2} x < \frac{3}{2} \del{ x + \frac{1}{3} }  < \frac{18 - 2x}{4}
      \]

    \question[10]
      \[ 
        2x + 1 < \frac{1}{3} \text { \hspace{0.1cm} or \hspace{0.1cm} } 3x - \frac{3}{2} > 1 
      \]

      \begin{solution}
        \begin{align*}
          2x + 1 < \frac{1}{3} & \text{ or } 3x - \frac{3}{2} > 1 \\
          2x < -\frac{2}{3}    & \text{ or } 3x > \frac{5}{2} \\
          x < -\frac{1}{3}     & \text{ or } x > \frac{5}{6} \\
        \end{align*}
        In interval form: $ \intoo{ -\infty, -\frac{1}{3} } \cup \intoo{ \frac{5}{6}, \infty }$
      \end{solution}

    \question[10]
      \[ 
        \frac{-2x + 4}{3} \leq 2 \text{ \hspace{0.1cm} and \hspace{0.1cm} } \frac{2x + 7}{9} < 3 
      \]

      \begin{solution}
        \begin{align*}
          \frac{-2x + 4}{3} \leq 2                   & \text{ and } \frac{2x + 7}{9} \leq 3 \\
          3 \del{ \frac{-2x + 4}{3} } \leq 3 \cdot 2 & \text{ and } 9 \del{ \frac{2x + 7}{9} } \leq 9 \cdot 3 \\
          -2x + 4 \leq 6                             & \text{ and } 2x + 7 \leq 27 \\
          -2x \leq 2                                 & \text{ and } 2x \leq 20 \\
          x \geq 1                                   & \text{ and } x \leq 10 \\
        \end{align*}
        In interval form: $\intcc{ 1, 10 }$
      \end{solution}

    \question[10]
      \[ 
        \abs{ 2x - 3 } < 4 
      \]

      \begin{solution}
        \begin{align*}
          -(2x - 3) < 4    & \text{ and } 2x - 3 < 4 \\
          -2x + 3 < 4      & \text{ and } 2x < 7 \\
          -2x < 1          & \text{ and } x < \frac{7}{2} \\
          x > -\frac{1}{2} & \text{ and } x < \frac{7}{2} \\
        \end{align*}
        In interval form: $\intoo{ -\frac{1}{2}, \frac{7}{2} }$
      \end{solution}

      \question[10]
        \[ 
          \abs{ \frac{x - 1}{3} } - 1 \geq 2 
        \]
        \begin{solution}
          \begin{align*}
            -\del{ \frac{x - 1}{5} } \geq 2 & \text{ or } \frac{x - 1}{5} \geq 2 \\
            \frac{1 - x}{5} \geq 2          & \text{ or } x - 1 \geq 10 \\
            1 - x \geq 10                   & \text{ or } x \geq 11 \\
            -x \geq 9                       & \text{ or } x \geq 11 \\
            x \leq -9                       & \text{ or } x \geq 11 \\
          \end{align*}
          In interval form: $\intoc{ -\infty, -9 } \cup \intco{ 11, \infty }$
        \end{solution}

    % \question[10] Solve for $m$:
    %   \[ 
    %     y = mx + b
    %   \]

    %   \begin{solution}
    %     \begin{align*}
    %       y  & = mx + b \\
    %       mx & = y - b \\
    %       m  & = \frac{y - b}{x}
    %     \end{align*}
    %   \end{solution}

    \question[10]
    Eloise Midgen and Daphne Greengrass both accept new jobs immediately after graduating from
    {\em Hogwarts School of Witchcraft and Wizardry}. 
    
    Here are their contracts:

    \begin{itemize*}
      \item Eloise: 500 galleon signing bonus with a monthly salary of 200 galleons. 
      \item Daphne: 200 galleon signing bonus with a monthly salary of 300 galleons. 
    \end{itemize*}

    % For those of you who aren't familiar with the wizarding world, a galleon is the wizard currency.

    \begin{parts}
      \part[8 ]When will they both have made the same number of galleons from their jobs?
      
      \part[2] Who has the better deal in the long run?
    \end{parts}

    \question[15]
    After working for a few months, Daphne has saved up some money. She has 140 galleons to
    invest. She plans to put some of it in an account paying 6\% simple interest and the remainder
    in an account paying 9\% simple interest. 
    
    How much should she put in each account in order to make 10.05 galleons in interest per year?

    \begin{solution}
      If you let $x$ be the amount to put in the 6\% account:
      \begin{align*}
        0.06x + 0.09( 140 - x ) & = 10.05 \\
        x                       & = 85 \\
      \end{align*}
    \end{solution}

    % \question[7]
    %   An accountant has determined that the cost to produce and sell $x$ broomsticks is:
    %   \[
    %     C = 20x + 1000
    %   \]

    %   The revenue from $x$ broomsticks is:
    %   \[
    %     R = 70x
    %   \]

    %   How many broomsticks must the company sell in order to break even?

    \uplevel{ \section{Extra Credit} }

    \bonusquestion[15] 
      Daphne and Eloise use some of their earnings to buy flying brooms. 
      
      Daphne's broom's top speed is 20 mph slower than Eloise's broom so she is only able to travel
      8 miles in the time it takes Eloise to travel 18 miles. 
      
      What is the top speed of each broom?

      \begin{solution}
        If Daphne's broom speed is $x$, Eloise's broom speed is $x + 20$. Using 
        \begin{align*}
          d & = rt \\
          t & = \frac{d}{r} \\
        \end{align*}

        \begin{align*}
          \frac{x}{8}          & = \frac{x + 20}{18} \\
          72 \cdot \frac{x}{8} & = 72 \cdot \frac{x + 20}{18} \\
          9x                   & = 4 (x + 20) \\
          % 9x                   & = 4x + 80 \\
          x                    & = 16 \\
        \end{align*}

        Daphne's broom has a top speed of \fbox{ 16 mph } and Eloise's broom has a top speed of
        \fbox{ 36 mph }.

      \end{solution}

  \end{questions}

\end{document}


