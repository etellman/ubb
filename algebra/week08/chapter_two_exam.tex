% no answer key
% \documentclass[letterpaper]{exam}

% answer key
\documentclass[letterpaper, landscape]{exam}
\usepackage{2in1, lscape} 
\printanswers{}

\usepackage{units} 
\usepackage{xfrac} 
\usepackage[fleqn]{amsmath}
\usepackage{commath}
\usepackage{cancel}
\usepackage{float}
\usepackage{mdwlist}
\usepackage{booktabs}
\usepackage{cancel}
\usepackage{polynom}
\usepackage{caption}
\usepackage{fullpage}
\usepackage{comment}
\usepackage{enumerate}
\usepackage{graphicx}
\usepackage{mathtools} 
\usepackage{parskip} 

\newcommand{\degree}{\ensuremath{^\circ}} 
\everymath{\displaystyle}

\addpoints{}

\title{Algebra \\ Chapter Two Exam}
\author{}
\date{\today}

\begin{document}

  \maketitle

  % \begin{center}
  %   \gradetable[h][pages]
  %   \bonusgradetable[h][pages]
  % \end{center}

  \ifprintanswers{}
  \else
    \section{Instructions} % (fold)

    \begin{itemize*}
      \item Show your work.
      \item Check your answers.
      \item Write the solutions to inequalities in interval form. For instance, if the answer is 
        $x \geq 2$, the solution should be written as $\intco{ 2, \infty }$.
    \end{itemize*}
  \fi

  \section{Questions} % (fold)

  \begin{questions}
    
    % \question[5]
    %   \[ 
    %     2 (6 + x) = 9 - x
    %   \]

    \question[5]
      \[ 
        17 - 3(5 - 2x) = 8 + 4x
      \]

      \begin{solution}
        \begin{align*}
          17 - 3(5 - 2x) & = 8 + 4x \\
          17 - 15 + 6x   & = 8 + 4x \\
          2 + 6x         & = 8 + 4x \\
          2x             & = 6 \\
          x              & = \boxed{ 3 } \\
        \end{align*}
      \end{solution}

    \question[5]
      \[
        3(x - 2) - 6 = 4x - (3x - 1)
      \]

      \begin{solution}
        \begin{align*}
        3(x - 2) - 6 & = 4x - (3x - 1) \\
        3x - 6 - 6   & = 4x - 3x + 1 \\
        3x - 12      & = x + 1 \\
        x            & = \boxed{ \frac{13}{2} } \\
        \end{align*}
      \end{solution}

    \question[10]
      \[ 
        \frac{7}{3} (x + 2) = \frac{3}{4} (x + 3)
      \]

      \begin{solution}
        \begin{align*}
          \frac{7}{3} (x + 2) & = \frac{3}{4} (x + 3) \\
          28 (x + 2)          & = 9 (x + 3) \\
          28x + 56            & = 9x + 27 \\
          19x                 & = -29 \\
          x                   & = \boxed{ - \frac{29}{19} } \\
        \end{align*}
      \end{solution}

    \ifprintanswers{}
      \newpage
    \fi

    \question[10]
      \[ 
        \frac{4x - 8}{3} = \frac{4}{5} (x - 10)
      \]

      \begin{solution}
        \begin{align*}
          \frac{4x - 8}{3} & = \frac{4}{5} (x - 10) \\
          5 (4x - 8)       & = 12 (x - 10) \\
          20x - 40         & = 12x - 120 \\
          8x               & = -80 \\
          x                & = \boxed{ -10 } \\
        \end{align*}
      \end{solution}

    \question[10]
      \[
        \frac{3}{5} - \frac{4}{5} (x + 1) = 2 - \frac{3}{10} (3x - 4)
      \]

      \begin{solution}
        \begin{align*}
          \frac{3}{5} - \frac{4}{5} (x + 1) & = 2 - \frac{3}{10} (3x - 4) \\
          6 - 8(x + 1)                      & = 20 - 3 (3x - 4) \\
          6 - 8x - 8                        & = 20 - 9x + 12 \\
          - 8x - 2                          & = 32 - 9x \\
          x                                 & = \boxed{ 34 } \\
        \end{align*}
      \end{solution}

    \question[10] 
      \[
        0.02 (x - 0.5) - 0.04 (x + 0.25) = 0.6
      \]

      \begin{solution}
        \begin{align*}
          0.02 (x - 0.5) - 0.04 (x + 0.25) & = 0.6 \\
          2 (x - 0.5) - 4 (x + 0.25)       & = 60 \\
          2x - 1 - 4x - 1                  & = 60 \\
          -2x - 2                          & = 60 \\
          -2x                              & = 62 \\
          x                                & = \boxed{ -31 } \\
        \end{align*}
      \end{solution}

    \question[5]
      \[
        2x + 1 < 3x
      \]

      \begin{solution}
        \begin{align*}
          2x + 1 & < 3x \\
          x      & > 1 \\
        \end{align*}
        \[
          \boxed{ \intoo{ 1, \infty } }
        \]
    \end{solution}

    \ifprintanswers{}
      \newpage
    \fi

    \question[10]
      \[ 
        \frac{x - 2}{3} - \frac{x + 1}{4} \leq \frac{3}{2} 
      \]

      \begin{solution}
        \begin{align*}
          \frac{x - 2}{3} - \frac{x + 1}{4}            & \leq \frac{3}{2} \\
          12 \del{ \frac{x - 2}{3} - \frac{x + 1}{4} } & \leq 12 \del{ \frac{3}{2} } \\
          4(x - 2) - 3(x + 1)                          & \leq 18 \\
          4x - 8 - 3x - 3                              & \leq 18 \\
          x - 11                                       & \leq 18 \\
          x                                            & \leq 29 \\
        \end{align*}
        
        \[
          \boxed{ \intoc{ -\infty, 29 } }
        \]

      \end{solution}

    \ifprintanswers{}
      \newpage
    \fi

    \question[10]
      \[ 
         \frac{3}{2} - \frac{1}{2} x < \frac{3}{2} \del{ x + \frac{1}{3} }  < \frac{18 - 2x}{4}
      \]

      \begin{solution}
        \begin{align*}
           \frac{3}{2} - \frac{1}{2} x & < \frac{3}{2} \del{ x + \frac{1}{3} } \\
           3 - x                       & < 3x + 1 \\
           2                           & < 4x \\
           x                           & > \frac{1}{2} \\
           \\
           \frac{3}{2} \del{ x + \frac{1}{3} } & < \frac{18 - 2x}{4} \\
           6 \del{ x + \frac{1}{3} }           & < 18 - 2x \\
           6x + 2                              & < 18 - 2x \\
           8x                                  & < 16 \\
           x                                   & < 2 \\
        \end{align*}

        \[
          \boxed{ \intoo{ \frac{1}{2}, 2 } }
        \]

      \end{solution}

    \ifprintanswers{}
      \newpage
    \fi

    \question[10]
      \[ 
        2x + 1 < \frac{1}{3} \text { \hspace{0.1cm} or \hspace{0.1cm} } 3x - \frac{3}{2} > 1 
      \]

      \begin{solution}
        \begin{align*}
          2x + 1 < \frac{1}{3} & \text{ or } 3x - \frac{3}{2} > 1 \\
          2x < -\frac{2}{3}    & \text{ or } 3x > \frac{5}{2} \\
          x < -\frac{1}{3}     & \text{ or } x > \frac{5}{6} \\
        \end{align*}
        \[
          \boxed{ \intoo{ -\infty, -\frac{1}{3} } \cup \intoo{ \frac{5}{6}, \infty } }
        \]
      \end{solution}

    \question[10]
      \[ 
        \frac{-2x + 4}{3} \leq 2 \text{ \hspace{0.1cm} and \hspace{0.1cm} } \frac{2x + 7}{9} < 3 
      \]

      \begin{solution}
        \begin{align*}
          \frac{-2x + 4}{3} \leq 2                   & \text{ and } \frac{2x + 7}{9} \leq 3 \\
          % 3 \del{ \frac{-2x + 4}{3} } \leq 3 \cdot 2 & \text{ and } 9 \del{ \frac{2x + 7}{9} } \leq 9 \cdot 3 \\
          -2x + 4 \leq 6                             & \text{ and } 2x + 7 \leq 27 \\
          -2x \leq 2                                 & \text{ and } 2x \leq 20 \\
          x \geq -1                                   & \text{ and } x \leq 10 \\
        \end{align*}

        \[
          \boxed{ \intcc{ -1, 10 } }
        \]

      \end{solution}

    \ifprintanswers{}
      \newpage
    \fi

    \question[10]
      \[ 
        \abs{ 2x - 3 } < 4 
      \]

      \begin{solution}
        \begin{align*}
          -(2x - 3) < 4    & \text{ and } 2x - 3 < 4 \\
          -2x + 3 < 4      & \text{ and } 2x < 7 \\
          -2x < 1          & \text{ and } x < \frac{7}{2} \\
          x > -\frac{1}{2} & \text{ and } x < \frac{7}{2} \\
        \end{align*}

        \[
          \boxed{ \intoo{ -\frac{1}{2}, \frac{7}{2} } }
        \]

      \end{solution}

    \ifprintanswers{}
      \newpage
    \fi

      \question[10]
        \[ 
          \abs{ \frac{x - 1}{3} } - 1 \geq 2 
        \]
        \begin{solution}
          The first thing you need to do is get the absolute value by itself on the left side of
          the equation. After you do this you can turn the original problem into two problems and
          solve them both.
          \begin{align*}
            \abs{ \frac{x - 1}{3} } - 1 & \geq 2 \\
            \abs{ \frac{x - 1}{3} }     & \geq 3 \\
            \\
            \frac{x - 1}{3} & \geq 3 \\
            x - 1           & \geq 9 \\
            x               & \geq 10 \\
            \\
            - \frac{x - 1}{3} & \geq 3 \\
            1 - x             & \geq 9 \\
            - x               & \geq 8 \\
            x                 & \leq -8 \\
          \end{align*}
          \[
            \boxed{ \intoc{ -\infty, -8 } \cup \intco{ 10, \infty } }
          \]

        \end{solution}

    % \question[10] Solve for $m$:
    %   \[ 
    %     y = mx + b
    %   \]

    %   \begin{solution}
    %     \begin{align*}
    %       y  & = mx + b \\
    %       mx & = y - b \\
    %       m  & = \frac{y - b}{x}
    %     \end{align*}
    %   \end{solution}

    \ifprintanswers{}
      \newpage
    \fi

    \question{}
    Eloise Midgen and Daphne Greengrass both accept new jobs immediately after graduating from
    {\em Hogwarts School of Witchcraft and Wizardry}. 
    
    Here are their contracts:

    \begin{itemize}
      \item Eloise: 500 galleon signing bonus with a monthly salary of 200 galleons. 
      \item Daphne: 200 galleon signing bonus with a monthly salary of 300 galleons. 
    \end{itemize}

    % For those of you who aren't familiar with the wizarding world, a galleon is the wizard currency.

    \begin{parts}
      \part[8 ]When will they both have made the same number of galleons from their jobs?
        \begin{solution}
          If $t$ is the number of months worked:
          \begin{align*}
            500 + 200t & = 200 + 300t \\
            100t       & = 300 \\
            t          & = \boxed{ \unit[3]{months} } \\
          \end{align*}
        \end{solution}
      
      \part[2] Who has the better deal in the long run?
        \begin{solution}
          After 3 months Daphne will make \newline 100 galleons more per month.
        \end{solution}

    \end{parts}

    \ifprintanswers{}
      \newpage
    \fi

    \question[15]
    After working for a few months, Daphne has saved up some money. She has 140 galleons to
    invest. She plans to put some of it in an account paying 6\% simple interest and the remainder
    in an account paying 9\% simple interest. 
    
    How much should she put in each account in order to make 10.05 galleons in interest per year?

    \begin{solution}
      If you let $x$ be the amount to put in the 6\% account:
      \begin{align*}
        0.06x + 0.09( 140 - x ) & = 10.05 \\
        x                       & = \boxed{ 85 } \\
      \end{align*}
    \end{solution}

    % \question[7]
    %   An accountant has determined that the cost to produce and sell $x$ broomsticks is:
    %   \[
    %     C = 20x + 1000
    %   \]

    %   The revenue from $x$ broomsticks is:
    %   \[
    %     R = 70x
    %   \]

    %   How many broomsticks must the company sell in order to break even?

    \uplevel{ \section{Extra Credit} }

    \bonusquestion[15] 
      Daphne and Eloise use some of their earnings to buy flying brooms. 
      
      Daphne's broom's top speed is 20 mph slower than Eloise's broom so she is only able to travel
      8 miles in the time it takes Eloise to travel 18 miles. 
      
      What is the top speed of each broom?

      \ifprintanswers{}
        \newpage
      \fi

      \begin{solution}
        \begin{align*}
          \frac{r_{d}}{8}          & = \frac{r_{d} + 20}{18} \\
          72 \cdot \frac{r_{d}}{8} & = 72 \cdot \frac{r_{d} + 20}{18} \\
          9r_{d}                   & = 4 (r_{d} + 20) \\
          r_{d}                    & = 16 \\
        \end{align*}

        Daphne's broom has a top speed of \fbox{ 16 mph } and Eloise's broom
        has a top speed of \fbox{ 36 mph }.
      \end{solution}

  \end{questions}

\end{document}


