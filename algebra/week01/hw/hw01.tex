% no answer key
\documentclass[letterpaper]{exam}

% answer key
% \documentclass[letterpaper, landscape]{exam}
% \usepackage{2in1, lscape} 
% \printanswers

\usepackage{units} 
\usepackage{xfrac} 
\usepackage[fleqn]{amsmath}
\usepackage{commath}
\usepackage{cancel}
\usepackage{float}
\usepackage{mdwlist}
\usepackage{booktabs}
\usepackage{cancel}
\usepackage{polynom}
\usepackage{caption}
\usepackage{fullpage}
\usepackage{comment}
\usepackage{enumerate}
\usepackage{graphicx}
\usepackage{mathtools} 

\newcommand{\degree}{\ensuremath{^\circ}} 
\everymath{\displaystyle}

\title{Algebra \\ Homework One \\ Section 2.1}
\author{}
\date{\today}

\begin{document}

    \maketitle

    \section{Section 2.1}
    Problems 1--5, 11--15, 21--25, 31--35, 41--45, 49--50, 59--60, 63, 66

    \ifprintanswers{}
        \begin{description}
            \item[1] $x = \boxed{ 4 }$
  
            \item[2] $x = \boxed{ 5 }$
  
            \item[3] $x = \boxed{ -3 }$
  
            \item[4] $x = \boxed{ -5 }$
  
            \item[5] $x = \boxed{ -14 }$
  
            \item[11] $x = \boxed{ 5 }$
  
            \item[12] $a = \boxed{ -4 }$
  
            \item[13] $y = \boxed{ - \frac{10}{3} }$
  
            \item[14] $y = \boxed{ - \frac{5}{2} }$
  
            \item[15] $x = \boxed{ 4 }$
  
            \item[21] $a = \boxed{ 8 }$
  
            \item[22] $a = \boxed{ 15 }$
  
            \item[23] $x = \boxed{ -9 }$
  
            \item[24] $x = \boxed{ -6 }$
  
            \item[25] $y = \boxed{ -3 }$
  
            \item[31] $x = \boxed{ -2 }$
  
            \item[32] $x = \boxed{ -7 }$
  
            \item[33] $x = \boxed{ - \frac{5}{3} }$
  
            \item[34] $x = \boxed{ - \frac{7}{5} }$
  
            \item[35] $x = \boxed{ \frac{33}{2} }$
  
            \item[41] $x = \boxed{ \frac{1}{6} }$
  
            \item[42] $x = \boxed{ - \frac{23}{4} }$
              
            \item[43] $x = \boxed{ 5 }$
  
            \item[44] $x = \boxed{ -17 }$
  
            \item[45] $x = \boxed{ -1 }$
  
            \item[49] $n = \boxed{ \frac{12}{7} }$
  
            \item[50] $n = \boxed{ - \frac{12}{11} }$
  
            \item[59]
                If his normal hourly rate is $r$, his pay is $p$, and the number of hours worked is $h$, the
                equation for his pay when he has worked overtime is:
                \[
                  p = 40r + 2r(h - 40)
                \]
  
                For this particular week:
                \begin{align*}
                  572 & = 40r + 2r(46 - 40) \\
                  r   & = \boxed{ \unit[\$11]{/hr} } \\
                \end{align*}
  
            \item[60]
                If the rate is $r$:
                \begin{align*}
                  130 & = 25 + 5r \\
                  r   & = \boxed{ \unit[\$21]{/hr} } \\
                \end{align*}
  
            \item[63]
                If the cost is $c$:
                \begin{align*}
                  750 & = 3c - 150 \\
                  c   & = \boxed{ \$ 200 } \\
                \end{align*}
  
            \item[66]
                If the number of bicycles is $b$ and he wants to make \$750 in the month:
                \begin{align*}
                  750 & = 300 + 15b \\
                  b   & = \boxed{ \unit[30]{bicycles} } \\
                \end{align*}
  
        \end{description}

    \fi

    \ifprintanswers{}
        \newpage
    \fi

    % \section{Problems}

    % \begin{questions}

    %   \question

    %     In 1772, the German astronomer Johann Elert Bode found a pattern in the
    %     distances of the planets from the sun.  At the time, only six planets
    %     were known.  The actual relative distances of the planets from the sun
    %     and his pattern are:

    %     \begin{tabular}{lrr}
    %       \toprule
    %       Planet  & Actual Distance & Bode's Pattern \\ 
    %       \midrule
    %       Mercury & 4               & \( 0 + 4 = 4 \) \\
    %       Venus   & 7               & \( 3 + 4 = 7 \) \\
    %       Earth   & 10              & \( 6 + 4 = 10 \) \\
    %       Mars    & 15              & \( 12 + 4 = 16 \) \\
    %       ?       & ?               & ? \\
    %       Jupiter & 52              & \( 48 + 4 = 52 \) \\
    %       Saturn  & 96              & \( 96 + 4 = 100 \) \\
    %       \bottomrule
    %     \end{tabular}

    %     \begin{parts}
    %       \part What equation belongs between Bode's equations for Mars and Jupiter?

    %       \begin{solution}
    %         After the first equation, the equations all look like: 
    %         \[ 
    %           2^n \cdot 3 + 4 
    %         \]
    %         where n = \{ 0, 1, 2, \ldots \}.  

    %         The missing equation is: $24 + 4 = 28$

    %       \end{solution}
    %       \part What equation belongs after Bode's equation for Saturn?

    %       \begin{solution}
    %         Using the same reasoning, the missing equation is: $192 + 4 = 196$

    %         In 1781 William Herschel discovered Uranus, the next planet beyond Saturn.  Because its
    %         distance of 192 units comes remarkably close to the number predicted by the equation,
    %         astronomers came to the conclusion that the equation between Bode's equations for Mars and
    %         Jupiter must also mean something.  In fact, in 1801, the asteroid Ceres was discovered at
    %         a distance of 28 units from the sun.

    %       \end{solution}

    %     \end{parts}

    %   \question

    %     In the late sixteenth century, the Italian scientist Galileo discovered an interesting
    %     relationship between the length of a pendulum and the time of the swing:

    %     \begin{tabular}{cc}
    %       \toprule
    %       Length of Pendulum & Time of Swing \\
    %       \midrule
    %       1 unit             & 1 second \\
    %       4 units            & 2 seconds \\
    %       9 units            & 3 seconds \\
    %       16 units           & 4 seconds \\
    %       \bottomrule
    %     \end{tabular}

    %     \begin{parts}
    %     \part
    %       Write an equation for the length of the pendulum in terms of the time of the swing.

    %       \begin{solution}
    %         The length is always the time of the swing squared.  So the equation is: 
    %         \[
    %           \boxed{ L = T^2 } 
    %         \]
    %       \end{solution}

    %     \part
    %       What would be length of a pendulum with a swing of 8 seconds?

    %       \begin{solution}
    %         Substituting 8 into the equation gives: 
    %         \begin{align*}
    %           L & = T^2 \\
    %             & = 8^2 \\
    %             & = \boxed{ 64 } \\
    %         \end{align*}
    %       \end{solution}

    %     \end{parts}

    \ifprintanswers{}
        \newpage
    \fi

    \section{Extra Credit}

    \begin{questions}
  
        \question{}
          You have 9 metal balls and a balance which allows you to compare the weight of two balls or
          two groups of balls.  You know that one of the balls is slightly heaver than the others but
          don't know which one is the heavier ball.  What is the smallest number of weighings required
          to identify the heavier ball?
  
          \begin{solution}
              To minimize the number of weighings, you want to eliminate as many balls as possible with each
              weighing.  This approach allows you to eliminate two thirds of the balls each time, resulting
              in two weighings.
    
              \begin{itemize}
                \item 
                  Divide the balls into three groups of three.  Select two of the groups and put them on
                  the balance.  If one of the groups is heavier, the heavy ball is in that group.
                  Otherwise, the heavy ball is in the group you didn't weigh.
    
                \item 
                  Select two balls from the group containing the heavy ball and put them on the balance.
                  If one of them is heavier, it, obviously, is the heavy ball.  Otherwise, the third ball
                  is the heavy ball. 
    
              \end{itemize}
    
              This procedure works for any number of balls that is a power of three. The number of
              weighings required is equal to the exponent:
    
              \begin{tabular}{rrr}
                  \toprule
                  Exponent & Number of Balls & Number of Weighings \\
                  \midrule
                  0        & $3^0 = 1$       & 0 \\
                  1        & $3^1 = 3$       & 1 \\
                  2        & $3^2 = 9$       & 2 \\
                  3        & $3^3 = 27$      & 3 \\
                  4        & $3^4 = 81$      & 4 \\
                  \vdots   & \vdots          & \vdots \\
                  \bottomrule
              \end{tabular}
    
          \end{solution}
  
        \question{}
  
          You throw away your balance and buy a scale instead.  But you are now faced with a different
          problem.  
  
          You have 10 stacks of coins, each of which contains 10 coins.  You know that one stack is full
          of counterfeit coins while all the other stacks contain real coins.  You also know that a real
          coin weighs 10 ounces while a counterfeit coin weighs 11 ounces.  What is the smallest number
          of weighings required to identify the stack of counterfeit coins?
  
          \begin{solution}
  
            The best approach only requires one weighing.  The key is to notice that you can select
            different numbers of coins from each stack.  So what you do is select 1 coin from the
            first stack, 2 coins from the second stack, 3 coins from the third stack, etc.\ and weigh
            them all at once.
  
            If all of the coins were real coins, you would expect the total to be: 
            \[ 
              10 \del{ 1 + 2 + 3 + 4 + 5 + 6 + 7 + 8 + 9 + 10 }  = 10 \cdot 55 = 550 
            \]
  
            The actual total will be more than the expected total by the stack number.  
  
            For example, suppose stack 7 contains the counterfeit coins.  The seven coins you included from
            stack 7 all weigh one ounce more than they should, so the total in this case will be: 
            \[ 
              550 + 7 = 662 
            \]
  
            To get the number of the stack containing the counterfeit coins, you just need to subtract 550
            from whatever it says on the scale.
  
          \end{solution}
  
    \end{questions}

    \ifprintanswers{}
    \else
        \vspace{6 cm}
        \begin{quote}
            \begin{em}
                No individual gets up and says, I'm going to take this because I want it. He'd say, I'm
                going to take it because it really belongs to me and it would be better for everyone if I
                had it. It's true of children fighting over toys. And it's true of governments going to war.
                Nobody is ever involved in an aggressive war; it's always a defensive war---on both sides.
            \end{em}
        \end{quote}
        \hspace{1 cm} --Noam Chomsky
    \fi

\end{document}


