\documentclass[fleqn,addpoints]{exam}
\usepackage{amsmath}

\title{Math 113 Homework One}
\author{}
\date{January 20, 2010}

% \oddsidemargin 0in
% \topmargin -0.5in
% \textwidth 6.5in

\extrawidth{-1 in}
\setlength{\mathindent}{0in}

% \printanswers

\begin{document}

\maketitle

\section{Reading}

Read Chapter 1.  Most of it should be review, but take a glance at it anyway to make sure you are familiar with
everything.  

\section{Problems}

The answers to the odd numbered problems are in the back of the book.

If you have trouble with anything, ask a TA or another student for some assistance.  We'll go over some of the problems
and answer any questions in class next week. 

\begin{questions}

\question

From the text:
\begin{itemize}

  \item pp 9-10, problems 1-6, 36-38

  \item pp 40-42, problems 1, 15-20, 25-30, 33-38, 53, 56

\end{itemize}

\question

In 1772, the German astronomer Johann Elert Bode found a pattern in the distances of the planets from the sun.  At the
time, only six planets were known.  The actual relative distances of the planets from the sun and his pattern are:

\vspace{0.5 cm}
  \begin{tabular}{|c|c|c|}
    \hline
    Planet  & Actual Distance & Bode's Pattern \\ 
    \hline
    Mercury & 4               & \( 0 + 4 = 4 \) \\
    Venus   & 7               & \( 3 + 4 = 7 \) \\
    Earth   & 10              & \( 6 + 4 = 10 \) \\
    Mars    & 15              & \( 12 + 4 = 16 \) \\
            &                 &  \\
    Jupiter & 52              & \( 48 + 4 = 52 \) \\
    Saturn  & 96              & \( 96 + 4 = 100 \) \\
            &                 &  \\
    \hline
  \end{tabular}
\vspace{0.5 cm}

\begin{parts}
  \part What equation belongs between Bode's equations for Mars and Jupiter?

  \begin{solution}
    The equations all look like: \( n + 4 = \ldots \).  The only thing that changes from equation to equation is the
    value of $n$.  After the first equation, each value of $n$ is twice its value from the preceding equation.  

    A slightly more formal way of saying the same thing is that after the first equation, the equations are  
    look like: \( 2^n \cdot 3 + 4 \), where n = \{ 0, 1, 2, \ldots \}.  

    Either way, the missing equation is: \( 24 + 4 = 28 \)

  \end{solution}
  \part What equation belongs after Bode's equation for Saturn?

  \begin{solution}
    Using the same reasoning, the missing equation is: \( 192 + 4 = 196 \)

  \end{solution}

\end{parts}

In 1781 William Herschel discovered Uranus, the next planet beyond Saturn.  Because its distance of 192 units comes
remarkably close to the number predicted by the equation, astronomers came to the conclusion that the equation between
Bode's equations for Mars and Jupiter must also mean something.  In fact, in 1801, the asteroid Ceres was discovered at
a distance of 28 units from the sun.

\question

In the late sixteenth century, the Italian scientist Galileo discovered an interesting relationship between the
length of a pendulum and the time of the swing:

\vspace{0.5 cm}
  \begin{tabular}{|c|c|}
    \hline
    Length of Pendulum  & Time of Swing \\ 
    \hline
    1 unit  & 1 second \\
    4 units & 2 seconds \\
    9 units & 3 seconds \\
    16 units & 4 seconds \\
    \hline
  \end{tabular}
\vspace{0.5 cm}

\begin{parts}
\part
Write an equation for the length of the pendulum in terms of the time of the swing.

\begin{solution}
The length is always the time of the swing squared.  So the equation is: \( L = T^2 \)
\end{solution}

\part
What would be length of a pendulum with a swing of 8 seconds?

\begin{solution}
Substituting 8 into the equation gives: 
\begin{eqnarray*}
  L & = & T^2 \\
    & = & 8^2 \\
    & = & 64 \\
\end{eqnarray*}
\end{solution}

\end{parts}

\section{Extra Credit}

\question

You have 9 metal balls and a balance which allows you to compare the weight of two balls or two groups of balls.  You know
that one of the balls is slightly heaver than the others but don't know which one is the heavier ball.  What is the
smallest number of weighings required to identify the heavier ball?

\begin{solution}
  To minimize the number of weighings, you want to eliminate as many balls as possible with each weighing.  This
  approach allows you to eliminate \(2/3\) of the balls each time, resulting in two weighings:

  \begin{itemize}
    \item 
      Divide the balls into three groups of three.  Select two of the groups and put them on the balance.  If one of
      the groups is heavier, the heavy ball is in that group.  Otherwise, the heavy ball is in the group you didn't
      weigh.

    \item 
      Select two balls from the group containing the heavy ball and put them on the balance.  If one of them is heavier,
      it, obviously, is the heavy ball.  Otherwise, the third ball is the heavy ball. 
  \end{itemize}

  This procedure works for any number of balls that is a power of three.  The number of weighings required is equal to
  the exponent:

\begin{center}
  \begin{tabular}{|c|c|c|}
    \hline
    Exponent & Number of Balls & Number of Weighings \\
    \hline
    0 & $3^0 = 1$ & 0 \\
    1 & $3^1 = 3$ & 1 \\
    2 & $3^2 = 9$ & 2 \\
    3 & $3^3 = 27$ & 3 \\
    4 & $3^4 = 81$ & 4 \\
    \vdots & \vdots & \vdots \\
    \hline

  \end{tabular}
\end{center}

\end{solution}

\question

You throw away your balance and buy a scale instead.  But you are now faced with a different problem.  

You have 10 stacks of coins, each of which contains 10 coins.  You know that one stack is full of counterfeit coins
while all the other stacks contain real coins.  You also know that a real coin weighs 10 ounces while a counterfeit coin
weighs 11 ounces.  What is the smallest number of weighings required to identify the stack of counterfeit coins?

\begin{solution}

The best approach only requires one weighing.  The key is to notice that you can select
different numbers of coins from each stack.  So what you do is select 1 coin from the first stack, 2 coins from the
second stack, 3 coins from the third stack, etc. and weigh them all at once.

If all of the coins were real coins, you would expect the total to be: 
\[ 10 \cdot (1 + 2 + 3 + 4 + 5 + 6 + 7 + 8 + 9 + 10)  = 10 \cdot 55 = 550 \]

The actual total will be more than the expected total by the stack number.  

For example, suppose stack 7 contains the counterfeit coins.  The seven coins you included from stack 7 all weigh one
ounce more than they should, so the total in this case will be: \( 550 + 7 = 662 \).

To get the number of the stack containing the counterfeit coins, you just need to subtract 550 from whatever it says
on the scale.

\end{solution}


\end{questions}

\end{document}


