\documentclass[fleqn]{article}
\usepackage{amsmath}

\title{Math 113--Week One Notes}
\author{}
\date{January 20, 2010}

\oddsidemargin 0in
\topmargin -0.5in
\textwidth 6.5in

\setlength{\mathindent}{1in}

\begin{document}

\maketitle

\section{Reading}

This information is covered in the text from pages 1-40.

\section{Sets}

\begin{itemize}
  \item A {\em set} is an unordered collection of elements.  

  \item Two sets are equal if they contain the same elements.

  \item There are several ways to define a set.  These all define the set of positive odd integers less than 20:
    \begin{eqnarray*}
      A & = & \{ 1, 3, 5, 7, 9, 11, 13, 15, 17, 19 \}          \\
      A & = & \{ 19, 3, 11, 9, 15, 1, 13, 7, 17, 5 \}          \\
      A & = & \{ 1, 3, 5, \ldots, 17, 19 \}                    \\
      A & = & \{ x | \text{ x is odd and between 1 and 20} \}
    \end{eqnarray*}

  \item One set is a {\em subset} of another set if all of the elements in the first set are contained in the second set.
    \begin{eqnarray*}
      \{ 3, 5 \} & \subseteq & \{ 1, 3, 5, 7 \}          \\
      \{ 3, 5 \} & \subseteq & \{ 3, 5 \} \\
      \emptyset  & \subseteq & \{ 3, 5 \} \\
    \end{eqnarray*}
\end{itemize}

\section{Types of Real Numbers}

For most of this course we will be using {\em real numbers}.  There are several types of real numbers:

\begin{itemize}
  \item {\em integers} are the numbers \( \{ \ldots, -3, -2, -1, 0, 1, 2, 3, \ldots \} \)
  \item {\em rational numbers} are the numbers which can be expressed as a ratio of two integers: 
    1/2, 3/4, 2, 3.5, etc.
  \item {\em irrational numbers} are numbers which can't be expressed as a ratio of two integers:
    \( \sqrt{2} \), \( \pi \), \( e \), \( \sqrt{17}\), etc.
\end{itemize}

\section{Absolute Value}

The absolute value of a number is the number without its sign.

\begin{eqnarray*}
  | 17 |  & = & 17 \\
  | -17 | & = & 17 \\
  | 0 | & = & 0 \\
\end{eqnarray*}

\section{Number Properties}

The text book describes several properties of numbers, most of which are fairly obvious (for example: if \( a = b \)
then \( b = a \)).  You should read over the chapter to make sure the rest of them are also familiar to you.  

Here are a few of the properties which are slightly less obvious and quite useful:

\begin{eqnarray*}
  a (b + c) & = & ab + ac \\
  a + (b + c) & = & (a + b) + c \\
  (ab)c & = & a(bc) \\
\end{eqnarray*}

\section{Algebraic Expressions}

An algebraic expression usually contains several {\em term}.  A {\em term} is a number and some variables multiplied
together.  Here are some terms: \( x^2 \), $ 3xy $,  $ 47 $, $ 12x^3y^2z^4 $.


Because of the distributive property, terms with exactly the same variables (including exponents) are called {\em like}
or {\em similar} terms and may be added together:

\begin{eqnarray*}
  3x^2 + 7x^2 & = & x^2(3 + 7) \\
              & = & 10x^2  \\
\end{eqnarray*}

To evaluate an algebraic expression for a particular value for variable, you substitute the value in for the variable
everywhere the variable appears in the equation.  Usually it's easiest to simplify the equation as much as you can before
you do this.

\end{document}


