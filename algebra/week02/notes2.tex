\documentclass[fleqn]{article}
\usepackage{amsmath}

\title{Math 113--Week Two Notes}
\author{}
\date{January 27, 2010}

\oddsidemargin 0in
\topmargin -0.5in
\textwidth 7in
\textheight 9in

\setlength{\mathindent}{1in}

\begin{document}

\maketitle

\section{Solving Linear Equations}

A linear equation is an equation in which highest exponent on any variable is 1.  These are all linear equations:

\begin{eqnarray*}
  x = 5 \\
  x + 2 = 17 \\
  \frac{x}{5} - 3 = x + 12 \\
\end{eqnarray*}

Equations like this are called {\em linear} equations because when you graph them, you get a straight line.

To solve a linear equation, you need to get the variable on one side of the equation and a number on the other side of
the equation.  Of course, there is a rule you need to follow to make things come out OK--the equation always needs to remain
{\em balanced}.  This means that whatever you do to one side of the equation, you also have to do to the other side of
the equation.  

Here are some things you can do which keep the equation balanced:

\begin{itemize}
  \item add the same number to each side of the equation
  \item subtract the same number from each side of the equation
  \item multiply each side of the equation by the same non-zero number
  \item divide each side of the equation by the same non-zero number
\end{itemize}

Of course, multiplying each side of the equation by zero keeps the equation balanced.  But it destroys the usefulness of
the equation since you end up with the not-very-useful equation: \( 0 = 0 \).

\section{Linear Equations with Fractions}

When you are faced with a linear equation which contains fractions, you have a few choices.  You should use
whichever approach works best for you for a particular problem.

One approach is to multiply each side of the equation by the {\em Least Common Multiple (LCM)} of the denominators which
appear in the equation.  This makes all the fractions go away, and you can just work with an equation full of integers.

Another approach is to make the fractions go away one at a time.  You can do this by multiplying each side by one of the
denominators.  The fraction with that denominator will go away, and then you can simplify the
remaining fractions and decide how to deal with them.

Or, if you don't mind working with fractions, you can just leave them in the equation and just start isolating the variable
you are solving for.

\end{document}


