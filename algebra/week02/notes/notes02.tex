\documentclass[letterpaper, landscape]{exam}
\usepackage{2in1, lscape} 
\printanswers{}

\usepackage{units} 
\usepackage{xfrac} 
\usepackage[fleqn]{amsmath}
\usepackage{commath}
\usepackage{cancel}
\usepackage{float}
\usepackage{mdwlist}
\usepackage{booktabs}
\usepackage{cancel}
\usepackage{polynom}
\usepackage{caption}
\usepackage{fullpage}
\usepackage{comment}
\usepackage{enumerate}
\usepackage{graphicx}
\usepackage{mathtools} 

\newcommand{\degree}{\ensuremath{^\circ}} 
\everymath{\displaystyle}
\setlength{\mathindent}{0in}

\title{Math 113--Week Two Notes}
\author{}
\date{\today}

\begin{document}

  \maketitle

  \section{Section 2.2} % (fold)
  
  \subsection{Adding Fractions}

  \begin{align*}
    \frac{1}{3} + \frac{1}{2}   & = \frac{5}{6} \\
    \frac{2}{5} + \frac{3}{10}  & = \frac{7}{10} \\
    \frac{3}{10} + \frac{4}{15} & = \frac{17}{30} \\
  \end{align*}

  rules:
  \begin{itemize*}
    \item factor all the denominators
    \item LCD is smallest set of factors which is superset of all denominators.
      Including all the factors always works, but may not be the LCD\@.
    \item once all the denominators match, sum of numerators on left is equal
      to numerator on right
  \end{itemize*}

  \subsection{Equations with Fractions} % (fold)

  \begin{description}
    \item[0]
      \begin{align*}
        \frac{x}{3} + \frac{x}{2} & = 15 \\
        \frac{5x}{6}              & = 15 \\
        x                         & = 18 \\
      \end{align*}

    \item[1]
      \begin{align*}
        \frac{3}{4} x & = 9 \\
        3x            & = 36 \\
        x             & = \boxed{ 12 } \\
      \end{align*}

    \item[2] 
      \begin{align*}
        \frac{2}{3} x & = -14 \\
        2x            & = -42 \\
        x             & = \boxed{ -21 } \\
      \end{align*}

    \item[6]
      \[
        \frac{x}{4} - \frac{5}{6} = \frac{5}{12}
      \]

    \item[7]
      \[
        \frac{5x}{6} - \frac{x}{8} = - \frac{17}{12}
      \]

    \item[8]
      \[
        \frac{2x}{5} - \frac{x}{6} = - \frac{7}{10}
      \]

    \item[9]
      \[
        \frac{x}{4} - 1 = \frac{x}{3} + 2
      \]

    \item[10]
      \[
        \frac{3x}{7} - 1 = \frac{x}{3}
      \]

    \item[17]
      \[
        \frac{x + 2}{2} - \frac{x - 1}{5} = \frac{3}{5}
      \]

    \item[18]
      \[
        \frac{2x + 1}{3} - \frac{x + 1}{7} = \frac{-1}{3}
      \]

    \item[19]
      \[
        \frac{x + 2}{4} - \frac{2x - 1}{3} = \frac{1}{6}
      \]

    \item[26]
      \[
        \frac{3x + 1}{9} + 2 = \frac{x - 1}{4}
      \]

    \item[27]
      \[
        \frac{2x - 3}{6} + \frac{3x - 2}{4} + \frac{5x + 6}{12} = 4
      \]

    \item[28]
      \[
        \frac{3x - 1}{4} + \frac{x - 2}{3} - \frac{x - 1}{5} = \frac{21}{30}
      \]

    \item[29]
      \[
        x + \frac{3x - 1}{9} - 4 = \frac{3x + 1}{3}
      \]

    \item[37]
      \[
        \frac{1}{2} (2x - 1) - \frac{1}{3} (5x + 2) = 3
      \]

    \item[38]
      \[
        \frac{2}{5} (4x - 1) + \frac{1}{4} (5x + 2) = -1
      \]

  \end{description}

  \subsection{Word Problems} % (fold)

  \begin{description}
      
    \item[46]
      if $x$ is his normal hourly rate:
      \begin{align*}
        40x + 4 \cdot \frac{3}{2} x & = 276 \\
        46x                         & = 276 \\
        x                           & = \boxed{ 6 } \\
      \end{align*}

    \item[47]
      \begin{align*}
        x & = \frac{2}{3} (20 - x) \\
        x & = \boxed{ \unit[8]{ft} } \\
      \end{align*}

    \item[50]
      If $j$ is Josephine's current age:
      \begin{align*}
        \frac{2}{3} j + 12 + j + 12 & = 54 \\
        j                           & = 18 \\
      \end{align*}

      Josephine is 18 and Annilee is 12. In 12 years, Josephine will be 30 and
      Annilee will be 24.

  \end{description}

  \section{Section 2.3} % (fold)

  \subsection{Problems} % (fold)
  
  \begin{description}

    \item[1] 
      \[
        0.14x = 2.8
      \]

    \item[5] 
      \[
        x + 0.4x = 56
      \]

    \item[5] 
      \[
        x + 0.4x = 56
      \]

    \item[11] 
      \[
        0.11x + 0.12(900 - x) = 104
      \]

    \item[12] 
      \[
        0.09x + 0.11 (500 - x) = 51
      \]

    \item[21] 
      \[
        0.12x + 0.1(5000 - x) = 500 
      \]

    \item[22] 
      \[
        0.10x + 0.12 (x + 1000) = 560
      \]

    \item[25] 
      \[
        0.3(2x + 0.1) = 8.43
      \]

    \item[26] 
      \[
        0.5(3x + 0.7) = 20.6
      \]

  \end{description}
  
  \subsection{Word Problems} % (fold)

  \begin{description}

    \item[29]
      \begin{align*}
        0.8x & = 72 \\
        x    & = \boxed{ 90 } \\
      \end{align*}

    \item[32]
      \begin{align*}
        72 - .35 \cdot 72 & = 0.65 \cdot 71 \\
                          & = \boxed{ \$46.8 } \\
      \end{align*}

    \item[44]
      if $x$ is the amount invested at 8\%:
      \begin{align*}
        .08x + 0.09(4000 - x) & = 350 \\
        x                     & = \boxed{ \$1000 } \\
      \end{align*}

    \item[46]
      \begin{align*}
        1500 \cdot 0.06 + x \cdot 0.09 & = 301.5 \\
        x                              & = \boxed{ \$2,350 } \\
      \end{align*}

  \end{description}
  
\end{document}


