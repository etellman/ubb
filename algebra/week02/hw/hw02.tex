% no answer key
\documentclass[letterpaper]{exam}

% answer key
% \documentclass[letterpaper, landscape]{exam}
% \usepackage{2in1, lscape} 
% \printanswers{}

\usepackage{units} 
\usepackage{parskip} 
\usepackage{xfrac} 
\usepackage[fleqn]{amsmath}
\usepackage{commath}
\usepackage{cancel}
\usepackage{float}
\usepackage{mdwlist}
\usepackage{booktabs}
\usepackage{cancel}
\usepackage{polynom}
\usepackage{caption}
\usepackage{fullpage}
\usepackage{comment}
\usepackage{enumerate}
\usepackage{graphicx}
\usepackage{mathtools} 

\newcommand{\degree}{\ensuremath{^\circ}} 
\everymath{\displaystyle}

\title{Algebra \\ Homework Two \\ Sections 2.2}
\author{}
\date{\today}

\begin{document}

  \maketitle

  \section{Homework}
  Section 2.2: 3--5, 11--16, 20--25, 30--36, 39--40, 45, 49, 53

  \section{Problems}

  \begin{questions}

    \question{}
    Alice, Carol, Ted, and Bob all work for the university.  Ted is a
    custodian, Alice is a math teacher, Carol is the university president, and
    Bob is the football coach.   

    Of course they all have different pay scales.  Alice makes two times what
    Ted makes.  Carol makes four times what Alice makes.  And Bob makes 5
    times what Carol makes.

    The sum of the four employees' hourly pay is \$765.

    \begin{solution}
      Since we know the total of the hourly pay is \$765, if we let Ted's pay
      be $x$, the solution is:  
      \begin{align*}
        x + 2x + 4 \cdot 2x + 5 \cdot 4 \cdot 2x & = 765 \\
        3x + 8x + 40x                            & = 765 \\
        51x                                      & = 765 \\
        x                                        & = 15 \\
      \end{align*}

    Ted makes \$15/hour, Alice makes \$30/hour, Carol makes \$120/hour, and
    Bob makes \$600/hour.

    Check:
      \begin{align*}
        15 + 30 + 120 + 600 & = 765 \\
        15 \cdot 2          & = 30  \\
        30 \cdot 4          & = 120 \\
        120 \cdot 5         & = 600 \\
      \end{align*}

    \end{solution}

    \begin{parts}
      \part{}
        If the custodian works a four hour football game, how much does he get
        paid for the game?
        \begin{solution}
          For four hours work, Ted makes:
          \[
            4 \cdot 15 = \$60
          \]
        \end{solution}

      \part{} If the coach works the same game, how much does he get paid for
      the game?
        \begin{solution}
            For four hours work, Bob makes:
            \[
              4 \cdot 600 = \$2,400
            \]
        \end{solution}

      \part{} How much does the teacher get paid for four hours teaching?
        \begin{solution}
            For four hours work, Alice makes: 
            \[
              4 \cdot 30 = \$120
            \]
        \end{solution}

      \part{} How much does the president get paid for four hours spent doing
        whatever it is university presidents do?

        \begin{solution}
            For four hours work, Carol makes:
            \[
              4 \cdot 120 = \$480
            \]
        \end{solution}

    \end{parts}

    \question{}
      Some students at Northwestern University have been examining death
      penalty cases.  Using DNA evidence they have discovered that in about 1
      out of 20 of the cases they have examined, the person being imprisoned
      is not guilty of the crime.  

      \begin{parts}

        \part{} They have exonerated 15 people so far.  How many cases have
        they looked at?
          \begin{solution}
            Letting $x$ be the number of cases examined:

            \begin{align*}
              \frac{1}{20} \cdot x & = 15 \\
              x                    & = 300 \\
            \end{align*}

            I actually rounded the numbers a little.  The real numbers are 14
            people exonerated and released out of 288 cases examined.

          \end{solution}

        \part{}
          Michael Evans was imprisoned for quite a long time for a crime which
          he didn't commit.  In fact, at the time he was released he he had
          been in prison for five years more than half his life. He went to
          prison when he was 17.  How long was he in prison? 

          \begin{solution}
            Letting $x$ be Evans' age when released:

            \begin{align*}
              x           & = 17 + \frac{x}{2} + 5 \\
              x           & = \frac{x}{2} + 22 \\
              \frac{x}{2} & = 22 \\
              x           & = 44 \\
            \end{align*}

            He was 44 years old when he was released.  We now need to figure
            out how long he was in prison. 

            Letting $p$ be the length of time he was in prison:
            \begin{align*}
              44 & = 17 + p \\
              p  & = 27 \\
            \end{align*}

            Paul Terry was incorrectly imprisoned for the same crime and also
            spent 27 years in prison.  

            After Evans and Terry were released, they filed a civil rights
            suit against the Chicago Police Department.  Evans lost his suit
            because his imprisonment was judged to be a result of police
            negligence and not misconduct.  Terry won his suit and was awarded
            \$2.7 million.  Perhaps he gave some of it to Evans.

          \end{solution}

      \end{parts}

    \ifprintanswers{}
    \else
      \pagebreak
    \fi

    \section{Extra Credit}

    \question{}

    Ted is a resident at WSR\@.  One day, he notices that a door is open and
    decides to take an unauthorized stroll into the lovely neighboring
    community of Monroe.  Unimpressed, he keeps walking.

    The prison staff notices Ted is missing.  Naturally they are concerned,
    because they know the prison choir concert is this Saturday, and they know
    how much Ted has been looking forward to the performance.  They send a
    officer, Larry, and the prison dog, Fido, out to look for Ted.

    Because Larry is hurrying, he walks 1 mph faster than Ted.  Fido is an
    exuberant dog, and runs at 10 mph.  Fido is fond of both Ted and Larry, so
    he runs ahead to catch up to Ted, says hello, then runs back to Larry, then
    runs ahead to Ted, etc.  Fido doesn't spend any time with either man.  He
    turns around and runs back immediately after he arrives.

    Since Larry is walking faster, he eventually catches up to Ted.  Ted
    thanks him for reminding him about the concert, but says that he is really
    enjoying his walk and would prefer to keep going.  Larry and Fido return
    to the prison just in time to catch the performance themselves.

    If Ted is 3 miles ahead when Larry and Fido set out, how far does Fido run
    before Larry catches up to Ted?

    \begin{solution}
      Since Larry walks 1 mph faster than Ted, and Larry has a 3 mile head
      start, it takes Larry 3 hours to catch up to Ted.  Fido is running 10 mph
      for the entire 3 hours, so Fido runs for 30 miles.
    \end{solution}

  \end{questions}
  \ifprintanswers{}

  \section{Section 2.2} % (fold)
  
    \begin{description}
      \item[3] 
        \begin{align*}
          \frac{-2x}{3} & = \frac{2}{5} \\
          -10x          & = 6 \\
          x             & = \boxed{ - \frac{3}{5} } \\
        \end{align*}

      \item[4] 
        \begin{align*}
          \frac{-5x}{4} & = \frac{7}{2} \\
          -5x           & = 14 \\
          x             & = \boxed{ - \frac{14}{5} } \\
        \end{align*}

      \item[5] 
        \begin{align*}
          \frac{n}{2} - \frac{2}{3} & = \frac{5}{6} \\
          3n - 4                    & = 5 \\
          n                         & = \boxed{ 3 } \\
        \end{align*}

      \item[11] 
        \begin{align*}
          \frac{h}{4} + \frac{h}{5} & = 1 \\
          5h + 4h                   & = 20 \\
          h                         & = \boxed{ \frac{20}{9} } \\
        \end{align*}

      \item[12] 
        \begin{align*}
          \frac{h}{6} + \frac{3h}{8} & = 1 \\
          4h + 9h                    & = 24 \\
          h                          & = \boxed{ \frac{24}{13} } \\
        \end{align*}

      \item[13] 
        \begin{align*}
          \frac{h}{2} - \frac{h}{3} + \frac{h}{6} & = 1 \\
          3h - 2h + h                             & = 6 \\
          h                                       & = \boxed{ 3 } \\
        \end{align*}

      \item[14] 
        \begin{align*}
          \frac{3h}{4} + \frac{2h}{5} & = 1 \\
          15h + 8h                    & = 20 \\
          h                           & = \boxed{ \frac{20}{23} } \\
        \end{align*}

      \item[15] 
        \begin{align*}
          \frac{x - 2}{3} + \frac{x + 3}{4} & = \frac{11}{6} \\
          4(x - 2) + 3(x + 3)               & = 22 \\
          x                                 & = \boxed{ 3 } \\
        \end{align*}

      \item[16] 
        \begin{align*}
          \frac{x + 4}{5} + \frac{x - 1}{4} & = \frac{37}{10} \\
          4 (x + 4) + 5 (x - 1)             & = 74 \\
          x                                 & = \boxed{ 7 } \\
        \end{align*}

      \item[20] 
        \begin{align*}
          \frac{n - 1}{9} - \frac{n + 2}{6}            & = \frac{3}{4} \\
          % 36 \del{ \frac{n - 1}{9} - \frac{n + 2}{6} } & = \frac{3}{4} \cdot 36 \\
          4(n - 1) - 6(n + 2)                          & = 27 \\
          n                                            & = \boxed{ - \frac{43}{2} } \\
        \end{align*}

      \item[21] 
        \begin{align*}
          \frac{y}{3} + \frac{y - 5}{10} & = \frac{4y + 3}{5} \\
          10y + 3(y - 5)                 & = 6 (4y + 3) \\
          y                              & = \boxed{ -3 } \\
        \end{align*}

      \item[22] 
        \begin{align*}
          \frac{y}{3} + \frac{y - 2}{8} & = \frac{6y - 1}{12} \\
          8y + 3(y - 2)                 & = 2(6y - 1) \\
          y                             & = \boxed{ -4 } \\
        \end{align*}

      \item[23] 
        \begin{align*}
          \frac{4x - 1}{10} - \frac{5x + 2}{4} & = -3 \\
          2(4x - 1) - 5(5x + 2)                & = -60 \\
          x                                    & = \boxed{ \frac{48}{17} } \\
        \end{align*}

      \item[24] 
        \begin{align*}
          \frac{2x - 1}{2} - \frac{3x + 1}{4} & = \frac{3}{10} \\
          10 (2x - 1) -  5 (3x + 1)           & = 6 \\
          x                                   & = \boxed{ \frac{21}{5} } \\
        \end{align*}

      \item[25] 
        \begin{align*}
          \frac{2x - 1}{8} - 1 & = \frac{x + 5}{7} \\
          2x - 1 - 8           & = \frac{8(x + 5)}{7} \\
          x                    & = \boxed{ \frac{103}{6} } \\
        \end{align*}

      \item[30]
        \begin{align*}
          \frac{2x + 7}{8} + x - 2 & = \frac{x-1}{2} \\
          2x + 7 + 8(x - 2)        & = 4(x - 1) \\
          x                        & = \boxed{ \frac{5}{6} } \\
        \end{align*}

      \item[31]
        \begin{align*}
          \frac{x + 3}{2} + \frac{x + 4}{5} & = \frac{3}{10} \\
          5(x + 3) + 2(x + 4)               & = 3 \\
          x                                 & = \boxed{ -\frac{20}{7} } \\
        \end{align*}


      \item[32]
        \begin{align*}
          \frac{x - 2}{5} - \frac{x - 3}{4} & = -\frac{1}{20} \\
          4(x - 2) - 5(x - 3)               & = -1 \\
          x                                 & = \boxed{ 8 } \\
        \end{align*}

      \item[33]
        \begin{align*}
          n + \frac{2n - 3}{9} - 2 & = \frac{2n + 1}{3} \\
          9n + (2n - 3) - 18       & = 2 (2n + 1) \\
          n                        & = \boxed{ \frac{24}{5} } \\
        \end{align*}

      \item[34]
        \begin{align*}
          n - \frac{3n + 1}{6} - 1 & = \frac{2n + 4}{12} \\
          12n - 2(3n + 1) - 12     & = 2n + 4 \\
          n                        & = \boxed{ \frac{9}{2} } \\
        \end{align*}

      \item[35]
        \begin{align*}
          \frac{3}{4}(t - 2) - \frac{2}{5}(2t + 3) & = \frac{1}{5} \\
          15(t - 2) - 8(2t + 3)                    & = 4 \\
          t                                        & = \boxed{ -10 } \\
        \end{align*}

      \item[36] 
        \begin{align*}
          \frac{2}{3} (2t + 1) - \frac{1}{2} (3t - 2) & = 2 \\
          4(2t + 1) - 3(3t - 2)                       & = 12 \\
          t                                           & = \boxed{ -2 } \\
        \end{align*}

      % \item[37]
      %   \begin{align*}
      %     \frac{1}{2}(2x - 1) - \frac{1}{3}(5x + 2) & = 3 \\
      %     3(2x - 1) - 2(5x + 2)                     & = 18 \\
      %      x                                        & = \boxed{ - \frac{25}{4} } \\
      %   \end{align*}

      \item[39]
        \begin{align*}
          3x - 1 + \frac{2}{7}(7x - 2) & = - \frac{11}{7} \\
          7(3x - 1) + 2(7x - 2)        & = -11 \\
          x                            & = \boxed{ 0 } \\
        \end{align*}

      \item[40]
        \begin{align*}
          2x + 5 + \frac{1}{2}(6x - 1) & = -\frac{1}{2} \\
          2x + 5 + 3x - \frac{1}{2}    & = -\frac{1}{2} \\
          5x + 5                       & = 0 \\
          x                            & = \boxed{ -1 } \\
        \end{align*}

      % \item[42]
      %   \begin{align*}
      %     \frac{x}{2} + \frac{3x}{4}            & = \frac{4x}{3} + 2 \\
      %     % 12 \del{ \frac{x}{2} + \frac{3x}{4} } & = 12 \del{ \frac{4x}{3} + 2 } \\
      %     6x + 9x                               & = 16x + 24 \\
      %     x                                     & = \boxed{ -24 } \\
      %   \end{align*}

      \item[45]
        \begin{align*}
          x + \frac{1}{3} (x + 1) + \frac{3}{8} (x + 2)            & = 25 \\
          % 24 \del{ x + \frac{1}{3} (x + 1) + \frac{3}{8} (x + 2) } & = 24 \cdot 25 \\
          24x + 8(x + 1) + 12(x + 2)                               & = 600 \\
          x                                                        & = \boxed{ 14 } \\
        \end{align*}

      \item[49] \ 
      If $A$ is Angie's age:
      \begin{align*}
        \frac{3}{5}(64 - A + 8)       & = A + 8 \\
        \frac{3}{5} (72 - A)          & = A + 8 \\
        3(72 - A)                     & = 5(A + 8) \\
        A                             & = 22 \\
      \end{align*}
      \fbox{ Angie is 22 and her mother is 42. }  
      
      \item[53] 
        The scores for the three exams are: $x$, $x + 10$ and $x + 14$.
        \begin{align*}
          \frac{ x + (x + 10) + (x + 14) }{3} & = 88 \\
          x                                   & = 80 \\
        \end{align*}
        \fbox{ The three scores are 80, 90 and 94 }
    \end{description}

  \fi

  \ifprintanswers{}
  \else
    \vspace{10 cm}
    \begin{quote}
      \begin{em}
        Politics is the art of making the people believe that they are in
        power, when in fact, they have none.        
      \end{em}
    \end{quote}
    \hspace{1 cm} --Mumia Abu-Jamal
  \fi

\end{document}


