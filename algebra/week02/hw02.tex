\documentclass[fleqn,addpoints]{exam}
\usepackage{amsmath}
\usepackage{mdwlist}

\title{Math 113 Homework Two}
\author{}
\date{\today}

\extrawidth{-1 in}
\setlength{\mathindent}{0in}

% \printanswers

\begin{document}

\maketitle

%% \section{Homework One Corrections}

%% When I was grading homework one, I realized to my dismay that the answer key had a few issues.  In fact, several
%% students did better than I did.  Here are the mistakes:

%% page 42, number 28: the correct answer is \(-13a + 4\)

%% page 42, number 38: the correct answer is \(-44\)

\section{From the Book}

\begin{itemize*}
\item Read pages 43-66.
\item pp. 50-51: 1-5, 26-30, 46, 47, 51, 53, 62
\item pp 58-59: 1-5
\end{itemize*}

\section{Additional Problems}

\begin{questions}

\question

Alice, Carol, Ted, and Bob all work for the university.  Ted is a custodian, Alice is a math teacher, Carol is the
university president, and Bob is the football coach.   

Of course they all have different pay scales.  Alice makes two times what Ted makes.  Carol makes four times what Alice
makes.  And Bob makes 5 times what Carol makes.

The sum of the four employees' hourly pay is \$765.

\begin{solution}
  Since we know the total of the hourly pay is \$765, if we let Ted's pay be $x$, the solution is:  
  \begin{eqnarray*}
    x + 2x + 4 \cdot 2x + 5 \cdot 4 \cdot 2x & = & 765 \\
    3x + 8x + 40x & = & 765 \\
    51x & = & 765 \\
    x & = & 15 \\
  \end{eqnarray*}

Ted makes \$15/hour, Alice makes \$30/hour, Carol makes \$120/hour, and Bob makes \$600/hour.

Check:
  \begin{eqnarray*}
    15 + 30 + 120 + 600 & = & 765 \\
    15 \cdot 2 & = & 30  \\
    30 \cdot 4 & = & 120 \\
    120 \cdot 5 & = & 600 \\
  \end{eqnarray*}

\end{solution}

\begin{parts}
\begin{part}
    If the custodian works a four hour football game, how much does he get paid for the game?

\begin{solution}
  For four hours work, Ted makes \( 4 \cdot 15 = 60 \) dollars.
\end{solution}

\end{part}

\begin{part}
    If the the coach works the same game, how much does he get paid for the game?

\begin{solution}
    For four hours work, Bob makes \( 4 \cdot 600 = 2,400 \) dollars.
\end{solution}

\end{part}

\begin{part}
    How much does the teacher get paid for four hours teaching?

\begin{solution}
    For four hours work, Alice makes \( 4 \cdot 30 = 120 \) dollars.
\end{solution}

\end{part}

\begin{part}
    How much does the president get paid for four hours spent doing whatever it is university presidents do?

\begin{solution}
    For four hours work, Carol makes \( 4 \cdot 120 = 480 \) dollars.
\end{solution}
\end{part}

\end{parts}

\question

Some students at Northwestern University have been examining death penalty cases.  Using DNA evidence they have
discovered that in about 1 out of 20 of the cases they have examined, the person being imprisoned is not the actual perpetrator of the
crime for which he is being detained.  

\begin{parts}

\begin{part}
They have exonerated 15 people so far.  How many cases have they looked at.

\begin{solution}
  Letting $x$ be the number of cases examined:

  \begin{eqnarray*}
    \frac{1}{20} \cdot x & = & 15 \\
    x & = & 300 \\
  \end{eqnarray*}

Check: \( 300 / 20 = 15 \).

I actually rounded the numbers a little.  The real numbers are 14 people exonerated and released out of 288 cases examined.

\end{solution}

\end{part}

\begin{part}

Michael Evans was imprisoned for quite a long time for a crime which he didn't commit.  In fact, at the
time he was released he he had been in prison for five years more than half his life. He went to prison when he was
17.  How long was he in prison? 

\begin{solution}
  Letting $x$ be Evans' age when released:
  \begin{eqnarray*}
    x & = & 17 + \frac{x}{2} + 5 \\
    x & = & \frac{x}{2} + 22 \\
    \frac{x}{2} & = & 22 \\
    x & = & 44 \\
  \end{eqnarray*}

check: \( 44/2 + 5 + 17 = 22 + 5 + 17 = 22 + 22 = 44 \)

  So he was 44 years old when he was released.  We now need to figure out how long he was in prison. 

  Letting $p$ be the length of time he was in prison:
  \begin{eqnarray*}
    44 & = & 17 + p \\
    p & = & 27 \\
  \end{eqnarray*}
Check: \( 27 = 44 / 2 + 5  \)

\vspace{.1 in}

Paul Terry was also incorrectly imprisoned for the same crime and also spent 27 years in prison.  

After Evans and Terry were
released, they filed a civil rights suit against the Chicago Police Department.  Evans lost his suit because his
imprisonment was judged to be a result of police negligence and not misconduct.  Terry won his suit and was awarded
\$2.7 million.  Perhaps he gave some of it to Evans.

\end{solution}

\end{part}

\end{parts}

\ifprintanswers
\else
\pagebreak
\fi

\section{Extra Credit}

\question

Ted is a prisoner at WSR.  One day, he notices that a door is open and decides to take an unauthorized stroll into the
lovely neighboring community of Monroe.  Unimpressed, he keeps walking.

The prison staff notices Ted is missing.  Naturally they are concerned, because they know the prison choir concert is
this Saturday, and they know how much Ted has been looking forward to the performance.  They send a guard, Larry, and the
prison dog Fido, out to look for Ted.

Because Larry is hurrying, he walks 1 mph faster than Ted.  Fido is a very exuberant dog, and runs at 10 mph.  Fido is
fond of both Ted and Larry, so he runs ahead to catch up to Ted, says hello, then runs back to Larry, then runs ahead to
Ted, etc.  Fido doesn't spend any time with either man.  He turns around and runs back immediately after he arrives.

Since Larry is walking faster, he eventually catches up to Ted.  Ted thanks him for reminding him about the concert, but
says that he is really enjoying his walk and would prefer to just keep going.  So Larry and Fido return to the prison
just in time to catch the performance themselves.

If Ted is 3 miles ahead when Larry and Fido set out, how far does Fido run before Larry catches up to Ted?

\begin{solution}

Since Larry walks 1 mph faster than Ted, and Larry has a 3 mile head start, it takes Larry 3 hours to catch up to Ted.
Fido is running 10 mph for the entire 3 hours, so Fido runs for 30 miles.

\end{solution}

\end{questions}

\end{document}


