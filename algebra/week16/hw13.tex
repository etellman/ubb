
\documentclass[fleqn,addpoints]{exam}

\usepackage{amsmath}
\usepackage{graphics}
\usepackage{cancel}
\usepackage{polynom}
\usepackage{mdwlist}

% \printanswers

\ifprintanswers
\usepackage{2in1, lscape}
\fi

\title{Math 113 Homework 13}
\author{}
\date{\today}

\begin{document}

\maketitle

\section{From the Book}

% Read pages 257-262.

% The problems from Chapter 5 are the last set of problems from last week, so you may have already done them.

\begin{itemize*}
  \item pp. 218-220 (Chapter 4 again): 46 and 50
  \item pp. 262-263: 20-24, 31-34, 47, 48, 52
\end{itemize*}


\section{Additional Problems}

\subsection{Ratio Problems}
\begin{questions}

\question
A quality control engineer found two defective units in a sample of 50.  At this rate, what is the expected number of
defective units in a shipment of 10,000 units?

\begin{solution}

If we let $x$ be the expected number of defective units in the shipment, we can use this ratio:
\begin{align*}
  \frac{2}{50} &= \frac{x}{10,000} \\
  20,000 &= 50x \\
  x = 400
\end{align*}

\end{solution}

\question
A pump can fill a 750-gallon tank in 35 minutes.  How long will it take to fill a 1,000-gallon tank with this pump?

\begin{solution}

If we let $x$ be the time to fill the larger tank, we can use this ratio:
\begin{align*}
  \frac{35}{750} &= \frac{x}{1,000} \\
  35,000 = 750x \\
  x = \frac{140}{3}
\end{align*}

So it will take 46 2/3 minutes or 46 minutes and 40 seconds.  I think the original take was perhaps supposed to have
been 700, which would have made the calculations a bit simpler.

\end{solution}

\question
The ratio of cement to sand in an 80-pound bag of dry mix is 1 to 4.  Find the number of pounds of sand in the bag.
Dry mix is composed only of cement and sand.

\begin{solution}

If we let $x$ be the number of pounds of sand in an 80-pound bag, the rest of the bag is $80-x$.  The rest of the bag
contains cement, so we can use this ratio to figure out how much sand there is:

\begin{align*}
  \frac{1}{4} &= \frac{80-x}{x} \\
  x = 320 - 4x \\
  5x = 320 \\
  x = 64 \\
\end{align*}

An alternate approach would be to say that if the ratio of sand to dry mix is 1 to 4, the ratio of sand to total bag
contents is 4 to 5.  This is true because for every pound of cement you have 4 pounds of sand, so out of 5 pounds of dry
mix you have 1 pound of cement and 4 pounds of sand.  This will let you use this ratio, which gives the same answer, of
course: 

\begin{align*}
  \frac{4}{5} &= \frac{x}{80} \\
  320 &= 5x \\
  x &= 64 \\ 
\end{align*}

\end{solution}

\subsection{Mixture Problems}
\question
A person has 20 coins in nickels and dimes with a combined value of \$1.60.  Determine the number of coins of each type.

\begin{solution}
  If we let $x$ be the number of nickels, all the other coins, or $20-x$ are dimes.  This leads to this equation:
\begin{align*}
  5x + 10(20-x) &= 160 \\
  5x + 200 - 10x &= 160 \\
  -5x &= -40 \\
  x &= 8 \\
\end{align*}

So the person has 8 nickels and 12 dimes.  This is \$1.60, so the answer is correct.

\end{solution}

\question
The cooling system in a truck contains 4 gallons of coolant that is 30\% antifreeze.  How much must be withdrawn and
replaced with 100\% antifreeze to bring the coolant in the system to 50\% antifreeze?

\begin{solution}
If we let $x$ be the amount of 100\% antifreeze, everything else in the final mixture, or $4-x$ will be 30\%
antifreeze.  The final solution will contain 50\% antifreeze.  For four gallons of solution, this is 2 gallons of
antifreeze.  This leads to this equation:

\begin{align*}
  x + .3(4-x) &= 2 \\
  x + \frac{3}{10}(4-x) &= 2 \\
  10x + 3(4-x) &= 20 \\
  10x + 12 - 3x &= 20 \\
  7x + 12 &= 20 \\
  7x  &= 8 \\
  x  &= \frac{8}{7} \\
\end{align*}

So the final mixture will contain $\dfrac{8}{7}$ gallons of 100\% solution and $4 - \dfrac{8}{7} = \dfrac{20}{7}$
gallons of 30\% solution.  To check:

\[
  \frac{8}{7} + \frac{3}{10} \cdot \frac{20}{7} = \frac{8}{7} + \frac{6}{7} = \dfrac{14}{7} = 2
\]

So there are two gallons of antifreeze in the final mixture.

\end{solution}


\subsection{Rate Problems}
\question
You can mow a lawn in 2 hours using a riding mower and in 3 hours using a push mower.  Using both machines together, how
long will it take you and a friend to mow the lawn?

\begin{solution}
The two rates are:
\begin{itemize*}
  \item $r_1 = \dfrac{1}{2}$ lawns/hours
  \item $r_2 = \dfrac{1}{3}$ lawns/hours
\end{itemize*}

So the combined rate is $\dfrac{1}{2} + \dfrac{1}{3} = \dfrac{5}{6}$ lawns/hour.

Since $t = \dfrac{d}{r}$, it will take them $\frac{6}{5}$ hour to mow one lawn.

\end{solution}

\question
Two cars start at the same point and travel in the same direction at average speeds of 40 mph and 55 mph.  How much time
must elapse before the two cars are 5 miles apart?

\begin{solution}
Here are the facts from the problem:
\begin{itemize*}
  \item $r_1 = 40$ mph
  \item $r_2 = 55$ mph
  \item $d_2 - d_1 = 5$ miles
  \item $t_1 = t_2$ hours
\end{itemize*}

We can get both distances in terms of the time
\begin{itemize*}
  \item $d_1 = 40t$
  \item $d_2 = 55t$
\end{itemize*}

And plug this into the distance equation to find the time:
\begin{align*}
  d_2 - d_1 &= 5 \\
  55t - 40t &= 5 \\
  15t  &= 5 \\
  t  &= \frac{1}{3} \\
\end{align*}

So in 1/3 hour or 20 minutes the two cars will be 5 miles apart.

\end{solution}

\question
On the first part of a 225-mile automobile trip you averaged 55 mph.  On the last part of the trip you averaged 48 mph
because of increased traffic.  The total trip took 4 hours and 15 minutes.  Find the travel time for each part of the
trip.

\begin{solution}
Here are the facts from the problem:
\begin{itemize*}
  \item $r_1 = 55$ mph
  \item $r_2 = 48$ mph
  \item $t_1 + t_2 = 4 \dfrac{1}{4}$ hours
  \item $d_1 + d_2 = 225$ miles
\end{itemize*}

We can get both distances in terms of $t_1$:
\begin{itemize*}
  \item $d_1 = r_1t_1 = 55t_1$ miles
  \item $t_2 = \dfrac{5}{4} - t_1$ hours
  \item $d_2 = r_2t_2 = 48 \left( \dfrac{17}{4} - t_1 \right)$ miles
\end{itemize*}

Now we can plug the distances into the last equation and solve:
\begin{align*}
  d_1 + d_2 &= 225 \\
  55t_1 + 48(\dfrac{17}{4} - t_1) &= 225 \\
  55t_1 + 204 - 48t_1 &= 225 \\
  7t_1 &= 21 \\
  t_1 &= 3 \\
\end{align*}

So you spent 3 hours traveling at 55 mph and 1 hour and 15 minutes traveling at 48 mph.

\end{solution}

\subsection{Problems with Radicals}

If you didn't do these last week, here they are again...

\question
The time for a pendulum of length $L$ (in feet) to go through one complete cycle is given by:  $t = 2 \pi \sqrt{\dfrac{L}{32}}$

How long is the pendulum of a clock with a cycle of 3 seconds?

\begin{solution}

We can solve the original equation for $L$:
\begin{align*}
  t &= 2 \pi \sqrt{\frac{L}{32}} \\
  t^2 &= 4 \pi^2 \cdot \frac{L}{32} \\
  t^2 &= \frac{\pi^2L}{8} \\
  8t^2 &= \pi^2L\\
  L &= \frac{8t^2}{\pi^2} \\
\end{align*}

Now we have an equation for $L$, we can plug in 3 seconds to get the answer we are looking for:

\[
  L = \frac{8t^2}{\pi^2} = \frac{8 \cdot 3^2}{\pi^2} = \frac{72}{\pi^2} \approx 7.3
\]
\end{solution}

So the pendulum needs to be about 7.3 feet long, which would make a pretty large clock.

\question
The time $t$ (in seconds) for a free-falling object to fall $d$ feet is given by: $t = \sqrt{\dfrac{d}{16}}$

If an object is dropped and hits the ground 3 seconds later, what is the height from which it was dropped?

\begin{solution}

We can solve the original equation for $d$:
\begin{align*}
  t &= \sqrt{\dfrac{d}{16}} \\
  t^2 &= \dfrac{d}{16} \\
  d &= 16t^2 \\
\end{align*}

Now we have an equation for $d$, we can plug in 3 seconds to get the answer we are looking for:

\[
  d = 16t^2 = 16 \cdot 3^2 = 144
\]

So the object was dropped from a height of 144 feet.

\end{solution}

\section{Extra Credit}

{\em For this problem, don't try to come up with a single equation.  Think about the various possibilities for the
  scoring system (number of points for first, second and third) and number of events.  Only one scoring system and
  number of events can explain the meet results, and this leads to the answer to the problem.}

\vspace{0.5 cm}

Three high schools---Washington, Lincoln, and Roosevelt---competed in a track meet.  Each school entered one man, and one
only, in each event.  Susan, a student at Lincoln High, sat in the bleachers to cheer her boyfriend, the school's
shot-put champion.

When Susan returned home later in the day, her father asked her how her school had done.

``We won the shot-put all right,'' she said, ``but Washington High won the track meet.  They had a final score of 22.
We only finished with 9.  So did Roosevelt High.''

``How were the events scored?'' her father asked.

``I don't remember exactly,'' Susan replied, ``but there was a certain number of points for the winner of each event, a
smaller number for second place and a still smaller number for third place.  The numbers were the same for all events.''
(By ``number'' Susan of course meant a positive integer.)

``How many events were there altogether?''

``Gosh, I don't know, Dad.  All I watched was the shot-put.''

``Was there a high jump,'' asked Susan's brother.

Susan nodded.

``Who won it?''

Susan didn't know.

Incredible as it may seem, this last question can be answered with only the information given.  Which school won the
high jump?

\begin{solution}

Here's how I solved this problem.  There are probably other approaches that also work.

First I thought about the minimum number of points that could be awarded for each event.  This would happen with a
scoring system that allocated 1 point for third place, 2 points for second place, and 3 points for first place.  This
leads to 6 points per event.

Then I thought about how many points were available in the entire meet.  The winning team had 22 points and each of the
other teams had 9 points, giving 40 possible points for the meet.

We know that there were more than two events, since at least two different teams won events and the meet didn't end in a
tie for first place.

Since each event is worth the same number of points, the number of points per event must be a divisor of 40.  There are
only a few possibilities that fit with the constraint of at least 6 points per event and more than two events:

\begin{itemize*} 
  \item 4 events worth 10 points each
  \item 5 events worth 8 points each 
\end{itemize*} 

If we explore the first possibility we find that there are several ways that each event could be worth 10 points:

\begin{tabular}{ | c | c | c | p{4 cm} | }
  \hline
  First & Second & Third & Notes \\
  \hline
  7 & 2 & 1 & No way to get 9 points in 4 events.   \\
  6 & 3 & 1 & No way to get 22 points in 4 events.   \\
  5 & 3 & 2 & No way to get 22 points in 4 events.   \\
  5 & 4 & 1 & No way to get 22 points in 4 events.   \\
  \hline
\end{tabular}

So there must be 5 events worth 8 points each.  The possibilities are:

\begin{tabular}{ | c | c | c | p{4 cm} | }
  \hline
  First & Second & Third & Notes \\
  \hline
  5 & 2 & 1 &  Seems to work. \\
  4 & 3 & 1 & No way to get 9 points in 5 events.   \\
  \hline
\end{tabular}

With the scoring system that works:
\begin{itemize*}
  \item team one (Washingon) won 4 events and got second in one event for 22 points
  \item team two (Lincoln) won 1 event and got third in 4 events for 9 points
  \item team three (Roosevelt) got second in 4 events and third in one event for 9 points
\end{itemize*}

Washington won every event but the shot put, which Lincoln won, so Washington won the high jump. 

\end{solution}

\end{questions}


\ifprintanswers

\begin{description}

\subsection{Pages 218-220}
\item[46]

The facts from the problem are:
\begin{itemize*}
  \item $d_w = 30$ miles
  \item $d_k = 20$ miles
  \item $r_w = r_k + 5$ mph
  \item the times are equal
\end{itemize*}

We can get each rate in terms of the time:
\begin{itemize*}
  \item $r_w = \dfrac{d_w}{t} = \dfrac{30}{t}$ mph
  \item $r_k = \dfrac{d_k}{t} = \dfrac{20}{t}$ mph
\end{itemize*}

Then we can use the rates in the rate equation and solve for the time:
\begin{align*}
  r_w &= r_k + 5 \\
  \dfrac{30}{t} &= \dfrac{20}{t} + 5 \\
  30 &= 20 + 5t \\
  5t &= 10 \\
  t &= 2 \\
\end{align*}

We can then plug the time back into the rate equations to get the rates:
\begin{itemize*}
  \item $r_w = \dfrac{d_w}{t} = \dfrac{30}{2} = 15$ mph
  \item $r_k = \dfrac{d_k}{t} = \dfrac{20}{2} = 10$ mph
\end{itemize*}

\item[50]

The facts from the problem are:
\begin{itemize*}
  \item $t_w = 1$ hour
  \item $t_m = \dfrac{50}{60} = \dfrac{5}{6}$ hour
  \item $t_m = \dfrac{1}{2}$
  \item $d_m + d_{mw} = 1$ lawn
\end{itemize*}

We can rearrange things a little to get the rates
\begin{itemize*}
  \item $r_w = 1$ lawn/hour
  \item $r_m = \dfrac{6}{5}$ lawn/hour
  \item $r_{mw} = 1 + \dfrac{6}{5} = \dfrac{11}{5}$ lawn/hour
\end{itemize*}

Since we know Malik worked for a half hour by himself before he was joined by Walt, here is the equation we need to
solve: 
\begin{align*}
  r_mt_m + r_{mw}t_{mw} &= 1 \\
  \frac{6}{5} \cdot \frac{1}{2} + \frac{11}{5}t_{mw} &= 1 \\
  \frac{3}{5} + \frac{11}{5}t_{mw} &= 1 \\
  3 + 11t &= 5 \\
  11t &= 2 \\
  t &= \frac{2}{11}
\end{align*}

So they worked together for $\dfrac{2}{11}$ hours.

\subsection{Pages 262-263}

\item[20]
\begin{align*}
  \sqrt{7x-6} - \sqrt{5x+2} &= 0 \\
  \sqrt{7x-6} &= \sqrt{5x+2} \\
  7x-6 &= 5x+2 \\
  2x &= 8 \\
  x &= 4 \\
\end{align*}

check: $  \sqrt{7 \cdot 4 - 6} - \sqrt{5 \cdot 4 + 2} = 0 $

\item[21]
\begin{align*}
  5\sqrt{t-1} &= 6 \\
  \sqrt{t-1} &= \frac{6}{5} \\
  t-1 &= \frac{36}{25} \\
  t &= \frac{36}{25} + 1\\
  t &= \frac{61}{25}\\
\end{align*}

check: $5\sqrt{\dfrac{61}{25}-1} = 6 $
\item[22]
\begin{align*}
  4 \sqrt{t+3} &= 6 \\
   \sqrt{t+3} &= \frac{6}{4} \\
   \sqrt{t+3} &= \frac{3}{2} \\
   t+3 &= \frac{9}{4} \\
   t &= - \frac{3}{4}
\end{align*}

check: $4 \sqrt{- \dfrac{3}{4}+3} = 6$

\item[23]
\begin{align*}
  \sqrt{x^2+7} &= 4 \\
  x^2+7 &= 16 \\
  x^2-9 &= 0 \\
  (x+3)(x-3) &= 0 \\
\end{align*}

$x = \{-3, 3\}$

check: $\sqrt{9 + 7} = 4$

\item[24]
\begin{align*}
  \sqrt{x^2+3} - 2 &= 0 \\
  \sqrt{x^2+3} &= 2 \\
  x^2+3 &= 4 \\
  x^2 - 1 &= 0 \\
  (x+1)(x-1) &= 0 \\
\end{align*}

$x = \{-1, 1\}$

check: $\sqrt{1+3} - 2 = 0$

\item[31]
\begin{align*}
  \sqrt{-4x+17} &= x-3 \\
  -4x+17 &= x^2-6x + 9 \\
  x^2 - 2x - 8 &= 0 \\
  (x-4)(x+2) &= 0 \\
\end{align*} \\

$x = \{4, -2 \}$

check:
\begin{align*}
  \sqrt{-4 \cdot 4 + 17} &= 4 - 3 \\
  \sqrt{-4 \cdot -2 + 17} &\neq -2 - 3 \\
\end{align*}

So 4 is the only solution.

\item[32]
\begin{align*}
  \sqrt{2x-1} &= x-2 \\
  2x-1 &= x^2 - 4x + 4 \\
  x^2 - 6x + 5 &= 0 \\
  (x-1)(x-5) &= 0 \\
\end{align*}

$x = \{1, 5\}$

check:
\begin{align*}
  \sqrt{2 \cdot 1-1} &\neq 1-2 \\
  \sqrt{2 \cdot 5-1} &= 5-2 \\
\end{align*}

So 5 is the only solution.

\item[33]
\begin{align*}
  \sqrt{n+4} &= n+4 \\
  n+4 &= n^2 + 8n + 16 \\
  n^2+7n+12 &= 0 \\
  (n+3)(n+4) &= 0
\end{align*}

$n = \{-3, -4\}$

check:
\begin{align*}
  \sqrt{-3+4} &= -3+4 \\
  \sqrt{-4+4} &= -4+4 \\
\end{align*}

\item[34]
\begin{align*}
  \sqrt{n+6} &= n+6 \\
  n+6 &= n^2 + 12n + 36 \\
  n^2 + 11n + 30 &= 0 \\
  (n+6)(n+5) &= 0 \\  
\end{align*}

$n = \{-6, -5\}$

check:
\begin{align*}
  \sqrt{-6+6} &= -6+6 \\
  \sqrt{-5+6} &= -5+6 \\
\end{align*}

\item[47]
\begin{align*}
  \sqrt{3x+1} + \sqrt{2x+4} &= 3 \\
  \sqrt{3x+1} &= 3 - \sqrt{2x+4} \\
  3x+1 &= 9 - 6\sqrt{2x+4} + 2x + 4 \\
  x-12 &= -6\sqrt{2x+4} \\
  x^2-24x+144 &= 36(2x+4) \\
  x^2-96x &= 0 \\
  x(x-96) &= 0 \\
\end{align*}

$x = \{0, 96\}$

check:
\begin{align*}
  \sqrt{1} + \sqrt{4} &= 3 \\
  \sqrt{3 \cdot 96 + 1} + \sqrt{2 \cdot 96 + 4} &\neq 3 \\
\end{align*}

So $x=0$ is the only solution.

\item[48]
\begin{align*}
  \sqrt{2x-1} - \sqrt{x+3} &= 1 \\
  \sqrt{2x-1}  &= 1 + \sqrt{x+3} \\
  2x-1  &= 1 + 2\sqrt{x+3} + x + 3 \\
  x-5  &= 2\sqrt{x+3} \\
  x^2-10x+25  &= 4(x+3) \\
  x^2-14x+13 &= 0 \\
  (x-13)(x-1) &= 0 \\
\end{align*}

$n = \{1, 13 \}$

check:
\begin{align*}
  \sqrt{2-1} - \sqrt{2+3} &\neq 1 \\
  \sqrt{26-1} - \sqrt{16} &= 5-4 = 1 \\
\end{align*}

So $x=13$ is the only solution.

\item[52]
\begin{align*}
  \sqrt{t+7} - 2\sqrt{t-8} &= \sqrt{t-5} \\
  (\sqrt{t+7} - 2\sqrt{t-8})(\sqrt{t+7} - 2\sqrt{t-8}) &= (\sqrt{t-5})^2 \\
  t+7 - 4\sqrt{(t-8)(t+7)} + 4(t-8) &= t-5 \\
  t+7 - 4\sqrt{t^2-t-56} + 4t - 32 &= t-5 \\
  5t -25 - 4\sqrt{t^2-t-56}  &= t-5 \\
  -4\sqrt{t^2-t-56}  &= -4t+20 \\
  -4\sqrt{t^2-t-56}  &= -4(t-5) \\
  \sqrt{t^2-t-56}  &= t-5 \\
  t^2-t-56  &= t^2-10t+25 \\
  -t-56  &= -10t+25 \\
  9t-56  &= 25 \\
  9t  &= 81 \\
  t  &= 9 \\
\end{align*}

check:
\[
  \sqrt{9+7} - 2\sqrt{9-8} = \sqrt{16} - 2 \sqrt{1} = 4 - 2 = \sqrt{9-5} \\
\]

\end{description}

\fi

\ifprintanswers
\else
\vspace{0.3 in}

{\em
\begin{verse}
I wear the black for the poor and the beaten down \\
Living in the hopeless, hungry side of town \\
I wear it for the prisoner who has long paid for his crime \\
But is there because he's a victim of the times
\end{verse}
}

\hspace{1 in} --Johnny Cash, {\em Man in Black}

\fi

\end{document}

