
\documentclass[fleqn,addpoints]{exam}

\usepackage{amsmath}
\usepackage{graphics}
\usepackage{cancel}
\usepackage{polynom}
\usepackage{2in1, lscape}

\title{Math 113 Chapter Four Study Guide}
\author{}
\date{April 21, 2010}

\begin{document}

\maketitle

\section{Section 4.1--Simplifying Rational Expressions}

\subsection{Procedure}

\begin{enumerate}
  \item factor the numerator and denominator
  \item cancel any of the factors which appear in both the numerator and denominator
\end{enumerate}

Remember that you can only cancel things which are multiplied together.  For example, in this expression:

\( \displaystyle \frac{xy+y^2}{x^2-y^2} \)

you can't cancel out $y^2$ because it is added to $xy$ on the top and subtracted from $x^2$ on the bottom.

\subsection{Examples}

\begin{itemize}
  \item \( \displaystyle \frac{32a^2b^3}{16a^4b} \)
  \item \( \displaystyle \frac{x^2-4}{x^2+2x} \)
  \item \( \displaystyle\frac{5x^2+6n-8}{10n^2-3n-4} \)
  \item \( \displaystyle\frac{4x^2y+8xy^2-12y^3}{18x^3y-12x^2y^2-6xy^3} \)
\end{itemize}

\section{Section 4.2--Multiplying and Dividing Rational Expressions}

\subsection{Procedure}

\begin{enumerate}
  \item factor the numerators and denominators
  \item if it is a division problem, flip the second fraction over (``copy, change flip'')
  \item write a single fraction which contains the numerators and denominators of the original fractions multiplied together
  \item cancel any of the factors which appear in both the numerator and denominator
\end{enumerate}

\subsection{Examples}
\begin{itemize}
  \item \( \displaystyle \frac{5a^2b^2}{11ab} \cdot \frac{22a^3}{15ab^2} \)
  \item \( \displaystyle \frac{x^2-4xy+4y^2}{7xy^2} \div \frac{4x^2-3xy-10y^2}{20x^2y+25xy^2} \)
  \item \( \displaystyle \frac{4xy^2}{7x} \cdot \frac{14x^3y}{12x} \div \frac{7y}{9x^3} \)
\end{itemize}

\section{4.3-4.4--Adding and Subtracting Rational Expressions}

\subsection{Procedure}

\begin{enumerate}
  \item factor the numerators and denominators
  \item find a common denominator which includes the factors from all the denominators
  \item give each fraction the common denominator by multiplying top and bottom by the factors which are missing in that
    fraction
  \item form a single fraction with the common denominator on the bottom and the sum (or difference) of the numerators
    on the top
\end{enumerate}

When subtracting, make sure that you apply the subtraction to the entire numerator, not just the first term.  For
example:

\( \displaystyle \frac{x}{x+2} - \frac{2x - 1}{x+2} = \frac{x - (2x - 1)}{x+2} = \frac{x-2x+1}{x+2} = \frac{-x+1}{x+2} \)

A common mistake would be to forget that the minus sign also applies to the $-1$ which would give you the incorrect
answer of: \( \displaystyle \frac{-x-1}{x+2} \).

\subsection{Examples}
\begin{itemize}
  \item \( \displaystyle \frac{2x}{x-1} + \frac{4}{x-1} \)
  \item \( \displaystyle \frac{2n}{n^2-25} - \frac{3}{4n+20} \)
  \item \( \displaystyle \frac{2n^2}{n^4-16} - \frac{n}{n^2-4} + \frac{1}{n+2}\)
\end{itemize}

\section{4.4--Complex Fractions}

\subsection{Procedure}
A complex fraction is a fraction which includes at least one fraction in its numerator or denominator.  There are two
approaches for simplifying complex fractions.  Either one is fine for any problem and you should use the approach you
find easiest.

\subsubsection{Approach One}

One approach is to treat the numerator and denominator as separate problems:
\begin{itemize}
  \item find a common denominator for the fractions in the numerator and add/subtract them
  \item find a common denominator for the fractions in the denominator and add/subtract them
  \item flip the denominator over and multiply 
  \item combine like terms and simplify
\end{itemize}

\subsubsection{Approach Two}

Another approach is to get rid of all the fractions at once:
\begin{itemize}
  \item find a common denominator for all the fractions in the numerator and denominator
  \item multiply top and bottom by the common denominator.  This will make all the fractions go away.
  \item combine like terms and simplify
\end{itemize}

\subsection{Examples}

\begin{itemize}
  \item \( \displaystyle \cfrac{ \cfrac{6}{a} - \cfrac{5}{b^2} } { \cfrac{12}{a^2} + \cfrac{2}{b} } \)
  \item \( \displaystyle \cfrac{\cfrac{2}{x-3} - \cfrac{3}{x+3} } { \cfrac{5}{x^2-9} - \cfrac{2}{x-3} } \)
  \item \( \displaystyle \cfrac{a}{\cfrac{1}{a}+4} + 1 \)  
\end{itemize}

\section{4.5--Polynomial Division}

\subsection{Procedure}

See the book or answer key.

\subsection{Examples}
\begin{itemize}
  \item \( \displaystyle \frac{9x^4 + 18x^3}{3x} \)
  \item \( \displaystyle \frac{x^2-7x-78}{x+6} \)
  \item \( \displaystyle (x^3-8) \div (x-4) \)
\end{itemize}

\section{4.6-4.7--Fractional Equations and Applications}

\subsection{Procedures}

\subsubsection{Fractional Equations}

\begin{enumerate}
  \item note any values for the variable which would make a denominator zero.  These are values which aren't allowed as solutions.
  \item find a common denominator for the fractions which appear in the equation
  \item multiply both sides by the LCD you found in step 2.  This will make the fractions go away.
  \item follow the usual procedures to solve the remaining equation
  \item discard any of the solutions which are not allowed because they would make the denominator zero.
\end{enumerate}

\subsection{Examples}

\begin{itemize}
  \item \( \displaystyle n + \frac{1}{n} = \frac{17}{4} \)
  \item \( \displaystyle \frac{5}{7x-3} = \frac{3}{4x-5} \)
  \item \( \displaystyle \frac{x}{4x-4} + \frac{5}{x^2 - 1} = \frac{1}{4} \)
  \item A sum of \$1,750 is to be divided between two people in the ratio of 3 to 4.  How much does each person receive?
  \item Ken drives his Mazda 270 miles in the same time that it takes Dave to drive his Nissan 250 miles.  If Kent
    averages 4 miles per hour faster than Dave find their rates.
\end{itemize}

\subsection{Rate Problems}

Many problems are generalizations of the $d = rt$ problems we solved earlier.  The difference is that the ``distance''
might be something like ``lawns mowed,'' ``words typed,'' or ``dollars earned'' instead of an actual distance.  The same
basic approach works whenever something is being done at a steady rate for a length of time.

There are several ways of approaching this type of problem.  Here's an approach that often works for me.

The problem is number 58 on page 220: 

{\em Felipe jogs for 10 miles and then walks another 10 miles.  He jogs 2.5 miles per
hour faster than he walks, and the entire distance of 20 miles takes 6 hours.  Find the rate at which he walks and the
rate at which he jogs}.

The first thing I do is write down all the facts from the problem as variables or very short expressions.  For this
problem, the facts are:

\begin{itemize}
  \item $d_{jog} = 10$
  \item $d_{walk} = 10$
  \item $r_{walk}$
  \item $\displaystyle r_{jog} = r_{walk} + \frac{5}{2}$  
  \item $t_{jog} + t_{walk} = 6$
\end{itemize}

You may prefer to use a table format for this step.

The next thing I try to do is pick a single variable and get all the other variables in terms of that variable by
rearranging things and substituting.

Since $d = rt$, $\displaystyle t = \frac{d}{r}$.  So $\displaystyle t_{jog} = \frac{d_{jog}}{r_{jog}}$ 
and $ \displaystyle t_{walk} = \frac{d_{walk}}{r_{walk}}$ 
And we can use this to change the last equation into: $\displaystyle \frac{10}{r_{jog}} + \frac{10}{r_{walk}} = 6$

Next we can plug the expression for $r_{jog}$ into this equation, which gives us:

\( \displaystyle \frac{10}{r_{walk} + \cfrac{5}{2}} + \frac{10}{r_{walk}} = 6 \)

We can solve this equation to find that: $\displaystyle r_{walk} = \frac{5}{2}$.

Once we know $r_{walk}$, we can figure out all the other values by plugging the value for $r_{walk}$ into the original
equations.  In this case, we have 
\begin{itemize}
  \item $\displaystyle r_{jog} = r_{walk} + \frac{5}{2} = 5$
  \item $ \displaystyle t_{walk} = \frac{10}{r_{walk}} = \frac{10}{\cfrac{5}{2}} = 10 \cdot \frac{2}{5} = 4$
  \item $ \displaystyle t_{jog} = \frac{10}{r_{jog}} = \frac{10}{5} = 2$
\end{itemize}

The last thing to do is to check the answer to see if it all fits together.  Felipe jogged for 2 hours at 5 mph, giving
him a total of 10 miles jogging.  Felipe walked for 4 hours at $2.5$ mph, which gives him a total of 10 miles
walking.

The numbers are also fairly reasonable.  A reasonable jogging pace is between 5 and 10 mph and a reasonable walking pace
is between 2 and 5 mph.  So while Felipe is not the speediest guy in the world, we don't have him jogging faster than a car
or slower than a turtle.

\end{document}