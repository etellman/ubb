% no answer key
% \documentclass[letterpaper]{exam}

% answer key
\documentclass[letterpaper, landscape]{exam}
\usepackage{2in1, lscape} 
\printanswers{}

\usepackage{units} 
\usepackage{xfrac} 
\usepackage[fleqn]{amsmath}
\usepackage{commath}
\usepackage{cancel}
\usepackage{float}
\usepackage{mdwlist}
\usepackage{booktabs}
\usepackage{cancel}
\usepackage{polynom}
\usepackage{caption}
\usepackage{fullpage}
\usepackage{comment}
\usepackage{enumerate}
\usepackage{graphicx}
\usepackage{mathtools} 
\usepackage{parskip} 

\newcommand{\degree}{\ensuremath{^\circ}} 
\everymath{\displaystyle}

\title{Algebra \\ Homework 7 \\ Sections 3.1 and 3.2 }
\author{}
\date{\today}


\begin{document}

  \maketitle

  \section{Homework}

  \begin{itemize*}
      \item Section 3.1: 11--20, 26--30, 36--40, 36--40, 50--55, 61--68
      \item Section 3.2: 5--10, 21--25, 50--55, 65--70, 81--84, 92
  \end{itemize*}

  \section{Extra Credit}

  A silver prospector was unable to pay his March rent in advance.  He owned a bar of pure silver, 31 inches long, so he
  made the following arrangement with his landlady.  He would cut the bar, he said, into smaller pieces.  On the first day
  of March he would give the lady an inch of the bar, and on each succeeding day he would add another inch to the amount
  of silver.  She would keep this silver as security.  At the end of the month, when the prospector expected to be able to
  pay his rent in full, she would return the pieces to him.

  March has 31 days, so one way to cut the bar would be to cut it into 31 sections, each an inch long.  But since it
  required considerable labor to cut the bar, the prospector wished to carry out his arrangement with the fewest possible
  number of pieces.  For example, he might give the lady an inch on the first day, another inch the second day, then on
  the third day he could take back the two pieces and give her a solid 3-inch section.

  Assuming the portions of the bar are traded back and forth in this fashion, what is the smallest number of pieces into
  which the prospector needs to cut his silver bar?

  (Martin Gardner)

  \begin{solution}

      The most efficient approach is that each time you run out of pieces
      you cut off a new piece exactly as long as you need to meet that
      day's rent.  

      \begin{itemize}
          \item On the first day, you have zero pieces, so you cut off a piece of size 1.  
          \item On the second day, you need a piece of size 2, so you cut off a piece of size 2.
          \item On the third day, you give the landlady both pieces.
          \item On the fourth day, you need another piece, so you cut off a piece of size 4.
          \item On days 5--7, you use the pieces you already have.
          \item On the eighth day, you need another piece, so you cut off a piece of size 8.
          \item On days 9--15, you use the pieces you already have.
          \item On the sixteenth day, you need another piece.  Fortunately, the rest of the bar is size 16, so you use that.
          \item On days 17--31, you use the pieces you already have.
      \end{itemize}

      You need to cut the bar into pieces of size 1, 2, 4, 8, and 16 and four cuts are required.

  \end{solution}
  \ifprintanswers{}
    \begin{description}
      \item[15] $y^2 + 6y - 55$

      \item[16]
      \( (y - 3)(y + 9) = y^2 - 3y + 9y - 27 = y^2 + 6y - 27 \)

      \item[17]
      \( (n + 2)(n - 7) = n^2 + 2n - 7n - 14 = n^2 - 5n - 14 \)

      \item[18]
      \( (n + 3)(n - 12) = n^2 + 3n - 12n - 36 = n^2 - 9n - 36 \)

      \item[19]
      \( (x + 6)(x - 6) = x^2 + 6x - 6x - 36 = x^2 - 36 \)

      \item[20]
      \( (t + 8)(t - 8) = t^2 + 8t - 8t - 64 = t^2 - 64 \)

      \item[53]
      \begin{eqnarray*}
        && (t + 3)(t^2 - 3t - 5) \\
        &=& t^3 - 3t^2 - 5t + 3t^2 - 9t - 15 \\
        &=& t^3 - 14t - 15 \\
      \end{eqnarray*}

      \item[54]
      \begin{eqnarray*}
        && (t - 2)(t^2 + 7t + 2) \\
        &=& t^3 + 7t^2 + 2t - 2t^2 - 14t - 4 \\
        &=& t^3 + 5t^2 - 12t - 4 \\
      \end{eqnarray*}

      \item[55]
      \begin{eqnarray*}
        && (x - 4)(x^2 + 5x -4) \\
        &=& x^3 + 5x^2 - 4x - 4x^2 - 20x + 16 \\
        &=& x^3 + x^2 - 24x + 16 \\
      \end{eqnarray*}

      \item[56]
      \begin{eqnarray*}
        && (x+6)(2x^2 - x - 7) \\
        &=& 2x^3 - x^2 - 7x + 12x^2 - 6x - 42 \\
        &=& 2x^3 + 11x^2 - 13x - 42 \\
      \end{eqnarray*}

      \item[62]
      \begin{eqnarray*}
        && (x^2 - x + 6)(x^2 - 5x - 8) \\
        &=& x^4 - 5x^3 - 8x^2 - x^3 + 5x^2 + 8x + 6x^2 - 30x - 48 \\
        &=& x^4 - 6x^3 + 3x^2 - 22x - 48 \\
      \end{eqnarray*}

      \item[63]
      \begin{eqnarray*}
        && (2x^2 + 3x - 4)(x^2 - 2x - 1) \\
        &=& 2x^4 - 4x^3 - 2x^2 + 3x^3 - 6x^2 - 3x - 4x^2 + 8x + 4 \\
        &=& 2x^4 - x^3 - 12x^2 + 5x + 4 \\
      \end{eqnarray*}

      \item[86]

      The area of the first rectangle is \( 3(x + 4) \) and the area of the second rectangle is \( 4(x + 6) \).
        The area of the sum is:

      \[ 3(x + 4) + 4(x + 6) = 3x + 12 + 4x + 24 = 7x + 36 \]

    \end{description}
  \fi
  \ifprintanswers{}
  \else
    \vspace{3 cm}
    \begin{quote}
      \begin{em}
        Any man more right than his neighbor is a majority of one already.
      \end{em}
    \end{quote}
    \hspace{2 cm}--Henry David Thoreau
  \fi

\end{document}

