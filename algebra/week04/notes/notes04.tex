\documentclass[fleqn]{article}
\usepackage{amsmath}

\title{Math 113--Week Four Notes}
\author{}
\date{February 10, 2010}

\oddsidemargin 0in
\topmargin -0.5in
\textwidth 6.5in
\textheight 9in

\setlength{\mathindent}{1in}

\begin{document}

\maketitle

\section{Chapter Two Test}

Next week we will have the {\em Chapter Two} test.  You can prepare by reviewing the homework and doing some of the review
problems and problems from the sample test at the end of the chapter.  The format will be
similar to the {\em Placement Exam} with about 20 questions. 

\section{New Terminology}

\begin{description}
  \item[$(a, b)$] \(a < x < b\).  Parentheses mean that the endpoints are not included.
  \item{$[a, b]$} \(a \leq x \leq b\).  Square brackets mean that the endpoints are included.
  \item[\(C = A \cap B\)] intersection.  $A$, $B$, and $C$ are sets where $C$ contains the elements which are in {\em both} $A$ {\em and} $B$.
  \item[\(C = A \cup B\)] Aunion.  $A$, $B$, and $C$ are sets where $C$ contains the elements which are in {\em either} $A$
    {\em or} $B$.
\end{description}

\section{Solving Inequalities}

When solving equations with inequalities, almost all the same rules apply as when working with equalities.  You need to
make sure to keep the equation balanced by always doing the same thing to both sides.  

The only difference is that when you are working with inequalities, you need to reverse the direction of the inequality
when you multiply both sides by a negative number. For example: \( -x < 8 \) becomes \( x > -8 \).

To solve two equations combined with an {\em and}:
\begin{itemize}
  \item first solve each equation separately
  \item the solution is the intersection of the two equations' solutions.  In other words, the solution is all the numbers which are
    solutions to both equations.
\end{itemize}

To solve two equations combined with an {\em or}:
\begin{itemize}
  \item first solve each equation separately
  \item the solution is the union of the two equations' solutions.  In other words, the solution is all the numbers which are
    solutions to either equation.
\end{itemize}

\section{Inequalities With Absolute Value}

\begin{itemize}
  \item \( |x| < k \) is equivalent to $x > -k$ {\em and} $x < k$, or, more concisely, $-k < x < k$, where $k$ is a positive number.
  \item \( |x| > k \) is equivalent to $x > k$ {\em or} $x < -k$, where $k$ is a positive number.
\end{itemize}

\end{document}
