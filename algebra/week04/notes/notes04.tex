\documentclass[letterpaper, landscape]{exam}
\usepackage{2in1, lscape} 
\printanswers{}

\usepackage{units} 
\usepackage{xfrac} 
\usepackage[fleqn]{amsmath}
\usepackage{commath}
\usepackage{cancel}
\usepackage{float}
\usepackage{mdwlist}
\usepackage{booktabs}
\usepackage{cancel}
\usepackage{polynom}
\usepackage{caption}
\usepackage{fullpage}
\usepackage{comment}
\usepackage{enumerate}
\usepackage{graphicx}
\usepackage{mathtools} 

\newcommand{\degree}{\ensuremath{^\circ}} 
\everymath{\displaystyle}

\title{Week Four Notes}
\author{}
\date{\today}

\begin{document}

  \maketitle

  \section{Terminology}

  Solutions to inequalities are ranges rather than single numbers
  \begin{itemize*}
    \item parentheses for open intervals (endpoint not included)
    \item square brackets for closed intervals (endpoint included)
    \item draw various intervals and number lines
  \end{itemize*}

  \section{Rules} % (fold)
  
  \subsection{Addition} % (fold)

  You can add the same thing to both sides of an inequality. It doesn't matter if $c$ is positive or
  negative.

  \begin{align*}
    a < b & \iff a + c < b + c \\
    a > b & \iff a + c > b + c \\
  \end{align*}
    
  examples:
  \begin{enumerate}
    \item 
      \begin{align*}
        2 & < 3 \\
        \\
        2 + 10 & < 3 + 10 \\
        12     & < 13 \\
        \\
        2 - 5 & < 3 - 5 \\
        -3    & < -2 \\
      \end{align*}

    \item 
      \begin{align*}
        x + 3 & < 7 \\
        x     & < 4 \\
        x     & \in \boxed{ \intoo{ -\infty, 4 } } \\
      \end{align*}

    \item 
      \begin{align*}
        x - 5 & > -2 \\
        x     & > 3 \\
        x     & \in \boxed{ \intoo{ 3, \infty } } \\
      \end{align*}

    \item 
      \begin{align*}
        2x - 5 & \geq x + 1 \\
        2x     & \geq x + 6 \\
        x      & \geq 6 \\
        x      & \in \boxed{ \intco{ 6, \infty } } \\
      \end{align*}

      Show how gap increases as $x = \cbr{ 5, 6, 7, 8, \ldots}$.

  \end{enumerate}

  \subsection{Multiplication} % (fold)

  You can multiply both sides of an inequality by a positive number 

  $c > 0$:
  \begin{align*}
    a < b & \iff ac < bc \\
    a > b & \iff ac > bc \\
  \end{align*}

  examples:
  \begin{align*}
    4                   & < 8 \\
    2 \cdot 4           & < 2 \cdot 8 \\
    \frac{1}{2} \cdot 4 & < \frac{1}{2} 8 \\
    \\
    x  & < 10 \\
    2x & < 20 \\
  \end{align*}

  When you multiply by a negative number, you need to switch the direction of
  the inequality:
  
  $c < 0$:
  \begin{align*}
    a < b & \iff ac > bc \\
    a > b & \iff ac < bc \\
  \end{align*}

  examples:
  \begin{align*}
    1  & < 2 \\
    -1 & > -2 \\
    \\
    x  & < 0 \\
    -x & > 0 \\
    \\
    x  & > 10 \\
    -x & < 10 \\
  \end{align*}
    
  \section{Examples} % (fold)

  \begin{enumerate}
    \item $4x < 12$

    \item $x - 5 \geq 7$

    \item $2x + 7 < 3 + 4x$

    \item $6x - 4 \leq 2 + 8x$

    \item $\frac{1}{2} (8x + 1) \geq 3x + \frac{5}{2}$

    \item $4(x + 1) < 2x + 3$

    \item $2x - 7 < 6x + 13$

    \item $2x - 3 < 7x + 22$

    \item $4(x - 3) \leq -2(x + 1)$

    \item $3(x - 1) \geq -(x + 4)$

    \item $4(2x - 1) - 3(3x + 4) \geq 0$

    \item $-5(3x + 4) < -2(7x - 1)$

    \item $-3(2x + 1) > 2(x + 4)$

  \end{enumerate}
  

\end{document}
