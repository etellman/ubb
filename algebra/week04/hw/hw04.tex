% no answer key
% \documentclass[letterpaper]{exam}

% answer key
\documentclass[letterpaper, landscape]{exam}
\usepackage{2in1, lscape} 
\printanswers{}

\usepackage{units} 
\usepackage{xfrac} 
\usepackage[fleqn]{amsmath}
\usepackage{commath}
\usepackage{cancel}
\usepackage{float}
\usepackage{mdwlist}
\usepackage{booktabs}
\usepackage{cancel}
\usepackage{polynom}
\usepackage{caption}
\usepackage{fullpage}
\usepackage{comment}
\usepackage{enumerate}
\usepackage{graphicx}
\usepackage{mathtools} 
\usepackage{parskip} 

\newcommand{\degree}{\ensuremath{^\circ}} 
\everymath{\displaystyle}

\title{Algebra \\ Homework Four \\ Section 2.5}
\author{}
\date{\today}


\begin{document}

  \maketitle

  \section{Section 2.5}
  1--15, 21--25, 31--40, 46--50, 61--65, 69--70, 74

  \section{Other Problems}

  Diophantes was a Greek mathematician who lived in the third century C.E.  The only details
  historians know about his life are provided by a puzzle one of his friends wrote about him after
  he died.  Here it is:

  \begin{itemize*}
    \item His boyhood lasted \sfrac{1}{6} of his life
    \item He grew a beard after another \sfrac{1}{12}
    \item After \sfrac{1}{7} more he married
    \item And had a son 5 years later
    \item The son lived to half his father's final age
    \item The father died 4 years later.
  \end{itemize*}

  How old was Diophantes when he died, and how old was he when each of the events mentioned occurred?

  \begin{solution}
    If we let $x$ be Diophantes' age when he died, the equation is:
    \begin{align*}
      \frac{x}{6} + \frac{x}{12} + \frac{x}{7} + 5 + \frac{x}{2} + 4 & = x \\
      x                                                              & = \boxed{ \unit[84]{years} } \\
    \end{align*}

    Here is when everything happened:
    \begin{itemize*}
      \item His boyhood lasted until he was 14
      \item He grew a beard when he was 21 
      \item He married when he was 33
      \item He had a son when he was 38
      \item The son lived to age 42, when the father was 80
      \item The father died 4 years later at age 84
    \end{itemize*}

  \end{solution}

  \ifprintanswers{}
  \else
    \newpage
  \fi

  \section{Extra Credit} % (fold)

  One morning, exactly at sunrise, a Buddhist monk began to climb a tall mountain.  The narrow path, no more than a
  foot or two wide, spiraled around the mountain to a glittering temple at the summit.

  The monk ascended the path at varying rates of speed, stopping many times along the way to rest and to eat the dried
  fruit he carried with him.  He reached the temple shortly before sunset.  After several days of fasting and
  meditation, he began his journey back along the same path, starting at sunrise and again walking at variable speeds
  with many pauses along the way.  His average speed descending was, of course, greater than his average climbing
  speed.

  Prove that there is a spot along the path that the monk will occupy on both trips at precisely the same time of day.

  (Martin Gardner)

  \begin{solution}
    Picture two monks making the trip on the same day, with one monk starting at the top of the mountain and one monk
    starting at the bottom of the mountain.  Clearly, they will meet on the path some time during the day, since the
    path is narrow and they must pass each other on the way to their respective destinations.  When this happens, both
    monks will be at the same place at the same time of day.  

    For any two possible trips up and down the mountain, there is always a point where the trips overlap.  If a
    single monk makes the two trips separated by a night, his trips still must overlap at some point.

  \end{solution}

  \ifprintanswers{}

    \section{Section 2.5 Solutions} 

    \begin{description}
      \item[1] $\intoo{ 1, \infty }$

      \item[2] $\intoo{ -2, \infty }$

      \item[3] $\intco{ -1, \infty }$

      \item[4] $\intco{ 3, \infty }$

      \item[5] $\intoo{ -\infty, -2 }$

      \item[6] $\intoo{ -\infty, 1 }$

      \item[7] $\intoc{ -\infty, 2 }$

      \item[8] $\intoc{ -\infty, 0 }$
        
      \item[9] $x < 4$

      \item[10] $x < -2$

      \item[11] $x \leq -7$

      \item[12] $x \leq 9$

      \item[13] $x > 8$

      \item[14] $x > -5$

      \item[15] $x \geq -7$

      \item[21] 
        \begin{align*}
          5x & \leq -10 \\
          x  & \leq -2 \\
        \end{align*}
        $\boxed{ \intoc{ -\infty, -2 } }$

      \item[22] 
        \begin{align*}
          4x & \geq -4 \\
          x  & \geq -1 \\
        \end{align*}
        $\boxed{ \intco{ -1, \infty } }$

      \item[23] 
        \begin{align*}
          2x + 1 & < 5 \\
          x      & < 2 \\
        \end{align*}
        $\boxed{ \intoo{ -\infty, 2 } }$

      \newpage

      \item[24] 
        \begin{align*}
          2x + 2 & > 4 \\
          x      & > 1 \\
        \end{align*}
        $\boxed{ \intoo{ 1, \infty } }$

      \item[25] 
        \begin{align*}
          3x - 2 & > -5 \\
          x      & > -1 \\
        \end{align*}
        $\boxed{ \intoo{ -1, \infty } }$

      % \item[30]
      %   \begin{align*}
      %     1 + 6x & > -17 \\
      %     6x     & > -18 \\
      %     x      & > -3 \\
      %   \end{align*}
      %   $\boxed{ (-3, \infty) }$

      \item[31]
        \begin{align*}
          5 - 3x & < 11 \\
          -3x    & < 6 \\
          x      & > -2 \\
        \end{align*}
        $\boxed{ \intoo{ -2, \infty } }$

      \item[32]
        \begin{align*}
          4 - 2x & < 12 \\
          -2x    & < 8 \\
          x      & > -4 \\
        \end{align*}
        $\boxed{ \intoo{ -4, \infty } }$ 

      \item[33]
        \begin{align*}
          15     & < 1 - 7x \\
          1 - 7x & > 15 \\
          -7x    & > 14 \\
          x      & < -2 \\
        \end{align*}
        $\boxed{ \intoo{ -\infty, -2 } }$

      \item[34]
        \begin{align*}
          12     & < 2 - 5x \\
          2 - 5x & > 12 \\
          -5x    & > 10 \\
          x      & < -2 \\
        \end{align*}
        $\boxed{ \intoo{ -\infty, -2 } }$ 

      \item[35]
        \begin{align*}
          -10    & \leq 2 + 4x \\
          2 + 4x & \geq -10 \\
          x      & \geq -3 \\
        \end{align*}
        $\boxed{ \intco{ -3, \infty} }$

      \item[36]
        \begin{align*}
          -9     & \leq 1 + 2x \\
          1 + 2x & \geq -9 \\
          x      & \geq -5 \\
        \end{align*}
        $\boxed{ \intco{ -5, \infty } }$

      \item[37]
        \begin{align*}
          3 (x + 2) & > 6 \\
          x         & > 0 \\
        \end{align*}
        $\boxed{ \intoo{ 0, \infty } }$

      \item[38]
        \begin{align*}
          2 (x - 1) & < -4 \\
          x         & < -1 \\
        \end{align*}
        $\boxed{ \intoo{ -\infty, -1 } }$

      \item[39]
        \begin{align*}
          5x + 2 & \geq 4x + 6 \\
          x      & \geq 4 \\
        \end{align*}
        $\boxed{ \intco{ 4, \infty } }$

      \item[40]
        \begin{align*}
          6x - 4 & \leq 5x - 4 \\
          6x & \leq 5x \\
          x      & \leq 0 \\
        \end{align*}
        $\boxed{ \intoc{ -\infty, 0 } }$

      \item[46]
        \begin{align*}
          -2(3x + 2) & \leq 18 \\
          -6x - 4    & \leq 18 \\
          x          & \geq - \frac{11}{3} \\
        \end{align*}
        $\boxed{ \intco{ -\frac{11}{3}, \infty } }$

      \item[47]
        \begin{align*}
          4 (3x - 2) & \geq -3 \\
          12x - 8    & \geq -3 \\
          x          & \geq \frac{5}{12} \\
        \end{align*}
        $\boxed{ \intco{ \frac{5}{12}, \infty } }$

      \item[48]
        \begin{align*}
          3 (4x - 3) & \leq -11 \\
          12x - 9    & \leq -11 \\
          x          & \leq - \frac{1}{6} \\
        \end{align*}
        $\boxed{ \intoc{ -\infty, - \frac{1}{6} } }$

      \item[49]
        \begin{align*}
          6x - 2 & > 4x - 14 \\
          2x     & > -12 \\
          x      & > -6 \\
        \end{align*}
        $\boxed{ \intco{ -6, \infty } }$

      \item[50]
        \begin{align*}
          9x + 5 & < 6x - 10 \\
          3x     & < -15 \\
          x      & < -5 \\
        \end{align*}
        $\boxed{ \intoo{ -\infty, -5 } }$

      % \item[60]
      %   \begin{align*}
      %     3(x - 1) - (x - 2) & > -2(x + 4) \\
      %     3x - 3 - x + 2     & > -2x - 8 \\
      %     2x - 1             & > -2x - 8 \\
      %     4x - 1             & > -8 \\
      %     4x                 & > -7 \\
      %     x                  & > -\frac{7}{4} \\
      %   \end{align*}
      %   $\boxed{ \del{-\frac{7}{4}, \infty} }$

      \item[61]
        \begin{align*}
          7(x + 1) - 8(x - 2) & < 0 \\
          7x + 7 - 8x + 16    & < 0 \\
          -x + 23             & < 0 \\
          x                   & > 23 \\
        \end{align*}
        $\boxed{ (23, \infty) }$

      \item[62]
        \begin{align*}
          5(x - 6) -6(x + 2) & < 0 \\
          5x - 30 - 6x -12   & < 0 \\
          -x - 42            & < 0 \\
          x                  & > -42 \\
        \end{align*}
        $\boxed{ (-42, \infty) }$

      \item[63]
        \begin{align*}
          -5(x - 1) + 3 & > 3x - 4 -4x \\
          -5x + 5 + 3   & > -x - 4 \\
          -4x           & > -12 \\
          x             & < 3 \\
        \end{align*}
        $\boxed{ (-\infty, 3) }$

      \item[64]
        \begin{align*}
          3(x + 2) + 4 & < -2x + 14 + x \\
          3x + 10      & < -x + 14 \\
          x            & < 1 \\
        \end{align*}
        $\boxed{ (-\infty, 1) }$

      \item[65]
        \begin{align*}
          3(x - 2) - 5(2x - 1) & \geq 0 \\
          3x - 6 - 10x + 5     & \geq 0 \\
          -7x - 1              & \geq 0 \\
          x                    & \leq - \frac{1}{7} \\
        \end{align*}
        $\boxed{ \intoc{ -\infty, - \frac{1}{7} } }$

      \item[69]
        \begin{align*}
          -3(x + 2) & > 2(x - 6) \\
          -3x - 6   & > 2x - 12 \\
          -5x       & > -6 \\
          x         & < \frac{6}{5} \\
        \end{align*}
        $\boxed{ \intoc{ -\infty, \frac{6}{5} } }$

      \item[70]
        \begin{align*}
          -2(x - 4) & < 5(x - 1) \\
          -2x + 8   & < 5x - 5 \\
          -7x       & < -13 \\
          x         & > \frac{13}{7} \\
        \end{align*}
        $\boxed{ \intoo{ \frac{13}{7}, \infty } }$

      \item[74]
        \begin{enumerate}[(a)]
          \item
            \begin{align*}
              5x - 2 & > 5x + 3 \\
              -2     & > 3 \\
            \end{align*}
            \fbox{ empty set }.

          \item
            \begin{align*}
              3x - 4 & < 3x + 7 \\
              -4     & < 7 \\
            \end{align*}
            $\boxed{ \intoo{ -\infty, \infty } }$.

          \item
            \begin{align*}
              4(x + 1) & < 2(2x + 5) \\
              % 4x + 4   & < 4x + 5 \\
              4        & < 5 \\
            \end{align*}
            $\boxed{ \intoo{ -\infty, \infty } }$.

          \item
            \begin{align*}
              -2(x - 1) & > 2(x + 7) \\
              -2x + 2   & > 2x + 14\\
              % -4x       & > 12\\
              x         & < -3\\
            \end{align*}
            $\boxed{ \intoo{ -\infty, -3 } }$.

          \item
            \begin{align*}
              3(x - 2) & < -3(x + 1) \\
              3x - 6   & < -3x - 3 \\
              % 6x       & < 3 \\
              x        & < \frac{1}{2} \\
            \end{align*}
            $\boxed{ \intoo{ -\infty, \frac{1}{2} } }$.

          \item
            \begin{align*}
              2(x + 1) + 3(x + 2) & < 5(x - 3) \\
              2x + 2 + 3x + 6     & < 5x - 15 \\
              % 5x + 8              & < 5x - 15 \\
              8                   & < -15 \\
            \end{align*}
            \fbox{ empty set }

        \end{enumerate}
    \end{description}
  \fi

  \ifprintanswers{}
  \else
    \vspace{12 cm}

    \begin{verse}
      seeker of truth

      follow no path \\
      all paths lead where

      truth is here 
    \end{verse}
    \hspace{2 cm} {\em e.e.\ cummings\/}

    \newpage

  \fi

\end{document}


