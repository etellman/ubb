% no answer key
\documentclass[letterpaper]{exam}

% answer key
% \documentclass[letterpaper, landscape]{exam}
% \usepackage{2in1, lscape} 
% \printanswers{}

\usepackage{units} 
\usepackage{xfrac} 
\usepackage[fleqn]{amsmath}
\usepackage{commath}
\usepackage{cancel}
\usepackage{float}
\usepackage{mdwlist}
\usepackage{booktabs}
\usepackage{cancel}
\usepackage{polynom}
\usepackage{caption}
\usepackage{fullpage}
\usepackage{comment}
\usepackage{enumerate}
\usepackage{graphicx}
\usepackage{mathtools} 
\usepackage{parskip} 

\newcommand{\degree}{\ensuremath{^\circ}} 
\everymath{\displaystyle}

\title{Algebra \\ Homework Four \\ Section 2.5}
\author{}
\date{\today}


\begin{document}

  \maketitle

  \section{From the Book}

  \begin{itemize*}
    \item pp. 85--86: 30--36, 60--65, 74
    \item pp. 94--95: 5, 6, 40--44, 53, 56, 58, 59, 62
    \item pp. 101--102: 30--36
  \end{itemize*}

  \section{Problems}

  \begin{questions}

    \question{}
    Diophantes was a Greek mathematician who lived in the third century C.E.  The only details
    historians know about his life are provided by a puzzle one of his friends wrote about him after
    he died.  Here it is:

    \begin{itemize*}
      \item His boyhood lasted \sfrac{1}{6} of his life
      \item He grew a beard after another 1/12
      \item After 1/7 more he married
      \item And had a son 5 years later
      \item The son lived to half his father's age
      \item And the father died 4 years later.
    \end{itemize*}

    How old was Diophantes when he died, and how old was he when each of the events mentioned occurred?

    \begin{solution}
    If we let $x$ be Diophantes' age when he died, the equation is:
    \begin{align*}
      x/6 + x/12 + x/7 + 5 + x/2 + 4 &=& x \\
      2x/12 + x/12 + x/7 + 6x/12 + 9 &=& x \\
      9x/12 + x/7 + 9 &=& x \\
      3x/4 + x/7 + 9 &=& x \\
      28(3x/4 + x/7 + 9) &=& 28x \\
      21x + 4x + 252 &=& 28x \\
      25x + 252 &=& 28x \\
      -3x &=& -252 \\
      x &=& 84 \\
    \end{align*}

    So here is when everything happened:
    \begin{itemize*}
      \item His boyhood lasted until he was 14
      \item He grew a beard when he was 21 
      \item He married when he was 33
      \item He had a son when he was 38
      \item The son lived to age 42, when the father was 80
      \item The father died 4 years later at age 84
    \end{itemize*}

    \end{solution}
    \question[]

    This problem uses the following two formulas:
    \begin{align*}
      V & = & \frac{4}{3} \pi r^3 \\
      A & = & 4 \pi r^2 \\
    \end{align*}
    Where $V$ is the volume of a sphere and $A$ is the surface area of a sphere.

    In H.G. Well's novel, {\em The First Men in the Moon}, the moon is found to be inhabited by intelligent
    insect-like creatures who live in caves below the surface.  These creatures, let us assume, have a unit of distance that we
    shall call a ``lunar.''  It was adopted because the moon's surface area, if expressed in square lunars, exactly equals
    the moon's volume in cubic lunars.  The moon's radius is 1,080 miles.  How many miles long is a lunar?

    \begin{solution}
    Since the volume in cubic lunars must equal the area in square lunars, we can set the two formulas equal to each other
    and solve for $r$.  This will tell us how long the radius is in lunars:
    \begin{align*}
      \frac{4}{3} \pi r^3 &=& 4 \pi r^2 \\
      \frac{4}{3} r &=& 4 \\
      \frac{r}{3} &=& 1 \\
      r &=& 3 \\
    \end{align*}

    So the radius of the moon is 3 lunars. Consequently, 3 lunars are 1,080 miles and one lunar is 360 miles.

    \end{solution}

    \pagebreak

    \section{Extra Credit}

    \question{}

    One morning, exactly at sunrise, a Buddhist monk began to climb a tall mountain.  The narrow path, no more than a foot
    or two wide, spiraled around the mountain to a glittering temple at the summit.

    The monk ascended the path at varying rates of speed, stopping many times along the way to rest and to eat the dried
    fruit he carried with him.  He reached the temple shortly before sunset.  After several days of fasting and meditation,
    he began his journey back along the same path, starting at sunrise and again walking at variable speeds with many pauses
    along the way.  His average speed descending was, of course, greater than his average climbing speed.

    Prove that there is a spot along the path that the monk will occupy on both trips at precisely the same time of day.

    \begin{solution}

    Picture two monks making the trip on the same day, with one monk starting at the top of the mountain and one monk
    starting at the bottom of the mountain.  Clearly, they will meet on the path some time during the day, since the path is
    narrow and they must pass each other on the way to their respective destinations.  When this happens,
    both monks will be at the same place at the same time of day.  

    So for any two possible trips up and down the mountain, there is always a point where the trips overlap.  If a single
    monk makes the two trips separated by several days, his trips still must overlap at some point.

    \end{solution}

  \end{questions}
  \ifprintanswers{}

    \subsection{Pages 85--86} 

    \begin{description}

      \item[30]
      \begin{align*}
        1 + 6x & > & -17 \\
        6x     & > & -18 \\
        x      & > & -3 \\
      \end{align*}
      $\boxed{ (-3, \infty) }$

      \item[31]
        \begin{align*}
          5 - 3x & < & 11 \\
          -3x    & < & 6 \\
          x      & > & -2 \\
        \end{align*}
      The solution is: \( (-2, \infty) \)

      check: \( 5 - 3 \cdot (-2) = 11\) and anything greater than $-2$ will produce a smaller number on the left side.

      \item[32]
      \begin{align*}
        4 - 2x &<& 12 \\
        -2x &<& 8 \\
        x &>& -4 \\
      \end{align*}
      The solution is: \( (-4, \infty) \)

      check: \( 4 - 2 \cdot (-4) = 12\) and anything greater than $-4$ will produce a smaller number on the left side.

      \item[33]
      \begin{align*}
        15 &<& 1 - 7x \\
        14 &<& -7x \\
        -7x &>& 14 \\
        x &<& -2 \\
      \end{align*}
      The solution is: \( (-\infty, -2) \)

      check: \( 1 - 7 \cdot (-2) = 15\) and anything smaller than $-2$ will produce a larger number on the right side.

      \item[34]
      \begin{align*}
        12 &<& 2 - 5x \\
        10 &<& -5x \\
        -5x &>& 10 \\
        x &<& -2 \\
      \end{align*}
      The solution is: \( (-\infty, -2) \)

      check: \( 2 - 5 \cdot (-2) = 12\) and anything smaller than $-2$ will produce a larger number on the right side.

      \item[35]
      \begin{align*}
        -10 &\leq& 2 + 4x \\
        -12 &\leq& 4x \\
        4x &\geq& -12 \\
        x &\geq& -3 \\
      \end{align*}
      The solution is: \( [-3, \infty) \)

      check: \( 2 + 4 \cdot (-3) = -10\) and anything larger than $-3$ will produce a larger number on the right side.

      \item[36]
      \begin{align*}
        -9 &\leq& 1 + 2x \\
        -10 &\leq& 2x \\
         2x &\geq& -10 \\
         x &\geq& -5 \\
      \end{align*}
      The solution is: \( [-5, \infty) \)

      check: \( 1 + 2 \cdot (-5) = -9\) and anything larger than $-5$ will produce a larger number on the right side.

      \item[60]
      \begin{align*}
        3(x - 1) - (x - 2) &>& -2(x + 4) \\
        3x - 3 - x + 2 &>& -2x - 8 \\
        2x - 1 &>& -2x - 8 \\
        4x - 1 &>& -8 \\
        4x &>& -7 \\
        x &>& -7/4 \\
      \end{align*}
      The solution is: \( (-7/4, \infty) \)

      check: \( 3(-7/4 - 1) - (-7/4 - 2) = -2(-7/4 + 4)  \) and anything larger than $-4/7$ will produce a larger number on
      the left side and a smaller number on the right side.

      \item[61]
      \begin{align*}
        7(x + 1) - 8(x - 2) &<& 0 \\
        7x + 7 - 8x + 16 &<& 0 \\
        -x + 23 &<& 0 \\
        -x &<& -23 \\
        x &>& 23 \\
      \end{align*}
      The solution is: \( (23, \infty) \)

      check: \( 7(23 + 1) - 8(23 - 2) = 0 \) and anything larger than $23$ will produce a smaller number on
      the left side.

      \item[62]
      \begin{align*}
        5(x - 6) -6(x + 2) &<& 0 \\
        5x - 30 - 6x -12 &<& 0 \\
        -x - 42 &<& 0 \\
        -x &<& 42 \\
        x &>& -42 \\
      \end{align*}
      The solution is: \( (-42, \infty) \)

      check: \( 5(-42 - 6) -6(-42 + 2) = 0 \) and anything larger than $-42$ will produce a smaller number on
      the left side.

      \item[63]
      \begin{align*}
        -5(x - 1) + 3 &>& 3x - 4 -4x \\
        -5x + 5 + 3 &>& -x - 4 \\
        -5x + 8 &>& -x - 4 \\
        -4x + 8 &>& -4 \\
        -4x  &>& -12 \\
        x  &<& 3 \\
      \end{align*}
      The solution is: \( (-\infty, 3) \)

      check: \( -5(3 - 1) + 3 = 3(3) - 4 -4(3) \) and anything less than $3$ will make the left side greater than the right side.

      \item[64]
      \begin{align*}
        3(x + 2) + 4 &<& -2x + 14 + x \\
        3x + 6 + 4 &<& -x + 14 \\
        3x + 10 &<& -x + 14 \\
        4x + 10 &<& 14 \\
        4x &<& 4 \\
        x &<& 1 \\
      \end{align*}
      The solution is: \( (-\infty, 1) \)

      check: \( 3(1 + 2) + 4 = -2 + 14 + 1 \) and anything less than $1$ will make the left side less than the right side.

      \item[65]
      \begin{align*}
        3(x - 2) - 5(2x - 1) &\geq& 0 \\
        3x - 6 - 10x + 5 &\geq& 0 \\
        -7x - 1 &\geq& 0 \\
        -7x  &\geq& 1 \\
        x  &\leq& -1/7 \\
      \end{align*}
      The solution is: \( (-\infty, -1/7] \)

      check: \( 3(-1/7 - 2) - 5(2(-1/7) - 1) = 0 \) and anything less than $-1/7$ will make the left side greater than zero.

      \item[74]
        \begin{description}
        \item[a]
        \begin{align*}
          5x - 2 &>& 5x + 3 \\
          -2 &>& 3 \\
        \end{align*}
      There is no value of $x$ which will make this statement true, so the solution set is the empty set.

        \item[b]
        \begin{align*}
          3x - 4 &<& 3x + 7 \\
          -4 &<& 7 \\
        \end{align*}
      Any value of $x$ makes this statement true, so the solution set is $(-\infty, \infty)$.

        \item[c]
        \begin{align*}
          4(x + 1) &<& 2(2x + 5) \\
          4x + 4 &<& 4x + 5 \\
          4 &<& 5 \\
        \end{align*}
      Any value of $x$ makes this statement true, so the solution set is $(-\infty, \infty)$.

        \item[d]
        \begin{align*}
          -2(x - 1) &>& 2(x + 7) \\
          -2x + 2 &>& 2x + 14\\
          -2x &>& 2x + 12\\
          -4x &>& 12\\
          x &<& -3\\
        \end{align*}
      The solution is: $(-\infty, -3)$.

      check: \( -2(-3 - 1) = -2(-4) = 2(-3 + 7) = 2(-4)\) and anything less than $-3$ will make the left side larger and the right side smaller.

        \item[e]
        \begin{align*}
          3(x - 2) &<& -3(x + 1) \\
          3x - 6 &<& -3x - 3 \\
          6x - 6 &<& - 3 \\
          6x &<& 3 \\
          x &<& 1/2 \\
        \end{align*}
      $\boxed{ (-\infty, 1/2) }$.

      \item[f]
        \begin{align*}
          2(x + 1) + 3(x + 2) & < 5(x - 3) \\
          2x + 2 + 3x + 6     & < 5x - 15 \\
          5x + 8              & < 5x - 15 \\
          8                   & < -15 \\
        \end{align*}

        There is no value of $x$ which will make this statement true, so the solution set is the empty set.

        \end{description}

      \subsection{Pages 94--95} 

      \item[5]
        \begin{align*}
          \frac{x-2}{3} + \frac{x+1}{4}     & \geq \frac{5}{2} \\
          12(\frac{x-2}{3} + \frac{x+1}{4}) & \geq 12(\frac{5}{2}) \\
          4x - 8 + 3x + 3                   & \geq 30 \\
          7x - 5                            & \geq 30 \\
          7x                                & \geq 35 \\
          x                                 & \geq 5 \\
        \end{align*}
        $\boxed{ [5, \infty) }$

      \item[6]
        \begin{align*}
          15(\frac{x-1}{3} + \frac{x+2}{5}) & \leq & 15(\frac{3}{5}) \\
          5x - 5 + 3x + 6                   & \leq & 9 \\
          8x + 1                            & \leq & 9 \\
          8x                                & \leq & 8 \\
          x                                 & \leq & 1 \\
        \end{align*}
        $ \boxed{ (-\infty, 1] } $

      \item[40]
        \begin{align*}
          3x + 2 > 17 & and x \geq 0 \\
          3x > 15     & and x \geq 0 \\
          x > 5       & and x \geq 0 \\
        \end{align*}
      $ \boxed{ (5, \infty) } $

      \item[41]
        \begin{align*}
          5x - 2 < 0 &and& 3x - 1 > 0 \\
          5x < 2 &and& 3x > 1 \\
          x < 2/5 &and& x > 1/3 \\
        \end{align*}
        $\boxed{ (1/3, 2/5) }$

      \item[42]
      \begin{align*}
        x + 1 > 0 &and& 3x - 4 < 0 \\
        x > -1 &and& 3x < 4 \\
        x > -1 &and& x < 4/3 \\
      \end{align*}
      The solution is: \( (-1, 4/3) \)

      \item[43]
      \begin{align*}
        3x + 2 < -1 &or& 3x + 2 > 1 \\
        3x < -3 &or& 3x > -1 \\
        x < -1 &or& x > -1/3 \\
      \end{align*}
      The solution is: \( (-\infty, -1) \cup (-1/3, \infty) \)

      \item[44]
      \begin{align*}
        5x - 2 < -2 &or& 5x - 2 > 2 \\
        5x < 0 &or& 5x > 4 \\
        x < 0 &or& x > 4/5 \\
      \end{align*}
      The solution is: \( (-\infty, 0) \cup (4/5, \infty) \)

      \item[53]
      \begin{align*}
        -4 \leq & \frac{x - 1}{3} & \leq 4 \\
        -12 \leq & x - 1 & \leq 12 \\
        -11 \leq & x & \leq 13 \\
      \end{align*}
      The solution is: \( [-11, 13] \)

      \item[56]
      \begin{align*}
        -4 < & 3 - x & < 4 \\
        -7 < & -x & < 1 \\
        7 >  & x & > -1 \\
        -1 < & x & < 7 \\
      \end{align*}
      The solution is: \( [-1, 7] \)

      \item[58]
      \begin{align*}
        100 \cdot 0.08 + 0.9x &>& 26 \\
        8 + 0.9x &>& 26 \\
        0.9x &>& 18 \\
        x &>& 200 \\
      \end{align*}
      The solution is: \( (200, \infty) \)

      \item[59]

      6 feet 8 inches is $72 + 8 = 80$ inches.

      6 feet 4 inches is $72 + 4 = 76$ inches.

      \begin{align*}
        \frac{3 \cdot 80 + 2x}{5} &\geq& 76 \\
        \frac{240 + 2x}{5} &\geq& 76 \\
        240 + 2x &\geq& 380 \\
        2x &\geq& 140 \\
        x &\geq& 70 \\
      \end{align*}

      The solution is: \( [70, \infty] \), although you would probably have a hard time finding a basketball player
      that was close to $\infty$ inches tall, and if you did he might have a hard time fitting into the gym.

      \item[62]

      \begin{align*}
        \frac{95 + 82 + 93 + 84 + x}{5} &\geq& 90 \\
        354 + x &\geq& 450 \\
        x &\geq& 96 \\
      \end{align*}
      The solution is \( [96, 100] \), assuming there isn't any extra credit and 100 is the highest possible score.

      \subsection{Pages 101--102} 

      \item[30]

      \( | 3 - 4x | = 8 \)

      \begin{align*}
        3 - 4x = 8 &or& -(3 - 4x) = 8 \\
         -4x = 5 &or& -3 + 4x = 8 \\
         x = -5/4 &or& 4x = 11 \\
         x = -5/4 &or& x = 11/4 \\
      \end{align*}
      The solution is \( \{-5/4, 11/4\} \).

      \item[31]

      \( | 2 - x | > 4 \)

      \begin{align*}
        -(2 - x) > 4 &or& 2 - x > 4 \\
        -2 + x > 4 &or&  -x > 2 \\
        x > 6 &or&  x < -2 \\
      \end{align*}
      $ \boxed{ (-\infty, -2) \cup (6, \infty) } $

      \item[32]

      \( | 4 - x | > 3 \)

      \begin{align*}
        -(4 - x) > 3 &or& 4 - x > 3 \\
        -4 + x > 3 &or&  -x > -1 \\
        x > 7 &or&  x < 1 \\
      \end{align*}
      The solution is \( (-\infty, 1) \cup (7, \infty) \).

      \item[33]

      \( | 1 - 2x | < 2 \)
      \begin{align*}
        -(1 - 2x) < 2 &and& 1 - 2x < 2 \\
        -1 + 2x < 2 &and& -2x < 1 \\
        2x < 3 &and& x > -1/2 \\
        x < 3/2 &and& x > -1/2 \\
      \end{align*}
      The solution is \( (-1/2, 3/2) \)

      \item[34]

      \( | 2 - 3x | < 5 \)

      \begin{align*}
        -(2 - 3x) < 5 &and& 2 - 3x < 5 \\
        -2 + 3x < 5 &and& -3x < 3 \\
        3x < 7 &and& x > -1 \\
        x < 7/3 &and& x > -1 \\
      \end{align*}
      The solution is \( (-1, 7/3) \)

      \item[35]

      \( | 5x + 9 | \leq 16 \)

      \begin{align*}
        -(5x + 9) \leq 16 &and& 5x + 9 \leq 16 \\
        -5x - 9 \leq 16 &and& 5x \leq 7 \\
        -5x \leq 25 &and& x \leq 7/5 \\
        x \geq -5 &and& x \leq 7/5 \\
      \end{align*}
      The solution is \( [-5, 7/5] \)

      \item[36]

      \( | 7x - 6 | \geq 22 \)

      \begin{align*}
        -(7x - 6) \geq 22 & or & 7x - 6 \geq 22 \\
        -7x + 6 \geq 22   & or & 7x \geq 28 \\
        -7x \geq 16       & or & x \geq 4 \\
        x \leq -16/7      & or & x \geq 4 \\
      \end{align*}

      $\boxed{ (-\infty, -16/7] \cup [4, \infty) }$

    \end{description}
  \fi


  \ifprintanswers{}
  \else
  \vspace{5 in}

  \begin{verse}
  seeker of truth

  follow no path \\
  all paths lead where

  truth is here 
  \end{verse}
  \hspace{1 in} e.e. cummings

  \fi

\end{document}


