
\documentclass[fleqn,addpoints]{exam}

\usepackage{amsmath}
\usepackage{graphics}
\usepackage{cancel}
\usepackage{polynom}
\usepackage{mdwlist}

% \printanswers

\ifprintanswers
\usepackage{2in1, lscape}
\fi

\title{Math 113 Homework 16}
\author{}
\date{\today}

\begin{document}

\maketitle

\section{From the Book}

Read pages 296-309.
\begin{itemize*}
  \item pp. 300-301: 25-29, 34-38, 44, 45, 47, 50, 60
  \item pp. 308-309: 1-5, 20-24, 38-42, 49, 50
\end{itemize*}

\section{Extra Credit}

If you are given the solutions to an equation, you can find the equation by working backwards.  Find an equation whose
solutions are $3 + 2i$ and $3-2i$.

\ifprintanswers
\pagebreak
\fi

\begin{solution}
Working backwards, we can start with a factored equation and then use FOIL to get the original equation:
\begin{align*}
  (x-(3+2i))(x-(3-2i)) &= 0 \\
  x^2 - x(3-2i) - x(3+2i)+(3+2i)(3-2i) &= 0 \\
  x^2 - 3x + 2ix - 3x - 2ix + 9 + 6i - 6i - 4i^2 &= 0 \\
  x^2 - 6x + 9 + 4 &= 0 \\
  x^2 - 6x + 13 &= 0 \\
\end{align*}

We can check the solutions in the equation:
\begin{itemize}
  \item $3+2i + 3 - 2i = 6$
  \item $(3+2i)(3 - 2i) = 9 - 4i^2 = 9 + 4 = 13$
\end{itemize}

Of course, there is more than one equation with these two solutions.  Any equation of the form:
\[ k(x^2 - 6x + 13) = 0 \]
 where $k$ is any number, will also work.

\end{solution}

\ifprintanswers

\section{Pages 300-301}
\begin{description}

\item[25] 
\begin{align*}
  n^2 + 2n + 6 &= 0 \\
  n^2 + 2n &= -6 \\
  n^2 + 2n + 1 &= -6 + 1 \\
  (n+1)^2 &= -5 \\
  n+1 &= \pm \sqrt{-5} \\
  n &= - 1 \pm i\sqrt{5} \\
\end{align*}

\item[26] 
\begin{align*}
  n^2 + n - 1 &= 0 \\
  n^2 + n  &= 1 \\
  n^2 + n  + \frac{1}{4} &= 1 + \frac{1}{4} \\
  \left( n + \frac{1}{2} \right)^2 &= \frac{5}{4} \\
  n + \frac{1}{2} &= \frac{\pm \sqrt{5}}{2} \\
  n &= \frac{-1 \pm \sqrt{5}}{2} \\
\end{align*}

\item[27] 
\begin{align*}
  x^2 + 3x - 2 &= 0 \\
  x^2 + 3x &= 2 \\  
  x^2 + 3x + \frac{9}{4} &= 2 + \frac{9}{4} \\  
  \left( x + \frac{3}{2} \right)^2 &= \frac{17}{4} \\
  x + \frac{3}{2} &= \frac{\pm \sqrt{17}}{2} \\
  x &= \frac{-3 \pm \sqrt{17}}{2} \\
\end{align*}

\item[28] 
\begin{align*}
  x^2 + 5x - 3 &= 0 \\
  x^2 + 5x &= 3 \\
  x^2 + 5x + \frac{25}{4} &= 3 + \frac{25}{4} \\
  \left( x + \frac{5}{2} \right)^2 &= \frac{37}{4} \\
  x + \frac{5}{2} &= \frac{\pm \sqrt{37}}{2} \\
 x &= \frac{-5 \pm \sqrt{37}}{2} \\
\end{align*}

\item[29] 
\begin{align*}
  x^2 + 5x + 1 &= 0 \\
  x^2 + 5x &= -1 \\
  x^2 + 5x + \frac{25}{4} &= \frac{25}{4} - 1 \\
  \left( x + \frac{5}{2} \right)^2 &= \frac{21}{4} \\
  x + \frac{5}{2} &= \frac{\pm \sqrt{21}}{2} \\
  x &= \frac{-5 \pm \sqrt{21}}{2} \\
\end{align*}

\item[34] 
\begin{align*}
  2t^2 - 4t + 1 &= 0 \\
  2t^2 - 4t &= -1 \\
  t^2 - 2t &= - \frac{1}{2} \\
  t^2 - 2t + 1 &= 1 - \frac{1}{2} \\
  (t-1)^2 &= \frac{1}{2} \\
  t-1 &= \pm \sqrt{\frac{1}{2}} \\
  t-1 &= \frac{\pm \sqrt{2}}{2} \\
  t &= 1 \pm \frac{\sqrt{2}}{2} \\
\end{align*}

\item[35] 
\begin{align*}
  3n^2 - 6n + 5 &= 0 \\
  3n^2 - 6n  &= -5 \\
  n^2 - 2n  &= - \frac{5}{3} \\
  n^2 - 2n + 1 &= 1 - \frac{5}{3} \\
  (n-1)^2 &= - \frac{2}{3} \\
  n &= 1 \pm \frac{i \sqrt{6}}{3}
\end{align*}

\item[36] 
\begin{align*}
  3x^2 + 12x - 2 &= 0 \\
  3x^2 + 12x &= 2 \\
  x^2 + 4x &= \frac{2}{3} \\
  x^2 + 4x + 4 &= 4 + \frac{2}{3} \\
  (x+2)^2 &= \frac{14}{3} \\
  x+2 &= \pm \sqrt{\frac{14}{3}} \\
  x &= -2 \pm {\frac{\sqrt{42}}{3}} \\
\end{align*}

\item[37] 
\begin{align*}
  3x^2 + 5x - 1 &= 0 \\
  3x^2 + 5x &= 1 \\
  x^2 + \frac{5x}{3} &= \frac{1}{3} \\
  x^2 + \frac{5x}{3} \frac{25}{36} &= \frac{25}{36} + \frac{1}{3} \\
  \left( x + \frac{5}{6} \right)^2 &= \frac{37}{36} \\
  x + \frac{5}{6} &= \frac{\pm \sqrt{37}}{6} \\
  x &= \frac{-5 \pm \sqrt{37}}{6} \\
\end{align*}

\item[38] 
\begin{align*}
  2x^2 + 7x - 3 &= 0 \\
  2x^2 + 7x  &= 3 \\
  x^2 + \frac{7x}{2}  &= \frac{3}{2} \\
  x^2 + \frac{7x}{2} + \frac{49}{16} &= \frac{3}{2} + \frac{49}{16}\\
  \left( x + \frac{7}{4} \right)^2 &= \frac{73}{16} \\
  x + \frac{7}{4} &= \frac{\pm \sqrt{73}}{4} \\
  x  &= \frac{-7 \pm \sqrt{73}}{4} \\
\end{align*}

\item[44] 
\[
  (5x+2)(x-4) = 0 
\]

$x = \left \{ - \dfrac{2}{5}, 4 \right \}$

\item[45] 
\begin{align*}
  (x+2)(x-7) &= 10 \\
  x^2 - 5x - 14 &= 10 \\
  x^2 - 5x - 24 &= 0 \\
  (x-8)(x+3) &= 0
\end{align*}

$x = \{-3, 8\}$

\item[47] 
\begin{align*}
  (x-3)^2 &= 12 \\
  x - 3 &= \pm \sqrt{12} \\
  x = 3 \pm 2 \sqrt{3} \\
\end{align*}

\item[50] 
\begin{align*}
  2n^2 - 2n - 1 &= 0 \\
  2n^2 - 2n &= 1 \\
  n^2 - n &= \frac{1}{2} \\
  n^2 - n + \frac{1}{4} &= \frac{1}{4} + \frac{1}{2} \\
  \left( n-\frac{1}{2} \right)^2 &= \frac{3}{4} \\
  n - \frac{1}{2} &= \frac{\pm \sqrt{3}}{2} \\
  n &= \frac{1 \pm \sqrt{3}}{2} \\
\end{align*}

\item[50] 
\begin{align*}
  5(x+2)^2 + 1 &= 16 \\
  5(x+2)^2 &= 15 \\
  (x+2)^2 &= 3 \\
  x+2 &= \pm \sqrt{3} \\
  x &= -2 \pm \sqrt{3} \\
\end{align*}

\end{description}

\section{Pages 308-309}
\begin{description}

\item[1] \( x^2 + 4x - 21 = 0 \)

The discriminant is \(4^2 - (4)(-21) = 100\).  The discriminant is positive so there are two real solutions.

\begin{align*}
  x &= \frac{-4 \pm \sqrt{100}}{2} \\
  &= \frac{-4 \pm 10}{2} \\
  &= \{ -7, 3 \} \\
\end{align*}
check:
\begin{itemize}
  \item \( 7-3 = 4 \)
  \item \( (7)(-3) = -21 \)
\end{itemize}

\item[2] \( x^2 - 3x - 54 = 0 \)

The discriminant is \( (-3)^2 - (4)(-54) = 225\).  The discriminant is positive so there are two real solutions.

\begin{align*}
  x &= \frac{3 \pm \sqrt{225}}{2} \\
  &= \frac{3 \pm 15}{2} \\
  &= \{-6, 9\} \\
\end{align*}
check:
\begin{itemize}
  \item \( \dfrac{3+3}{2} = 3 \)
  \item \( \dfrac{9 - 225}{4} = -54 \)
\end{itemize}


\item[3] \( 9x^2-6x+1 = 0 \)
The discriminant is \( (-6)^2 - (4)(9) = 0 \).  The discriminant is zero so there is one real solution with a
multiplicity of two. 

\[
  x = \frac{6}{18} = \frac{1}{3}
\]

check: 
\[
  9 \left(\frac{1}{3} \right)^2 - 6 \left( \frac{1}{3} \right) + 1 = 1 - 2 + 1 = 0
\]

\item[4] \( 4x^2 + 20x + 25 = 0 \)

The discriminant is \( 20^2 - (4)(4)(25) = 0 \).  The discriminant is zero so there is one real solution with a
multiplicity of two. 

\[
  x = \frac{-20}{8} = - \frac{5}{2}\\
\]

check:
\begin{itemize}
  \item \( \dfrac{-5-5}{2} = -2 \)
  \item \( \left( - \dfrac{5}{2} \right)^2 = \dfrac{25}{4} \)
\end{itemize}

\item[5] \( x^2 - 7x + 13 = 0 \)

The discriminant is \( (-7)^2 - (4)(13) = -3\).  The discriminant is negative so there are two nonreal solutions.

\begin{align*}
  x &= \frac{7 \pm \sqrt{-3}}{2} \\
  &= \frac{7 \pm i \sqrt{3}}{2} \\
\end{align*}

check:
\begin{itemize}
  \item \( \dfrac{7+7}{2} = 7\)
  \item \( \dfrac{49 + 3}{4} = 13 \)
\end{itemize}

\item[20] \( x^2 + 19x + 70 = 0 \)
\begin{align*}
  x &= \frac{-19 \pm \sqrt{19^2 - 4 \cdot 70}}{2} \\
  x &= \frac{-19 \pm \sqrt{81}}{2} \\
  x &= \frac{-19 \pm 9}{2} \\
  x &= \{-14, -5 \} \\
\end{align*}

check:
\begin{itemize}
  \item \( -14-5=19 \)
  \item \( (-14)(-5) = 70 \)
\end{itemize}

\item[21] 
\begin{align*}
  -y^2 &= -9y + 5 \\
  y^2 - 9y + 5 &= 0 \\
\end{align*}

\begin{align*}
  y &= \frac{9 \pm \sqrt{(-9)^2 - (4)(5)}}{2} \\
    &= \frac{9 \pm \sqrt{61}}{2}
\end{align*}

check:
\begin{itemize}
  \item \( \dfrac{9 + 9}{2} = 9\)
  \item \( \dfrac{81-61}{4} = 5 \)
\end{itemize}

\item[22] \(  \)
\begin{align*}
  -y^2 + 7y &= 4 \\
  y^2 - 7y + 4 &= 0 \\
\end{align*}

\begin{align*}
  y &= \frac{7 \pm \sqrt{(-7)^2 - (4)(4)}}{2} \\
  &= \frac{7 \pm \sqrt{33}}{2}
\end{align*}

check:
\begin{itemize}
  \item \( \dfrac{7 + 7}{2} = 7 \)
  \item \( \dfrac{49-33}{4} = 4 \)
\end{itemize}

\item[23] \( 2x^2+x-4=0 \)
\begin{align*}
  x &= \frac{-1 \pm \sqrt{1 - (4)(2)(-4)}}{4} \\
  &= \frac{-1 \pm \sqrt{33}}{4} \\
\end{align*}

check:
\begin{itemize}
  \item \( \dfrac{-1 - 1}{4} = - \dfrac{1}{2} \)
  \item \( \dfrac{1 - 33}{16} = -2 \)
\end{itemize}

\item[24] \( 2x^2 + 5x - 2 = 0 \)
\begin{align*}
  x &= \frac{-5 \pm \sqrt{5^2 - (4)(2)(-2)}}{4} \\
  &= \frac{-5 \pm \sqrt{41}}{4} \\
\end{align*}

check:
\begin{itemize}
  \item \( \dfrac{-5-5}{4} = \dfrac{-5}{2} \)
  \item \( \dfrac{25-41}{16} = -1 \)
\end{itemize}

\item[38] \( 7x^2+12x = 0 \)
\begin{align*}
  x &= \frac{-12 \pm \sqrt{12^2}}{14} \\
  &= \frac{-12 \pm 12}{14} \\
  &= \left \{ - \frac{12}{7}, 0 \right \} \\
\end{align*}

check:
\begin{itemize}
  \item \( - \dfrac{12}{7} + 0 = - \dfrac{12}{7} \)
  \item \( - \dfrac{12}{7} \cdot 0 = 0 \)
\end{itemize}


\item[39] 
\begin{align*}
  3x^2 &= 5 \\
  3x^2 - 5 &= 0 \\
\end{align*}

\begin{align*}
  x &= \frac{\pm \sqrt{- (4)(3)(-5)}}{6} \\
  &= \frac{\pm \sqrt{15}}{3}
\end{align*}

check: \( 3 \left( \dfrac{\sqrt{15}}{3} \right)^2 = \dfrac{15}{3} = 5 \)

\item[40]
\begin{align*}
  4x^2 &= 3 \\
  4x^2 - 3 &= 0 \\
\end{align*}

\begin{align*}
  x &= \frac{\pm \sqrt{- (4)(4)(-3)}}{8} \\
  &= \frac{\pm 4 \sqrt{3}}{8} \\
  &= \frac{\pm \sqrt{3}}{2} \\
\end{align*}

check: \( 4 \left( \dfrac{\sqrt{3}}{2} \right)^2 = 3 \)

\item[41] \( 6t^2 + t - 3 = 0 \)
\begin{align*}
  t &= \frac{-1 \pm \sqrt{1 - (4)(6)(-3)}}{12} \\
  &= \frac{-1 \pm \sqrt{73}}{12} \\
\end{align*}

check:
\begin{itemize}
  \item \( \dfrac{-1-1}{12} = - \dfrac{1}{6} \)
  \item \( \dfrac{1-73}{144} = - \dfrac{1}{2} \)
\end{itemize}

\item[42] \( 2t^2 + 6t - 3 = 0 \)
\begin{align*}
  t &= \frac{-6 \pm \sqrt{6^2 - (4)(2)(-3)}}{4} \\
  &= \frac{-6 \pm \sqrt{60}}{4} \\
  &= \frac{-3 \pm \sqrt{15}}{2} \\
\end{align*}

check:
\begin{itemize}
  \item \( \dfrac{-3-3}{2} = -3 \)
  \item \( \dfrac{9-15}{4} = - \dfrac{3}{2} \)
\end{itemize}

\item[49] \( -6x^2 + 2x + 1 = 0 \)
\begin{align*}
  x &= \frac{-2 \pm \sqrt{2^2 - (4)(-6)}}{-12} \\
  &= \frac{-2 \pm \sqrt{28}}{-12} \\
  &= \frac{1 \pm \sqrt{7}}{6} \\
\end{align*}

check:
\begin{itemize}
  \item \( \dfrac{1+1}{6} = \dfrac{1}{3} \)
  \item \( \dfrac{1-7}{36} = - \dfrac{1}{6} \)
\end{itemize}

\item[50] \( -2x^2 + 4x + 1 = 0 \)
\begin{align*}
  x &= \frac{-4 \pm \sqrt{4^2 - (4)(-2)}}{-4} \\
  &= \frac{-4 \pm \sqrt{24}}{-4} \\
  &= \frac{2 \pm \sqrt{6}}{2} \\
\end{align*}

check:
\begin{itemize}
  \item \( \dfrac{2 + 2}{2} = 2 \)
  \item \( \dfrac{4 - 6}{4} = - \dfrac{1}{2} \)
\end{itemize}

\end{description}

\else
\vspace{4 in}

\begin{em}
  I heartily accept the motto, ``That government is best which governs least''; and I should like to see it acted up to
  more rapidly and systematically. Carried out, it finally amounts to this, which also I believe---``That government is
  best which governs not at all''; and when men are prepared for it, that will be the kind of government which they will
  have.
\end{em}

\vspace{0.1 in}
\hspace{0.5 in} --Henry David Thoreau

\fi

\end{document}

