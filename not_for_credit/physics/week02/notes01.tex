\documentclass{article}

\usepackage{graphicx}
\usepackage[fleqn]{amsmath}
\usepackage{cancel}
\usepackage{polynom}
\usepackage{float}
\usepackage{mdwlist}
\usepackage{booktabs}
\usepackage{cancel}
\usepackage{polynom}
\usepackage{caption}

\newcommand{\degree}{\ensuremath{^\circ}} 

\oddsidemargin 0in
\topmargin -0.5in
\textwidth 6.5in

\title{Introduction to Physics \\ Notes}
\date{September 12, 2011}
\author{Ed Tellman}

\begin{document}

\maketitle

\section{Vectors}
A vector describes something with both a size and a direction.  Examples of vectors are velocity, acceleration, and force.
\begin{itemize} 
  \item Vectors are represented in text like this: $\vec{A}$.  Sometimes people use a bold font ($\mathbf{A}$) instead
    of or in addition to the arrow.
  \item The length of $\vec{A}$ is usually represented by $A$ (without the arrow).
  \item A vector really represents two numbers, an $x$ part and a $y$ part.  If $\theta$ is the angle the vector makes
    with the $x$ axis, you can find each part this way: 
    \begin{align*}
      A_x &= A \cos \theta \\
      A_y &= A \sin \theta \\
    \end{align*}
  \item If you have the two parts, you can use the quadratic formula to find the length of the vector.  \[A = \sqrt{A_x^2 + A_y^2}\]
  \item The only thing that matters about a vector is its size and direction.  You can move a vector around at it is still the same vector.
  \item You can add vectors $\vec{A}$ and $\vec{B}$ graphically by
    \begin{itemize*}
      \item place the head of $\vec{B}$ on the tail of $\vec{A}$.
      \item draw a new vector from the tail of $\vec{A}$ to the head of $\vec{B}$
    \end{itemize*}
  \item If you multiply a vector by a {\em scalar} the length of the vector changes and the direction remains the same.
  \item Multiplying a vector by a negative number reverses its direction.
  \item The {\em unit vectors}, $\hat{x}$ and $\hat{y}$ are vectors of magnitude 1 pointing along the x and y axes, respectively.
  \item Any vector can be written as a combination of unit vectors.  $3 \hat{x} + 4 \hat{y}$, for example, is a vector
    with an x component of 3 and a y component of 4.
\end{itemize} 

\section{Velocity and Acceleration}
\begin{itemize}
  \item In physics, {\em velocity} is different from {\em speed}.  Velocity includes both speed and direction,
    while speed only indicates how fast something is going without including the direction.  Since velocity
    includes both speed and direction, velocity is a vector. 
  \item In physics, if an object is accelerating its velocity is changing.  This happens when:
    \begin{itemize*}
      \item the object is speeding up
      \item the object is slowing down
      \item the object is changing direction
    \end{itemize*}

  \item In physics, a car stopping at a stop sign is accelerating because its velocity is changing.  Of course, in this
    case the accelration is negative.

  \item In physics, a car traveling an a constant speed in a sircle is accelerating because its velocity is changing.  In this
    case the the acceleration is towards the center of the circle.

  \item Since acceleration includes a magnitude (how fast is the velocity changing?) and a direction (how is the
    direction of the velocity changing), acceleration is also a vector.
\end{itemize}

\section{Zero Acceleration}
\begin{itemize}
  \item An item moving in a straight line at a constant speed has a constant velocity and is undergoing non-accelerated
    motion. 
  \item For non-accelerated motion:
    \begin{align*}
      a &\text{ is 0} \\
      v &\text{ is constant} \\
      x &= x_0 + vt \\
    \end{align*}
\end{itemize}

\section{Constant Acceleration in Straight Line}
\begin{itemize}
  \item An object moving in a straight line while its speed changes at a constant rate is undergoing
    motion in a straight line with constant acceleration.
  \item The relevant equations are:
    \begin{align*}
      a & \text{ is a constant} \\
      v &= v_0 + at \\
      x &= x_0 + v_0t + \frac{1}{2} at^2 \\
    \end{align*}
\end{itemize}

\pagebreak

\section{Constant Acceleration in Two Dimensions}
\begin{itemize}
  \item An object moving while its velocity (speed and/or direction) changes at a constant rate is undergoing motion
    with constant acceleration.
  \item The easiest way to analyze motion in two dimensions is to analyze the x and y directions separately using the
    same equations you used for one dimension.
  \item Another way to look at it is with vector equations.
    \begin{align*}
      \vec{a} & \text{ is a constant} \\
      \vec{v} &= \vec{v}_0 + \vec{a}t \\
      \vec{r} &= \vec{r}_0 + \vec{v}_0 t + \dfrac{1}{2} \vec{a}t^2 \\
    \end{align*}
    Each of these equations really represents two equations---one equation for the $x$ direction and one equation for the
    $y$ direction.
\end{itemize}

\end{document}

