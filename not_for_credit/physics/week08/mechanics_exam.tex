
\documentclass[fleqn,addpoints]{exam}

\usepackage{graphicx}
\usepackage{float}
\usepackage{amsmath}
\usepackage{cancel}
\usepackage{polynom}
\usepackage{caption}
\usepackage{mdwlist}

\newcommand{\degree}{\ensuremath{^\circ}} 

\printanswers

\everymath{\displaystyle}

\ifprintanswers 
\usepackage{2in1, lscape} 
\fi

\title{Physics \\ Mechanics Exam}
\date{October 24, 2010}

\begin{document}

\maketitle

\ifprintanswers
\else
\begin{center}
\gradetable[h][pages]
\bonusgradetable[h][pages]
\end{center}

\fi

\ifprintanswers
\else
\section{Notes}

\begin{itemize*}
  \item Feel free to use a calculator or borrow one from the UBB office.
  \item Use your textbook or notes for the required formulas.
  % \item If you get stuck on something, you can ask another student for assistance. 
  \item Unless the question says otherwise, assume there is no air resistance or
    friction.
  \item Some questions are marked ``bonus''.  These questions are a bit harder
    and are extra-credit.
  \item There's no class on Oct 31, so this is due on Nov 7.
\end{itemize*}

\fi
\section{Questions}

\begin{questions}

\subsection{Linear Motion}

\question
A projectile is launched with an initial horizontal velocity of $v_x$ and an initial vertical velocity of $v_y$.

\begin{parts}

\part[2]
What is its vertical velocity the highest point in its path?

\begin{solution}
At the highest point in its path it isn't going up any more so its vertical velocity is zero.
\end{solution}

\part[2]
What is its horizontal velocity the highest point in its path?

\begin{solution}
The horizontal velocity doesn't change throughout the flight so it is still $v_x$.
\end{solution}

\part[2]
What is its acceleration at the highest point in its path?

\begin{solution}
The acceleration also doesn't change throughout the flight so it is still $-9.8 \ m/s^2$.
\end{solution}

\end{parts}

\question
A projectile is launched into the air and lands some distance away.  During its flight:
\begin{parts}
\part[2] Is its velocity ever parallel to its acceleration?

\begin{solution}
The acceleration is always straight down, so the only way the velocity could be parallel to the acceleration is if the
velocity was either straight down or straight up at some point in the flight.  This could only happen if the projectile
was launched straight up.  But the question says the projectile lands some distance away, so it wasn't launched straight
up and the velocity is never parallel to the acceleration.
\end{solution}

\part[2] Is its velocity ever perpendicular to its acceleration?
\begin{solution}
Since the acceleration is always straight down, for the velocity to be perpendicular to the acceleration, the projectile
must be moving horizontally.  This happens at the top of the flight when $v_y$ is zero.
\end{solution}
\end{parts}

\question[3]
When you play darts in an elevator moving up at a constant speed, how should you adjust your aim compared to how you
play when standing on the ground?

\begin{solution}
An elevator moving 
velocity is added to the darts' vertical velocity and everything behaves exactly as it would in a stationary elevator.
\end{solution}

\question[3]
When you play darts in an elevator moving up with a constant acceleration, how should you adjust your aim compared to how you
play when standing on the ground?

\begin{solution}
The constant acceleration of the elevator adds to the natural acceleration of gravity.  If the elevator is accelerating
up, you will need to aim a bit higher, since the total acceleration will be higher.  

\end{solution}

\question
A diver runs off a 6 meter diving board with an initial speed of 2 m/s.
\begin{parts}
\part[5]
How long does it take for the diver to hit the water?

\begin{solution}
The diver needs to fall 6 m.  This takes:
\begin{align*}
  y &= \frac{1}{2} at^2 \\
  6 \ m &= \frac{1}{2} (9.8 \ m/s^2)t^2 \\
  t &\approx 1.11 \ s
\end{align*}

\end{solution}

\part[3]
\begin{solution}
What is her vertical speed when she hits the water?
\begin{align*}
  v &= v_0 + at \\
    &= 0 + (9.8 \ m/s^2)(1.11 \ s) \\
    &\approx 10.84 \ m/s \\
\end{align*}
\end{solution}

\part[2]
What is her horizontal speed when she hits the water?

\begin{solution}
  Her horizontal speed hasn't changed and she is still moving at 2 m/s.
\end{solution}

\end{parts}

\question[5]
A ball is thrown at a $45 \degree$ angle and reaches a height of 30 meters.  What was its initial velocity?

\begin{solution}
The initial velocity in the y direction is the same is if the ball had been dropped from 30 meters.  We can figure this
out by first finding out how long the ball was in flight and the finding out how fast it is going after this much time
has elapsed.
\begin{align*}
  y &= \frac{1}{2} at^2 \\
  30 \ m &= \frac{1}{2} (9.8 \ m/s^2)(t^2) \\
  t & \approx 2.47 \ s \\
\end{align*}
After 2.47 s:
\begin{align*}
  v &= v_0 + at \\
   &= (9.8 \ m/s^2)(2.47 \ s) \\
  &\approx 24.2 \ m/s \\
\end{align*}

This is only the vertical part of the motion.  There are two ways to figure out the total velocity.  One approach is to
notice that since the ball was thrown at a $45 \degree$ angle, its vertical and horizontal velocity's are the same.
Knowing this we can use the Pythagorean theorem to find the total velocity:
\begin{align*}
  v^2 &= v_x^2 + v_y^2 \\
  v & 34.3 \ m/s \\
\end{align*}
The other approach is to note that:
\begin{align*}
  \sin \theta &= \frac{v_y}{v} \\
  v &= \frac{v_y}{\sin \theta} \\
    & 34.3 \ m/s \\
\end{align*}

\end{solution}

\question[5]
A ball is thrown at a $45 \degree$ angle and travels 100 meters.  What was its initial velocity?

\begin{solution}

I probably should have made this problem extra credit, as it's pretty elaborate.  On the other hand, it's exactly like
the pumpkin launching problem from homework 1, so if you remembered that, you could refer back to the homework one
answer key to see how to do it.

Here's the solution, copied from week 2's homework with the word ``pumpkin'' replaced by ``ball''

Since velocity is a vector, the x and y components are given by:
\begin{align*}
  v_{0x} &= v_0 \cos \theta \\
  v_{0y} &= v_0 \sin \theta \\
\end{align*}
where $\mathbf{v_0}$ is the initial velocity and $\theta$ is the launch angle.

The first thing to do is to figure out how long the ball will be in the air.  There are two ways to do this.

One option is to figure out when the y coordinate is zero.  This happens at time 0, of course, before the launch.  It
also happens when the ball hits the ground at the end of its flight.
\begin{align*}
  y = v_{0y} t + \frac{1}{2} at^2 \\
  0 = v_0 \sin \theta t + \frac{1}{2} at^2 \\
  0 = t \left( v_0 \sin \theta + \frac{1}{2} at \right) \\
\end{align*}
The solution at the end of the flight is given by:
\begin{align*}
  v_0 \sin \theta + \frac{1}{2} at &= 0 \\
  % at &= -2 v_0 \sin \theta \\
  t &= - \frac{2 v_0 \sin \theta}{a} \\
\end{align*}
The other way to figure out how long the ball is in the air is to figure out when $v_y$ is zero at the top of the
flight and multiply by 2.  Here's that approach:
\begin{align*}
  v_y &= v_{0y} + at \\
  0 &= v_0 \sin \theta + at \\
  t &= - \frac{v_0 \sin \theta}{a} \\
\end{align*}
Since this is the time to reach the top of the flight, you have to multiply by 2 to get the total flight time, so the
two approaches agree on the flight time.

Now we know the flight time, we need to figure out how far the ball travels in the x direction in this time.  The
velocity in the x direction is constant and we can substitute in the expression for the total flight time.
\begin{align*}
  x &= v_{0x}t \\
   &= v_0 \cos \theta t \\
   &= v_0 \cos \theta \left( - \frac{2 v_0 \sin \theta}{a} \right) \\
   &= - \frac{2 v_0^2 \sin \theta \cos \theta}{a} \\
\end{align*}
For this problem, we have a launch angle of $45 \degree$.  So now we can plug in all the numbers and solve for $v_0$:
\begin{align*}
   100 \ m &= - \frac{2 v_0^2 \sin 45 \degree \cos 45 \degree}{-9.8 \ m/s^2} \\
   v_0 &\approx 31.3 \ m/s
\end{align*}

\end{solution}
\bonusquestion[5]
A mountain goat leaps horizontally across a 3 meter crevasse, landing 2 meters below his starting point.  What is the
minimum initial horizontal velocity required for him to make it across?

\begin{solution}
The first thing to do is find out how long the flight lasts.  Since the goat jumps horizontally, he has time to fall 2
meters before landing.  
\begin{align*}
  y &= \frac{1}{2} at^2 \\
  2 \ m &= \frac{1}{2} (9.8 \ m/s^2)t^2 \\
  t &\approx 0.64 \ s
\end{align*}

He needs to travel 3 meters horizontally in this time and he's not accelerating in the horizontal direction.
\begin{align*}
  x &= x_0 + v_xt \\
  3 \ m &= (v_x)(0.64 \ s) \\
  v_x &\approx 4.7 \ m/s
\end{align*}

\end{solution}

\subsection{Newton's Laws}

%% \question[5]
%% What are the action/reaction forces for a bat hitting a baseball?

\question[5]
If you push a 5 kg block on a flat frictionless surface with a constant force of 10 N, how far does it travel in 
3 seconds?

\begin{solution}
The first thing to do is to find the acceleration:
\begin{align*}
  F &= ma \\
  10 \ N &= (5 \ kg) a \\
  a &= 2 \ m/s^2 \\
\end{align*}

Now we can figure out how far it went:
\begin{align*}
  x &= \frac{1}{2} at^2 \\
    &= \frac{1}{2} (2 \ m/s^2)(3 \ s)^2 \\
    &= 9 \ m \\
\end{align*}

\end{solution}

\question
When you lift a weight with a force of 100 N, it accelerates upward at a rate of $5 \ m/s^2$.

\begin{solution}
I should have put the two parts in the opposite order since you need to find the mass in order to find the weight.  The
net force which is producing the $5 \ m/s^2$ acceleration is the force you are applying minus the weight:
\begin{align*}
  F_{net} &= F_{applied} - mg \\
  ma &= F_{applied} - mg \\
  ma + mg &= F_{applied} \\
  m &= \frac{F_{applied}}{a + g} \\
   &= \frac{100 \ N}{5 \ m/s^2 + 9.8 \ m/s^2} \\
   &\approx 6.76 \ kg \\
\end{align*}
The weight is:
\begin{align*}
  W &= mg \\
    &= (6.76 \ kg)(9.8 \ m/s^2) \\
    &\approx 66.2 \ N \\
\end{align*}
To check to see if this is correct:
\begin{align*}
  F_{net} &= 100 \ N - 66.2 \ N = 33.8 \ N \\
  m &= 6.76 \ kg \\
  \\
  F &= ma \\
  33.8 \ N &= (6.76 \ kg)(5 \ m/s^2)
\end{align*}
so everything checks out OK.

\end{solution}

\begin{parts}
\part[3] What is its weight?

\begin{solution}
  66.2  N
\end{solution}

\part[3] What is its mass?

\begin{solution}
  6.76 \ kg
\end{solution}

\end{parts}

\question There are three blocks sitting next to each other on a frictionless
surface.  The first has a mass of 1 kg, the second has a mass of 3 kg, and the
third has a mass of 2 kg.  If you push on the first block with a force of 6 N,
what is the force between:

\begin{solution}
The strategy for this one is to find the acceleration for the three blocks together and then use $F=ma$ on the blocks
mentioned for each part to find the forces between the blocks.

The total acceleration is:
\begin{align*}
  F &= ma \\
  6 \ N &= (6 \ kg) a \\
  a &= 1 \ m/s^2 \\
\end{align*}
\end{solution}

\begin{parts}
\part[3]
block one and block two?

\begin{solution}
The force between blocks one and two needs to accelerate blocks 2 and 3 which have a total mass of 5 kg.  So the force
required is:
\begin{align*}
  F &= ma \\
    &= (5 \ kg)(1 \ m/s^2) \\
    &= 5 \ N \\
\end{align*}
\end{solution}

\part[3]
block two and block three?

\begin{solution}
The force between blocks two and three only needs to accelerate block 3 which has a mass of 2 kg.  So the force
required is:
\begin{align*}
  F &= ma \\
    &= (2 \ kg)(1 \ m/s^2) \\
    &= 2 \ N \\
\end{align*}
\end{solution}

\end{parts}

\question[5]
A rocket has an acceleration of $30 \ m/s^2$ at launch.  What force does the floor of the rocket exert on an 80 kg
astronaut during the launch?

\begin{solution}
To accelerate an 80 kg astronaut at $30 \ m/s^2$, you need a net force of:
\[
  F_{net} = ma = (80 \ kg)(30 \ m/s^2) = 2,400 \ N 
\]
The force exerted by the floor also needs to counteract the astronaut's weight, so the force from the floor is:
\begin{align*}
  F_{net}   &= F_{floor} - W \\
  F_{floor} &= F_{net} + W \\
           &= 2,400 \ N + (80 \ kg)(9.8 \ m/s^2) \\
           &= 3,184 \ N \\
\end{align*}

\end{solution}

\bonusquestion
A pitcher accelerates a 0.15 \ kg baseball from 0 to 45 m/s in a distance of 2 m.

\begin{parts}
\bonuspart[5]
Use Newton's laws to find the average force required.

\begin{solution}
To use Newton's laws, we need to find the acceleration.  This requires using two of the motion equations, with the
initial velocities and positions zero.
\begin{align*}
  v &= at \\
  t &= \frac{v}{a} \\
  \\
  x &= \frac{1}{2} at^2 \\
    &= \frac{1}{2} a \left( \frac{v}{a} \right)^2 \\
  x  &= \frac{v^2}{2a} \\
  a &= \frac{v^2}{2x} \\
  \\
  a &= \frac{(45 \ m/s)^2}{2 (2 \ m)} \\
    &= 506.25 \ m/s^2 \\
\end{align*}
Now we can find the force:
\begin{align*}
  F &= ma \\
    &= (0.15 \ kg)(506.25 \ m/s^2) \\
    &\approx 76 \ N
\end{align*}
\end{solution}

\bonuspart[5]
Use work and the change in kinetic energy to find the average force required.
\begin{solution}
Doing it this way is much easier.  The initial kinetic energy is zero since the ball isn't moving at first.  The final
kinetic energy is:
\[
  K = \frac{1}{2} mv^2 = \frac{1}{2} (0.15 \ kg)(45 \ m/s)^2 \approx 151.9 \ J
\]

This is the work done and the distance was 2 m.  So the force was:
\begin{align*}
  W &= Fd \\
  151.9 \ J &= F (2 \ m) \\
  F &\approx 76 \ N
\end{align*}
 
\end{solution}
\end{parts}

\subsection{Circular Motion}

\question[5]
What is the centripetal force required to keep a 1,000 kg car on the road while traveling around a curve with
a radius of 60 meters at a speed of 15 m/s?

\begin{solution}
First find the centripetal acceleration required:
\[
  a = \frac{v^2}{r} = \frac{(15 \ m/s)^2}{60 \ m} = 3.75 \ m/s^2
\]

Then find the force:
\[
  F = ma = (1,000 \ kg)(3.75 \ m/s^2) = 3,750 \ N
\]

\end{solution}

\question
A passenger of mass m travels on a ferris wheel at a constant speed v.

\begin{solution}
  The forces involved are the passenger's weight, the force exerted by the seat and the centripetal force required to
  keep the passenger moving in a circle.  The net force must be the centripetal force.
\begin{align*}
  W &= mg \\
  F_{cp} &= m a_{cp} = \frac{mv^2}{r}
\end{align*}
\end{solution}

\begin{parts}

\part[3]
What is the force exerted on the rider by his seat at the bottom of the ride?

\begin{solution}
  At the bottom of the ride, the centripetal force points up while the weight points down.
\begin{align*}
  F_{cp}   &= F_{seat} - W \\
  F_{seat} &= F_{cp} + W \\
          &= \frac{mv^2}{r} + mg \\
\end{align*}
\end{solution}

\part[3]
What is the force exerted on the rider by his seat at the top of the ride?

\begin{solution}
  At the top of the ride, both the centripital force and the weight are in the same direction, with the force from the
  seat pointing the other way:
\begin{align*}
  F_{cp}   &= W - F_{seat} \\
  F_{seat} &= W - F_{cp} \\
          &= mg - \frac{mv^2}{r} \\
\end{align*}
\end{solution}

\end{parts}

\subsection{Work and Energy}
\question[2]
Can you do work on a stationary object?

\begin{solution}
  No.  Work is $\mathbf{W} = \mathbf{F}d$ so if $d$ is 0 so is $W$.
\end{solution}

\question[3] A 75 N force is used to push a 20 kg block 5 meters on a horizontal
surface.  How much work was done by the force?

\begin{solution}

\[
  W = Fd = (75 \ N)(5 \ m) = 375 \ J
\]

The mass doesn't matter.

\end{solution}

%% \question[2]
%% What is the kinetic energy of a 0.15 kg baseball traveling at 45 m/s?

\question[5] A 1,300 kg car coasts from 20 m/s to 10 m/s while traveling 30 m.
How much work was done by friction to slow the car?

\begin{solution}
There are two ways to do this one.

The easy way is to find the change in kinetic energy since this will also be the work done by friction.
\begin{align*}
  \Delta K &= \frac{1}{2} mv^2 - \frac{1}{2} m v_0^2 \\
           &= \frac{1}{2} m (v^2 - v_0^2) \\
           &= \left( \frac{1,300 \ kg}{2} \right) ((20 / m/s)^2 - (10 \ m/s)^2) \\
           &= 195,000 \ J
\end{align*}

The hard way is to find the average friction force and then multiply by the distance traveled.

To do it this way, we first need to find out how long the process took:
\begin{align*}
  v &= v_0 + at \\
  a &= \frac{v - v_0}{t} \\
  \\
  x &= v_0t + \frac{1}{2} at^2 \\
    &= v_0t + \frac{1}{2} \left( \frac{v - v_0}{t} \right) t^2 \\
    &= \frac{t(v + v_0)}{2} \\
  t &= \frac{2x}{v + v_0} \\
  \\
  t &= \frac{2 (30 \ m)}{20 \ m/s + 10 \ m/s} \\
  t &= 2 \ s
\end{align*}

Now we know how long it took, we can find the acceleration:
\[
  a = \frac{v - v_0}{t} = \frac{10 m/s}{2 \ m/s} = 5 \ m/s^2
\]

This gives us the force:
\[
  F = ma = (1,300 \ kg)(5 \ m/s^2) = 6,500 \ N
\]

Which gives us the work:
\[
  W = Fd = (6,500 \ N)(30 \ m) = 195,000 \ J 
\]

\end{solution}

\question[5]
Lifting a box 5 meters requires 200 J of work.  What is the box's mass?

\begin{solution}

The work required is the same as the change in potential energy.
\begin{align*}
  U &= mgh \\
  m &= \frac{U}{gh} \\
    &= \frac{200 \ J}{(9.8 \ m/s^2)(5 \ m)} \\
    &\approx 4.1 \ kg \\
\end{align*}

\end{solution}

\question[5]
A child slides down a frictionless water slide starting from a height of 10 m.  How fast is she going when she enters the pool?

\begin{solution}
All of the original potential energy is converted to kinetic energy.
\begin{align*}
  mgh &= \frac{1}{2} mv^2 \\
  v^2 &= 2gh \\
  v   &= \sqrt{2gh} \\
      &= \sqrt{2 (9.8 \ m/s^2)(10 \ m)} \\
      &= 14 \ m/s \\
\end{align*}

\end{solution}

\subsection{Momentum, Impulse, and Collisions}

\question[5]
A cart of mass $m$ moving with velocity $v$ collides and sticks to a stationary cart with mass $2m$.  What is the final
velocity of the pair?

\begin{solution}
Since only one cart is moving initially, the initial momentum is: $p_{initial} = mv$.

The final momentum is the same:
\begin{align*}
  3mv_{final} &= mv \\
  3v_{final} &= v \\
  v_{final} &= \frac{v}{3} \\   
\end{align*}

\end{solution}

\question[5] When a batter hits a fastball, the bat is in contact with the ball for
0.002 seconds.  If the pitch is a 45 m/s fastball and leaves the bat at 60 m/s,
how much force was exerted by the bat?

\begin{solution}
I inadvertently neglected to include the baseball's mass in the problem statement.  You can get it from problem 6, if
you noticed that that problem also included a baseball.

If you define the final velocity as being in the positive direction, the initial and final momentums are:
\begin{align*}
  p_{initial} &= m (-45 \ m/s) \\
  p_{final}   &= m (60 \ m/s) \\
\end{align*}
The total change in momentum is:
\[
  \Delta p = (60 + 45 )(m) \ kg \cdot m/s = (105)(m) \ kg \cdot m/s 
\]

Impulse is the change in the momentum and also the force multiplied by the time.
\begin{align*}
  I &= \Delta p = 105m \ kg \cdot m/s \\
  \\
  I &= Ft \\
  F &= \frac{I}{t} \\
    &= \frac{105 m \ kg \cdot m/s}{0.002 \ s} \\
    &= 52,500 m \ N\\
\end{align*}

This is as far as you can go without the mass.  If you got the mass from problem 6, you could get the final answer of:
\[
  F = 52,000 \cdot 0.15 \ N = 7,875 \ N
\]

\end{solution}

\bonusquestion[5] A moving cart with mass $m$ and velocity $v$ collides and
bounces off an identical cart with moving at velocity $-2v$ in a perfectly
elastic collision.  What is the velocity of each cart after the collision?

\begin{solution}
There are several ways to solve this one.  The first approach uses Feynman's idea of using different points of view to
observe the same event and doesn't require much math, but does require visualizing a few different collisions.

All of the collisions involve identical objects.

\begin{enumerate}
\item In a perfectly elastic collision between two objects with with velocities $v$ and $-v$, the two objects bounce
  off each other and move in the opposite direction with their original velocities.  Because the situation is symmetric,
  both objects will be moving at the same speed after the collision.  They just change direction in the collision.

\item If you observe the preceding collision from a coordinate system moving at velocity $v$, the first object appears to be
  initially stationary and the second object appears to be initially moving at velocity $2v$.  From this point of view,
  after the collision the second object appears to be stationary and the first object appears to be moving at a velocity
  $2v$.  So you can conclude that when a moving object has a perfectly elastic collision with a stationary object, the
  stationary object gets all of the moving object's velocity and the moving object stops.

\item If you observe the collision described in the problem from a coordinate system moving at velocity $v$, the first
  cart appears to be stationary and the second cart appears to be moving at velocity $3v$.  Applying what we just
  learned about a collision between a stationary object and a moving object, from this point of view, after the
  collision the first object appears to be moving at velocity $3v$ and the second object appears to be stationary.  From
  the point of view of a stationary coordinate system, the first object is moving at velocity $2v$ and the second object
  is moving at velocity $-v$, which is the answer we are looking for.
\end{enumerate}

The other way to solve this problem requires more algebra\ldots

In an elastic collision, both momentum and kinetic energy are conserved.  The initial values for each are:
\begin{align*}
  K &= \frac{1}{2} mv^2 + \frac{1}{2} m (2v)^2 \\
    &= \frac{5}{2} mv^2 \\
  \\
  p &= mv - 2mv \\
    &= -mv \\
\end{align*}
If we say that after the collision, the first cart is moving at velocity $v_1$ and the second cart is moving at velocity
$v_2$, the values for kinetic energy and momentum after the collision are:
\begin{align*}
  K &= \frac{1}{2} mv_1^2 + \frac{1}{2} mv_2^2 \\
  p &= mv_1 + mv_2 \\
\end{align*}
$K$ and $p$ don't change, so:
\begin{align*}
  \frac{1}{2} mv_1^2 + \frac{1}{2} mv_2^2 &= \frac{5}{2} mv^2 \\
  v_1^2 + v_2^2 &= 5v^2 \\
  \\
  mv_1 + mv_2 &= -mv \\
  v_1 + v_2 &= v \\
  v_2 &= v - v_1 \\
  \\
  v_1^2 + (v - v_1)^2 &= 5v^2 \\
  v_1^2 + v^2 - 2vv_1 + v_1^2 &= 5v^2 \\
  %% 2v_1^2 - 2vv_1 + v^2 &= 5v^2 \\
  %% 2v_1^2 - 2vv_1 - 4v^2 &= 0 \\
  v_1^2 - vv_1 - 2v^2 &= 0 \\
  (v_1 - 2v)(v_1 + v) &= 0 \\
  v_1 = 2v &\text{ or } v_1 = -v \\
\end{align*}
The interesting value is $v_1 = 2v$, since the other value was the velocity before the collision.  Substituting back in
the momentum equation gives $v_2 = -v$.  The two carts swapped velocities in the collision.

\end{solution}

\subsection{Gravity}

\question[3]
What is the gravitational force between earth and the sun?
\begin{align*}
  m_{earth} &= 5.97 \cdot 10^{24} \ kg \\
  m_{sun} &= 2.00 \cdot 10^{30} \ kg \\
  G &= 6.67 \cdot 10^{-11} \ N \cdot m^2/kg^2 \\
  r &= 1.50 \cdot 10^{11} \ m \\
\end{align*}

\begin{solution}
\begin{align*}
  F &= \frac{Gm_1m_2}{r^2} \\
    &= \frac{(6.67 \cdot 10^{-11} \ N \cdot m^2/kg^2)(5.97 \cdot 10^{24} \ kg )(2.00 \cdot 10^{30} \ kg)}
            {(1.50 \cdot 10^{11} \ m)^2} \\
    &= 3.54 \cdot 10^{22} \ N \\
\end{align*}
\end{solution}
\subsection{Rotational Motion}

\question[2] The minute hand and hour hand of a clock have the same mass.  The
minute hand is long and thin and the hour hand is short and fat.  Which one has
the larger moment of inertia?

\begin{solution}
Since the rotational inertia increases the farther the mass is from the radius, the minute hand has a larger moment of
inertia because it is longer. 
\end{solution}

\question The angular speed of a wheel increases with constant acceleration
from 10 rad/s to 20 rad/s in 2 revolutions.

\begin{parts}

\part[3] How long does it take?
\begin{solution}

First find $\alpha$ in terms of $\omega$ and $t$:
\begin{align*}
  \omega &= \omega_0 + \alpha t \\
  \alpha &= \frac{\omega - \omega_0}{t} \\
\end{align*}

Then plug $\alpha$ in to the angle equation to find $t$:
\begin{align*}
  \theta &= \omega_0 t + \frac{1}{2} \alpha t^2 \\
         &= \omega_0 t + \frac{1}{2} \left( \frac{\omega - \omega_0}{t} \right) t^2 \\
         &= \left( \omega_0 + \frac{\omega - \omega_0}{2} \right) t \\
         &= \left( \frac{\omega + \omega_0}{2} \right) t \\
       t &= \frac{2 \theta}{\omega + \omega_0} \\
         &= \frac{2 \cdot 4 \cdot \pi \ rad}{20 \ rad/s + 10 \ rad/s} \\
         &\approx 0.838 \ s
\end{align*}
\end{solution}

\part[3] What is its angular acceleration?
\begin{solution}
\[
  \alpha = \frac{\omega - \omega_0}{t} = \frac{20 \ rad/s - 10 \ rad/s}{0.838 \ s} = 11.9 \ rad/s^2
\]
\end{solution}

\end{parts}

\question[3] A torque of $15 \ N \cdot m$ is required to tighten a nut.  If the
wrench is 30 cm long, what is the minimum force needed?

\begin{solution}
\begin{align*}
  \tau &= Fr \\
   F &= \frac{\tau}{r} \\
     &= \frac{15 \ N \cdot m}{0.30 \ m} \\
     &= 50 \ N \\
\end{align*}
\end{solution}

\question A torque of $100 \ N \cdot m$ is applied to a bicycle wheel of radius
35 cm and mass 1 kg.
\begin{parts}

\part[3] If you treat the wheel as a hoop ($I = mr^2$), what is the angular acceleration?
\begin{solution}
\begin{align*}
   \alpha &= \frac{\tau}{I} \\
          &= \frac{100 \ N \cdot m}{ (1 \ kg)(0.35 \ m)^2} \\
          &\approx 816 \ rad/s \\
\end{align*}
\end{solution}

\part[3] If you treat the wheel as a uniform disk ($I = \frac{1}{2} mr^2$), what is the angular acceleration?
\begin{solution}
If you divide the rotational inertia by 2, you get twice as much acceleration, or: $\alpha = 1,633 \ rad/s$
\end{solution}

\end{parts}

\question
A 15 g record with a radius of 15 cm rotates with an angular speed of 33 1/3 rpm.  
\begin{parts}

\part[3] Find the record's angular momentum, assuming the record is a uniform disk.
\begin{solution}
For a disk:
\[
  I = \frac{1}{2} mr^2 = \frac{1}{2} (0.015 \ kg) (0.15 \ m)^2 = 1.69 \cdot 10^{-4} \ kg \cdot m^2
\]

The angular velocity, in rad/s is:
\[
  \omega = \frac{33.33 \ rev}{1 \ minute} \cdot \frac{1 \ minute}{60 \ s} \cdot \frac{2 \pi \ rad}{1 \ rev} \approx 3.49 \ rad/s
\] 

So the record's angular momentum is:
\begin{align*}
  L &= \omega I \\
    &= (3.49 \ rad/s)(1.69 \cdot 10^{-4} \ kg \cdot m^2) \\
    &\approx 5.9 \cdot 10^{-4} \ kg \cdot m^2/s \\
\end{align*}
\end{solution}

\part[3] Find the angular momentum of a 2 g fly sitting on the rim of the record.
\begin{solution}
The fly's rotational inertia is:
\[
  I = mr^2 = (0.002 \ kg) (0.15 \ m)^2 = 4.5 \cdot 10^{-5} \ kg \cdot m^2
\]

The fly's angular momentum is:
\begin{align*}
  L &= \omega I \\
    &= (3.49 \ rad/s)(4.5 \cdot 10^{-5} \ kg \cdot m^2) \\
    &\approx 1.57 \cdot 10^{-4} \ kg \cdot m^2/s \\ 
\end{align*}
\end{solution}
\end{parts}

\end{questions}

\end{document}

