
\documentclass[fleqn,addpoints]{exam}

\usepackage{graphicx}
\usepackage{float}
\usepackage{amsmath}
\usepackage{cancel}
\usepackage{polynom}
\usepackage{caption}
\usepackage{mdwlist}

\newcommand{\degree}{\ensuremath{^\circ}} 

% \printanswers

\ifprintanswers 
\usepackage{2in1, lscape} 
\fi

\everymath{\displaystyle}


\title{Physics \\ Mechanics Review}
\date{October 24, 2010}

\begin{document}

\maketitle


\begin{questions}

\section{Linear Motion}

\question
A ball is dropped from a height of 10 meters. 
\begin{parts}

\part How fast is the ball going after 1 second?

\part When does it hit the ground?
\end{parts}

\question
The distance from the pitcher to the catcher is 18 meters.  How far does a 45 m/s fastball thrown horizontally fall
before reaching the catcher?

\section{Newton's Laws}

\question
What are the action/reaction forces for the basket in a hot air balloon?

\question
If you push a 2 kg block on a flat frictionless surface with a constant force of 5 N, how fast is it going after 4 seconds?

\question 

There are two blocks sitting next to each other on a frictionless surface.  The
first has a mass of 2 kg and the second has a mass of 5 kg.  If you push on the
first block with a force of 10 N, what is the force between the two blocks?

\question What force is required to launch a 10 kg model rocket with an
acceleration of $15 \ m/s^2$?

\section{Circular Motion}

\question
A race track has four sections:
\begin{itemize*}
\item two straight sections, each 5/8 mile long.
\item two circular curved sections, each 5/8 mile long.
\end{itemize*}

The cars travel 190 mph around the curves.  What is the centripital force
required to keep a 1,000 lb car on the track while traveling around the curve?

In standard units:
\begin{align*}
  5/8 \ mile &\approx 1,000 \ meters \\
  190 \ mph &\approx 85 \ m/s \\
  1,000 \ lb &\approx 450 \ kg \\
\end{align*}

\section{Gravity}

\question
Why don't heavier objects fall more quickly than lighter ones?

\question On the planet Zircon, a 5 kg rock weighs 10 N.  What is the
acceleration due to gravity on Zircon?

\section{Work and Energy}

\question
\label{work:1}
A 10 N force is used to push a 5 kg block 7 meters on a horizontal surface.
How much work was done by the force?

\question If the force in question \ref{work:1} is opposed by a friction force
of 2 N, how much work is done by the friction force?

\question
What is the kinetic energy of a 1,000 kg car traveling at 30 m/s?

\question
A ball is launched upward on the planet Zircon with a velocity of 7 m/s and rises to a height of 50 meters.  What is the acceleration due to gravity on Zircon?

\section{Momentum, Impulse, and Collisions}

\question A cart of mass $m$ moving with velocity $v$ collides and sticks to
a stationary cart and the pair of carts continues moving at speed $\frac{v}{4}$.
What is the mass of the stationary cart relative to the moving cart?

\question In a golf swing, the club is in contact with the ball for 0.001
seconds.  The 45 g ball acquires a speed of 70 m/s.  How much force is exerted
by the club on the ball?

\section{Rotational Motion}

\question
Two metal spheres have identical masses but one is hollow and one is solid.  How can you determine which one is solid?

\question The angular speed of a wheel increases with constant acceleration
from 10 rad/s to 20 rad/s in 2 seconds.
\begin{parts}
\part What is its angular acceleration?
\part What angle does it travel through in this time?
\end{parts}

\question
A wheel decelerates from 6 rad/s to 0 rad/s in 15 revolutions.
\begin{parts}
\part What was its average angular acceleration?
\part How long did it take?
\end{parts}

\question A force of 5 N is applied to the end of a 20 cm wrench.  What is the torque on the wrench?

\question A CD accelerates uniformly from 0 to 450 rev/min in 3.0
revolutions.  If a CD has a radius of 6 cm and a mass of 20 g, what is the
torque applied?

\end{questions}

\end{document}

