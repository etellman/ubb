
\documentclass{exam}

\usepackage{graphicx}
\usepackage[fleqn]{amsmath}
\usepackage{cancel}
\usepackage{polynom}
\usepackage{float}
\usepackage{mdwlist}
\usepackage{booktabs}
\usepackage{cancel}
\usepackage{polynom}
\usepackage{caption}

\newcommand{\degree}{\ensuremath{^\circ}} 

\everymath{\displaystyle}

% \oddsidemargin .5in
% \topmargin -1in
\textwidth 6.5 in

% \printanswers

\ifprintanswers 
\usepackage{2in1, lscape} 
\fi

% \begin{tabular}{cc}
% \toprule
% disk with axis of rotation at center & $I = \dfrac{1}{2} mr^2$ \\
% \midrule
% disk with axis of rotation at edge & $I = \dfrac{3}{2} mr^2$. \\
% \bottomrule
% \end{tabular}
 

\title{Physics \\ Mechanics Summary}
\date{October 24, 2011}
% \author{Ed Tellman}

\begin{document}

\maketitle

% \tableofcontents
 
\section{Overview}
This is a short summary of what we've covered so far.  I did my best to get all the equations on a
single page, but didn't quite make it.

\section{Linear Motion}

When an object moves with constant acceleration $a$, with starting position $x_0$ and starting velocity $v_0$, its position
and velocity at a later time can be found with two equations:

\begin{align*}
  v &= v_0 + at \\
  x &= x_0 + v_0t + \frac{1}{2}at^2 \\
\end{align*}

\section{Newton's Laws}

Newton's three laws are:
\begin{enumerate*}

\item If something is stationary or moving in a straight line at a constant speed, it will continue doing so unless a
  force acts on it.  This is actually just the second law with $\mathbf{F} = 0$.

\item The force, acceleration, and mass of an object are related by: $\mathbf{F} = m \mathbf{a}$.

\item Every action has an equal an opposite reaction.  If object one pushes on object two with force $\mathbf{F}$, object
  two pushes on object one with force $-\mathbf{F}$.

\end{enumerate*}

\section{Gravity}
Two objects wih masses $m_1$ and $m_2$ separated by a distance $r$ exert a force on each other:

\[
  F_{gravity} = \frac{Gm_1m_2}{r^2}
\]

$G$ is the universal graviatationl constant and has a value of: $G = 6.67 \cdot 10^{-11} \ N \cdot m^2/kg^2$.

\section{Work and Power}

\subsection{Work}
Work is a force applied over a distance: 
\[
\mathbf{W} = \mathbf{F}d
\]

Some rules apply:
\begin{itemize}
\item Only force in the line of motion counts.  Work is positive if it is in the direction of motion and negative
  if it opposes the motion.  For an object in free fall with air resistance, for example, gravity does positive work
  while the air resistance does negative work.

% \item If the work is only partially in the direction of motion and the force makes an angle $\theta$ with the line of
%   motion, the formula for work is:  $W = Fd \sin \theta$.  $d \sin \theta$ is the component of the force in the line
%   of motion.

\item When you do work, you change the energy of a system.  So another way to look at work is: $W = \Delta E$.  Some
  problems are much easier when you use this formula instead of $W = \mathbf{F}d$.
\end{itemize}

\subsection{Power}
Power is work done over time:
\[
  P = \frac{W}{t}
\]

Another way of looking at it is force times velocity:
\[
  P = \frac{W}{t} = \frac{Fd}{t} = F \frac{d}{t} = Fv
\]

\subsection{Energy}

Energy comes in many different forms.  The ones we talked about are:
\begin{description}

\item[Kinetic Energy] $K = \dfrac{1}{2} mv^2$ is the energy an object has because it is moving.

\item[Gravitational Potential Energy] $U = mgh$ is the energy an object has because of its position in a gravitational
  field.

\item[Spring Energy] $E = \dfrac{1}{2} kx^2$ is the energy of a compressed or stretched spring, where $k$ is the spring
constant, which is different for every spring, and $x$ is the distance the spring is compressed or stretched from its
neutral position.

\item[Heat Energy] If a moving object has friction or air resistance opposing its motion, some of its energy is
  converted to heat.  Heat is actually just the kinetic energy of the molecules in an object.

\end{description}

The total energy of a closed system remains constant, although it may change from one form to another.  

Since the universe is an example of a closed system, the energy in the universe is constant.  For the universe, you also
have to know that $e = mc^2$ and mass is another form of energy (we'll talk about this later). 

\section{Momentum, Impulse, and Collisions}

\subsection{Momentum}
An object's momentum depends on its mass and velocity: 
\[
\mathbf{p} = m \mathbf{v}
\]

The total momentum of a closed system remains constant.  You can see why this is by thinking about Newton's third law.
If object one pushes on object two and changes its momentum, object two pushes back on object one, giving it an equal
and opposite change in momentum.

Since the universe is an example of a closed system, the momentum in the universe is constant.

\subsection{Impulse}

An {\em impulse} is a force applied over a time: 
\[
  \mathbf{I} = \mathbf{F}t
\]

An impulse results in a change in an object's momentum, so impulse is also: 
\[
  \mathbf{I} = \Delta \mathbf{p}
\]
A large impulse results in a large change in momentum.

\subsection{Collisions}
Momentum is conserved in all collisions.  Kinetic energy is also conserved in perfectly elastic collisions.

\section{Rotational Motion}
All the equations for non-rotating motion have corresponding equations for rotating motion.  You just have to replace
the symbols with the rotational version.  Here are the translations:

\vspace{.2 in}

\begin{tabular}{lcc}
\toprule
Description & Linear & Rotational \\
\midrule
mass/moment of intertia  & $m$          & $I$   \\
force/torque & $\mathbf{F}$ & $\mathbf{\tau}$   \\
position     & $x$          & $\theta$          \\
velocity     & $\mathbf{v}$ & $\mathbf{\omega}$ \\
acceleration & $\mathbf{a}$ & $\mathbf{\alpha}$ \\
momentum     & $\mathbf{p}$ & $\mathbf{L}$ \\
\bottomrule
\end{tabular}

\vspace{.2 in}

The {\em moment of intertia} or {\em rotational inertia} depends on both the mass and shape of the object.  The values
for a few common shapes are:

\vspace{.2 in}

\begin{tabular}{ll}
\toprule
Shape & $I$ \\
\midrule
point mass  & $I = mr^2$ \\
hoop with all mass in rim   & $I = mr^2$ \\
disk or cylinder   & $I = \frac{1}{2}mr^2$ \\
stick with axis at end   & $I = \frac{1}{3}mr^2$ \\
\bottomrule
\end{tabular}

\vspace{.2 in}

For rotating motion:
\begin{itemize*}
\item the position ($\theta$) is usually measured in radians 
\item the velocity ($\omega$) is usually measured in $rad/s$
\item the acceleration ($\alpha$) is usually measured in $rad/s^2$
\end{itemize*}

You can go back and forth between the linear and rotational values:
\begin{align*}
  \tau &= Fr \text{ (for a tangential force)} \\
  v &= \omega r \\
  a &= \alpha r \\
  L &= p r \text{ (for a tangential momentum)} \\
\end{align*}

You can replace all the variables in the linear equations to get the rotational versions:

\vspace{.2 in}

\begin{tabular}{ll}
\toprule
Linear & Rotational \\
\midrule
$ \mathbf{F} = m \mathbf{a}$         & $\mathbf{\tau} = I \mathbf{\alpha}$               \\
$ v = v_0 + at$                      & $\omega = \omega_0 + \alpha t$  \\
$ x = x_0 + v_0t + \frac{1}{2}at^2$  & $\theta = \theta_0 + \omega_0 t + \frac{1}{2} \alpha t^2$ \\
$ K = \frac{1}{2} mv^2$              & $K = \frac{1}{2} I \omega^2$ \\
$ \mathbf{p} = m \mathbf{v}$         & $\mathbf{L} = I \mathbf{\omega}$ \\
\bottomrule
\end{tabular}


\end{document}

