
\documentclass{exam}

\usepackage{graphicx}
\usepackage{siunitx}
\usepackage[fleqn]{amsmath}
\usepackage{cancel}
\usepackage{polynom}
\usepackage{float}
\usepackage{mdwlist}
\usepackage{booktabs}
\usepackage{cancel}
\usepackage{polynom}
\usepackage{caption}

\newcommand{\degree}{\ensuremath{^\circ}} 
\everymath{\displaystyle}

% \oddsidemargin .5in
% \topmargin -1in
\textwidth 6.5 in

\printanswers

\ifprintanswers 
\usepackage{2in1, lscape} 
\fi

\title{Physics \\ Homework Seven \\ Electrostatics}
\date{November 7, 2011}

\begin{document}

\maketitle

\ifprintanswers
\else
\section{Notes}

This information is in the text, but I had to poke around a bit to find it.  Here it is to save you the trouble:

\begin{itemize*}
\item When rubbing a glass rod with silk, the rod becomes positively charged.
\item When rubbing a plastic or rubber rod with fur, the rod becomes negatively charged.
\end{itemize*}

\fi

\section{Homework}

\begin{itemize*}
  \item Read Chapter 12
  \item Chapter 12
    \begin{itemize*}
      \item questions 4, 11-12, 20, 22, 23, 25-28, 31
      \item exercises 3, 6, 8, 10-11, 13, 15-16
      \item synthesis problems: 1-3, 5
    \end{itemize*}
\end{itemize*}

\section{Extra Credit}
Three charges, $q_1$, $q_2$, and $q_3$ are in a line, with $q_1$ and $q_3$ $32 \ \centi \meter$ apart.  The magnitudes of the charges
are:
\begin{itemize*}
\item $q_1 = +12 \cdot 10^{-6} \ C$
\item $q_2 = -24 \cdot 10^{-6} \ C$
\item $q_3 = +36 \cdot 10^{-6} \ C$
\end{itemize*}

Find the position between $q_1$ and $q_3$ where $q_2$ experiences a net force of zero.

\begin{solution}
I accidentally made this one require the quadratic forumula.  I meant to have the equation be factorable but messed it up somehow.

The two charges on the ends are pulling in oposite directions.  Onlye the relative magnitude of the charges matters, so
I'll just use $q$ for the charge.  All the $q$s cancel out.  $d$ is the distance between $q_1$ and $q_2$.
\begin{align*}
  \frac{k q \cdot 2q}{d^2} &=   \frac{k 2q \cdot 3q}{(32 -d)^2} \\
  (32 - d)^2 &= 3d^2 \\
  d^2 + 32d - 512 &= 0 \\
\end{align*}
Using the quadratic forumla:
\[
 d = \frac{-32 \pm \sqrt{32^2 - (4)(-512)}}{2} = -16 \pm 16 \sqrt{3} \\
\]
One of the solutions is negative, so the solution we're looking for is: 
\[
  -16 + 16 \sqrt{3} = 11.71 \ \centi\meter
\]
Check:
\begin{align*}
  \frac{(12 \micro\coulomb) (24 \micro\coulomb)}{(.1171 \ \meter)^2} &\approx 2.1 \cdot 10^{-8} \ \newton \\
  \frac{(36 \ \micro\coulomb) (24 \ \micro\coulomb)}{(.32 - .1171 \ \meter)^2} &\approx 2.1 \cdot 10^{-8} \ \newton
\end{align*}

\end{solution}

\ifprintanswers
\pagebreak
\section{Questions}

\begin{description}

\item[Q4]
\begin{description}

\item[a]
The glass rod is positively charged and, because charge is conserved, the cloth is negatively charged.  So the ball that
touches the rod becomes positively charged and the ball that touches the cloth becomes negatively charged.

\item[b]
Since the balls have opposite charges, they will attract each other.

\end{description}

\item[Q11]

When the negatively charged rod is close to the ball, the electrons are forced to the side of the ball away from the
rod.  When you touch the ball, some of these electrons leave the ball and end up in you.  So in the end the ball has
fewer electrons than it started with and ends up positively charged.

\item[Q25]
Since the two charges are attracted to each other, moving them closer together reduces the potential energy of the
system.  This is similar to moving a heavy object closer to the earth, which reduces the potential energy of the object.

A little more mathematically, the change in potential energy is:
\[
  \Delta U = - W
\]

where $W$ is the work done by the force.  In this case we are moving the object in the direction of the force, so we are
doing positive work and the change in potential energy is negative.

\item[Q26]
This is the opposite of the situation in Q25, since the objects are repelled from each other, so the potential energy of
the system increases.

\item[Q27]
The electric field lines point from positive to negative charges.  So this is just like Q26 where the charge is moving
towards a negative charge, and its potential energy increases.

\item[Q28]
The electric potential decreases in the direction of the field lines, so it decreases in this case.

You can compare this with gravity as well.  The force of gravity is directed down, and as you move towards the ground,
your gravitational potential decreases.

\item[Q31]
A positive charge will move towards lower potential so a negative charge moves towards higher potential.  
\end{description}

\section{Exercises}

\begin{description}
\item[E3]
When the balls touch, an equal amount of charge ends up on each ball.  The original charge was 
$12 \ \micro\coulomb - 4 \ \micro\coulomb = 8 \ \micro\coulomb$, so each ball ends up with $4 \micro\coulomb$.

\item[E6]
\begin{align*}
   F &= \frac{kq_1q_2}{r^2} \\
     &= \frac{(9 \cdot 10^9 \ N \cdot m^2/C^2)(4 \cdot 10^{-6} \ C)^2}{(.1 \ m)^2} \\
     &= 14.4 \ N \\
\end{align*}

\item[E8]
\begin{align*}
   F &= \frac{kq_1q_2}{r^2} \\
     &= \frac{(9 \cdot 10^9 \ N \cdot m^2/C^2)(1.6 \cdot 10^{-19} \ C)^2}{(5 \cdot 10^{-11} \ m)^2} \\
     &= 9.2 \cdot 10^{-8} \ N \\
\end{align*}

\item[E10]
\begin{align*}
  E &= \frac{F}{q} \\
    &= \frac{12 \ N}{4 \cdot 10^{-6} \ C} \\
    &= 3 \cdot 10^{6} \ N/C \\
\end{align*}

The field is down, in the same direction as the force.

\item[E11]
The net force is $4 \ \newton$ East.

The field is:
\begin{align*}
  E &= \frac{F}{q} \\
    &= \frac{4 \ \newton}{1.5 \cdot 10^{-6} \ \coulomb} \\
    &= 2.7 \cdot 10^{6} \ \newton \per \coulomb \\
\end{align*}

The direction of the field is also East.

\item[E13]
The potential difference between the two points is 50 V.  So the change in potential energy is:
\begin{align*}
  \Delta U &= q \Delta V \\
           &= (.25 \ C)(50 \ V) \\
           &= 12.5 \ J \\
\end{align*}

\item[E15]
\begin{align*}
  \Delta U &= q \Delta V \\
  (.02 - .06 \ J) &= (2 \cdot 10^{-6} \ C) \Delta V \\
  \Delta V &= -20,000 \ V \\
\end{align*}

\item[E16]
\begin{align*}
  \Delta U &= q \Delta V \\
           &= (-5 \cdot 10^{-4} \ \coulomb) (400 \ \volt) \\
           &= -0.2 \ \joule \\
\end{align*}

Since the charge is negative, it has less potential energy after being moved.

\end{description}

\section{Synthesis Problems}

\begin{description}

\item[SP1]

\begin{description}
\item[a]
\[
  F = \frac{kq_1q_2}{r^2} = \frac{(9 \cdot 10^9 \ N \cdot m^2/C^2)(0.1 \ C)(0.02 \ C)}{(2 \ m)^2} = 4.5 \cdot 10^6 \ N
\]

to the right.

\item[b]
\[
  F = \frac{kq_1q_2}{r^2} = \frac{(9 \cdot 10^9 \ N \cdot m^2/C^2)(0.04 \ C)(0.02 \ C)}{(1 \ \meter)^2} = 7.2 \cdot 10^6 \ \newton
\]

to the left.

\item[c]
The charges are pushing in opposite directions, so the net force is:
\[
  F_{net} = 7.2 \cdot 10^6 - 4.5 \cdot 10^6 \ \newton = 2.7 \cdot 10^6 \ \newton
\]

to the left.

\item[d]
\[
  E = \frac{F}{q} = \frac{2.7 \cdot 10^6 \ N}{0.02 \ C} = 1.35 \cdot 10^8 \ \newton \per \coulomb \\
\]

to the left.

\item[e]
\[
  F = Eq = (1.35 \cdot 10^8 \ N/C)(-0.06 \ C) = 8.1 \cdot 10^{-7} \ \newton
\]

to the right.

\end{description}
\item[SP5]

\begin{description}
\item[a]
\[
  \Delta U = \Delta V \cdot q = (-400 \ V)(3 \cdot 10^{-4}) = -0.12 \ \joule
\]

\item[b]
The force is in the direction of decreasing potential towards the top plate.

\item[c]
The field is in the same direction as the force.

\item[d]
\begin{align*}
  V &= Ed \\
  E &= \frac{V}{d} \\
    &= \frac{400 \ \volt}{0.12 \ \meter} \\
    &\approx 3,333 \ \volt \per \meter \\
\end{align*}

\end{description}

\end{description}
\fi


\vspace{1 in}

\ifprintanswers
\else
{\em We have to fight against the evils of a society that has failed to produce brotherhood for every member of that
  society.  This in no way means that we're antiwhite, antiblue, antigreen, or antiyellow.  We're antiwrong.  We're
  antidescrimination.  We're antisegregation.  We're against anybody who wants to practice some form of segregation or
  descrimination because we don't happen to be a color that's acceptable to you.}
\vspace{.2 cm}

\hspace{1 cm} --Malcolm X
\fi

\end{document}

