\documentclass{exam}

\usepackage{graphicx}
\usepackage[fleqn]{amsmath}
\usepackage{cancel}
\usepackage{polynom}
\usepackage{float}
\usepackage{mdwlist}
\usepackage{booktabs}
\usepackage{cancel}
\usepackage{polynom}
\usepackage{caption}

\newcommand{\degree}{\ensuremath{^\circ}} 

% \oddsidemargin .5in
% \topmargin -1in
\textwidth 6.5 in

\printanswers

\ifprintanswers 
\usepackage{2in1, lscape} 
\fi

\title{Physics \\ Homework Two}
\date{September 19, 2011}
% \author{Ed Tellman}

\begin{document}

\maketitle

\section{From the Book}

\begin{itemize*}
  \item Read Chapters 4 and 5
  \item Chapter 4
    \begin{itemize*}
      \item questions 20, 23, 30, 31
      \item exercises 1, 2, 3, 7, 9, 15
      \item synthesis problems 1, 2, 5-7
    \end{itemize*}

%   \item Chapter 5
%     \begin{itemize*}
%       \item questions 3, 4, 5, 12, 22, 25
%       \item exercises 1, 4, 6, 11, 12
%       \item synthesis problems 1-5
%     \end{itemize*}

\end{itemize*}

\section{Extra Credit}

\begin{questions}

\question
A 1000 kg drag racer deploys its drag chute to slow down after it crosses the finish line.  What force must the chute
exert on the car to slow it from from 100 m/s to 25 m/s in 200 m? 

\begin{solution}
From the information provided and the equations of motion with constant acceleration, we have two equations and two unknowns:
\begin{align*}
  v &= v_0 + at \\
  x &= v_0t + \frac{1}{2} at^2 \\
\end{align*}

The two unknowns are $t$ and $a$---everything else is provided in the problem.  $a$ is what we are looking for, since
once we have $a$ we can multiply it by the mass to find the force.  So we can solve the first equation for $t$ and
substitute this value in the second equation.

\begin{align*}
  v &= v_0 + at \\
  t &= \frac{v-v_0}{a} \\
  \\
  x &= v_0 \left( \frac{v-v_0}{a} \right) + \frac{1}{2} a \left( \frac{v-v_0}{a} \right)^2 \\
    &= \frac{v_0v - v_0^2}{a} + \frac{(v-v_0)^2}{2a} \\
    &= \frac{v_0v - v_0^2}{a} + \frac{v^2 + 2v_0v + v_0^2}{2a} \\
    &= \frac{2v_0v - 2v_0^2 + v^2 - 2v_0v + v_0^2}{2a} \\
    &= \frac{v^2 - v_0^2}{2a} \\
  a &= \frac{v^2 - v_0^2}{2x} \\
\end{align*}

We have all of these numbers, so we can plug them in to find the acceleration:
\[
  a = \frac{(25 \text{ m/s})^2 - (100 \text{ m/s})^2}{2 \cdot (200 \text{ m})} \approx -23 \text{ m/s}^2 \\
\]

Then we can use Newton's second law to find the force:
\[
  F = ma = (1000 \text{ kg }) (23 \text{ m/s}^2) = 23,000 \text{ N}
\]

The force is in the opposite direction of the motion, since it is slowing the car down.

\end{solution}
% \question 
% You find yourself in the space shuttle trying to reach a satellite you are supposed to repair.  The
% satellite is in the same orbit as you but a few miles ahead.  You feel a bit frustrated, because if you fire your
% rocket, you shoot out to a higher orbit and if you flip around and fire your engine in reverse, you
% fall down to a lower orbit.  How can you catch up to the satellite so you can repair it?

% \begin{solution}
%   First flip around and fire your engine in reverse, so you drop down to a lower orbit.  In your new orbit, you will make
%   a trip around the earth more quickly than the satellite, since your new radius is smaller.  In a little while you will
%   catch up to the satellite, but be below it (closer to the earth).  When you catch up, fire your rocket again, in the
%   other direction, to boost yourself up to where the satellite is.
% \end{solution}

\end{questions}

\ifprintanswers

\section{Chapter 4}

\begin{description}

\item[Q20]
The force of gravity pulls the boy down and the normal force from the chair pushes up.  They exactly cancel, so the boy
doesn't go anywhere.  They don't form an action/reaction pair since they act on the same object.

\item[Q23]
\begin{description}
\item[a]
The force of gravity pulls down and the force from the string pulls up.

\item[b]
The net force is 0 because the ball is not accelerating.

\item[c]
The reaction force for gravity is the force from the ball pulling up on the earth.  The reaction force for the string is
the force the ball exerts on the string pulling the string down.

\end{description}

\item[Q30]
\begin{description}
\item[a] The blocks will accelerate because there is a net force operating on them.
\item[b] The tension in the string is less than the force.  The force from the string only accelerates the second block
  while the force $\vec{F}$needs to accelerate both blocks.
\end{description}

\item[Q31]
The force of gravity will always exceed the force from air resistance.  So the diver will always experience a net
downward force and will continue to accelerate indefinitely.

\item[E1]
\[
  a = \frac{F}{m} = \frac{40 \text{ N}}{5 \text{ kg}} = 8 \text{ m/s}^2
\]

\item[E2]
\[
  F = ma = (2.5 \text{ kg} \cdot (6 \text{ m/s}^2) = 15 \text{ N}
\]

\item[E3]
\[
  m = \frac{F}{a} = \frac{20 \text{ N}}{4 \text{ m/s}^2} = 5 \text{ kg}
\]

\item[E7]
The net force is 20N to the right.

\[
  m = \frac{F}{a} = \frac{20 \text{ N}}{4 \text{ m/s}^2} = 5 \text{ kg}
\]

\item[E9]
\begin{description*}
\item[a] The net force is: $F = -2 \text{ N} - 0.5 \text{ N}= -2.5N$ (opposing the motion)
\item[b] The acceleration is $a = \dfrac{-2.5 \text{ N}}{4 \text{ kg}} = -0.625 \text{ m/s}^2$
\end{description*}

\item[E15]
\begin{itemize}
  \item The force from gravity is: $(5 \text{ kg}) (-9.8 \text{ m/s}^2) = -49 N$ (down)
  \item The net force is: $-49 \text{ N}+ 15 \text{ N}= -34 \text{ N}$ (down)
  \item The acceleration is: $a = \dfrac{-34 \text{ N}}{4 \text{ kg}} = -8.5 \text{ m/s}^2$ (down)
\end{itemize}

\item[SP1]

\begin{description}
\item[a]
\[
  a = \frac{25 \text{ N}}{5 \text{ kg}} = 5 \text{ m/s}^2
\]

\item[b]
\[
  v = (5 \text{ m/s}^2) \cdot (3 \text{ s}) = 15 \text{ m/s}
\]

\item[c]
\[
  x = \frac{1}{2} \cdot (5 \text{ m/s}^2) \cdot (3 \text{ s})^2 = 22.5 \text{ m}
\]

\end{description}

\item[SP2]

\begin{description}
\item[a]
\[
  a = \frac{2 \text{ m/s}}{2 \text{ s}} = 1 \text{ m/s}^2
\]

\item[b]
\[
  F = (60 \text{ kg}) \cdot (1 \text{ m/s}^2) = 60 N
\]

\item[c]
\begin{align*}
  250 \text{ N} + F_{friction} &= 60 \text{ N} \\
  F_{friction} = -180 \text{ N} \\  
\end{align*}

\item[d]
For the box to move with constant velocity, the force from the rope would have to just cancel the force of friction, so
the force from the rope would need to be 180 N.

\end{description}

\pagebreak

\item[SP5]

\begin{description}
\item[a]
\[
  F = 30 \text{ N} - 14 \text{ N} = 16 \text{ N}
\]

\item[b]
\[
  a = \frac{F}{m} = \frac{16 \text{ N}}{6 \text{ kg}} \approx 2.66 \text{ m/s}^2
\]

\item[c]
\[
  F = ma = (2 \text{ kg})(2.66 \text{ m/s}^2) \approx 5.3 \text{ N}
\]

The force from the string also has to overcome the force of friction, so the total force from the string is $11.3 \text{ N}$.

\item[d]
\[
  F = 30 \text{ N} - 11.3 \text{ N} - 8 \text{ N} \approx 10.7 \text{ N}
\]

\[
  a = \frac{F}{m} = \frac{10.7 \text{ N}}{4 \text{ kg}} = 2.7 \text{ m/s}^2
\]

The two accelerations match. 

\end{description}

\item[SP6]

\begin{description}
\item[a]
\[
  F = ma = (60 \text{ kg})(-9.8 \text{ m/s}^2) = 588 \text{ N}
\]

\item[b]
\[
  F = ma = (60 \text{ kg})(1.4 \text{ m/s}^2) = -84 \text{ N (down)}
\]

\item[c]
\begin{align*}
  -588 \text{ N} + F &= -84 \text{ N} \\
  F &=  -84 \text{ N} + 588 \text{ N} \\
   &=  504 \text{ N} \\
\end{align*}

\item[d]
From part c, 504 N.

\item[e]
The man's apparent weight would increase by 84 N instead of decreasing by 84 N.

\end{description}

\item[SP7]

\begin{description}
\item[a]

\begin{itemize*}
  \item The force of air resistance is: $4 \cdot 100 \text{ N} = 400 \text{ N}$ (up)
  \item The force of gravity is: $-750$ N (down)
  \item The net force is: $400 \text{ N} - 750 \text{ N} = -350 \text{ N}$
\end{itemize*}

\item[b]
To figure out the acceleration, we first need to figure out the mass.

\[
  m = \frac{F}{a} = \frac{750 \text{ N}}{9.8 \text{ m/s}^2} \approx 77 \text{ kg}
\]

Now we can find the acceleration:
\[
  a = \frac{F}{m} = \frac{-350 \text{ N}}{77 \text{ kg}} = -4.6 \text{ m/s}^2 \text{ down}
\]

\item[c]
The terminal velocity happens when the force of the air resistance is equal to the sky diver's weight.  This happens at:
$v = 75 \text{ m/s}$ since the force from the air resistance is ten times the velocity.

\item[d]
The upward force would exceed the downward force and he would slow down to his terminal velocity again. 

\end{description}

\end{description}

\fi


%% \section{Other Questions}

%% \begin{questions}

%% \question
%% Two divers run horizontally off the edge of a low cliff.  Diver 2 runs with twice the speed of diver 1.  When the 
%% divers hit the water, is the horizontal distance covered by diver 2:
%% \begin{description*}
%%   \item[a] the same as
%%   \item[b] twice as much as
%%   \item[c] four times as much as
%% \end{description*}
%% the distance covered by diver 1?

%% \end{questions}


\vspace{1.5 in}

\ifprintanswers
\else
\begin{em}
But there was a wisdom in it all, as you'll see if you take a walk some night on a suburban street and pass house after
house on both sides of the street each with the lamplight of the living room, shining golden, and inside the little blue
square of the television, each living family riveting its attention on probably one show; nobody talking; silence in the
yards; dogs barking at you because you pass on human feet instead of wheels.  You'll see what I mean, when it begins to
appear like everybody in the world is soon going to be thinking in the same way...
\end{em}

\vspace{.2 cm}
\hspace{1.5 cm} --Jack Kerouac, {\em The Dharma Bums}

\fi

\end{document}

