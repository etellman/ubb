\documentclass{article}

\usepackage{graphicx}
\usepackage{amsmath}
\usepackage{cancel}
\usepackage{polynom}
\usepackage{float}
\usepackage{mdwlist}

\oddsidemargin 0in
\topmargin -0.5in
\textwidth 6.5in

\title{Introduction to Physics \\ Syllabus}
\date{August 29, 2010}
\author{Ed Tellman}

\begin{document}

\maketitle

\section{Introduction}
This course will provide an introduction to physics.  Since we aren't preparing for a specific Ohio University exam, I
thought we would try to do a high-level overview of multiple topics rather than cover a few things in extreme detail.

\section{Why to Take This Course}

Here are some of the things you may get out of this course:
\begin{itemize*}
  \item You'll learn some interesting things about physics.
  \item You'll learn something about why everyday things like bicycles, basketballs, and radios work the way they do.
  \item You'll practice using algebra to solve real-world problems.
  \item If you take a for-credit physics class later, it may be easier because the material will already be familiar
    to you.
%   \item If you are thinking about taking calculus, but wonder why you should bother, you'll learn a little about what
%     calculus is and how it is related to physics.
\end{itemize*}

\section{Homework and Exams}
You should expect to spend a few hours each week doing homework.  You are welcome to work together on homework.
If you are having difficulty with a problem feel free to consult with another student.

% Physics is like learning piano, basketball, or bicycle mechanics.  Watching someone else do it or reading about it in a
% book is helpful.  But you can't actually learn how to do it yourself unless you've practiced on your own.

Each section will be followed by an in-class examination to make sure that everyone is keeping up.

\section{Text Book}
We are a bit short of funds, so we were only able to purchase a limited number of text books.  We'll have to work out
some system for sharing.  

The homework will include some problems from the text.  But I'll include the actual questions (instead of just the
problem numbers) on the homework so you can still do the homework if you don't have a text book with you.

\section{Attendance}
I'll try to make the classes interesting and useful, so that that attending class helps you to understand the material
and do the homework.  If you find you are unable to attend, let me know know and I'll help you catch up on whatever
you missed.

% \section{Course Credit}
% Since we are a bit short on tuition funds at the moment, there won't be any Ohio University exam for credit
% for this course.

\section{Course Overview}

Here's what we'll plan to cover in this course.

\subsection{Mechanics}

\begin{itemize*}
  \item position, velocity, and acceleration and how they are related
  \item force and work
  \item gravity and gravitational potential energy
  \item angles, angular velocity, and angular acceleration, and torque
  \item momentum
\end{itemize*}

\subsection{Electricity and Magnetism}
\begin{itemize*}
  \item electric charge
  \item the force between two charged particles
  \item electric fields
  \item circuits
  \item magnetism and how it is related to electricity
\end{itemize*}

\subsection{Waves and Optics}
\begin{itemize*}
  \item waves, wavelengths, and periods
  \item sound and music
  \item electromagnetic waves and light
  \item reflection and refraction
\end{itemize*}

\subsection{Relativity}
\begin{itemize*}
  \item time dilation
  \item length contraction
  \item $e=mc^2$
  \item general relativity and gravity
\end{itemize*}


\end{document}

