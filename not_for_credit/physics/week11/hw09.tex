
\documentclass{exam}

\usepackage{siunitx}
\usepackage{graphicx}
\usepackage[fleqn]{amsmath}
\usepackage{cancel}
\usepackage{polynom}
\usepackage{float}
\usepackage{mdwlist}
\usepackage{booktabs}
\usepackage{cancel}
\usepackage{polynom}
\usepackage{caption}

\newcommand{\degree}{\ensuremath{^\circ}} 
\everymath{\displaystyle}

% \oddsidemargin .5in
% \topmargin -1in
\textwidth 6.5 in

\printanswers

\ifprintanswers 
\usepackage{2in1, lscape} 
\fi

\title{Physics \\ Homework Nine \\ Magnetism}
\date{November 21, 2011}

\begin{document}

\maketitle

\section{Homework}

\begin{itemize*}
  \item Read Chapter 14, sections 1-3
  \item Chapter 14
    \begin{itemize*}
      \item questions 8-11, 15, 17
      \item exercises 2, 4, 6-8
      \item synthesis problems: 1-2
    \end{itemize*}
\end{itemize*}

\section{Extra Credit}

The magnitude of a proton or electron's charge is $1.6 \cdot 10^{-19} \ \coulomb$.

\begin{questions}

  \question An electron moving at $10 \ \meter \per \second$ in the positive $x$ direction experiences no magentic
  force.  When it moves in the positive $y$ direction, it experiences a force of $2.0 \cdot 10^{-13} \ \newton$ in the
  negative $z$ direction.  What is the direction and magnitude of the electric field? (James Walker)

\begin{solution}

Because the electron experiences no force when moving parallel to the $x$ axis, the magnetic field must also be parallel
to the $x$ axis.  Because of the direction of the force experienced by the electron, the field is in the positive $x$
axis direction.

The magnitude is:
\begin{align*}
  F &= qvB \\
  B &= \frac{F}{qv} \\
    &= \frac{2.0 \cdot 10^{-13} \newton}{(1.6 \cdot 10^{-19} \ \coulomb)(10 \ \meter \per \second)} \\
    &= 1.25 \cdot 10^5 \ \tesla \\
\end{align*}

\end{solution}

\question 
When at rest, a proton experiences a net electromagnetic force of $8.0 \cdot 10^{-13} \ \newton$ in the
positive $x$ direction.  When the proton moves with a speed of $1.5 \cdot 10^6 \ \meter \per \second$ in the positive
$y$ direction, the net electromagnetic force on it decreases in magnitude to $7.5 \cdot 10^{-13} \ \newton$, still
pointing in the positive $x$ direction.  Find the magnitude and directions of the electric and magnetic fields. (James
Walker)

\begin{solution}
The force the proton experiences when at rest is electrostatic since only moving protons are affected by magnetism.  The electric field is:
\[
  E = \frac{F}{q} = \frac{8.0 \cdot 10^{-13} \ \newton}{1.6 \cdot 10^{-19} \ \coulomb} = 5.0 \cdot 10^{6} \ \newton \per \coulomb
\]

When the proton is moving, it is also affected by the magnetic field:
\begin{align*}
  F_{total} &= F_{B} + F_{E} \\
  7.5 \cdot 10^{-13} \ \newton &= F_{B} + 8.0 \cdot 10^{-13} \ \newton \\
  F_{B} &= -5.0 \cdot 10^{-14} \ \newton \\
\end{align*}

The field can be determined from the force and velocity:
\begin{align*}
  F_{B} &= qvB \\
  B &= \frac{F_{B}}{qv} \\
    &= \frac{5.0 \cdot 10^{-14}}{(1.6 \cdot 10^{-19} \ \coulomb)(1.5 \cdot 10^6 \meter \per \second)} \\
    &= 0.2 \ \tesla \\
\end{align*}

The force from the magenetic field must be in the opposite direction from the electrostatic force.  So the field is in the negative $z$ direction.


\end{solution}

\end{questions}

\ifprintanswers

\section{Questions}

\begin{description}

\item[Q8]
The needle will deflect about $90 \degree$.

\item[Q9]
The needle is already pointing in the direction of the field, so it won't deflect.

\item[Q10]
Each wire has a balance of positive and negative charges, so there aren't any electrostatic forces and the only
forces are magnetic.

\item[Q11]
There is a force pointing down.

\item[Q15]
The field points down at the center.

\item[Q17]
Using the right hand rule, the force at the top of the loop is to the left and the force at the bottom of the loop is to
the right.  This combination wil make the loop rotate counterclockwise.

\end{description}

\section{Exercises}

\begin{description}

\item[E2]
\[
   \frac{F}{l} = \frac{2k'I_1I_2}{r} = \frac{2 (1 \cdot 10^{-7} \ \newton/\ampere^2) (4 \ \ampere)^2}{0.1 \ \meter}
     = 3.2 \cdot 10^{-5} \ \newton \per \meter
\]

\item[E4]
\begin{align*}
  \frac{F}{l} &= \frac{2k'I_1I_2}{r} \\ 
    r &= \frac{2k'I_1I_2}{F/l} \\ 
     &= \frac{2 (1 \cdot 10^{-7} \ \newton/\ampere^2) (2 \ \ampere)^2}{1.6 \cdot 10^{-5} \ \newton \per \meter} \\
     &= 5 \ \centi\meter \\
\end{align*}

\item[E6]
\[
  F = qvB = (0.06 \ \coulomb)(600 \ \meter \per \second)(0.5 \ \tesla) = 18 \ \newton
\]

\item[E7]
\[
  F = IlB = (5 \ \ampere)(0.1 \ \meter)(0.6 \ \tesla) = 0.3 \ \newton
\]

\item[E8]
\begin{align*}
  F &= IlB \\
  B &= \frac{F}{Il} \\
    &= \frac{6 \ \newton}{(5 \ \ampere)(0.3 \ \meter)} \\
    &= 4 \ \tesla \\
\end{align*}

\end{description}

\section{Synthesis Problems}
\begin{description}

\item[SP1]
\begin{description}

\item[a]
\[
   \frac{F}{l} = \frac{2k'I_1I_2}{r} = \frac{2 (1 \cdot 10^{-7} \ \newton/\ampere^2) (5 \ \ampere)(10 \ \ampere)}{0.05 \ \meter}
     = 2 \cdot 10^{-4} \ \newton \per \meter
\]

\item[b]
Since the currents are in opposite directions, the wires are repelled from each other.

\item[c]
\[
  F = 2 \cdot 10^{-4} \ \newton \per \meter \cdot 0.3 \meter = 6 \cdot 10^{-5} \ \newton
\]

\item[d]
\begin{align*}
  F &= IlB \\
  B &= \frac{F}{Il} \\
    &= \frac{6 \cdot 10^{-5} \ \newton}{(10 \ \ampere)}{0.3 \ \meter} \\
    &= 2 \cdot 10^{-5} \ \tesla
\end{align*}


\item[e]
Using the right hand rule, the direction is down.

\end{description}

\item[SP2]
\begin{description}

\item[a]
\[
  F = qvB = (0.05 \ \coulomb)(200 \ \meter \per \second)(0.5 \ \tesla) = 5 \ \newton
\]

\item[b]
up

\item[c]
Since the force is perpendicular to the velocity, it changes the direction but not the magnitude of the velocity.

\item[d]
\[
  a = \frac{F}{m} = \frac{5 \ \newton}{0.025 \ \kilogram} = 200 \ \meter \per \second^2
\]

\item[e]
\begin{align*}
  a &= \frac{v^2}{r} \\
  r &= \frac{v^2}{a} \\
    &= \frac{(200 \ \meter \per \second)^2}{200 \ \meter \per \second^2} \\
    &= 200 \ \meter \\
\end{align*}

\end{description}

\end{description}

\else

\vspace{2 in}

{\em 
``Tell me one last thing,'' said Harry.  ``Is this real?  Or has it all been happening inside my head?''.  

Dumbeldore beamed at him and his voice sounded loud and strong in Harry's ears even though the bright mist was
descending again, obscuring his figure.

``Of course it is happening inside your head, Harry, but why on earth should that mean that it is not real?''
}

\vspace{.2 cm}

\hspace{1 cm} --J. K. Rowling, {\em Harry Potter and the Deathly Hallows}
\fi

\end{document}

