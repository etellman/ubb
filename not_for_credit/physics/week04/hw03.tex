
\documentclass{exam}

\usepackage{graphicx}
\usepackage[fleqn]{amsmath}
\usepackage{cancel}
\usepackage{polynom}
\usepackage{float}
\usepackage{mdwlist}
\usepackage{booktabs}
\usepackage{cancel}
\usepackage{polynom}
\usepackage{caption}

\newcommand{\degree}{\ensuremath{^\circ}} 

% \oddsidemargin .5in
% \topmargin -1in
\textwidth 6.5 in

\printanswers

\ifprintanswers 
\usepackage{2in1, lscape} 
\fi

\title{Physics \\ Homework Three}
\date{September 26, 2011}
% \author{Ed Tellman}

\begin{document}

\maketitle

\section{From the Book}

\begin{itemize*}
  \item Read Chapter 5 and 6.1-6.4
  \item Chapter 5
    \begin{itemize*}
      \item questions 3, 4, 5, 12, 22, 25
      \item exercises 1, 4, 6, 11, 12
      \item synthesis problems 1-5
    \end{itemize*}
  \item Chapter 6
    \begin{itemize*}
      \item questions 1, 3-5, 8, 12-14, 22, 30
      \item exercises 1-6, 8, 11-12, 14, 16
      \item synthesis problems 1-2, 5
    \end{itemize*}
\end{itemize*}

\section{Extra Credit}

\begin{questions}

\question 
You find yourself in the space shuttle trying to reach a satellite you are supposed to repair.  The
satellite is in the same orbit as you but a few miles ahead.  You feel a bit frustrated, because if you fire your
rocket, you shoot out to a higher orbit and if you flip around and fire your engine in reverse, you
fall down to a lower orbit.  How can you catch up to the satellite so you can repair it?

\begin{solution}
  First flip around and fire your engine in reverse, so you drop down to a lower orbit.  In your new orbit, you will make
  a trip around the earth more quickly than the satellite, since your new radius is smaller.  In a little while you will
  catch up to the satellite, but be below it (closer to the earth).  When you catch up, fire your rocket again, in the
  other direction, to boost yourself up to where the satellite is.
\end{solution}

\question 
A 1000 kg drag racer deploys its drag chute to slow down after it crosses the finish line.  What force must
the chute exert on the car to slow it from from 100 m/s to 25 m/s in 200 m?  ({\em This is last week's problem, but this
time solve it using work and kinetic energy instead of Newton's laws.  It's much easier this way.})

\begin{solution}

The initial and final kinetic energies are:
\begin{align*}
  K_0 &= \frac{1}{2} (1000 \text{ kg})(100 \text{ m/s})^2 = 5 \cdot 10^6 \text{ J} \\
  K_1 &= \frac{1}{2} (1000 \text{ kg})(25 \text{ m/s})^2 =  3.125 \cdot 10^5 \text{ J} \\
\end{align*}

The work is the change in energy, or: $W = - 4.69 \cdot 10^6 \text{ J}$.  And, since $W = Fd$:

\begin{align*}
  - 4.69 \cdot 10^6 \text{ J} &= F \cdot 200 \text{ m} \\
  F &= -23,437 \text{ N} \\
\end{align*}

which is the same as the answer from last week (although I included a few more significant digits this week).

\end{solution}

\question 

You just purchased a new slingshot and are curious to figure out how far it can shoot.  Of course, you don't
want to actually try it, and would much prefer to do a few calculations.  You know that if you pull the elastic back 0.3 m
and launch a 0.5 kg projectile straight up, the projectile will go to a height of 30 m.

\begin{parts} 

\part What is the spring coefficient k, assuming the elastic behaves like a spring?

\begin{solution}
At the peak of the flight, all the energy from the slingshot has been converted to potential energy.  At this point, it
has 
\[
  U = mgh = (.5 \text{ kg})(9.8 \text{ m/s}^2)(30 \text{ m}) = 147 \text{ J}
\]
of energy.

Originally, all this energy was in the slingshot:
\begin{align*}
  E_{spring} &= \frac{1}{2} kx^2 \\
  147 \text{ J} &= \frac{1}{2}k(.3 \text{ m})^2 \\
  k &= 3,267 \text{ N/m} \\
\end{align*}

\end{solution}

\part How far will the 0.5 kg projectile travel if launched at a $45 \degree$ angle?

\begin{solution}

Immediately after the projectile is launched, all the energy is kinetic energy.  We can use this information to figure
out the launch velocity:
\begin{align*}
  E_{kinetic} &= \frac{1}{2} mv^2 \\
  147 \text{ J} &= \frac{1}{2}(0.5 \text{ kg})v^2 \\
  v &\approx 24.2 \text{ m/s} \\
\end{align*}

The y-component of the velocity is $v_y =  (24 \text{ m/s})(\sin 45 \degree) = 17.1 \text{ m/s}$.  The x-component is the same,
since the launch angle is $45 \degree$.

The peak of the flight is reached when $v_y = 0$:
\begin{align*}
  v_y &= v_{0y} + at \\
  0 &= (17 \text{ m/s}) - (9.8 \text{ m/s}^2)t \\
  t &= 1.75 \text{ s} \\
\end{align*}

The total flight time is twice this, or 3.5 s.  In this time, the projectile travels:
\[
  x = (17.1 \text{ m/s}) \cdot (3.5 \text{ s}) \approx 60 \text{ m} 
\]

\end{solution}

\end{parts} 

\end{questions}

\ifprintanswers

\section{From the Book}

\subsection{Chapter 5}
\begin{description}

\item[Q3]
The faster moving car has experienced a greater change in velocity.  For example, it may have gone from traveling 60 mph
North to 60 mph East which is a bigger change than going from 1 mph North to 1 mph East.

\item[Q4]
The car traveling around the curve with the smaller radius experiences the larger change in velocity.

The bigger the radius the more closely the curve approximates a straight line.  A car traveling at a constant speed in a
straight line experiences no change in velocity, and the closer the curve is to a straight line, the smaller the change
in velocity.

\item[Q5]
The speed enters once because of the effect described in Q3.  The other reason the speed appears is that a faster moving object
moves farther in each time interval.  So, for example, if you look at object every second, a faster moving object will
have experienced a greater change than a slower moving object.

\item[Q12]
Since the object is moving at a constant speed, the centripetal force is constant throughout the motion.  At the bottom
of the circle, the object's weight contributes to the tension in the string.  At the top of the circle, the object's
weight subtracts from the tension in the string.  So the tension is greatest at the bottom of the circle.

\item[Q22]
Since $m_2$ is closer to $m_1$ than $m_3$, the attraction between $m_1$ and $m_2$ is greater than the attraction between $m_2$ and $m_3$, so
the net force is to the left.

\item[Q25]
We have a new moon when the moon is roughly between the earth and the sun.  So the new moon rises and sets at about the
same times the sun rises and sets.

\item[E1]
\[
  a = \frac{v^2}{r} = \frac{(5 \text{ m/s})^2}{.8 \text{ m}} = 31.25 \text{ m/s}^2
\]

\item[E4]
\begin{align*}
  a_{60} &= \frac{(60 \text{ m/h})^2}{r} = \frac{3600}{r} \text{m/h}^2 \\
  a_{30} &= \frac{(30 \text{ m/h})^2}{r} = \frac{900}{r} \text{m/h}^2 \\
  \frac{a_{60 mph}}{a_{30 mph}} &= \frac{3600}{900} = 4 \\ 
\end{align*}
The faster car has 4 times the acceleration.

\item[E6]
\begin{description}
\item[a]
\[
  a = \frac{v^2}{r} = \frac{(20 \text{ m/s})^2}{40 \text{ m}} = 10 \text{ m/s}^2
\]

\item[b]
\[
  F = ma = (1,200 \text{ kg}(10 \text{ m/s}^2) = 12,000 \text{ N}
\]

\end{description}

\item[E11]
Increasing the distance by a factor of 3 reduces the force by a factor of $3^2$ so the new force will be 
\[
  \frac{0.36 \text{ N}}{9} = 0.04 \text{ N}
\]

\item[E12]
\[
  F_g = \frac{Gm_1m_2}{r^2} = 6.67 \cdot 10^{-11} \cdot 200 \cdot 200 \text{ N} = 2.67 \cdot 10^{-6} \text{ N}
\]

\item[SP1]
\begin{description}
\item[a]
\[
  a = \frac{v^2}{r} = \frac{(4.0 \text{ m/s})^2}{.6 \text{ m}} = 26.67 \text{ m/s}^2
\]

\item[b]
\[
  F = ma = (0.2 \text{ kg})(26.7 \text{ m/s}^2) \approx 5.33 \text{ N}
\]

\item[c]
\[
 F = ma =  (0.2 \text{ kg})(9.8 \text{ m/s}^2) \approx 1.96 \text{ N}
\]

\item[d]
\[
  F = \sqrt{1.96^2 + 5.33^2} \text{ N} = 5.68 \text{ N}
\]
\end{description}

\item[SP2]
\begin{description}
\item[a]
\[
  s = \frac{d}{t} = \frac{2 \pi \cdot 12 \text{ m}}{8 \text{ s}} = 9.4 \text{ m/s}
\]

\item[b]
\[
 a = \frac{v^2}{r} = \frac{(9.4 \text{ m/s})^2}{12 \text{ m}} = 7.4 \text{ m/s}^2
\]

\item[c]
\[
  F = ma = (40 \text{ kg})(7.4 \text{ m/s}^2) \approx 296 \text{ N}
\]

Since the centripetal acceleration is less than the acceleration from gravity, the rider's weight is sufficient to
provide the centripetal acceleration.

\item[d]
The rider's weight is:
\[
  F = ma = (40 \text{ kg})(9.8 \text{ m/s}^2) = 392 \text{ N}
\]

The normal force is the difference between the weight and the centripetal force:
\[
  F_n = 392 - 296 \text{ N} = 96 \text{ N}
\]

\item[e]
If the rider's seat belt was not securely fastened, he would get launched off into the air, which would make the ride
much more exciting but might reduce the number of repeat customers.

\end{description}

\item[SP3]
\begin{description}
\item[a]
\[
  a = \frac{v^2}{r} = \frac{(25 \text{ m/s})^2}{60 \text{ m}} = 10.4 \text{ m/s}^2
\]

\item[b]
\[
  F = ma = (900 \text{ kg})(10.4 \text{ m/s}^2) = 9,360 \text{ N}
\]

\item[c]
\[
  F = ma = (900 \text{ kg})(9.8 \text{ m/s}^2) = 8,820 \text{ N}
\]

\item[d]
I can't draw a picture with the computer, but here's the calculation:
\begin{align*}
  \cos \theta &= \frac{weight}{F_{normal}} \\
  .97 &= \frac{8,820}{F_{normal}} \\
  F_{normal} &= 9,131 \text{ N} \\
\end{align*}

\item[e]
The normal force is smaller than the required centripetal force, so the the horizontal component by itself definitely
isn't enough nearly enough. 

\end{description}

\item[SP4]
\begin{description}
\item[a]
\[
  s = \frac{d}{t} = \frac{2 \pi \cdot 3 \text{ m}}{1 \text{ s}} = 18.85 \text{ m/s}
\]

\item[b]
\[
  a = \frac{v^2}{r} = \frac{(18.85 \text{ m/s})^2}{3 \text{ m}} = 118.4 \text{ m/s}^2
\]

It's much bigger than the acceleration due to gravity.

\item[c]
\[
  F = ma = (60 \text{ kg})(118.4 \text{ m/s}^2) = 7,106 \text{ N}
\]

The passenger's weight is only:
\[
  F = ma = (60 \text{ kg})(9.8 \text{ m/s}^2) = 588 \text{ N}
\]

If he doesn't have his seat belt fastened, he may be enjoying an exciting flight through the air.

\end{description}

\item[SP5]
\begin{description}

\item[a]
\[
  F = \frac{Gm_{sun}m_{earth}}{r_{sun/earth}^2} = 3.53 \cdot 10^{22} \text{ N}
\]

\item[b]
\[
  F = \frac{Gm_{moon}m_{earth}}{r_{moon/earth}^2} = 2.01 \cdot 10^{20} \text{ N}
\]

\item[c]
\[
  \frac{3.53 \cdot 10^{22}}{2.01 \cdot 10^{20}} = 175
\]

The force from the sun on the earth is about 175 times the force from the moon on the earth, so the moon doesn't have
much impact on the earth's motion.

\item[d]
\[
  F = \frac{Gm_{moon}m_{sun}}{r_{sun/earth}^2} = 4.34 \cdot 10^{20} \text{ N}
\]

The force from the sun on the moon is about twice the force from the earth on the moon (from part b), so the sun does
have a big impact on the moon's motion.  Both the earth and the moon orbit the sun while the moon orbits the earth.

\end{description}
\end{description}

\subsection{Chapter 6}

\begin{description}

\item[Q1]
The formula for work is $W = Fd$.  The mass isn't involved, so the mass that moves twice as far has twice as much work done on it.

\item[Q3]
\begin{description*}
\item[a] Part of the force is in the direction of motion, so the force does do work on the box.  
\item[b] Only part of the force is in the direction of motion, so only part of the force does do work on the box.  
\end{description*}

\item[Q4]
The frictional force is in the opposite direction of the motion so it does negative work on the box.

\item[Q5]
The normal force is perpendicular to the direction of the motion so it doesn't do work on the box.

\item[Q8]
She does the same amount of work:
\begin{align*}
  d_1 &= 4 d_2 \\
  F_1 &= \frac{F_2}{4} \\
  d_1 F_1 &= 4 d_2 \cdot \frac{F_2}{4} = d_2 F_2 \\
\end{align*}

\item[Q12]
Some of the work done by the string goes into overcoming the friction, so the change in kinetic energy is less than the
work done by the string. 

\item[Q13] 

No.  An object moving in a circle just has the centripetal force acting on it.  The centripetal changes the direction
but not the magnitude of the velocity, so there is no change in kinetic energy in this case.

Another way to look at it is that the centripetal force is always perpendicular to the motion.  It doesn't do any work
on the object, so there is no change in the kinetic energy.

\item[Q14]
The work is the change in the kinetic energy.
\begin{align*}
  v_2 &= 2 v_1 \\
  W_1 &= \frac{1}{2} mv_1^2 \\
  W_2 &= \frac{1}{2} mv_2^2 = \frac{1}{2} m (2v_1)^2 = 2 mv_1^2\\
\end{align*}

The second force does four times as much work to accelerate the ball to twice the velocity.

\item[Q22]
\begin{itemize*}
\item gravitational potential energy
\item At the bottom of the swing, all the gravitational potential energy has been converted to kinetic energy, so this is
  when the kinetic energy is the greatest.
\item At the top of the swing, all the kinetic energy has been converted to gravitational potential energy, so this is
  when the gravitational potential energy is the greatest.
\end{itemize*}

\item[E1]
\[
  W = Fd = (40 \text{ N})(2.5 \text{ m}) = 100 \text{ J}
\]
 
\item[E2]
\begin{align*}
  F &= \frac{W}{d} \\
   &= \frac{160 \text{ J}}{4 \text{ m}} \\
   &= 40 \text{ N}  
\end{align*}

\item[E3]
\begin{align*}
  d &= \frac{W}{F} \\
   &= \frac{300 \text{ J}}{60 \text{ N}} \\
   &= 5 \text{ m}  
\end{align*}

\item[E4]
\begin{description*}
\item[a] $W_{rope} = Fd = (180 \text{ N})(2 \text{ m}) = 360 \text{ J}$
\item[b] $W_{friction} = Fd = (-60 \text{ N})(2 \text{ m}) = -120 \text{ J}$
\item[c] $W_{total} = W_{rope} + W_{friction} = 360 \text{ J} - 120 \text{ J} = 240 \text{ J}$
\end{description*}

\item[E5]
\begin{description*}

\item[a] $W_{horizontal} = Fd = (40 \text{ N})(4 \text{ m}) = 160 \text{ J}$

\item[b] The vertical component of the force isn't in the direction of motion so it doesn't do any work.

\item[c] The total work is just the work done by the horizontal component, or 160 J.

\end{description*}

\item[E6]
\begin{description*}

\item[a] $W = Fd = (60 \text{ N})(10 \text{ m}) = 600 \text{ J}$    

\item[b] The increase in kinetic energy is the same as the work done, or 600 J.  You can also figure out the final
  velocity, if you're curious:
\begin{align*}
  K &= \frac{1}{2} mv^2 \\
  600 \text{ J} &= \frac{1}{2} \cdot (4 \text{ kg}) v^2 \\
  v^2 &= 300 \text{ J/kg} \\
  v &\approx 17 \text{ m/s}
\end{align*}

\item[E8]
\begin{description*}
  \item[a] $U = mgh = (5 \text{ kg})(10 \text{ m/s}^2)(2 \text{ m}) = 100 \text{ J}$

  \item[b] See part a
\end{description*}

\end{description*}

\item[SP1]

\begin{description}
\item[a]
\[
  W = Fd = (5 \text{ N})(1.5 \text{ m}) = 7.5 \text{ J}
\]

\item[b]
\[
  W = Fd = (3 \text{ N})(1.5 \text{ m}) = 4.5 \text{ J}
\]

\item[c]
The net force.  Some of the work from the 5 N force goes towards overcoming friction, not to speeding up the block.

\item[d]
Some of it shows up in the increase in kinetic energy and the rest of it turns into heat.

\item[e]
Since there was 4.5 J of work done by the net force, this is the kinetic energy at the end of the motion.  Its velocity
is:
\begin{align*}
  4.5 \text{ J} &= \frac{1}{2} \cdot (0.25 \text{ kg}) v^2 \\
  v &= 6 \text{ m/s} \\
\end{align*}

\end{description}

\item[SP2]
\begin{description}
\item[a]
\begin{align*}
  F &= ma \\
  50 \text{ N} &= (100 \text{ kg}) \cdot a \\
  a &= 0.5 \text{ m/s}^2 \\
\end{align*}

\item[b]
\[
  x = \frac{1}{2} at^2 = \frac{1}{2} \cdot (0.5 \text{ m/s}^2)(4 \text{ s})^2  = 4 \text{ m}
\]

\item[c]
\[
  W = Fd = (50 \text{ N})(4 \text{ m}) = 200 \text{ N}
\]

\item[d]
\[
  v = at = (0.5 \text{ m/s}^2)(4 \text{ s}) = 2 \text{ m/s}
\]

\item[e]
\[
  E_k = \frac{1}{2} mv^2 = \frac{1}{2} (100 \text{ kg}) (2 \text{ m/s})^2 = 200 \text{ J}
\]

The two values match since all the work went into increasing the kinetic energy.

\end{description}

\item[SP5]

\begin{description}
\item[a]
The initial potential energy is 
\[
  U = mgh = (40 \text{ kg})(9.8 \text{ m/s}^2)(40 \text{ m}) = 15,680 \text{ J}
\]

The potential energy at the second point is: 
\[
  U = mgh = (40 \text{ kg})(9.8 \text{ m/s}^2)(30 \text{ m}) = 11,760 \text{J}
\]

The difference is:
\[
  15,680 \text{ J} - 11,760 \text{J} = 3,920 \text{ J}
\]

Since only 2,000 J is lost to friction, there is plenty of energy to make it to the second hump.

\item[b]
The maximum energy that the sled could have on the second hump would be
\[
  15,680 \text{ J} - 2,000 \text{J} = 13,680 \text{ J}
\]

This corresponds to a height of:
\begin{align*}
  U &= mgh \\
  13,680 \text{ J} &= (40 \text{ kg})(9.8 \text{ m/s}^2) h \\
  h &\approx 35 \text{ m} \\
\end{align*}

In this case it will stop at the top of the second hump, so it will need a little push to continue to the bottom.

\end{description}
\end{description}

\fi


%% \section{Other Questions}

%% \begin{questions}

%% \question
%% Two divers run horizontally off the edge of a low cliff.  Diver 2 runs with twice the speed of diver 1.  When the 
%% divers hit the water, is the horizontal distance covered by diver 2:
%% \begin{description*}
%%   \item[a] the same as
%%   \item[b] twice as much as
%%   \item[c] four times as much as
%% \end{description*}
%% the distance covered by diver 1?

%% \end{questions}


\vspace{0.3 in}

\ifprintanswers
\else
\begin{em}
  My passion for social justice has often brought me into conflict with people, as did my aversion to any obligation and
  dependence I do not regard as absolutely necessary. 
  I always have a high regard for the individual and have an
  insuperable distaste for violence and clubmanship.  All these motives made me into a passionate pacifist and
  anti-militarist. I am against any nationalism, even in the guise of mere patriotism. Privileges based on position and
  property have always seemed to me unjust and pernicious, as did any exaggerated personality cult.
\end{em}

\vspace{.2 cm}
\hspace{1.5 cm} --Albert Einstein

\fi

\end{document}

