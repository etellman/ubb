
\documentclass{exam}

\usepackage{graphicx}
\usepackage[fleqn]{amsmath}
\usepackage{cancel}
\usepackage{polynom}
\usepackage{float}
\usepackage{mdwlist}
\usepackage{booktabs}
\usepackage{cancel}
\usepackage{polynom}
\usepackage{caption}

\newcommand{\degree}{\ensuremath{^\circ}} 

% \oddsidemargin .5in
% \topmargin -1in
\textwidth 6.5 in

\printanswers

\ifprintanswers 
\usepackage{2in1, lscape} 
\fi

\title{Physics \\ Homework Five \\Rotational Kinematics}
\date{October 10, 2011}
% \author{Ed Tellman}

\begin{document}

\maketitle

\section{From the Book}

\begin{itemize*}
  \item Read Chapter 8, Section 1
  \item Chapter 8
    \begin{itemize*}
      \item questions 3-6, 8
      \item exercises 2-7
    \end{itemize*}
\end{itemize*}

\section{Useful Formulas}

\begin{itemize}
\item The {\em tangential velocity} of a rotating object is $v_t = r \omega$, where $\omega$ is in {\em rad/s}.  With a
  constant angular velocity, the farther the object is from the center, the faster it moves.

\item The {\em tangential acceleration} of a rotating object is $a_t = r \alpha$, where $\alpha$ is in $\text{
  rad/s}^2$.  With a constant angular acceleration, the farther the object is from the center, the greater its
  tangential acceleration.

\item The {\em centripetal acceleration} of a rotating object in terms of its angular velocity is $a_{cp} = r \omega^2$.

\end{itemize}

\section{Exercises}

\begin{questions}
\item Find the angular velocity of the earth's rotation on its axis in rad/s.

\begin{solution}
One revolution takes 24 hours.  To convert this to rad/s:
\[
  \frac{1 \cancel{\text{ rev}}}{24 \cancel{\text{ h}}} 
      \cdot \frac{1 \cancel{\text{ h}}}{60 \cancel{\text{ min}}}
      \cdot \frac{1 \cancel{\text{ min}}}{60 \text{ s}}
      \cdot \frac{2 \pi \text{ rad}}{1 \cancel{\text{ rev}}}
      = 7.27 \cdot 10^{-5} \text{ rad/s}
\]

\end{solution}

\item The angular velocity of a wheel increases from 10 rad/s to 20 rad/s in 5 revolutions.  What is its angular
  acceleration?

\begin{solution}
5 revolutions is $5 \cdot 2 \pi \approx 31.42 \text{ rad}$

First find out how long it took:
\begin{align*}
  \theta &= v_{avg}t \\
  t &= \frac{\theta}{v_{avg}} \\
  t &= \frac{31.42 \text{ rad}}{15 \text{ rad/s}} \approx 2.09 \text{ s}
\end{align*}

Then find the acceleration:
\[
  \alpha = \frac{\Delta v}{t} = \frac{10 \text{ rad/s}}{2.09 \text{ s}} \approx 4.77 \text{ rad/s}^2
\]

check:
\begin{align*}
  \omega &= \omega_0 + \alpha t \\
  20 \text{ rad/s} &= 10 \text{ rad/s} + (4.77 \text{ rad/s}^2)(2.09 \text{ s}) \\ 
  20 \text{ rad/s} &\approx 19.97 \text{ rad/s} \\
  \\
   \theta &= \omega_0 t + \frac{1}{2} \alpha t^2 \\
  5 \cdot 2 \pi \text{ rad} &= (10 \text{ rad/s})(2.09 \text{ s}) + \frac{1}{2} (4.77 \text{ rad/s}^2) (2.09 \text{ s})^2 \\
  31.4 \text{ rad} &\approx 31.3 \text{ rad} \\
\end{align*}

\end{solution}


\item The angular velocity of a wheel increases from 10 rad/s to 20 rad/s in 5 seconds.  What angle did it turn through in
  this time.

\begin{solution}
First find the angular acceleration:

\[
  \alpha = \frac{\Delta v}{t} = \frac{10 \text{ rad/s}}{5 \text{ s}} = 2 \text{ rad/s}^2
\]

Then find the angle:
\begin{align*}
  \theta &= \omega_0 t + \frac{1}{2} \alpha t^2 \\
    &= (10 \text{ rad/s})(5 \text{ s}) + \frac{1}{2} (2 \text{ rad/s}^2)(5 \text{ s})^2 \\
    &= 75 \text{ rad} \\
\end{align*}


\end{solution}

\item If the second hand of a clock is 10 cm long, what is the linear velocity of the tip of the hand?
\begin{solution}
The angular velocity is:
\[
  \omega = \frac{1 \cancel{\text{ rev}}}{1 \cancel{\text{ m}}} 
      \cdot \frac{1 \cancel{\text{ m}}}{60 \text{ s}} 
      \cdot \frac{2 \pi \text{ rad}}{1 \cancel{\text{ rev}}} 
      \approx 0.105 \text{ rad/s}
\]

The linear velocity is:
\[
  v = \omega r = (0.105 \text{ rad/s})(10 \text{ cm}) = 1.05 \text{ cm/s}
\]

\end{solution}

\item The outer edge of a Frisbee with diameter .3 m has linear velocity of 3 m/s.  What is the Frisbee's angular velocity?

\begin{solution}
\begin{align*}
  v &= \omega r \\
  \omega &= \frac{v}{r} \\
  \\
  \omega &= \frac{3 \text{ m/s}}{.15 \text{ m}} = 20 \text{ rad/s}
\end{align*}

\end{solution}

\item A carousel rotates once every 45 seconds.
\begin{solution}
The angular velocity is:
\[
  \omega = \frac{1 \cancel{\text{ rev}}}{45 \text{ s}} \cdot \frac{2 \pi \text{ rad}}{\cancel{\text{rev}}} 
      \approx 0.14 \text{ rad/s}
\]

\end{solution}
\begin{parts}
  \part What is the linear velocity of a horse 1 m from the center?

\begin{solution}
\[
  v = \omega r = (0.14 \text{ rad/s}) \cdot (1 \text{ m}) = 0.14 \text{ m/s}
\]
\end{solution}

  \part What is the linear velocity of a horse 3 m from the center?
\begin{solution}
\[
  v = \omega r = (0.14 \text{ rad/s}) \cdot (3 \text{ m}) = 0.42 \text{ m/s}
\]
\end{solution}

\end{parts}

\item A wheel of radius .1 m spins at 1,000 rpm.  What is the centripetal acceleration at the edge of the wheel?

\begin{solution}
The angular velocity in rad/s is:
\[
  \omega = \frac{1,000 \cancel{\text{ rev}}}{\cancel{\text{ min}}} 
      \cdot \frac{2 \pi \text{ rad}}{1 \cancel{ \text{ rev}}} \cdot \frac{1 \cancel{\text{ min}}}{60 \text{ s}} 
      \approx 104.7 \text{ rad/s}
\]

The acceleration is:
\[
  a_{cp} = r \omega^2 = (.1 \text{ m})(104.7 \text{ rad/s})^2 \approx 1,096 \text{ m/s}^2
\]

\end{solution}

\item If the wheels of a car have a radius of .3 m, what is their angular speed when the car is driven at 20 m/s (about 45 mph)?

\begin{solution}
If the car was stationary while the wheel continued rotating at the same speed, a point on the edge of the wheel would
move at 20 m/s.  To find the angular speed, convert the linear speed to angular speed:
\begin{align*}
  \omega &= \frac{v}{r} \\
  &= \frac{20 \text{ m/s}}{.3 \text{ m}} \\
  &\approx 66.7 \text{ rad/s} \\ 
\end{align*}

\end{solution}

\section{Extra Credit}

\question 

At 3:00, the angle between the minute hand and the hour hand of a clock is $90 \degree$.
When is the first time after 3:00 when the hands make a $45 \degree$ angle? (\em{James Walker})

\begin{solution}

The angular velocity of the hour hand is 
\[
  \omega_h = \frac{1}{12} \text{ rev/h} = \frac{1}{12} \cdot 360 \text{ degree/h} = 30 \text{ degree/h}  
\]

The angular velocity of the minute hand is 
\[
  \omega_m = 1 \text{ rev/h} = 360 \text{ degree/h}
\]

The initial positions are:
\begin{itemize*}
  \item $\omega_{h0} = 90 \degree$
  \item $\omega_{m0} = 0 \degree$
\end{itemize*}

So the angular position formulas are:
\begin{itemize*}
  \item $\theta_h = 90 + 30t \text{ degree}$
  \item $\theta_m = 360t \text{ degree}$
\end{itemize*}
where $t$ is in hours.

We want the difference to be $45 \degree$ 
\begin{align*}
  45 \degree &= \theta_h - \theta_t \\
  45 \degree &= 90\degree + 30t - 360t \\
  t &= 0.1364 \text{ h}
\end{align*}

0.1364 hours is about 8 minutes and 11 seconds, so at 3:08:11 the hands will make a $45 \degree$ angle.

\end{solution}

\end{questions}

\ifprintanswers

\pagebreak

\section{Chapter 8}
\begin{description}

\item[Q3]
The coin does have a rotational acceleration since its number of revolutions per second is increasing as it rolls down the incline.

\item[Q4]
It does have a rotational acceleration.  Acceleration happens whenever the angular velocity changes, and can be negative.

\item[Q5]
The rotational velocity is the same.  They both make the same number of revolutions in each minute.

\item[Q6]
The linear velocity is different.  The closer you are to the edge the farther you have to go in each revolution.

\item[Q8]
\begin{description*}
\item[a] The ball experiences both kinds of acceleration.  It is making more revolutions per second and moving faster as
  it progresses down the incline.

\item[b]
Each revolution the ball travels the length of its circumference, or $2 \pi r$.

\item[c]
As we discussed in class, the linear velocity is: $v = \omega r$.

\end{description*}

\item[E2]
\begin{description*}

\item[a]
\[
  \frac{45 \cancel{\text{ rev}}}{1 \cancel{\text{ m}}} \cdot \frac{1 \cancel{\text{ m}}}{60 \text{ s}} = 0.75 \text{ rev/s}
\]

\item[b]
\[
  0.75 \text{ rev/s} \cdot 5 \text{ s} = 3.75 \text{ rev}
\]

\end{description*}

\item[E3]
\begin{description*}

\item[a]
\[
  \theta = 3 \cancel{\text{ rev}} \cdot \frac{2 \pi \text{ rad}}{1 \cancel{\text{ rev}}} = 6 \pi \text{ rad} 
  \approx 18.85 \text{ rad}
\]

\item[b]

\[
  \omega = \frac{18.85 \text{ rad}}{4 \text{ s}} = 4.71 \text{ rad/s}
\]
 
\end{description*}

\item[E4]
\begin{align*}
  \Delta \omega &= 1.8 \text{ rad/s} - 0.6 \text{ rad/s} = 1.2 \text{ rad/s} \\
  \alpha &= \frac{1.2 \text{ rad/s}}{4 \text{ s}} = 0.3 \text{ rad/s} \\
\end{align*}

\item[E5]
\begin{description}
\item[a]
\[
  \omega = \alpha t = (1.2 \text{ rev/s})(4 \text{ s}) = 4.8 \text{ rev/s}
\]

%% \begin{align*}
%%   \alpha &= \frac{1.2 \cancel{\text{ rev}}}{\text{s}^2} \cdot \frac{2 \pi \text{ rad}}{1 \cancel{\text{ rev}}} 
%%       \approx 7.54 \text{ rad/s}^2 \\
%%   \omega &= \alpha t = (7.54 \text{ rad/s}^2) (4 \text{ s}) \approx 30.16 \text{ rad/s}
%% \end{align*}

\item[b]
\[
  \theta = \frac{1}{2} \alpha t^2 = \frac{1}{2} (4.8 \text{ rev/s}^2) (4 \text{ s})^2 = 9.6 \text{ rev}
\]

%% \[
%%   \theta = \frac{1}{2} \alpha t^2 = \frac{1}{2} (7.54 \text{ rad/s}^2) (4 \text{ s})^2 \approx 60.32 \text{ rad}
%% \]

\end{description}

\item[E6]
The change in angular velocity, measured in revolutions per second is: 
\[
  6 \text{ rev/s} - 3 \text{ rev/s} = 3 \text{rev/s}
\]

%% We need to convert this to rad/s:
%% \[
%%   (3 \text{ rev/s})(2 \pi \text{ rad/rev}) \approx 18.85 \text{ rad/s}
%% \]

The change occurs in 12 s, so the angular acceleration is:
\[
  \alpha = \frac{3 \text{ rev/s}}{12 \text{ s}} = 0.25 \text{ rev/s}^2
\]

\item[E7]
%% Convert the angular acceleration from rev/s to rad/s:
%% \[
%%   (0.2 \text{ rev/s}^2)(2 \pi \text{ rad/rev}) \approx 1.26 \text{ rad/s}^2
%% \]

\begin{description}
\item[a]
\[
  \omega = \alpha t = (0.2 \text{ rev/s}^2)(5 \text{ s}) = 1 \text{ rev/s}
\]

\item[b]
The angle the merry-go-round moves through is:
\[
  \theta = \frac{1}{2} \alpha t^2 = \frac{1}{2} (1 \text{ rev/s}^2) (5 \text{ s})^2 = 2.5 \text{ rev}
\]

\end{description}


\end{description}

\fi


\vspace{4.5 in}

\ifprintanswers
\else
\begin{verse}
crazy jay blue) \\
demon laughshriek \\
ing at me \\
your scorn of easily 

hatred of timid \\
\& loathing for(dull all \\
regular righteous \\
comfortable)unworlds 

thief crook cynic \\
(swimfloatdrifting \\
fragment of heaven) \\
trickstervillain 

raucous rogue \& \\
vivid voltaire \\
you beautiful anarchist \\
(i salute thee
\end{verse}

\vspace{.2 cm}
\hspace{1 cm} --e.e. cummings

\fi

\end{document}

