
\documentclass{exam}

\usepackage{graphicx}
\usepackage{siunitx}
\usepackage[fleqn]{amsmath}
\usepackage{cancel}
\usepackage{polynom}
\usepackage{float}
\usepackage{mdwlist}
\usepackage{booktabs}
\usepackage{cancel}
\usepackage{polynom}
\usepackage{caption}

\newcommand{\degree}{\ensuremath{^\circ}} 
\everymath{\displaystyle}

% \oddsidemargin .5in
% \topmargin -1in
\textwidth 6.5 in

\printanswers

\ifprintanswers 
\usepackage{2in1, lscape} 
\fi

\title{Physics \\ Homework Eight \\ Electricity}
\date{November 14, 2011}

\begin{document}

\maketitle

\section{Homework}

\begin{itemize*}
  \item Read Chapter 13
  \item Chapter 13
    \begin{itemize*}
      \item questions 7, 13-16, 22
      \item exercises 2-6, 9-10, 12-14, 16-17
      \item synthesis problems: 1-2, 4-5
    \end{itemize*}
\end{itemize*}

\section{Extra Credit}
\begin{questions}

\question
In your spare time you are making a radio and find to your dismay that you need a $150 \ \ohm$ resistor but only have
$220 \ \ohm$, $79 \ \ohm$, and $92 \ \ohm$ resistors available.  How can you connect the resistors you have to get the
desired resistance? (James Walker)

\begin{solution}
The $220 \ \ohm$ resistor is too large to be connected in series with either of the other two.  If we connect it in
parallel with the  $79 \ \ohm$ resistor, the equivalent resistance is:
\[
  R = \frac{220 \cdot 79}{220 + 79} \ \ohm \approx 58 \ \ohm
\]

If this combination is connected in series with the remaining resistor, the total is the desired $150 \ \ohm$.

\end{solution}

\question How many $65 \ \watt$ light bulbs can be connected in parallel across a potential difference of $85 \ \volt$
before the total current exceeds $2.1 \ \ampere$? (James Walker)

\begin{solution}
Since the bulbs are all the same, an equal current goes through each.  If $n$ is the number of bulbs, the current through one bulb is:
\[
  I_{one bulb} = \frac{I_{total}}{n}
\]

We can use the power consumption of one bulb and the current and solve for $n$:
\begin{align*}
  P &= \frac{VI}{n} \\
  n &= \frac{VI}{P} \\
    &= \frac{(85 \ \volt)(2.1 \ \ampere)}{65 \ \watt} \\
    &\approx 2.7 \\
\end{align*}

You can't have a fraction of a light bulb, so 2 is the largest number of bulbs that will work.

The current through one bulb is:
\[
  I = \frac{P}{V} = \frac{65 \ \watt}{85 \ \volt} \approx 0.76 \ \ampere
\]

The total current for two bulbs is $2 \cdot 0.76 \ \ampere = 1.52 \ \ampere$.  

\end{solution}

\end{questions}

\ifprintanswers

\section{Questions}

\begin{description}

\item[Q7]
Circuit b will work because there is a complete circuit which includes the batter and the light.  Circuit a doesn't
provide a route for the charge to both enter and return to the battery.

\item[Q13]
\begin{description}
\item[a]
The same current flows through both resistors.

\item[b]
Since $R_2$ is larger and $V = IR$, $R_2$ has the largest voltage difference.
\end{description}

\item[Q14]
All of the current flows through $R_3$ while only a part of the current flows through each of $R_1$ and $R_2$, so $R_3$
has the largest current.

\item[Q15]
$R_3$ is the largest resistance and it has the largest current, so it has the largest voltage difference.

\item[Q16]
The current will decrease because there is one less route for it to follow.

\item[Q22]
No, it will quadruple:
\[ 
  P = (2I)^2 R = 4I^2R
\]

\end{description}

\section{Exercises}

\begin{description}

\item[E2]
\[
  Q = I \Delta t = (2.5 \ \ampere)(60 \ \second) = 150 \ \coulomb
\]

\item[E3]
\[
  I = \frac{V}{R} = \frac{6 \ \volt}{24 \ \ohm} = 0.25 \ \ampere
\]

\item[E4]
\[
  V = \frac{I}{R} = \frac{1.5 \ \ampere}{18 \ \ohm} = 27 \ \volt
\]

\item[E5]
\[
  R = \frac{V}{I} = \frac{120 \ \volt}{0.6 \ \ampere} = 200 \ \ohm
\]

\item[E6]
\begin{description}
\item[a]
\[
  I = \frac{V}{R} = \frac{6 \ \volt}{60 \ \ohm} = 0.1 \ \ampere
\]

\item[b]
\[
  V = IR = (0.1 \ \ampere)(20 \ \ohm) = 2 \ \volt
\]
\end{description}

\item[E9]
\begin{description}
\item[a]
The current is the same through all the resistors:
\[
  I = \frac{V}{R} = \frac{6 \ \volt}{60 \ \ohm} = 0.1 \ \ampere
\]

\item[b]
yes

\item[c]
\[
  V = IR = (0.1 \ \ampere)(20 \ \ohm) = 2 \ \volt
\]

\end{description}

\item[E10]
\begin{align*}
  \frac{1}{R} &= \frac{1}{8 \ \ohm} + \frac{1}{8 \ \ohm} \\
  \frac{1}{R} &= \frac{1}{4 \ \ohm} \\
  R           &= 4 \ \ohm \\
\end{align*}


\item[E12]
\begin{description}
\item[a]
\begin{align*}
  \frac{1}{R} &= 3 \cdot \frac{1}{24 \ \ohm} \\
  \frac{1}{R} &= \frac{1}{8 \ \ohm} \\
  R           &= 8 \ \ohm \\
\end{align*}

\item[b]
\[
  I = \frac{V}{R} = \frac{12 \ \volt}{8 \ \ohm} = 1.5 \ \ampere
\]

\item[c]
\[
  I = \frac{12 \ \volt}{24 \ \ohm} = 0.5 \ \ampere
\]
This is consistent with $1.5 \ \ampere$ total, since 1/3 of the current goes through each resistor.

\end{description}

\item[E13]
\[
  P = VI = (9 \ \volt)(1.5 \ \ampere) = 13.5 \ \watt
\]

\item[14]
\begin{description}

\item[a]
\[
  I = \frac{V}{R} = \frac{3 \ \volt}{30 \ \ohm} = 0.1 \ \ampere
\]

\item[b]
\[
  P = VI = {3 \ \volt}{0.1 \ \ampere} = 0.3 \ \watt
\]

\end{description}

\item[16]

\begin{description}

\item[a]
\[
  P = VI = (110 \ \volt)(7 \ \ampere) = 770 \ \watt
\]

\item[b]
\[
  R = \frac{V}{I} = \frac{110 \ \volt}{7 \ \ampere} \approx 15.7 \ \ohm
\]

\end{description}

\item[E17]
\[
  I = \frac{P}{V} = \frac{5,500 \ \watt}{220 \ \volt} = 25 \ \ampere
\]
\end{description}

\section{Synthesis Problems}

\begin{description}
\item[SP1]

\begin{description}
\item[a]
\begin{align*}
  \frac{1}{R} &= \frac{1}{6 \ \ohm} + \frac{1}{12 \ \ohm} \\
  \frac{1}{R} &= \frac{1}{4 \ \ohm} \\
  R &= 4 \ \ohm \\
\end{align*}

\item[b]
\[
  I = \frac{V}{R} = \frac{1.5 \ \volt}{12 \ \ohm} = 0.125 \ \ampere
\]

\item[b]
\[
  I = \frac{V}{R} = \frac{1.5 \ \volt}{12 \ \ohm} = 0.125 \ \ampere
\]

\item[c]
The voltage over the $8 \ \ohm$ resistor is:
\[
  V = IR = (0.125 \ \ohm)(8 \ \ohm) = 1 \ \volt
\]

So the voltage over the other two resistors is $0.5 \ \volt$, which gives a current through the $6 \ \ohm$ resistor of:
\[
  I = \frac{V}{R} = \frac{0.5 \ \volt}{6 \ \ohm} \approx 0.833 \ \ampere
\]

The current through the $12 \ \ohm$ resistor is
\[
  I = \frac{V}{R} = \frac{0.5 \ \volt}{12 \ \ohm} \approx 0.417 \ \ampere
\]

Adding the two currents together gives $0.125 \ \ampere$, so all the current is accounted for.

\item[d]
\[
  P = I^2R = (0.125 \ \ampere)^2(8 \ \ohm) = 0.125 \ \watt
\]

\item[e]
Greater--see part c.

\end{description}

\item[SP2]
\item[a]
The effective resistance of the three resistors is:
\begin{align*}
  \frac{1}{R} &= \frac{3}{30 \ \ohm} \\
  R &= 10 \ \ohm \\
\end{align*}

The current is:
\[
  I = \frac{V}{R} = \frac{1.5 \ \volt}{10 \ \ohm} = 0.15 \ \ampere
\]

\item[b]
1/3 of the current goes through each bulb, so each bulb gets $0.05 \ \ampere$.  An alternate approach is:
\[
  I = \frac{V}{R} = \frac{1.5 \ \volt}{30 \ \ohm} = 0.05 \ \ampere
\]

\item[c] 
A bulb burning out has no effect on the other bulbs.  Each bulb still has the same voltage difference and the
same current it had before.  Another way to look at it is that there is less total current than there was, but there
are fewer bulbs to share the current, and the net effect is no change.

\item[SP4]
\begin{description}
\item[a]
Since each of the parallel combinations involves identical resistors, the equivalent resistance is just the resistance
of one resistor divided by the number of resistors.  So the first one is $1.5 \ \ohm$ and the second one is $1 \ \ohm$.

\item[b]
\[
  R = 1.5 \ \ohm + 3 \ \ohm + 1 \ \ohm = 5.5 \ \ohm
\]

\item[c]
\[
  I = \frac{V}{R} = \frac{6 \ \volt}{5.5 \ \ohm} \approx 1.1 \ \ampere
\]

\item[d]
Each resistor gets 1/3 of the current, or $0.36 \ \ampere$

\end{description}

\item[SP5]

\begin{description}
\item[a]
\begin{tabular}{lr}
\toprule
Appliance & Current \\
\midrule
toaster        & $5.2 \ \ampere$\\
iron           & $10.4 \ \ampere$\\
food processor & $4.3 \ \ampere$\\
\bottomrule
\end{tabular}

\item[b]
If the iron and toaster are turned together, the total current will be more than $15 \
\ampere$, which would be a problem.

\item[c]
\[
  R = \frac{V}{I} = \frac{115 \ \volt}{10.4 \ \ampere} \approx 11.1 \ \ohm
\]

\end{description}


\end{description}
\else

\vspace{2.5 in}

% When a sixth of the population of a nation which has undertaken to be the refuge of liberty are slaves, and a whole
% country is unjustly overrun and conquered by a foreign army, and subjected to military law, I think that it is not too
% soon for honest men to rebel and revolutionize. What makes this duty the more urgent is the fact that the country so
% overrun is not our own, but ours is the invading army


{\em 

Our inventions are wont to be pretty toys, which distract our attention from serious things. They are but improved
means to an unimproved end, an end which it was already but too easy to arrive at.  

} 

\vspace{.2 cm}

\hspace{1 cm} --Henry David Thoreau
\fi

\end{document}

