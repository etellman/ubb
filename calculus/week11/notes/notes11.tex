\documentclass[letterpaper, landscape]{exam}
\usepackage{2in1, lscape} 
\printanswers

\usepackage{units} 
\usepackage[fleqn]{amsmath}
\usepackage{float}
\usepackage{mdwlist}
\usepackage{booktabs}
\usepackage{caption}
\usepackage{fullpage}
\usepackage{enumerate}
\usepackage{graphicx}
\usepackage[justification=justified]{caption}

\setcounter{tocdepth}{1}
\everymath{\displaystyle}

\author{}
\date{\today}
\title{Calculus I \\ Week Eleven}

\begin{document}

  \maketitle
  \tableofcontents

  \section{Trigonometry Review} % (fold)
  
  \begin{itemize*}
    \item definitions of all functions using the unit circle
    \item radians vs. degrees
    \item triangle definitions
  \end{itemize*}

  \section{Useful Limits} % (fold)
  
  \subsection{$\lim_{\theta \to 0} \frac{\sin \theta}{\theta}$} % (fold)
  
  Draw a unit circle with a wedge in the first quadrant. 

  The area of the small triangle is: $A_{inner} = \frac{1}{2} \sin
    \theta \cos \theta$

  The area of the sector is:
    \begin{align*}
      \frac{\theta}{2 \pi \cdot 1} &= \frac{A_{sector}}{\pi \cdot 1^2} \\
      A_{sector} &= \frac{\theta}{2} \\
    \end{align*}

  The area of the outer triangle is $A_{outer} = \frac{1}{2} \tan \theta$

  \begin{align*}
    \frac{1}{2} \sin \theta \cos \theta < \frac{\theta}{2} < \frac{1}{2} \tan \theta \\
    \sin \theta \cos \theta < \theta < \tan \theta \\
    \sin \theta \cos \theta < \theta < \frac{\sin \theta}{\cos \theta} \\
    \cos \theta < \frac{\theta}{\sin \theta} < \frac{1}{\cos \theta} \\
  \end{align*}

  By the Squeeze Theorem:
  \[
    \lim_{\theta \to 0} \frac{\sin \theta}{\theta} = 1
  \]

  \subsection{$\lim_{\theta \to 0} \frac{1 - \cos \theta}{\theta} = 0$} % (fold)

  \subsubsection{Proof} % (fold)
  
  \begin{align*}
    \lim_{x \to 0} \frac{1 - \cos x}{x} 
      & = \lim_{x \to 0} \frac{1 - \cos^2 x}{x(1 + \cos x)} \\
      & = \lim_{x \to 0} \frac{\sin^2 x}{x(1 + \cos x)} \\
      & = \lim_{x \to 0} \left[ \frac{\sin x}{x} \cdot \frac{\sin x}{1 + \cos x} \right] \\
      & = \lim_{x \to 0} \frac{\sin x}{1 + \cos x} \\
      & = 0 \\
  \end{align*}

  % \subsubsection{Alternate Proof} % (fold)
  
  % \begin{align*}
  %   \sin \frac{x}{2}   & = \sqrt{\frac{1 - \cos x}{2}} \\
  %   \sin^2 \frac{x}{2} & = \frac{1 - \cos x}{2} \\
  %   1 - \cos x         & = 2 \sin^2 \frac{x}{2} \\
  %   \\
  %   \lim_{x \to 0} \frac{1 - \cos x}{x} 
  %     & = \lim_{x \to 0} \frac{2 \sin^2 x/2}{x} \\
  %     & = \lim_{x \to 0} \left[ \sin x/2 \cdot \frac{\sin x/2}{x/2}  \right] \\
  %     & = \lim_{x \to 0} \sin x/2 \cdot \lim_{x \to 0} \frac{\sin x/2}{x/2} \\
  %     & = \lim_{x \to 0} \sin x/2 \cdot 1 \\
  %     & = 0 \\
  % \end{align*}

  \section{Identities} % (fold)
  
  \begin{align*}
    \sin(x + y) &= \sin x \cos y + \cos x \sin y \\
    \cos(x + y) &= \cos x \cos y - \sin x \sin y \\
    \cos x &= \sin \left( x + \frac{\pi}{2} \right) \\
  \end{align*}

  \section{Derivatives} % (fold)
  
  \subsection{Sine} % (fold)
 
  \begin{align*}
    \frac{d}{dx} \sin x & = \lim_{h \to 0} \frac{\sin(x + h) - \sin x}{h} \\
                        & = \lim_{h \to 0} \frac{\sin x \cos h + \cos x \sin h - \sin x}{h} \\
                        & = \lim_{h \to 0} \left[ \frac{\sin x \cos h - \sin x}{h} 
                            + \frac{\cos x \sin h}{h} \right] \\
                        & = \lim_{h \to 0} \left[ \sin x \frac{\cos h - 1}{h} 
                            + \cos x \frac{\sin h}{h} \right] \\
                        & = \cos x \\
  \end{align*}
  
  \subsection{Cosine} % (fold)
  \begin{align*}
    \frac{d}{dx} \cos x & = \lim_{h \to 0} \frac{\cos (x + h) - \cos x}{h} \\
                        & = \lim_{h \to 0} \frac{\cos x \cos h - \sin x \sin h - \cos x}{h} \\
                        & = \lim_{h \to 0} \left[ \frac{\cos x \cos h - \cos x}{h} - \frac{\sin x \sin h}{h} \right] \\
                        & = \lim_{h \to 0} \left[ \frac{\cos x (\cos h - 1)}{h} - \frac{\sin x \sin h}{h} \right] \\
                        & = - \sin x \\
  \end{align*}

  \subsection{Tangent} % (fold)
  
  \begin{align*}
    \frac{d}{dx} \tan x & = \frac{d}{dx} \left( \frac{\sin x}{\cos x}  \right) \\
                        & = \frac{\cos x \cos x - \sin x (- \sin x)}{\cos^2 x} \\
                        & = \frac{1}{\cos^2 x} \\
                        & = \sec^2 x \\
  \end{align*}
  
  \subsection{Secant} % (fold)

  \begin{align*}
    \frac{d}{dx} \sec x & = \frac{d}{dx} \left( \frac{1}{\cos x} \right) \\
                        & = \frac{\cos x \cdot 0 - (- \sin x)}{\cos^2 x} \\
                        & = \frac{\sin x}{\cos^2 x} \\
                        & = \tan x \sec x \\
  \end{align*}

  \subsection{Cosecant} % (fold)

  \begin{align*}
    \frac{d}{dx} \csc x & = \frac{d}{dx} \left( \frac{1}{\sin x}  \right) \\
                        & = \frac{\sin x \cdot 0 - \cos x}{\sin^2 x} \\
                        & = - \frac{\cos x}{\sin^2 x} \\
                        & = - \cot x \csc x \\
  \end{align*}

  \subsection{Cotangent} % (fold)
  \begin{align*}
    \frac{d}{dx} \cot x & = \frac{d}{dx} \left( \frac{\cos x}{\sin x}  \right) \\
                        & = \frac{\sin x (- \sin x) - \cos^2 x }{\sin^2 x} \\
                        & = \frac{- \left( \sin^2 x + \cos^2 x \right)}{\sin^2 x} \\
                        & = - \frac{1}{\sin^2 x} \\
                        & = -\csc^2 x \\
  \end{align*}
  
\end{document}

