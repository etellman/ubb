\documentclass{article}

\usepackage{parskip} 
\usepackage{graphicx}
\usepackage[fleqn]{amsmath}
\usepackage{cancel}
\usepackage{float}
\usepackage{mdwlist}
\usepackage{booktabs}
\usepackage{cancel}
\usepackage{polynom}
\usepackage{caption}
\usepackage{fullpage}
\usepackage{comment}

\newcommand{\degree}{\ensuremath{^\circ}} 
\everymath{\displaystyle}

\excludecomment{comment}

\author{}
\date{\today}
\title{Math 141 \\ Final Notes}

\begin{document}

\maketitle

Notes received via email\dots

Yes, the Chapter 1 material mostly should be review from Math 90. And - I'd say there is less of it on the final than
was on the Midterm 1. So although the students should make sure they can solve and simplify, they shouldn't get too
worried about all of it. I'm happy to take specific questions about Ch 1 if you have them once you meet with them.

A Review List for the final (which is just the review lists from each of the 3 tests put together --because my students
just get each test's review list and then tell them the final is over all 3 tests' material).  If you or your students
have any questions about an item on the list, feel free to email and I can clarify or give an example.

Midterms 1, 2, 3 from a few years ago - I apologize I don't have the answers typed neatly as you do (impressive!). But,
if you/students want to see answers to check, I can scan in my old solutions and send you.  Tell the students that these
3 tests combined is like a fairly comprehensive practice final. Hopefully having all 3 will give the students plenty of
review since it's been several months for them!

Looking at your tests versus mine, I think the biggest difference is that we don't cover Chapter 3 in Stewart in Math
141 as that material is in Math 142 (rationals, polynomials, etc). So - the final will actually be easier than the
students are prepared for! :)  Let them know that the focus of the final they'll be taking is on linear functions,
quadratic functions, exponential functions and logarithmic functions.

\end{document}
