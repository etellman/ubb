\documentclass{exam}

\usepackage{units} 
\usepackage{graphicx}
\usepackage[fleqn]{amsmath}
\usepackage{cancel}
\usepackage{float}
\usepackage{mdwlist}
\usepackage{booktabs}
\usepackage{cancel}
\usepackage{polynom}
\usepackage{caption}
\usepackage{fullpage}
\usepackage{comment}
\usepackage{enumerate}
\usepackage{xfrac}

\newcommand{\dg}{\ensuremath{^\circ}} 
\everymath{\displaystyle}

% \printanswers
\excludecomment{comment}

\ifprintanswers 
  \usepackage{2in1, lscape} 
\fi

\author{}
\date{October 30, 2013}
\title{Math 142 \\ Homework Nine}

\begin{document}

  \maketitle

  \section{Homework}
  Section 6.3: 1-15, 21-26, 33-46, 52-54, 65-66

  \section{Extra Credit}
  Section 6.3: 57 and 58

  \ifprintanswers
    \begin{description}
      \item[57]
        The area of the entire circle is: 
        \[
          A_{circle} = \pi 2^2 = 4 \pi
        \]

        The sector is one third of the circle or: 
        \[
          A_{sector} = \frac{4 \pi}{3}
        \]

        The area of the triangle is: 
        \[
          A_{triangle} = \frac{1}{2} \cdot 2^2 \sin 120 \dg = \sqrt{3}
        \]

        The area of the shaded region is the difference:
        \[
          A_{shaded} = A_{sector} - A_{triangle} = \frac{4 \pi}{3} - \sqrt{3} \approx \boxed{ 2.457 }
        \]

      \item[58]
        The area of the entire circle is: 
        \[
          A_{circle} = \pi 12^2 = 144 \pi
        \]

        The sector is one third of the circle or: 
        \[
          A_{sector} = \frac{144 \pi}{3}
        \]

        The area of the triangle is: 
        \[
          A_{triangle} = \frac{1}{2} \cdot 12^2 \sin \frac{\pi}{3} = 36 \sqrt{3}
        \]

        The area of the unshaded region is the difference between the area of the sector and the area of the triangle:
        \[
          A_{unshaded} = A_{sector} - A_{triangle} = \frac{144 \pi}{3} - 36 \sqrt{3} \approx 88.44
        \]

        The area of the shaded region is the difference between the area of the circle and the area of the unshaded
        region:
        \[
          A_{shaded} = A_{circle} - A_{unshaded} = 144 \pi - 88.44 \approx \boxed{ 363.95 }
        \]
      \end{description}

  \fi

  \section{Review}
  Find the period and graph:

  \begin{enumerate}
    \item $y = \tan 2 \pi x$
    \item $y = 2 \sec 3x$
  \end{enumerate}

  \ifprintanswers
    \section{Section 6.3}

    \begin{description}

      \item[1] 
        \begin{enumerate}[(a)]
          \item $180 \dg - 150 \dg = \boxed{ 30 \dg }$
          \item $360 \dg - 330 \dg = \boxed{ 30 \dg }$
          \item $0 \dg - (-30 \dg) = \boxed{ 30 \dg }$
        \end{enumerate}

      \item[2] 
        \begin{enumerate}[(a)]
          \item $180 \dg - 120 \dg = \boxed{ 60 \dg }$
          \item $180 \dg - 210 \dg = \boxed{ 60 \dg }$
          \item $780 \dg - 720 \dg = \boxed{ 30 \dg }$
        \end{enumerate}

      \item[3] 
        \begin{enumerate}[(a)]
          \item $225 \dg - 180 \dg = \boxed{ 45 \dg }$
          \item $810 \dg - 720 \dg = \boxed{ 90 \dg }$
          \item $180 \dg - 105 \dg = \boxed{ 75 \dg }$
        \end{enumerate}

      \item[4] 
        \begin{enumerate}[(a)]
          \item $180 \dg - 99 \dg = \boxed{ 81 \dg }$
          \item $199 \dg - 180 \dg = \boxed{ 19 \dg }$
          \item $360 \dg - 359 \dg = \boxed{ 1 \dg }$
        \end{enumerate}

      \item[5] 
        \begin{enumerate}[(a)]
          \item $\frac{12 \pi}{4} - \frac{11 \pi}{4} = \boxed{ \frac{\pi}{4} }$
          \item $\frac{12 \pi}{6} - \frac{11 \pi}{6} = \boxed{ \frac{\pi}{6} }$
          \item $\frac{12 \pi}{3} - \frac{11 \pi}{3} = \boxed{ \frac{\pi}{3} }$
        \end{enumerate}

      \item[6] 
        \begin{enumerate}[(a)]
          \item $\frac{4 \pi}{3} - \frac{3 \pi}{3} = \boxed{ \frac{\pi}{3} }$
          \item $\frac{33 \pi}{4} - \frac{32 \pi}{4} = \boxed{ \frac{\pi}{4} }$
          \item $\frac{24 \pi}{6} - \frac{23 \pi}{6} = \boxed{ \frac{\pi}{6} }$
        \end{enumerate}

      \item[9]
        \[
          \sin 150 \dg = \sin 30 \dg = \boxed{ \frac{1}{2} }
        \]

      \item[10]
        \[
          \sin 225 \dg = - \sin 45 \dg = \boxed{ - \frac{\sqrt{2}}{2} }
        \]

      \item[11]
        \[
          \cos 135 \dg = - \sin 45 \dg = \boxed{ - \frac{\sqrt{2}}{2} }
        \]

      \item[12]
        \[
          \cos (-60 \dg) \dg = \cos 45 \dg = \boxed{ \frac{1}{2} }
        \]

      \item[13]
        \[
          \tan (-60 \dg) \dg = - \tan 60 \dg = \boxed{ - \sqrt{3} }
        \]

      \item[14]
        \[
          \sec (300 \dg) \dg = \sec 60 \dg = \boxed{ 2 }
        \]

      \item[15]
        \[
          \csc (-630 \dg) \dg = \csc 90 \dg = \boxed{ 1 }
        \]

      \item[21]
        \[
          \sin \frac{2 \pi}{3} = \sin \frac{\pi}{3} = \boxed{ \frac{\sqrt{3}}{2} }
        \]

      \item[22]
        \[
          \sin \frac{5 \pi}{3} = - \sin \frac{\pi}{3} = \boxed{ - \frac{\sqrt{3}}{2} }
        \]

      \item[23]
        \[
          \sin \frac{3 \pi}{2} = - \sin \frac{\pi}{2} = \boxed{ - 1 }
        \]

      \item[24]
        \[
          \cos \frac{7 \pi}{3} = \cos \frac{\pi}{3} = \boxed{ \frac{1}{2} }
        \]

      \item[25]
        \[
          \cos \frac{-7 \pi}{3} = \cos \frac{\pi}{3} = \boxed{ \frac{1}{2} }
        \]

      \item[26]
        \[
          \cos \frac{5 \pi}{6} = - \cos \frac{\pi}{3} = \boxed{ - \frac{\sqrt{3}}{2} }
        \]

      \item[33] III
      \item[34] IV
      \item[35] IV
      \item[36] II

      \item[37]
        \begin{align*}
          \tan^2 \theta + 1 & = \sec^2 \theta \\
          \tan^2 \theta + 1 & = \frac{1}{\cos^2 \theta} \\
          \tan^2 \theta     & = \frac{1}{\cos^2 \theta} - 1 \\
          \tan \theta       & = \boxed{ \sqrt{\frac{1}{\cos^2 \theta} - 1} } \\
        \end{align*}

      \item[38]
        \begin{align*}
          \cot^2 \theta + 1 & = \csc^2 \theta \\
          \cot^2 \theta + 1 & = \frac{1}{\sin^2 \theta} \\
          \cot^2 \theta     & = \frac{1}{\sin^2 \theta} - 1 \\
          \cot \theta       & = \boxed{ - \sqrt{ \frac{1}{\sin^2 \theta} - 1 } } \\
        \end{align*}

      \item[39]
        \begin{align*}
          \sin^2 \theta + \cos^2 \theta & = 1 \\
          \cos^2 \theta                 & = 1 - \sin^2 \theta \\
          \cos \theta                   & = \boxed{ \sqrt{ 1 - \sin^2 \theta } } \\
        \end{align*}

      \item[40]
        \begin{align*}
          \sin^2 \theta + \cos^2 \theta           & = 1 \\
          \sin^2 \theta + \frac{1}{\sec^2 \theta} & = 1 \\
          \frac{1}{\sec^2 \theta}                 & = 1 - \sin^2 \theta \\
          \sec^2 \theta                           & = \frac{1}{1 - \sin^2 \theta} \\
          \sec \theta                             & = \boxed{ \sqrt{ \frac{1}{1 - \sin^2 \theta} } } \\
        \end{align*}

      \item[41]
        \begin{align*}
          \sec^2 \theta & = 1 + \tan^2 \theta \\
          \sec \theta   & = \boxed{ - \sqrt{ 1 + \tan^2 \theta } } \\
        \end{align*}

      \item[42]
        \begin{align*}
          \csc^2 \theta & = 1 + \cot^2 \theta \\
          \csc \theta   & = \boxed{ - \sqrt{ 1 + \cot^2 \theta } } \\
        \end{align*}

      \item[43]
        \begin{tabular}[H]{cccccc}
          \toprule
          $\sin t$      & $\cos t$        & $\tan t$        & $\csc t$      & $\sec t$        & $\cot t$ \\
          \midrule
          $\frac{3}{5}$ & $- \frac{4}{5}$ & $- \frac{3}{4}$ & $\frac{5}{3}$ & $- \frac{5}{4}$ & $- \frac{4}{3}$ \\
          \bottomrule
        \end{tabular}

      \item[44]
        \begin{tabular}[H]{cccccc}
          \toprule
          $\sin t$                & $\cos t$         & $\tan t$              & $\csc t$                & $\sec t$        & $\cot t$ \\
          \midrule
          $-\frac{\sqrt{95}}{12}$ & $-\frac{7}{122}$ & $\frac{\sqrt{95}}{7}$ & $-\frac{12}{\sqrt{95}}$ & $-\frac{12}{7}$ & $\frac{7}{12}$ \\
          \bottomrule
        \end{tabular}

      \item[45]
        \begin{tabular}[H]{cccccc}
          \toprule
          $\sin t$       & $\cos t$      & $\tan t$       & $\csc t$       & $\sec t$      & $\cot t$ \\
          \midrule
          $-\frac{3}{5}$ & $\frac{4}{5}$ & $-\frac{3}{4}$ & $-\frac{5}{3}$ & $\frac{5}{4}$ & $-\frac{4}{3}$ \\
          \bottomrule
        \end{tabular}

      \item[46]
        \begin{tabular}[H]{cccccc}
          \toprule
          $\sin t$                & $\cos t$      & $\tan t$      & $\csc t$                 & $\sec t$ & $\cot t$ \\
          \midrule
          $-\frac{2 \sqrt{6}}{5}$ & $\frac{1}{5}$ & $-2 \sqrt{6}$ & $-\frac{5 \sqrt{6}}{12}$ & $5$      & $-\frac{\sqrt{6}}{12}$ \\
          \bottomrule
        \end{tabular}

      \item[52]
        \[
          A = \frac{1}{2} 7 \cdot 9 \sin 72 \dg \approx \boxed{ 29.96 }
        \]

      \item[53]
        \[
          A = \frac{1}{2} 10 \cdot 22 \sin 10 \dg \approx \boxed{ 19.10 }
        \]

      \item[54]
        \[
          A = \frac{1}{2} 10^2 \sin 60 \dg \approx \boxed{ 43.30 }
        \]

      \item[65]
        \begin{enumerate}[(a)]
          \item 
            \begin{tabular}[H]{lr}
              \toprule
              range  & $\unit[3.897]{ft}$ \\
              height & $\unit[0.5625]{ft}$ \\
              \bottomrule
            \end{tabular}

          \item 
            \begin{tabular}[H]{lr}
              \toprule
              range  & $\unit[23.98]{ft}$ \\
              height & $\unit[3.462]{ft}$ \\
              \bottomrule
            \end{tabular}
        \end{enumerate}

      \item[66]
        \[
          t = \sqrt{ \frac{2000}{16 \sin 30 \dg}} \approx \boxed{ \unit[15.81]{s} }
        \]

    \end{description}

  \else
    \vspace{7 cm}
    \begin{quote}
      \begin{em}
        The 20th century has been characterized by three developments of great political importance: The growth of
        democracy, the growth of corporate power, and the growth of corporate propaganda as a means of protecting
        corporate power against democracy
      \end{em}
    \end{quote}
    \hspace{1 cm} --Alex Carey
  \fi

\end{document}

