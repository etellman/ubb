\documentclass{exam}

\usepackage{units} 
\usepackage{graphicx}
\usepackage[fleqn]{amsmath}
\usepackage{cancel}
\usepackage{float}
\usepackage{mdwlist}
\usepackage{booktabs}
\usepackage{cancel}
\usepackage{polynom}
\usepackage{caption}
\usepackage{fullpage}
\usepackage{xfrac}
\usepackage{enumerate}

\newcommand{\dg}{\ensuremath{^\circ}} 
\everymath{\displaystyle}

\printanswers

\title{Math 142 Notes \\ Section 6.3}

\date{\today}

\begin{document}

  \maketitle
  \tableofcontents

  \section{Bonus Topics}
  \subsection{Imaginary Powers of e}

  for large $n$
  \begin{align*}
    e       & \approx \left( 1 + 1/n \right)^n \\
    \\
    e^{1/n} & \approx \left( 1 + 1/n \right)^{n/n} \\
            & \approx 1 + 1/n \\
    \\
    e^{i/1024} & \approx 1 + \frac{1}{1/1024} i \\
               & \approx 1 + 0.0009765625 i \\
    e^{i/512}  & = \left( e^{1/1024} \right)^2 \\
               & \approx 0.999998093 + 0.00195312376 i \\
    \vdots \\
    e^{i} &= 0.540302306 + 0.841470985 i \\
  \end{align*}

  Fill in graphs and plot and find that real part is cosine graph and imaginary part is sine graph:
  \begin{align*}
    e^{i \theta}  & = \cos \theta + i \sin \theta \\
    e^{\pi i}     & = -1 \\
    e^{\pi i} + 1 & = 0 \\
  \end{align*}

  Equation contains all important numbers: $e$, $i$, $\pi$, and $i$.

  \subsection{Eratosthenes Radius of Earth}

  \begin{enumerate}
    \item When the sun is directly overhead in Syene, in Alexandria there is a 1.2 foot shadow from a 10 foot pole.
    What is the angle to the sun?  Assume the sun is infinitely far away.

      \begin{solution}
        \begin{align*}
          \sin \theta &= \frac{1.2}{10} \\
          \theta &\approx 7 \dg \\
        \end{align*}
      \end{solution}

    \item If the distance between the two cities is 5000 stadia, what is the radius of the earth in stadia?
      \begin{solution}
        convert degrees to radians:
        \[
          7 \dg \approx 0.122
        \]

        Find the radius:
        \begin{align*}
          s & = r \theta \\
          r & = \frac{s}{\theta} \\
            & = \frac{5000}{0.122} \\
            & \approx \unit[41,000]{stadia} \\
        \end{align*}

        Actual stadium unit may have been 
        \begin{itemize*}
          \item 157 meters, leading to an estimate of 6400 km.  
          \item 185 meters, leading to an estimate of 7571 km.  
        \end{itemize*}

        The actual value for the radius is 6371 km.

        A flaw in the plan is that the cities aren't actually directly north of each other, so the apparent precision
        involved a bit of luck.

      \end{solution}
  \end{enumerate}

  \section{Trigonometric Functions of Angles}

  \subsection{Notes}
  \begin{itemize*}
    \item same as unit circle with larger radius
    \item reference angle determines value, quadrant determines sign
  \end{itemize*}

  \subsection{Examples}

  \begin{enumerate}
    \item find various reference angles in degrees and radians, positive and negative angles

    \item find values of sine, cosine, etc. of common angles ($30 \dg$, etc.)
      with angles greater than $90 \dg$, negative angles, etc.

    \item Given $\tan \theta = \frac{2}{3}$ or similar, find values of other trignometric functions.
  \end{enumerate}

  \section{Pythagorean Identities}

  \subsection{Notes}

  \begin{align*}
    x^2 + y^2                                                   & = r^2 \\
    \left( \frac{y}{r} \right)^2 + \left( \frac{x}{r} \right)^2 & = 1 \\
    \sin^2 \theta + \cos^2 \theta                               & = 1 \\
    \\
    \tan^2 \theta + 1 &= \sec^2 \theta \\
    \cot^2 \theta + 1 &= \csc^2 \theta \\
  \end{align*}

  \subsection{Examples}

  \begin{enumerate}
    \item $\cos \theta$ in terms of $\sin \theta$ Q-II
    \item $\tan \theta$ in terms of $\sin \theta$ Q-I
    \item $\sec \theta$ in terms of $\tan \theta$ Q-IV
    \item etc.
  \end{enumerate}

  \section{Area of Triangles}
  \subsection{Notes}

  \begin{itemize*}
    \item show that if $\theta$ is angle between $a$ and $b$, $A = \frac{1}{2} ab \sin \theta$ because $h = b \sin
      \theta$
    \item demonstrate when included angle is greater than or less than $90 \dg$
  \end{itemize*}

  \subsection{Examples}

  Find area of various triangles given two sides and included angle.

\end{document}
