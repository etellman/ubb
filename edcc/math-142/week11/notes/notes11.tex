\documentclass{exam}

\usepackage{units} 
\usepackage{graphicx}
\usepackage[fleqn]{amsmath}
\usepackage{cancel}
\usepackage{float}
\usepackage{mdwlist}
\usepackage{booktabs}
\usepackage{cancel}
\usepackage{polynom}
\usepackage{caption}
\usepackage{fullpage}
\usepackage{xfrac}
\usepackage{enumerate}

\newcommand{\dg}{\ensuremath{^\circ}} 
\everymath{\displaystyle}

\printanswers

\title{Math 142 Notes \\ Section 6.5}

\date{November 13, 2013}

\begin{document}

  \maketitle
  \tableofcontents

  \section{Homework 10}
  \begin{itemize}
    \item make sure reference angle is to closest x axis.  It should be always be between $0 \dg$ and $90 \dg$ or
      $0$ and $\pi$

    \item be careful about signs and quadrants
  \end{itemize}

  \section{Law of Cosines}

  \subsection{Definition and Proof}
  \begin{enumerate}
    \item Draw right triangle with $\angle A$ at the origin, hypotenuse labeled $b$, and top point labeled $C$.  Show that:
      \[
        C = (b \cos A, b \sin A)
      \]

    \item Draw second right triangle with common side at $h$, hypotenuse labeled $a$ and bottom side on the x axis.  The
      total of the two sides on the bottom is $c$.  The two triangles together make a new non-right triangle. 
      
      Show that:
      \[
        a^2 = \left( b \sin A \right)^2 + \left( c - b \cos A \right)^2 
      \]

    \item Simplify:
      \begin{align*}
        a^2 & = \left( b \sin A \right)^2 + \left( c - b \cos A \right)^2  \\
            & = b^2 \sin^2 A + c^2 - 2bc \cos A  + b^2 \cos^2 A \\
            & = b^2 \sin^2 A + c^2 - 2bc \cos A  + b^2 \cos^2 A \\
            & = b^2 + c^2 - 2bc \cos A \\
      \end{align*}

  \end{enumerate}
  
  \subsection{Notes}

  procedure:
  \begin{itemize*}
    \item Use the Law of Cosines to get a side
    \item Use the Law of Sines to find other sides.    
  \end{itemize*}

  Note that only one triangle fits with Law of Cosines--there is never ambiguity, but you have to be careful when you
  find the remaining sides using the Law of Sines.

  For example: $b = 3$, $c = 4$, $\angle A = 30 \dg$.  
  \begin{itemize*}
    \item Use Law of Cosines to find $a$

    \item Use Law of Sines to find $\angle C$.  The possibilities are $72 \dg$ and $180 - 72 \dg = 108 \dg$ and the
      second possibility is correct because it puts the largest angle across from the longest side.

    \item The final angle is $180 \dg - 30 \dg - 108 \dg = 42 \dg$.
  \end{itemize*}

  \subsection{Examples}
  \begin{enumerate}
    \item $a = 3$, $c = 4$, $\angle B = 100 \dg$

    \item $a = 5$, $c = 6$, $\angle B = 170 \dg$

  \end{enumerate}

  \subsection{SSS Case}
  The Law of Cosines can also be used to find an angle given all three sides:
  \begin{align*}
    a^2             & = b^2 + c^2 - 2bc \cos A \\
    a^2 - b^2 - c^2 & = - 2bc \cos A \\
    A               & = \cos^{-1} \left( \frac{b^2 + c^2 - a^2}{2bc} \right) \\
  \end{align*}

  Example: $a = 6$, $b = 2$, and $c = 5$

  \begin{itemize*}
    \item First solve for $\angle A$ first and use Law of Sines for others.
    \item Then solve for $\angle B$ first and use Law of Sines for others.  Note that you have to check both
      possibilities for ambiguous case for $\angle A$ because $\angle A$ is actually obtuse.
  \end{itemize*}

  \subsection{Examples}

  \begin{enumerate}

    \item $a = 4$, $c = 4$, $c = 5$

    \item $a = 8$, $c = 5$, $c = 6$

  \end{enumerate}

  \subsection{Word Problems from Other Books}

  \section{Bearing}
  \subsection{Notes}
  \begin{itemize*}
    \item N $30 \dg$ E
    \item S $15 \dg$ W
  \end{itemize*}

  Angle is always less than $90 \dg$.

  \subsection{Examples}
  \begin{enumerate}
    \item One plane flew due north at 500 mph.  A second plane left from the same airport and flew S $30 \dg$ E at 400
      mph.  How far apart are the planes after 3 hours?

    \item A plane flew north at 250 mph for 2 hours and then turned N $30 \dg$ W, slowed down to 200 mph, and flew for
      another hour.  How far is it from its starting point?

  \end{enumerate}

  \section{Heron's Formula for Area}

  \subsection{Formula}
  \begin{align*}
    A & = \sqrt{s (s - a)(s - b)(s - c)} \\
    s & = \frac{a + b + c}{2} \\
  \end{align*}

  \subsection{Examples}
  
  Find the area:
  \begin{enumerate}
    \item $a = 4$, $c = 7$, $c = 11$
    \item $a = 3$, $c = 2$, $c = 5$

    \item problem 61 from other book
  \end{enumerate}

\end{document}
