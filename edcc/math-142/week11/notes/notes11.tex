\documentclass{exam}

\usepackage{units} 
\usepackage{graphicx}
\usepackage[fleqn]{amsmath}
\usepackage{cancel}
\usepackage{float}
\usepackage{mdwlist}
\usepackage{booktabs}
\usepackage{cancel}
\usepackage{polynom}
\usepackage{caption}
\usepackage{fullpage}
\usepackage{xfrac}
\usepackage{enumerate}

\newcommand{\degree}{\ensuremath{^\circ}} 
\everymath{\displaystyle}

\printanswers

\title{Math 142 Notes \\ Section 6.5}

\date{November 13, 2013}

\begin{document}

  \maketitle
  \tableofcontents

  \section{Law of Cosines}

  \subsection{Proof}
  \begin{enumerate}
    \item Draw right triangle with $\angle A$ at the origin, hypotenuse labeled $b$, and top point labeled $C$.  Show that:
      \[
        C = (b \cos A, b \sin A)
      \]

    \item Draw second right triangle with common side at $h$, hypotenuse labeled $a$ and bottom side on the x axis.  The
      total of the two sides on the bottom is $c$.  The two triangles together make a new non-right triangle. 
      
      Show that:
      \[
        a^2 = \left( b \sin A \right)^2 + \left( c - b \cos A \right)^2 
      \]

    \item Simplify:
      \begin{align*}
        a^2 & = \left( b \sin A \right)^2 + \left( c - b \cos A \right)^2  \\
            & = b^2 \sin^2 A + c^2 - 2bc \cos A  + b^2 \cos^2 A \\
            & = b^2 \sin^2 A + c^2 - 2bc \cos A  + b^2 \cos^2 A \\
            & = b^2 + c^2 - 2bc \cos A \\
      \end{align*}

  \end{enumerate}

  The Law of Cosines can also be used to find an angle given all three sides:
  \begin{align*}
    a^2             & = b^2 + c^2 - 2bc \cos A \\
    a^2 - b^2 - c^2 & = - 2bc \cos A \\
    A               & = \cos^{-1} \left( \frac{b^2 + c^2 - a^2}{2bc} \right) \\
  \end{align*}

\end{document}
