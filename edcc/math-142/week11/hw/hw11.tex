\documentclass{exam}

\usepackage{units} 
\usepackage{graphicx}
\usepackage[fleqn]{amsmath}
\usepackage{cancel}
\usepackage{float}
\usepackage{mdwlist}
\usepackage{booktabs}
\usepackage{cancel}
\usepackage{polynom}
\usepackage{caption}
\usepackage{fullpage}
\usepackage{comment}
\usepackage{enumerate}
\usepackage{xfrac}

\newcommand{\dg}{\ensuremath{^\circ}} 
\everymath{\displaystyle}

\printanswers
\excludecomment{comment}

\ifprintanswers 
  \usepackage{2in1, lscape} 
\fi

\author{}
\date{\today}
\title{Math 142 \\ Homework Eleven}

\begin{document}

  \maketitle

  \section{Homework}
  Section 6.5: 

  \section{Extra Credit}
  Section 6.5: 35 and 36

  \ifprintanswers

    \section{Section 6.5}

    \begin{description}

      \item[1] 
        \begin{align*}
          x^2 & = 21^2 + 42^2 - 2 \cdot 21 \cdot 42 \cdot \cos 39 \dg \\
          x   & \approx \boxed{ 28.88 } \\
        \end{align*}

      \item[2] 
        \begin{align*}
          x^2 & = 15^2 + 18^2 - 2 \cdot 15 \cdot 18 \cdot \cos 108 \dg \\
          x   & \approx \boxed{ 26.76 } \\
        \end{align*}

      \item[3] 
        \begin{align*}
          x^2 & = 25^2 + 25^2 - 2 \cdot 25 \cdot 25 \cdot \cos 140 \dg \\
          x   & \approx \boxed{ 46.98 } \\
        \end{align*}

      \item[4] 
        \begin{align*}
          x^2 & = 8^2 + 2^2 - 2 \cdot 8 \cdot 2 \cdot \cos 88 \dg \\
          x   & \approx \boxed{ 8.18 } \\
        \end{align*}

      \item[9] 
        \begin{align*}
          c^2 & = 10^2 + 18^2 - 2 \cdot 10 \cdot 18 \cdot \cos 120 \dg \\
          c   & \approx \boxed{ 24.58 } \\
          \\
          \cos A & \approx \frac{ 10^2 + 24.58^2 - 18^2}{2 \cdot 10 \cdot 24.58} \\
          A      & \approx \boxed{ 20.6 \dg }
          \\
          B & \approx 180 \dg - 120 \dg - 20.6 \dg \\
            & = \boxed{ 39.4 \dg } \\
        \end{align*}

      \item[10] 
        \begin{align*}
          \cos A & \approx \frac{ 12^2 + 44^2 - 12^2}{2 \cdot 12 \cdot 44} \\
          A      & \approx \boxed{ 63 \dg }
          \\
          \cos B & \approx \frac{ 40^2 + 44^2 - 12^2}{2 \cdot 40 \cdot 44} \\
          B      & \approx \boxed{ 15.5 \dg }
          \\
          C & \approx 180 \dg - 63 \dg - 15.5 \dg \\
            & = \boxed{ 101.5 \dg } \\
        \end{align*}

      \item[11]
        \begin{align*}
          c        & \approx 3.25 \\
          \angle A & \approx 47.5 \dg \\
          \angle B & \approx 79.5 \dg \\
        \end{align*}

      \item[12]
        \begin{align*}
          a        & \approx 57.17 \\
          \angle B & \approx 80.46 \dg \\
          \angle C & \approx 29.54 \dg \\
        \end{align*}

      \item[13]
        \begin{align*}
          \angle A & \approx 49.87 \dg \\
          \angle B & \approx 72.89 \dg \\
          \angle C & \approx 57.25 \dg \\
        \end{align*}

      \item[14]
        \begin{align*}
          \angle A & \approx 38.62 \dg \\
          \angle B & \approx 48.51 \dg \\
          \angle C & \approx 92.87 \dg \\
        \end{align*}

      \item[15]
        \begin{align*}
          b        & \approx 104.1 \\
          \angle B & \approx 50.5 \dg \\
          \angle C & \approx 89.5 \dg \\
        \end{align*}

      \item[16]
        \begin{align*}
          b        & \approx 52.18 \\
          \angle A & \approx 78.97 \dg \\
          \angle B & \approx 49.03 \dg \\
        \end{align*}

      \item[17]
        \begin{align*}
          c        & \approx 54.74 \\
          \angle B & \approx 76.57 \dg \\
          \angle C & \approx 55 \dg \\
        \end{align*}

      \item[18]
        Find the third angle: $A = 180 \dg - 61 \dg - 83 \dg = 36 \dg$

        Use the Law of Sines to find the missing sides:
        \begin{align*}
          \frac{\sin 36 \dg}{73.5} & = \frac{\sin 83 \dg}{b} \\
          b                        & \approx 124.1 \\
          \\
          \frac{\sin 36 \dg}{73.5} & = \frac{\sin 61 \dg}{c} \\
          c                        & \approx 109.4 \\
        \end{align*}

      \item[19]
        Find the third angle: $B = 180 \dg - 85 \dg - 35 \dg = 60 \dg$

        Use the Law of Sines:
        \begin{align*}
          \frac{\sin 60 \dg}{3} & = \frac{\sin 35 \dg}{x} \\
          x                        & \approx \boxed{ 1.99 } \\
        \end{align*}

      \item[20]
        Use the Law of Cosines:
        \begin{align*}
          x^2 & = 10^2 + 18^2 - 2 \cdot 10 \cdot 18 \cdot \cos 40 \dg \\
              & \approx \boxed{ 25.77 } \\
        \end{align*}

      \item[21]
        Use the Law of Sines:
        \begin{align*}
          \frac{\sin 100 \dg}{50} & = \frac{\sin 30 \dg}{x} \\
          x                       & \approx \boxed{ 25.39 } \\
        \end{align*}

      \item[27]
        \begin{align*}
          s  & = \frac{1}{2}(9 + 12 + 15) \\
             & = 18 \\
          \\
          A  & = \sqrt{18 (18 - 9)(18 - 12) (18 - 15)}
             & = 54 \\
        \end{align*}

      \item[28]
        \begin{align*}
          s  & = \frac{1}{2}(1 + 2 + 2) \\
             & = 2.5 \\
          \\
          A  & = \sqrt{2.5 (2.5 - 1)(2.5 - 2) (2.5 - 2)}
             & \approx \boxed{ 0.9682 } \\
        \end{align*}

      \item[29]
        \begin{align*}
          s  & = \frac{1}{2}(7 + 8 + 9) \\
             & = 12 \\
          \\
          A  & = \sqrt{12 (12 - 7)(12 - 8) (12 - 9)}
             & \approx \boxed{ 26.83 } \\
        \end{align*}

      \item[30]
        \begin{align*}
          s  & = \frac{1}{2}(11 + 100 + 101) \\
             & = 106 \\
          \\
          A  & = \sqrt{106 (106 - 11)(106 - 100) (106 - 101)}
             & \approx \boxed{ 549.64 } \\
        \end{align*}

      \item[31]
        \begin{align*}
          s  & = \frac{1}{2}(4 + 3 + 6) \\
             & = 6.5 \\
          \\
          A  & = \sqrt{6.5 (6.5 - 4)(6.5 - 3) (6.5 - 6)}
             & \approx \boxed{ 5.33 } \\
        \end{align*}

      \item[32]
        Find the area of one of the two identical triangles:
        \begin{align*}
          s  & = \frac{1}{2}(5 + 5 + 2) \\
             & = 6 \\
          \\
          A  & = \sqrt{6 (6 - 5)(6 - 5) (6 - 2)}
             & \approx 4.899 \\
        \end{align*}

        Since there are two triangles, the total area is twice the area of one triangle: 
        \[
          A_{total} = 2 \cdot 4.99 \approx \boxed{ 9.798 }
        \]

      \item[33]
        Draw a diagonal from left to right in the figure to form two triangles.  Then use the Law of Cosines to find the
        length of the diagonal:
        \begin{align*}
          a^2 & = 5^2 + 6^2 - 2 \cdot 5 \cdot 6 \cdot \cos 100 \dg \\
          a   & \approx 8.45 \\
        \end{align*}

        Find the area of the top triangle:
        \begin{align*}
          s_1 & = \frac{1}{2}(5 + 6 + 8.45) \\
              & = 9.725 \\
          \\
          A_1 & = \sqrt{9.725 (9.725 - 5)(9.725 - 6) (9.725 - 8.45)}
              & \approx \boxed{ 14.77 } \\
        \end{align*}

        Find the area of the bottom triangle:
        \begin{align*}
          s_1 & = \frac{1}{2}(8 + 7 + 8.45) \\
              & = 11.72 \\
          \\
          A_2 & = \sqrt{11.72 (11.72 - 8)(11.72 - 7) (11.72 - 8.45)}
              & \approx 26.00 \\
        \end{align*}

        The total area is: 
        \[
          A_{total} \approx 14.77 + 26.00 = \boxed{ 40.77 }
        \]

      \item[34]
        If you split the $60 \dg$ angle in half, you get two identical triangles which each contain a $30 \dg$ angle.

        Use the Law of Sines to find another angle:
        \begin{align*}
          \frac{\sin 30 \dg}{3} & = \frac{\sin A}{4} \\
          A                     & \approx 41.82 \dg \\
        \end{align*}

        Since $\angle A > 90 \dg$ in the figure, $A = 90 \dg + 41.82 \dg = 131.82 \dg$.

        The remaining angle is:
        \[
          B = 180 \dg - 131.82 \dg - 30 \dg = 18.18 \dg
        \]

        Use the Law of Sines to find the missing side:
        \begin{align*}
          \frac{\sin 30 \dg}{3} & = \frac{\sin 18.18}{b} \\
          b                     & \approx 1.872 \\
        \end{align*}

        Find the area of one triangle:
        \begin{align*}
          s  & = \frac{1}{2}(4 + 3 + 1.872) \\
             & = 4.436 \\
          \\
          A  & = \sqrt{4.436 (4.436 - 4)(4.436 - 3) (4.436 - 1.872)}
             & \approx 2.67 \\
        \end{align*}

        Multiply by two to get the total area:
        \[
          A_{total} \approx 2 \cdot 2.67 = \boxed{ 5.337 }
        \]

      \item[37]
        Use the Law of Cosines:
        \begin{align*}
          c^2 & = 2.82^2 + 3.56^2 - 2 \cdot 2.82 \cdot 3.56 \cdot \cos 40.3 \dg \\
          c   & \approx \boxed{ \unit[2.30]{mi} } \\
        \end{align*}

      \item[39]
        First find out how far each car has travelled in the 30 minutes they've been driving:
        \begin{align*}
          d_1 &= 0.5 \cdot 50 = \unit[25]{mi} \\
          d_2 &= 0.5 \cdot 30 = \unit[15]{mi} \\
        \end{align*}

        Use the Law of Cosines to figure out how far apart they are:
        \begin{align*}
          d^2 & = 25^2 + 15^2 - 2 \cdot 25 \cdot 15 \cdot \cos 65 \dg \\
          d   & \approx \boxed{ \unit[23]{mi} } \\
        \end{align*}
    \end{description}

  \else
    \vspace{1 cm}
    \begin{quote}
      \begin{em}
        TO DO
      \end{em}
    \end{quote}
    \hspace{1 cm} --Source
  \fi

\end{document}

