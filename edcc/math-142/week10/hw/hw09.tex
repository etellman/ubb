\documentclass{exam}

\usepackage{units} 
\usepackage{graphicx}
\usepackage[fleqn]{amsmath}
\usepackage{cancel}
\usepackage{float}
\usepackage{mdwlist}
\usepackage{booktabs}
\usepackage{cancel}
\usepackage{polynom}
\usepackage{caption}
\usepackage{fullpage}
\usepackage{comment}
\usepackage{enumerate}
\usepackage{xfrac}

\newcommand{\dg}{\ensuremath{^\circ}} 
\everymath{\displaystyle}

\printanswers
\excludecomment{comment}

\ifprintanswers 
  \usepackage{2in1, lscape} 
\fi

\author{}
\date{\today}
\title{Math 142 \\ Homework Nine}

\begin{document}

  \maketitle

  \section{Homework}
  Section 6.4: 

  \section{Extra Credit}
  Section 6.4: 29 and 30

  \ifprintanswers

    \section{Section 6.4}

    \begin{description}

      \item[1] 
        \begin{align*}
          \frac{\sin 98.4 \dg}{376} & = \frac{\sin 57 \dg}{x} \\
          x                         & \approx \boxed{ 318.8 } \\
        \end{align*}

      \item[2] 
        \begin{align*}
          \frac{\sin 37.5 \dg}{17} & = \frac{\sin 114.4 \dg}{x} \\
          x                        & \approx \boxed{ 25.43 } \\
        \end{align*}

      \item[3] 
        \begin{align*}
          \frac{\sin 58 \dg}{26.7} & = \frac{\sin 52 \dg}{x} \\
          x                         & \approx \boxed{ 24.81 } \\
        \end{align*}

      \item[4] 
        \begin{align*}
          \frac{\sin 67 \dg}{80.2} & = \frac{\sin \theta \dg}{56.3} \\
          \sin \theta              & \approx 0.6462 \\
          \theta                   & \approx \boxed{ 40.25 \dg } \\
        \end{align*}

      \item[5] 
        \begin{align*}
          \frac{\sin 120 \dg}{45} & = \frac{\sin \theta \dg}{36} \\
          \sin \theta             & \approx 0.6928 \\
          \theta                  & \approx \boxed{ 43.85 \dg } \\
        \end{align*}

      \item[6] 
        \begin{align*}
          \frac{\sin 102 \dg}{185} & = \frac{\sin 50 \dg}{x} \\
          x                        & \approx \boxed{ 144.9 } \\
        \end{align*}

      \item[7] 
        \begin{align*}
          \frac{\sin 114 \dg}{65} & = \frac{\sin 20 \dg}{b} \\
          b                       & \approx \boxed{ 24.33 } \\
          \\
          \frac{\sin 114 \dg}{65} & = \frac{\sin 46 \dg}{a} \\
          a                       & \approx \boxed{ 51.18 } \\
        \end{align*}

      \item[8] 
        \begin{align*}
          \frac{\sin 50 \dg}{2} & = \frac{\sin 30 \dg}{a} \\
          a                       & \approx \boxed{ 1.305 } \\
          \\
          \frac{\sin 50 \dg}{2} & = \frac{\sin 100 \dg}{c} \\
          c                       & \approx \boxed{ 2.571 } \\
        \end{align*}

      \item[11] 
        \begin{align*}
          \frac{\sin 62 \dg}{230} & = \frac{\sin 50 \dg}{a} \\
          a                       & \approx \boxed{ 199.5 } \\
          \\
          \frac{\sin 62 \dg}{230} & = \frac{\sin 68 \dg}{b} \\
          b                       & \approx \boxed{ 231.5 } \\
        \end{align*}

      \item[12] 
        \begin{align*}
          \frac{\sin 47 \dg}{50} & = \frac{\sin 23 \dg}{a} \\
          a                      & \approx \boxed{ 63.39 } \\
          \\
          \frac{\sin 47 \dg}{50} & = \frac{\sin 110 \dg}{b} \\
          b                       & \approx \boxed{ 64.24 } \\
        \end{align*}

      \item[17] 
        \begin{align*}
          \frac{\sin 110 \dg}{28} & = \frac{\sin B}{15} \\
          B                       & \approx \boxed{ 30 \dg } \\
          \\
          C & = 180 \dg - 110 \dg - 30 \dg \\
            & = \boxed{ 40 \dg } \\
          \\
          \frac{\sin 110 \dg}{28} & = \frac{\sin 40 \dg}{c} \\
          c                       & \approx \boxed{ 19.1 } \\
        \end{align*}

      \item[18] 
        \begin{align*}
          \frac{\sin 37 \dg}{30} & = \frac{\sin C}{40} \\
          C                      & \approx \boxed{ 53 \dg } \\
          \\
          B & = 180 \dg - 37 \dg - 53 \dg \\
            & = \boxed{ 90 \dg } \\
          \\
          \frac{\sin 37 \dg}{30} & = \frac{\sin 90 \dg}{c} \\
          c                      & \approx \boxed{ 49.8 } \\
        \end{align*}

      \item[19] 
        \begin{align*}
          \frac{\sin 125 \dg}{20} & = \frac{\sin C}{45} \\
          \sin C                  & \approx 1.84 \\
        \end{align*}

        No solution

      \item[20] 
        \begin{align*}
          \frac{\sin 38 \dg}{42} & = \frac{\sin B}{45} \\
          B                      & \approx \boxed{ 41 \dg } \\
          \\
          C & = 180 \dg - 38 \dg - 41 \dg 
            & = \boxed{ 101 \dg } \\
          \\
          \frac{\sin 38 \dg}{42} & = \frac{\sin 101 \dg}{c} \\
          c                      & \approx \boxed{ 67.0 } \\
        \end{align*}

      \item[21] 
        solution 1:
        \begin{align*}
          \frac{\sin 25 \dg}{25} & = \frac{\sin C}{30} \\
          C                      & \approx \boxed{ 30 \dg } \\
          \\
          B & = 180 \dg - 25 \dg - 30 \dg \\
            & = \boxed{ 125 \dg } \\
          \\
          \frac{\sin 25 \dg}{25} & = \frac{\sin 125}{b} \\
          b                      & \approx \boxed{ 48.4 } \\
        \end{align*}

        solution 2:
        \begin{align*}
          \frac{\sin 25 \dg}{25} & = \frac{\sin C}{30} \\
          C                      & \approx \boxed{ 150 \dg } \\
          \\
          B & = 180 \dg - 25 \dg - 150 \dg \\
            & = \boxed{ 5 \dg } \\
          \\
          \frac{\sin 25 \dg}{25} & = \frac{\sin 5 \dg}{b} \\
          b                      & \approx \boxed{ 5.1 } \\
        \end{align*}

      \item[22] 
        solution 1:
        \begin{align*}
          \frac{\sin 30 \dg}{75} & = \frac{\sin B}{100} \\
          B                      & \approx \boxed{ 42 \dg } \\
          \\
          C & = 180 \dg - 30 \dg - 42 \dg \\
            & = \boxed{ 108 \dg } \\
          \\
          \frac{\sin 30 \dg}{75} & = \frac{\sin 108 \dg}{c} \\
          c                      & \approx \boxed{ 142.7 } \\
        \end{align*}

        solution 2:
        \begin{align*}
          \frac{\sin 30 \dg}{75} & = \frac{\sin B}{100} \\
          B                      & 180 \dg - 42 \dg \\
                                 & \approx \boxed{ 138 \dg } \\
          \\
          C & = 180 \dg - 30 \dg - 138 \dg \\
            & = \boxed{ 12 \dg } \\
          \\
          \frac{\sin 30 \dg}{75} & = \frac{\sin 12 \dg}{c} \\
          c                      & \approx \boxed{ 31.2 } \\
        \end{align*}

      \item[27] 
        Find $\angle CDA$ and $\angle BDC$
        \begin{align*}
          \frac{\sin 30 \dg}{20} & = \frac{\sin CDA}{28} \\
          \angle CDA             & \approx 180 \dg - 44 \dg \\
                                 & = 136 \dg \\
          \angle BDC             & = 44 \dg \\
        \end{align*}

        Use the Law of Sines again to find $\angle CBD$
        \begin{align*}
          \frac{\sin 44 \dg}{20} & = \frac{\sin CBD}{20} \\
          \angle CBD             & \approx 44 \dg \\
        \end{align*}

        The missing angles are the third angles in the two small triangles.  You can find them by knowing that the sum
        of the angles in a triangle is $180 \dg$:
        \begin{align*}
          \angle BCD &= 180 \dg - 44 \dg - 44 \dg = \boxed{ 92 \dg } \\
          \angle DCA &= 180 \dg - 136 \dg - 30 \dg = \boxed{ 14 \dg } \\
        \end{align*}

      \item[28] 
        Since the smaller triangle on the left is an isosceles triangle, $\angle BCD = \angle CBD = 25 \dg$ 

        The third angle in this triangle is: $\angle BDC = 180 \dg - 25 \dg - 25 \dg = 130 \dg$

        Use the Law of Sines to find $BC$:
        \begin{align*}
          \frac{\sin 130 \dg}{BC} & = \frac{\sin 25 \dg}{12} \\
          \angle BC               & \approx 21.8 \\
        \end{align*}

        The third angle in the large triangle is: $\angle BAC = 180 \dg - 25 \dg - 50 \dg = 105 \dg$. 
        
        Use the Law of Sines to find $AB$:
        \begin{align*}
          \frac{\sin 105 \dg}{21.8} & = \frac{\sin 50 \dg}{AB} \\
          \angle AB                 & \approx 17.3 \\
        \end{align*}

        Now we can find the missing length:
        \[
          DA = AB - BD = 17.3 - 12 = \boxed{ 5.3 }
        \]

    \end{description}

  \else
    \vspace{1 cm}
    \begin{quote}
      \begin{em}
        TO DO
      \end{em}
    \end{quote}
    \hspace{1 cm} --Source
  \fi

\end{document}

