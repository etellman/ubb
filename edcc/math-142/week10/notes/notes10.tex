\documentclass{exam}

\usepackage{units} 
\usepackage{graphicx}
\usepackage[fleqn]{amsmath}
\usepackage{cancel}
\usepackage{float}
\usepackage{mdwlist}
\usepackage{booktabs}
\usepackage{cancel}
\usepackage{polynom}
\usepackage{caption}
\usepackage{fullpage}
\usepackage{xfrac}
\usepackage{enumerate}

\newcommand{\dg}{\ensuremath{^\circ}} 
\everymath{\displaystyle}

\printanswers

\title{Math 142 Notes \\ Section 6.3}

\date{\today}

\begin{document}

  \maketitle
  \tableofcontents

  \section{Homework Eight Notes}
  \begin{itemize*}
    \item OK to leave things as $2 \sqrt{13}$, etc.
    \item round, don't truncate
    \item angle of depression (problems 56 and 57)
    \item do both extra credit problems
    \item do second review problem (bicycle wheel)
  \end{itemize*}

  \section{Solving Triangles}

  As long as you have three sides/angles with at least one side, you can find all the other sides/angles.

  \subsection{Right Triangle Review}

  With right triangles, you have one angle already, so you need either:
  \begin{itemize*}
    \item 2 sides
    \item 1 side and one more angle
  \end{itemize*}

  Do examples of both

  \subsection{Cases and Terminology}

  \begin{itemize*}
    \item ASA, SAA, SSA, SAS, SSS
    \item upper case for angles, lower case for sides
  \end{itemize*}

  Law of Sines is good for ASA, SAA, and SSA cases.  Law of Cosines (next section) covers the remaining cases.

  \section{Law of Sines}

  \subsection{Definition and Proof}

  \[
    \frac{1}{2} ab \sin C = \frac{1}{2} bc \sin A = \frac{1}{2} ac \sin B \\
  \]

  Multiply by $\frac{2}{abc}$ to get the Law of Sines.

  \subsection{ASA and SAA Examples}
  Do various examples with ASA and SAA.

  \section{SSA/Ambiguous Case}
  With SSA, there will be zero, one (right), or two triangles that work.

  \subsection{No Solution}
  When the side isn't long enough to complete the triangle, there is no solution.

  For example: $\angle A = 60$, $a = 5$, $b = 10$:

  \begin{align*}
    \frac{\sin 60 \dg}{5} & = \frac{\sin B}{10} \\
    \frac{\sqrt{3}}{10}   & = \frac{\sin B}{10} \\
    \sin B                & = \sqrt{3} \\
  \end{align*}

  You can detect this situation because you'll get a value for sine greater than 1 or less than $-1$.

  \subsection{One Right Triangle Solution}
  For example: $\angle A = 60$, $a = 5 \sqrt{3}$, $b = 10$:

  \begin{align*}
    \frac{\sin 60 \dg}{5 \sqrt{3}} & = \frac{\sin B}{10} \\
    \sin B                         & = 1 \\
    B                              & = 90 \dg \\
    \\
    (5 \sqrt{3})^2 + c^2           & = 10^2 \\
    c^2                            & = 25 \\
    c                              & = 5 \\
    \\
    C                              & = 30 \dg \\
  \end{align*}

  Verify with other right triangle approaches.

  \subsection{Two Solutions}
  For example: $\angle A = 60$, $a = 9$, $b = 10$:

  \begin{align*}
    \frac{\sin 60 \dg}{9} & = \frac{\sin B}{10} \\
    \sin B                & \approx 0.96225\\
    B                     & \approx 74.21 \dg \\
    \\
    C                     & \approx 180 \dg - 60 \dg - 74.21 \dg \\
                          & = 45.79 \dg \\
    \\
    \frac{\sin 60 \dg}{9} & = \frac{\sin 45.79 \dg}{c} \\
    c                     & \approx 7.3485 \\
  \end{align*}

  Other possibility for $\angle B$ is $180 \dg - 74.21 \dg = 105.79 \dg$ since both of these angle have the same sine.

  Draw unit circle picture.

  \begin{align*}
    B                     & \approx 105.79 \dg \\
    \\
    C                     & \approx 180 \dg - 60 \dg - 105.79 \dg \\
                          & = 14.21 \dg \\
    \\
    \frac{\sin 60 \dg}{9} & = \frac{\sin 14.21 \dg}{c} \\
    c                     & \approx 2.5511 \\
  \end{align*}

  \subsection{One Non-Right Triangle Solution}
  For example: $\angle A = 60$, $a = 20$, $b = 10$:

  \begin{align*}
    \frac{\sin 60 \dg}{20} & = \frac{\sin B}{10} \\
    \sin B                & \approx 0.433\\
    B                     & \approx 25.66 \dg \\
    \\
    C                     & \approx 180 \dg - 60 \dg - 25.66 \dg \\
                          & = 94.34 \dg \\
    \\
    \frac{\sin 60 \dg}{20} & = \frac{\sin 94.34 \dg}{c} \\
    c                     & \approx 23.03 \\
  \end{align*}

  The other possibility for $\angle B$ would be $180 \dg - 25.66 \dg = 154.34$, but with the $60 \dg$ we started with,
  this would put the total for the triangle over $180 \dg$.

  \subsection{Summary}
  \begin{itemize*}
    \item ASA and SAA always have exactly one solution

    \item SSA has zero, one, or two solutions:
      \begin{itemize}
        \item $\sin B > 1$: no solutions
        \item $\sin B = 1$: one right triangle solution
        \item $\sin B < 1$: one or two solutions.  Check angle and supplement.
      \end{itemize}

  \end{itemize*}

  \section{Examples}
  Do examples from book, etc.

\end{document}
