\documentclass{exam}

\usepackage{units} 
\usepackage{graphicx}
\usepackage[fleqn]{amsmath}
\usepackage{cancel}
\usepackage{float}
\usepackage{mdwlist}
\usepackage{booktabs}
\usepackage{cancel}
\usepackage{polynom}
\usepackage{caption}
\usepackage{fullpage}
\usepackage{comment}
\usepackage{enumerate}
\usepackage{xfrac}

\newcommand{\dg}{\ensuremath{^\circ}} 
\everymath{\displaystyle}

\printanswers
\excludecomment{comment}

\ifprintanswers 
  \usepackage{2in1, lscape} 
\fi

\author{}
\date{\today}
\title{Math 142 \\ Homework Twelve}

\begin{document}

  \maketitle

  \section{Homework}
  Section 7.1: 

  \section{Extra Credit}
  Section 7.1: 

  \ifprintanswers

    \begin{description}
      \item[35]
        The triangle formed has sides $a = 9$, $b = 10$, and $c = 11$.

        Use the Law of Cosines to find the angles in the triangle:
        \begin{align*}
          \angle A & \approx 50.48 \dg \\
          \angle B & \approx 58.99 \dg \\
          \angle C & \approx 70.53 \dg \\
        \end{align*}

        Convert to radians:
        \begin{align*}
          \angle A & \approx 0.8810 \\
          \angle B & \approx 1.030 \dg \\
          \angle C & \approx 1.231 \dg \\
        \end{align*}

        Find the areas of the three sectors:
        \begin{align*}
          A_1 & \approx \frac{1}{2} 0.881 \cdot 6^2 \\
              & \approx 15.858 \\
          \\
          A_2 & \approx \frac{1}{2} 1.03 \cdot 5^2 \\
              & \approx 12.875 \\
          \\
          A_3 & \approx \frac{1}{2} 1.231 \cdot 4^2 \\
              & \approx 9.848 \\
        \end{align*}

        Find the area of the triangle:
        \begin{align*}
          s  & = \frac{1}{2}(9 + 10 + 11) \\
             & = 15 \\
          \\
          A_{triangle} & = \sqrt{15 (15 - 9)(15 - 10) (15 - 11)} \\
                       & \approx 42.4264 \\
        \end{align*}

        The final area is the difference between the area of the triangle and the area of the three sectors:
        \begin{align*}
          Area & \approx 42.4264 - 15.858 - 12.875 - 9.848 \\
               & = \boxed{ \unit[3.85]{cm^2} } \\
        \end{align*}

    \end{description}

    \pagebreak
  \fi

  \section{Review}

  \begin{enumerate}
    \item What is the angle in degrees between the hour and minute hands of a clock at 12:20?

      \begin{solution}
        12:20 is one third of an hour past 12:00.  
        
        The hour hand has travelled one third of one twelfth of the way around:
        \[
          \angle H = \frac{1}{3} \cdot \frac{1}{12} \cdot 360 \dg = 10 \dg
        \]

        The minute hand has traveled one third of the way around:
        \[
          \angle M = \frac{1}{3} \cdot 360 \dg = 120 \dg
        \]

        The difference is:
        \[
          120 \dg - 10 \dg = 110 \dg
        \]

      \end{solution}

  \end{enumerate}

  \ifprintanswers
    \section{Section 7.1}
    \begin{description}

      \item[1] 
        \begin{align*}
          \cos t \tan t & = \cos t \cdot \frac{\sin t}{\cos t} \\
                        & = \sin t \\
        \end{align*}

      \item[2] 
        \begin{align*}
          \cos t \csc t & = \cos t \cdot frac{1}{\sin t} \\
                        & = \cot t \\
        \end{align*}

      \item[3] 
        \begin{align*}
          \sin \theta \sec \theta & = \sin \theta \cdot \frac{1}{\cos \theta} \\
                                  & = \tan t \\
        \end{align*}

      \item[4] 
        \begin{align*}
          \tan \theta \csc \theta & = \frac{\sin \theta}{\cos \theta} \cdot \frac{1}{\sin \theta} \\
                                  & = \sec t \\
        \end{align*}

      \item[5] 
        \begin{align*}
          \tan^2 x - \sec^2 x & = \frac{\sin^2 x}{\cos^2 x} - \frac{1}{\cos^2 x} \\
                              & = \frac{\sin^2 x - 1}{\cos^2 x} \\
                              & = \frac{- \cos^2 x}{\cos^2 x} \\
                              & = -1 \\
        \end{align*}

      \item[6] 
        \begin{align*}
          \frac{\sec x}{\csc x} & = \frac{\sin x}{\cos x} \\
                                & = \tan x \\
        \end{align*}

      \item[7] 
        \begin{align*}
          \sin u + \cot u \cos u & = \sin u + \frac{\cos u}{\sin u} \cdot \cos u \\
                                 & = \sin u + \frac{\cos^2 u}{\sin u} \\
                                 & = \frac{\sin^2 u + \cos^2 u}{\sin u} \\
                                 & = \frac{1}{\sin u} \\
                                 & = \csc u \\
        \end{align*}

      \item[8] 
        \begin{align*}
          \cos^2 \theta \left( 1 + \tan^2 \theta \right) & = \cos^2 \theta \left( 1 + \frac{\sin^2 \theta}{\cos^2 \theta} \right) \\
                                                         & = \cos^2 \theta \cdot \frac{\sin^2 \theta + \cos^2 \theta}{\cos^2 \theta} \\
                                                         & = 1 \\
        \end{align*}

      \item[9] 
        \begin{align*}
          \frac{\sec \theta - \cos \theta}{\sin \theta} & = \frac{\sfrac{1}{\cos \theta} - \cos \theta}{\sin \theta} \\
                                                        & = \frac{\sfrac{1 - \cos^2 \theta}{\cos \theta}}{\sin \theta} \\
                                                        & = \frac{\sin^2 \theta}{\sin \theta \cos \theta} \\
                                                        & = \frac{\sin \theta}{\cos \theta} \\
                                                        & = \tan \theta \\
        \end{align*}

      \item[10] 
        \begin{align*}
          \frac{\cot \theta}{\csc \theta - \sin \theta} & = \frac{\sfrac{\cos \theta}{\sin \theta}}{\sfrac{1}{\sin \theta} - \sin \theta} \\
                                                        & = \frac{\cos \theta}{1 - \sin^2 \theta} \\
                                                        & = \frac{\cos \theta}{\cos^2 \theta} \\
                                                        & = \frac{1}{\cos^2 \theta} \\
                                                        & = \sec \theta \\
        \end{align*}

      \item[11] 
        \begin{align*}
          \frac{\sin x \sec x}{\tan x} & = \frac{\sin x}{\cos x \cdot \sfrac{\sin x}{\cos x}} \\
                                       & = \frac{\sin x}{\sin x} \\
                                       & = 1 \\
        \end{align*}

      \item[12] 
        \begin{align*}
          \cos^3 x + \sin^2 x \cos x & = \cos x \left( \cos^2 x + \sin^2 x \right) \\
                                     & = \cos x \\
        \end{align*}

      \item[13] 
        \begin{align*}
          \frac{1 + \cos y}{1 + \sec y} & = \frac{1 + \cos y}{1 + \sfrac{1}{\cos y}} \\
                                        & = \frac{1 + \cos y}{\sfrac{\cos y + 1}{\cos y}} \\
                                        & = \cos y \\
        \end{align*}

    \end{description}

  \else
    \vspace{5 cm}

    \begin{quote}
      \begin{em}
        Our inventions are wont to be pretty toys, which distract our attention from serious things. They are but
        improved means to an unimproved end, an end which it was already but too easy to arrive at.
      \end{em}
    \end{quote}
    \hspace{1 cm} --Henry David Thoreau
  \fi

\end{document}

