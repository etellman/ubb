\documentclass{exam}

\usepackage{units} 
\usepackage{graphicx}
\usepackage[fleqn]{amsmath}
\usepackage{cancel}
\usepackage{float}
\usepackage{mdwlist}
\usepackage{booktabs}
\usepackage{cancel}
\usepackage{polynom}
\usepackage{caption}
\usepackage{fullpage}
\usepackage{xfrac}
\usepackage{enumerate}

\newcommand{\dg}{\ensuremath{^\circ}} 
\everymath{\displaystyle}

\printanswers
\usepackage{2in1, lscape} 

\title{Math 142 Notes \\ Section 6.2}

\date{October 23, 2013}

\begin{document}

  \maketitle
  \tableofcontents

  \section{Trigonometric Ratios}

  \subsection{Notes}
  \begin{itemize*}
    \item Show sine, cosine, etc. as ratios of sides of right triangles.
    \item use $3/4/5$ and $1/2/\sqrt{5}$ triangles
  \end{itemize*}

  \subsection{Examples}
  Draw a triangle that matches and find all the other trigonometric ratios:
  \begin{enumerate}
    \item $\sin \theta = \frac{2}{3}$  
    \item $\tan \theta = 3$
    \item etc.
  \end{enumerate}

  \section{Special Triangles}

  \begin{itemize*}
    \item $45 \dg$ from 1x1 square 
    \item $30 \dg$ and $60 \dg$ from equilateral triangle with side length 1 (use perpendicular bisector)
  \end{itemize*}

  \section{Applications}

  \subsection{Notes}
  ``Solve a triangle'' means find all sides and angles given some sides and angles.  Future sections will include
  techniques for solving non-right triangles.

  In this section, you can find everything about a triangle given at least one side and one other angle or side.  Two
  angles don't uniquely determine everything.

  If you have two sides, you can use the inverse trigonometric functions on a calculator to find the angles.

  \subsection{Examples}

  \begin{enumerate}
    \item Solve example right triangles given any one angle and any side.  Include a few examples with radians.

    \item If tree casts a 50 foot shadow with an angle of elevation of $30 \dg$, how tall is the tree?
      \begin{solution}
        \begin{align*}
          \tan 30 \dg & = \frac{h}{50} \\
          h           & \approx \unit[29]{ft} \\
        \end{align*}
      \end{solution}

    \item If kite's string is 200 feet long and the angle the string makes with the ground is $60 \dg$, how high is the
      kite?
      \begin{solution}
        \begin{align*}
          \sin 60 \dg & = \frac{h}{200} \\
          h           & \approx \unit[173]{ft} \\
        \end{align*}
      \end{solution}

    \item A 20 foot ladder is 7 feet from the base of a building
      \begin{parts}
        \part What angle does the ladder make with the ground?
          \begin{solution}
            \begin{align*}
              \cos \theta & = \frac{7}{20} \\
              \theta      & \approx 70 \dg \\
            \end{align*}
          \end{solution}

        \part How high does the ladder reach on the building?
          \begin{solution}
            \begin{align*}
              \sin 70 \dg & = \frac{h}{20} \\
              h           & \approx \unit[18.8]{ft} \\
            \end{align*}
          \end{solution}
      \end{parts}

    \item From the top of a fire lookout tower on the top of a 1 mile peak, the angle of depression to a fire is
      $25 \dg$.  How far away is the fire?

      \begin{solution}
        \begin{align*}
          \tan 25 \dg & = \frac{1}{d} \\
          d           & \approx \unit[2.1]{mi} \\
        \end{align*}
      \end{solution}

    \item Geostationary Dish TV satellite in orbit 35,000 km.
      \begin{parts}
        \part If pointing error is $0.000278 \dg$, how big would the satellite have to be?
          \begin{solution}
            \begin{align*}
              \tan 0.000278 \dg & = \frac{x}{35,000} \\
              x                 & \approx \unit[170]{m} \\
            \end{align*}

            Satellite would need to be twice as big, or $\unit[340]{m}$.
          \end{solution}

        \part If the satellite is only $\unit[10]{m}$ long, how accurate would the antenna have to be?
          \begin{solution}
            Divide by two because of $\pm$.

            \begin{align*}
              \tan \theta & = \frac{0.005}{35,000} \\
              \theta      & \approx \pm 8.2 \dg \times 10^{-6} \\
            \end{align*}

          \end{solution}

      \end{parts}

    \item $65 \dg$ ski slope from top of 4000 ft mountain.  How long is the run?
      \begin{solution}
        \begin{align*}
          \sin 65 \dg & = \frac{4000}{d} \\
          d           & \approx \unit[4413]{ft}
        \end{align*}
      \end{solution}

    \item An airline is 5000 feet from the runway coming in for a landing with a $3 \dg$ glide path.  What is the
      plane's altitude?

      \begin{solution}
        \begin{align*}
          \tan 3 \dg & = \frac{h}{5000} \\
          d           & \approx \unit[262]{ft}
        \end{align*}
      \end{solution}

    \item An airline is at an altitude of 450 feet when it is 5200 feet from the runway.  What is the glide angle? 
      plane's altitude?

      \begin{solution}
        \begin{align*}
          \tan \theta & = \frac{450}{5200} \\
          \theta      & \approx 5.1 \dg \\
        \end{align*}
      \end{solution}

    \item From one point angle of elevation to the top of a mountain is $45 \dg$.  1000 feet closer, the angle is $47 \dg$.  How
      tall is the mountain?

      \begin{solution}
        \begin{align*}
          \tan 45 \dg &= \frac{h}{x + 1000} \\
          \tan 47 \dg &= \frac{h}{x} \\
          \\
          x &\approx \unit[14,818]{ft} \\
        \end{align*}
      \end{solution}

    \item Two fire lookouts spot the same fire.  The two towers are one mile apart.  One lookout measures an angle of
      $45 \dg$ and the other measures an angle of $30 \dg$.  Find all the distances.  See figure for problem 60.

      \begin{solution}
        Let x be the distance from one of the towers to the perpendicular bisector.

        \begin{align*}
          \tan 30 \dg & = \frac{h}{x} \\
          \tan 45 \dg & = \frac{h}{1 - x} \\
          x           & \approx \unit[0.63]{mi} \\
          \\
          h & = x \tan 30 \dg \\
            & \approx \unit[0.37]{mi} \\
          \\
          r_1 & = \frac{h}{\sin 30 \dg} \\
              & \approx \unit[0.73]{mi} \\
          \\
          r_2 & = \frac{h}{\sin 45 \dg} \\
              & \approx \unit[0.52]{mi} \\
        \end{align*}
      \end{solution}

  \end{enumerate}

  \section{Eratosthenes Distance to Sun}

  \begin{enumerate}
    \item When the sun is directly overhead in Syene, in Alexandria there is a 1.2 foot shadow from a 10 foot pole.
    What is the angle to the sun?  Assume the sun is infinitely far away.

      \begin{solution}
        \begin{align*}
          \sin \theta &= \frac{1.2}{10} \\
          \theta &\approx 7 \dg \\
        \end{align*}
      \end{solution}

    \item If the distance between the two cities is 5000 stadia, what is the radius of the earth in stadia?
      \begin{solution}
        convert degrees to radians:
        \[
          7 \dg \approx 0.122
        \]

        Find the radius:
        \begin{align*}
          s & = r \theta \\
          r & = \frac{s}{\theta} \\
            & = \frac{5000}{0.122} \\
            & \approx \unit[41,000]{stadia} \\
        \end{align*}

        Actual stadium unit may have been 
        \begin{itemize*}
          \item 157 meters, leading to an estimate of 6400 km.  
          \item 185 meters, leading to an estimate of 7571 km.  
        \end{itemize*}

        The actual value for the radius is 6371 km.

        A flaw in the plan is that the cities aren't actually directly north of each other, so the apparent precision
        involved a bit of luck.

      \end{solution}
  \end{enumerate}
\end{document}
