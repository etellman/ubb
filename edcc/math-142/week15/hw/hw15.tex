\documentclass{exam}

\usepackage{units} 
\usepackage{graphicx}
\usepackage[fleqn]{amsmath}
\usepackage{cancel}
\usepackage{float}
\usepackage{mdwlist}
\usepackage{booktabs}
\usepackage{cancel}
\usepackage{polynom}
\usepackage{caption}
\usepackage{fullpage}
\usepackage{comment}
\usepackage{enumerate}
\usepackage{xfrac}
\usepackage{parskip}

\newcommand{\dg}{\ensuremath{^\circ}} 
\everymath{\displaystyle}

\printanswers
\excludecomment{comment}

\ifprintanswers 
  \usepackage{2in1, lscape} 
\fi

\author{}
\date{\today}
\title{Math 142 \\ Homework Fifteen}

\begin{document}

  \maketitle

  \section{Homework}
  Section 7.4:

  \section{Extra Credit}
  Section 7.4: 48 and 56

  \ifprintanswers
    \begin{description}
      \item[48]
        \begin{align*}
          \sin x &= \pm \frac{1}{\sqrt{1 + \sfrac{1}{\tan^2 x}}} \\
          \cos x &= \pm \frac{1}{\sqrt{1 + \tan^2 x}} \\
          \\
          \sin(x - y) &= \sin x \cos y - \cos x \sin y \\
                      &= \begin{cases}
                            - \sqrt{ \frac{1 - x^2}{1 + \sfrac{1}{x^2}}} + \frac{x}{\sqrt{1 + x^2}} & \mbox{if } -1 \leq x < 0 \\
                            0                                                                       & \mbox{if } x = 0 \\
                            \sqrt{ \frac{1 - x^2}{1 + \sfrac{1}{x^2}}} + \frac{x}{\sqrt{1 + x^2}}   & \mbox{if } 0 < x \leq 1 \\
                          \end{cases}
        \end{align*}

      \item[56]
        Use radians for $\theta$ to make the arc length calculations easier.

        \begin{parts}
          \part 
            \begin{align*}
              \cos \theta & = \frac{r}{h + r} \\
              \theta      & = \boxed{ \cos^{-1} \frac{r}{h + r} } \\
            \end{align*}

          \part
            \[
              s(\theta) = \boxed{ r \theta }
            \]

          \part
            \[
              s(h) = \boxed{ r \cos^{-1} \frac{r}{h + r} }
            \]

          \part
            \begin{align*}
              s(100) & = 3960 \cos^{-1} \frac{3960}{100 + 3960} \\
                     & \approx \boxed{ \unit[881]{mi} } \\
            \end{align*}

          \part
            \begin{align*}
              s                & = r \cos^{-1} \frac{r}{h + r} \\
              % \frac{s}{r}      & = \cos^{-1} \frac{r}{h + r} \\
              % \cos \frac{s}{r} & = \frac{r}{h + r} \\
              h                & = r \left( \sec \frac{s}{r} - 1 \right) \\
              \\
              h(2450) & = 3960 \left( \sec \frac{2450}{3960} - 1 \right) \\
                      & \approx \boxed{ \unit[901]{mi} } \\
            \end{align*}

        \end{parts}

    \end{description}
  \fi

  % $x = 0$ is a special case to avoid dividing by zero.

  % \section{Review}

  \ifprintanswers
    \section{Section 7.4}
    \begin{description}

      \item[1] 
        \begin{parts}
          \part $\sin^{-1} \frac{1}{2} = \boxed{ \frac{\pi}{6} }$
          \part $\cos^{-1} \frac{1}{2} = \boxed{ \frac{\pi}{3} }$
          \part $\cos^{-1} 2$ is not defined
        \end{parts}

      \item[2] 
        \begin{parts}
          \part $\sin^{-1} \frac{\sqrt{3}}{2} = \boxed{ \frac{\pi}{3} }$
          \part $\cos^{-1} \frac{\sqrt{3}}{2} = \boxed{ \frac{\pi}{6} }$
          \part $\cos^{-1} \frac{- \sqrt{3}}{2} = \boxed{ \frac{5 \pi}{6} }$
        \end{parts}

      \item[3] 
        \begin{parts}
          \part $\sin^{-1} \frac{\sqrt{2}}{2} = \boxed{ \frac{\pi}{4} }$
          \part $\cos^{-1} \frac{\sqrt{2}}{2} = \boxed{ \frac{\pi}{4} }$
          \part $\sin^{-1} \frac{- \sqrt{2}}{2} = \boxed{ - \frac{\pi}{4} }$
        \end{parts}

      \item[4] 
        \begin{parts}
          \part $\tan^{-1} \sqrt{3} = \boxed{ \frac{\pi}{3} }$
          \part $\tan^{-1} (- \sqrt{3}) = \boxed{ -\frac{\pi}{3} }$
          \part $\sin^{-1} \sqrt{3}$ is not defined
        \end{parts}

      \item[5] 
        \begin{parts}
          \part $\sin^{-1} 1 = \boxed{ \frac{\pi}{2} }$
          \part $\cos^{-1} 1 = \boxed{ 0 }$
          \part $\cos^{-1} (-1) = \boxed{ \pi }$
        \end{parts}

      \item[6] 
        \begin{parts}
          \part $\tan^{-1} 1 = \boxed{ \frac{\pi}{4} }$
          \part $\tan^{-1} (-1) = \boxed{ \frac{- \pi}{4} }$
          \part $\tan^{-1} 0 = \boxed{ 0 }$
        \end{parts}

      \item[7] 
        \begin{parts}
          \part $\tan^{-1} \frac{\sqrt{3}}{3} = \boxed{ \frac{\pi}{6} }$
          \part $\tan^{-1} \left( - \frac{\sqrt{3}}{3} \right) = \boxed{ - \frac{\pi}{4} }$
          \part $\sin^{-1} (-2)$ is not defined
        \end{parts}

      \item[8] 
        \begin{parts}
          \part $\sin^{-1} 0 = \boxed{ 0 }$
          \part $\cos^{-1} 0 = \boxed{ \frac{\pi}{2} }$
          \part $\cos^{-1} \left( - \frac{1}{2} \right) = \boxed{ \frac{2 \pi}{3} }$
        \end{parts}

      \item[13] 
        \[
          \sin \left( \sin^{-1} \frac{1}{4} \right) = \boxed{ \frac{1}{4} }
        \]

      \item[14] 
        \[
          \cos \left( \cos^{-1} \frac{2}{3} \right) = \boxed{ \frac{2}{3} }
        \]

      \item[15] 
        \[
          \tan \left( \tan^{-1} 5 \right) = \boxed{ 5 }
        \]

      \item[16] $\sin \left( \sin^{-1} 5 \right)$ is not defined

      \item[17] 
        \[
          \cos{-1} \left( \cos \frac{\pi}{3} \right) = \boxed{ \frac{\pi}{3} }
        \]

      \item[18] 
        \[
          \tan{-1} \left( \tan \frac{\pi}{6} \right) = \boxed{ \frac{\pi}{6} }
        \]

      \item[19] 
        \[
          \sin{-1} \left( \sin \left( -\frac{\pi}{6} \right) \right) = \boxed{ - \frac{\pi}{6} }
        \]

      \item[20] 
        \[
          \sin{-1} \left( \sin \left( \frac{5 \pi}{6} \right) \right) = \boxed{ \frac{\pi}{6} }
        \]

      \item[21] 
        \[
          \tan{-1} \left( \tan \left( \frac{2 \pi}{3} \right) \right) = \boxed{ - \frac{\pi}{3} }
        \]

      \item[22] 
        \[
          \cos{-1} \left( \cos \left( - \frac{\pi}{4} \right) \right) = \boxed{ \frac{\pi}{4} }
        \]

      \item[23] 
        \[
          \tan \left( \sin^{-1} \frac{1}{2} \right) = \boxed{ \frac{\sqrt{3}}{3} }
        \]

      \item[24] 
        \[
          \sin \left( \sin^{-1} 0 \right) = \boxed{ 0 }
        \]

      \item[25] 
        \[
          \cos \left( \sin^{-1} \frac{\sqrt{3}}{2} \right) = \boxed{ \frac{1}{2} }
        \]

      \item[26] 
        \[
          \tan \left( \sin^{-1} \frac{\sqrt{2}}{2} \right) = \boxed{ 1 }
        \]

      \item[27] 
        \begin{align*}
          \tan^{-1} \left( 2 \sin \frac{\pi}{3} \right) & = \tan^{-1} \sqrt{3} \\
                                                        & = \boxed{ \frac{\pi}{3} }
        \end{align*}

      \item[28] 
        \begin{align*}
          \cos^{-1} \left( \sqrt{3} \sin \frac{\pi}{6} \right) & = \cos^{-1} \frac{\sqrt{3}}{2} \\
                                                               & = \boxed{ \frac{\pi}{6} } \\
        \end{align*}

      \item[29]
        \[
          \sin \left( \cos^{-1} \frac{3}{5} \right) = \boxed{ \frac{4}{5} } 
        \]

      \item[30]
        \[
          \tan \left( \sin^{-1} \frac{4}{5} \right) = \boxed{ \frac{4}{3} } 
        \]

      \item[31]
        \[
          \sin \left( \tan^{-1} \frac{12}{5} \right) = \boxed{ \frac{12}{13} } 
        \]

      \item[32]
        \[
          \cos \left( \tan^{-1} 5 \right) = \boxed{ \frac{\sqrt{26}}{26} } 
        \]

      \item[33]
        \[
          \sec \left( \sin^{-1} \frac{12}{13} \right) = \boxed{ \frac{13}{5} } 
        \]

      % \item[37]
      %   \begin{align*}
      %     \sin \left( 2 \cos^{-1} \frac{3}{5} \right) & = 2 \sin \left( \cos^{-1} \frac{3}{5} \right) \cos \left( \cos^{-1} \frac{3}{5} \right) \\
      %                                                 & = 2 \cdot \frac{4}{5} \cdot \frac{3}{5} \\
      %                                                 & = \boxed{ \frac{24}{25} } \\
      %   \end{align*}

      % \item[38]
      %   \begin{align*}
      %     \tan \left( 2 \tan^{-1} \frac{5}{13} \right) & = \frac{ 2 \tan \left( \tan^{-1} \sfrac{5}{13} \right) }
      %                                                           { 1 - \tan^2 \left( \tan^{-1} \sfrac{5}{13} \right) } \\
      %                                                  & = \frac{ 2 \cdot \sfrac{5}{13} }{1 - \left( \sfrac{5}{13} \right)^2 } \\
      %                                                  % & = \frac{10}{13 ( 1 - \sfrac{25}{13^2} )} \\
      %                                                  % & = \frac{10}{13 - \sfrac{25}{13} } \\
      %                                                  & = \boxed{ \frac{65}{72} } \\
      %   \end{align*}

      \item[41]
        \begin{align*}
          \cos x &= \pm \sqrt{1 - \sin^2 x } \\
          \cos \left( \sin^{-1} x \right) & = \boxed{ \sqrt{ 1 - x^2} } \\
        \end{align*}

      \item[42]
        \begin{align*}
          \sin^2 x & = \frac{\sin^2 x}{\cos^2 x} \cdot \cos^2 x \\
                   & = \tan^2 x \cdot \cos^2 x \\
                   & = \frac{\tan^2 x}{\sec^2 x} \\
                   & = \frac{\tan^2 x}{1 + \tan^2 x} \\
          \sin x   & = \pm \frac{ \tan x }{ \sqrt{1 + \tan^2 x} } \\
          \\
          \sin \left( \tan^{-1} x \right) &= \boxed{ \frac{x}{\sqrt{1 + x^2}} } \\
        \end{align*}

      \item[43]
        \begin{align*}
          \tan x & = \frac{\sin x}{\sqrt{1 - \sin^2 x}} \\
          \\
          \tan \left( \sin^{-1} x \right) &= \boxed{ \frac{x}{\sqrt{1 - x^2}} } \\
        \end{align*}

      \item[44]
        \begin{align*}
           \sec^2 x &= \tan^2 x + 1 \\
           \frac{1}{\cos^2 x} &= \tan^2 x + 1 \\
           \cos x &= \frac{1}{\sqrt{tan^2 x + 1}} \\
           \\
           \cos \left( \tan^{-1} x \right) &= \frac{1}{\sqrt{x^2 + 1}} \\
        \end{align*}

      \item[45]
        \begin{align*}
          \cos 2x & = 2 \cos^2 x - 1 \\
                  & = \frac{1}{\tan^2 x + 1} - 1 \\
                  & = \frac{1- \tan^2 x}{\tan^2 x + 1} \\
          \\
          \cos \left( 2 \tan^{-1} x \right) & = \boxed{ \frac{1 - x^2}{1 + x^2} } \\
        \end{align*}

      \item[46]
        \begin{align*}
          \sin 2x & = 2 \sin x \cos x \\
                  & = 2 \sin x \sqrt{1 - \sin^2 x} \\
                  \\
          \sin \left( 2 \sin^{-1} x \right) & = \boxed{ 2x \sqrt{1 - x^2} } \\
        \end{align*}

      \item[47]
        \begin{align*}
          \cos(x + y) & = \cos x \cos y - \sin x \sin y \\
                      & = \cos x \sqrt{1 - \sin^2 y} -  \sin y \sqrt{1 - \sin^2 x} \\
                      \\
          \cos(\cos^{-1} x + \sin^{-1} y) & = x \sqrt{1 - x^2} - x \sqrt{1 - x^2} \\
                                          & = \boxed{ 0 } \\
        \end{align*}

      \item[53]
        \begin{parts}
          \part
            \begin{align*}
              \tan \theta & = \frac{h}{2} \\
              h           & = \boxed{ 2 \tan \theta } \\
            \end{align*}

          \part
            \begin{align*}
              \tan \theta & = \frac{h}{2} \\
              \theta      & = \boxed{ \tan^{-1} \frac{h}{2} } \\
            \end{align*}

        \end{parts}

      \item[54]
        \begin{parts}
          \part
            \begin{align*}
              \tan \theta & = \frac{50}{s} \\
              \theta      & = \boxed{ \tan^{-1} \frac{50}{s} } \\
            \end{align*}

          \part
            \begin{align*}
              \theta & = \tan^{-1} \frac{50}{20} \\
                     & \approx \boxed{ 69.2 \dg } \\
            \end{align*}

        \end{parts}

      \item[55]
        \begin{parts}
          \part
            \begin{align*}
              \sin \theta & = \frac{h}{680} \\
              \theta & = \boxed{ \sin^{-1} \frac{h}{680} } \\
            \end{align*}

          \part
            \begin{align*}
              \theta & = \sin^{-1} \frac{500}{680} \\
                     & \approx \boxed{ 47.3 \dg } \\
            \end{align*}

        \end{parts}

    \end{description}

  \else
    \vspace{5 cm}

    \begin{quote}
      \begin{em}
      \end{em}
    \end{quote}
    \hspace{1 cm} --TO DO
  \fi

\end{document}

