\documentclass{exam}

\usepackage{units} 
\usepackage{graphicx}
\usepackage[fleqn]{amsmath}
\usepackage{cancel}
\usepackage{float}
\usepackage{mdwlist}
\usepackage{booktabs}
\usepackage{cancel}
\usepackage{polynom}
\usepackage{caption}
\usepackage{fullpage}
\usepackage{xfrac}
\usepackage{enumerate}

\newcommand{\dg}{\ensuremath{^\circ}} 
\everymath{\displaystyle}

\printanswers

\title{Math 142 Notes \\ Section 7.4}

\date{\today}

\begin{document}

  \maketitle
  \tableofcontents

  \section{Inverse Trigonometric Functions}

  \begin{itemize*}
    \item draw picture with sets
    \item draw vertical line picture
    \item analogy with square root of negative numbers
  \end{itemize*}

  \begin{tabular}[H]{lrr}
    \toprule
           & domain    & range \\
    \midrule
      sine & $[-1, 1]$ & $[- \frac{\pi}{2}, \frac{\pi}{2}]$ \\
    \midrule
      cos  & $[-1, 1]$ & $[0, \pi]$ \\
    \midrule
      tan  & $[-\infty, \infty]$ & $[- \frac{\pi}{2}, \frac{\pi}{2}]$ \\
    \bottomrule
  \end{tabular}

  \begin{align*}
    \cos(\cos^{-1} x) & = x \\
    \sin(\sin^{-1} x) & = x \\
    \tan(\tan^{-1} x) & = x \\
  \end{align*}

  \begin{itemize*}
    \item The other direction only works if $x$ is in the proper range.
    \item examples where large angle gets mapped to different angle in proper range after arcsine of sine, etc.:
      \[
        \sin^{-1}(\sin \sfrac{9 \pi}{4}) = \sfrac{\pi}{4}
      \]
  \end{itemize*}

  \section{Composing Inverse Trigonometric Functions}

  \subsection{Triangles}
  Find $\cos \sin^{-1}{\sfrac{2}{3}}$ by drawing a triangle, finding the hypotenuse, and then finding the cosine.

  \subsection{Algebraic}

  Find $\cos \left( \sin^{-1}{\sfrac{2}{3}} \right)$ by drawing a triangle, finding the hypotenuse, and then finding the
  cosine.

  \begin{itemize*}
    \item write cosine in terms of sine
    \item substitute number
    \item check sign
  \end{itemize*}

  \begin{align*}
    \cos x                           & = \sqrt{1 - \sin^2 x} \\
    \cos ( \sin^{-1}{\sfrac{2}{3}} ) & = \sqrt{1 - \frac{4}{9}} \\
                                     & = \frac{\sqrt{5}}{3} \\
  \end{align*}
\end{document}
