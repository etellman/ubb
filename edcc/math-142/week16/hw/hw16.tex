\documentclass{exam}

\usepackage{units} 
\usepackage{graphicx}
\usepackage[fleqn]{amsmath}
\usepackage{cancel}
\usepackage{float}
\usepackage{mdwlist}
\usepackage{booktabs}
\usepackage{cancel}
\usepackage{polynom}
\usepackage{caption}
\usepackage{fullpage}
\usepackage{comment}
\usepackage{enumerate}
\usepackage{xfrac}
\usepackage{parskip}

\newcommand{\dg}{\ensuremath{^\circ}} 
\everymath{\displaystyle}

\printanswers
\excludecomment{comment}

\ifprintanswers 
  \usepackage{2in1, lscape} 
\fi

\author{}
\date{\today}
\title{Math 142 \\ Homework Sixteen}

\begin{document}

  \maketitle

  \section{Homework}
  Section 7.5:

  \section{Extra Credit}
  Section 7.5: 85 part a

  \ifprintanswers
    \begin{description}
      \item[85a]
        Since the two pulleys have the same angles and different radii, you only need to analyze one of them and then
        use the same equation with the other radius.

        Find the length of belt touching the pulley.  Let $\alpha$ be the interior angle:
        \[
          \alpha = \frac{\pi}{2} - \frac{\theta}{2} = \frac{\pi - \theta}{2}
        \]
        The belt touching the pulley is:
        \[
          2 \pi r - 2r \alpha = 2 \pi r - \pi r - \theta r = r (\theta + \pi)
        \]

        If $L_r1$ is the length of belt between the pulley and the crossing point:
        \begin{align*}
          \tan \frac{\theta}{2} & = \frac{r}{L_{r1}} \\
          L_{r1}                & = r \cot \frac{\theta}{2} \\
        \end{align*}

        The total belt for one side is:
        \begin{align*}
          L_r & = r (\theta + \pi) + 2 r \cot \frac{\theta}{2} \\
              & = r \left( \theta + \pi + 2 \cot \frac{\theta}{2} \right) \\
        \end{align*}

        The total belt is:
        \begin{align*}
          L                                 & = (R + r) \left( \theta + \pi + 2 \cot \frac{\theta}{2} \right) \\
          \theta  + 2 \cot \frac{\theta}{2} & = \frac{L}{R + r} - \pi \\
        \end{align*}

    \end{description}

    \pagebreak

    \section{Section 7.5}
    \begin{description}

      \item[1] 
        \begin{align*}
          \cos x + 1 & = 0 \\
          \cos x     & = -1 \\
          x          & = \boxed{ \pi + 2 k \pi } \\
        \end{align*}

      \item[2] 
        \begin{align*}
          \sin x + 1 & = 0 \\
          \sin x     & = -1 \\
          x          & = \boxed{ \frac{3 \pi}{2} + 2 k \pi } \\
        \end{align*}

      \item[3] 
        \begin{align*}
          2 \sin x - 1 & = 0 \\
          \sin x       & = \frac{1}{2} \\
          x            & = \boxed{ \left\{ \frac{\pi}{6} + 2 k \pi, \frac{5 \pi}{6} + 2 k \pi \right\} } \\
        \end{align*}

      \item[4] 
        \begin{align*}
          \sqrt{2} \cos x - 1 & = 0 \\
          \cos x              & = \frac{\sqrt{2}}{2} \\
          x                   & = \boxed{ \left\{ \frac{\pi}{4} + 2 k \pi, \frac{7 \pi}{4} + 2 k \pi \right\} } \\
        \end{align*}

      \item[5] 
        \begin{align*}
          \sqrt{3} \tan x + 1 & = 0 \\
          \tan x              & = - \frac{\sqrt{3}}{3} \\
          x                   & = \boxed{ \frac{5 \pi}{6} + k \pi } \\
        \end{align*}

      \item[6] 
        \begin{align*}
          \cot x + 1 & = 0 \\
          \cot x     & = -1 \\
          x          & = \boxed{ \frac{3 \pi}{4} + k \pi } \\
        \end{align*}

      \item[7] 
        \begin{align*}
          4 \cos^2 x - 1 & = 0 \\
          \cos x         & = \pm \frac{1}{2} \\
          x              & = \boxed{ \left\{ \frac{\pi}{3} + k \pi, \frac{2 \pi}{3} + k \pi \right\} } \\
        \end{align*}

      \item[8] 
        \begin{align*}
          2 \cos^2 x - 1 & = 0 \\
          \cos x         & = \pm \frac{\sqrt{2}}{2} \\
          x              & = \boxed{ \left\{ \frac{\pi}{4} + k \pi, \frac{3 \pi}{4} + k \pi \right\} } \\
        \end{align*}

      \item[9] 
        \begin{align*}
          \sec^2 x - 2 & = 0 \\
          \sec x         & = \pm 2 \\
          \cos x         & = \pm \frac{\sqrt{2}}{2} \\
          x              & = \boxed{ \left\{ \frac{\pi}{4} + k \pi, \frac{3 \pi}{4} + k \pi \right\} } \\
        \end{align*}

      \item[10] 
        \begin{align*}
          \csc^2 x - 4 & = 0 \\
          \csc x         & = \pm 2 \\
          \sin x         & = \pm \frac{1}{2} \\
          x              & = \boxed{ \left\{ \frac{\pi}{6} + k \pi, \frac{5 \pi}{6} + k \pi \right\} } \\
        \end{align*}

      \item[11] 
        \begin{align*}
          3 \csc^2 x - 4 & = 0 \\
          \csc x         & = \pm \frac{2}{\sqrt{3}} \\
          \sin x         & = \pm \frac{\sqrt{3}}{2} \\
          x              & = \boxed{ \left\{ \frac{\pi}{3} + k \pi, \frac{2 \pi}{3} + k \pi \right\} } \\
        \end{align*}

      \item[12] 
        \begin{align*}
          1 - \tan^2 & = 0 \\
          \tan x         & = \pm 1 \\
          x              & = \boxed{ \left\{ \frac{\pi}{4} + k \pi, \frac{3 \pi}{4} + k \pi \right\} } \\
        \end{align*}

      \item[13] 
        \begin{align*}
          \cos x & (2 \sin x + 1) = 0 \\
          \\
          \cos x & = 0 \\
          x      & = \left\{ \frac{\pi}{2} + 2k \pi, \frac{3 \pi}{2} + 2k \pi \right\} \\
          \\
          2 \sin x + 1 & = 0 \\
          \sin x       & = - \frac{1}{2} \\
          x            & = \left\{ \frac{7 \pi}{6} + 2k \pi, \frac{13 \pi}{6} + 2k \pi \right\} \\
          \\
          x & = \boxed{ \left\{ \frac{\pi}{2} + 2k \pi, \frac{3 \pi}{2} + 2k \pi,
                        \frac{7 \pi}{6} + 2k \pi, \frac{13 \pi}{6} + 2k \pi \right\} } \\
        \end{align*}

      \item[14] 
        \begin{align*}
          \sec x (2 \cos x - \sqrt{2}) & = 0 \\
          \\
          2 \cos x - \sqrt{2} & = 0 \\
          \cos x              & = \frac{\sqrt{2}}{2} \\
          x                   & = \boxed{ \left\{ \frac{\pi}{4} + 2 k \pi, \frac{7 \pi}{4} + 2 k \pi \right\} } \\
        \end{align*}

      \item[15] 
        \begin{align*}
          (\tan x + \sqrt{3}) (\cos x + 2) & = 0 \\
          \\
          \tan x + \sqrt{3} & = 0 \\
          x                 & = \boxed{ \frac{2 \pi}{3} + k \pi } \\
        \end{align*}

      \item[16] 
        \begin{align*}
          (2 \cos x & + \sqrt{3}) (2 \sin x - 1) = 0 \\
          \\
          2 \cos x + \sqrt{3} & = 0 \\
          x                   & = \left\{ \frac{5 \pi}{6} + 2 k \pi, \frac{7 \pi}{6} + 2 k \pi \right\} \\
          \\
          2 \sin x - 1 & = 0 \\
          x                   & = \left\{ \frac{\pi}{6} + 2 k \pi, \frac{5 \pi}{6} + 2 k \pi \right\} \\
          \\
          x & = \boxed{ \left\{ \frac{\pi}{6} + 2 k \pi, \frac{5 \pi}{6} + 2 k \pi, \frac{7 \pi}{6} + 2 k \pi, \right\} } \\
        \end{align*}

      \item[17] 
        \begin{align*}
          \cos x \sin x - 2 \cos x & = 0 \\
          \cos x (\sin x - 2)      & = 0 \\
          \\
          \cos x & = 0 \\
          x      & = \boxed{ \left\{ \frac{\pi}{2} + 2k \pi, \frac{3 \pi}{2} + 2k \pi \right\} } \\
        \end{align*}

      \item[18] 
        \begin{align*}
          \tan x \sin x + \sin x & = 0 \\
          \sin x (\tan x + 1)    & = 0 \\
          \\
          \sin x & = 0 \\
          x      & = \pi + k \pi  \\
          \\
          \tan x + 1 & = 0 \\
          x      & = \frac{3 \pi}{4} + k \pi \\
          \\
          x      & = \boxed{ \left\{ \frac{3 \pi}{4} + k \pi ,\pi + k \pi \right\} } \\
        \end{align*}

      \item[19] 
        \begin{align*}
          4 \cos^2 x - 4 \cos x + 1 & = 0 \\
          (2 \cos x - 1)^2 & = 0 \\
          \\
          2 \cos x - 1 & = 0 \\
          x            & = \boxed{ \left\{ \frac{\pi}{3} + 2k \pi, \frac{5 \pi}{3} + 2k \pi \right\} } \\
        \end{align*}

      \item[20] 
        \begin{align*}
          2 \sin^2 x - \sin x - 1 & = 0 \\
          (2 \sin x + 1) (\sin x - 1) & = 0 \\
          \\
          2 \sin x + 1 & = 0 \\
          x            & = \left\{ \frac{7 \pi}{6} + 2k \pi, \frac{11 \pi}{6} + 2k \pi \right\} \\
          \\
          \sin x - 1 & = 0 \\
          x          & = \frac{\pi}{2} + 2k \pi \\
          \\
          x & = \boxed{ \left\{ \frac{\pi}{2} + 2k \pi, \frac{7 \pi}{6} + 2k \pi, \frac{11 \pi}{6} + 2k \pi \right\} } \\
        \end{align*}

      \item[25] 
        \begin{align*}
          2 \sin 3x + 1 & = 0 \\
          3x            & = \left\{ \frac{7 \pi}{6} + 2k \pi, \frac{11 \pi}{6} + 2k \pi \right\} \\
          x             & = \boxed{ \left\{ \frac{7 \pi}{18} + \frac{2k \pi}{3}, \frac{11 \pi}{18} + \frac{2k \pi}{3} \right\} } \\
        \end{align*}

      \item[26] 
        \begin{align*}
          2 \cos 2x + 1 & = 0 \\
          2x            & = \left\{ \frac{2 \pi}{3} + 2k \pi, \frac{4 \pi}{3} + 2k \pi \right\} \\
          x             & = \boxed{ \left\{ \frac{\pi}{3} + k \pi, \frac{2 \pi}{3} + k \pi \right\} } \\
        \end{align*}

      \item[27] 
        \begin{align*}
          \sec 4x - 2 & = 0 \\
          4x          & = \left\{ \frac{\pi}{3} + 2k \pi, \frac{5 \pi}{3} + 2k \pi \right\} \\
          x           & = \boxed{ \frac{\pi}{12} + \frac{k \pi}{6} } \\
        \end{align*}

      \item[28] 
        \begin{align*}
          \sqrt{3} \tan 3x + 1 & = 0 \\
          3x                   & = \frac{5 \pi}{6} + k \pi \\
          x                    & = \boxed{ \frac{\pi}{18} + \frac{k \pi}{3} } \\
        \end{align*}

      \item[29] 
        \begin{align*}
          \sqrt{3} \sin 2x & = \cos 2x \\
          \sqrt{3} \tan 2x & = 1 \\
          2x               & = \frac{\pi}{6} + k \pi \\
          x                & = \boxed{ \frac{\pi}{12} + \frac{k \pi}{2} } \\
        \end{align*}

      \item[30] 
        \begin{align*}
          \cos 3x & = \sin 3x \\
          \tan 3x & = 1 \\
          3x      & = \frac{\pi}{4} + k \pi \\
          x       & = \boxed{ \frac{\pi}{12} + \frac{k \pi}{2} } \\
        \end{align*}

      \item[31] 
        \begin{align*}
          \cos \frac{x}{2} - 1 & = 0 \\
          \frac{x}{2}          & = 2k \pi \\
          x                    & = \boxed{ 4k \pi } \\
        \end{align*}

      \item[32] 
        \begin{align*}
          2 \sin \frac{x}{3} + \sqrt{3} & = 0 \\
          \frac{x}{3}                   & = \left\{ \frac{4 \pi}{3} + 2k \pi, \frac{5 \pi}{3} + 2k \pi \right\} \\
          x                             & = \boxed{ \left\{ 4 \pi + 6k \pi, 5 \pi + 6k \pi \right\} } \\
        \end{align*}

      \pagebreak

      \item[35] 
        \begin{align*}
          \tan^5 x - 9 \tan x                                            & = 0 \\
          \tan x \left( \tan^4 x - 9 \right)                             & = 0 \\
          \tan x \left( \tan^2 x + 3 \right) \left( \tan^2 x - 3 \right) & = 0 \\
          \tan x \left( \tan^2 x + 3 \right) (\tan x + \sqrt{3}) (\tan x - \sqrt{3}) & = 0 \\
        \end{align*}
        \begin{align*}
          \tan x & = 0 \\
          x      & = k \pi \\
          \\
          \tan x + \sqrt{3} & = 0 \\
          x                 & = \frac{2 \pi}{3} + k \pi \\
          \\
          \tan x - \sqrt{3} & = 0 \\
          x                 & = \frac{\pi}{3} + k \pi \\
          \\
          x & = \boxed{ \frac{k \pi}{3} } \\
        \end{align*}

      \item[43] 
        \begin{align*}
          2 \sin x \tan x - \tan x                & = 1 - 2 \sin x \\
          2 \sin x \tan x + 2 \sin x - 1 - \tan x & = 0  \\
          2 \sin x (\tan x + 1) - (\tan x + 1)    & = 0  \\
          (2 \sin x - 1) (\tan x + 1)             & = 0  \\
        \end{align*}
        \begin{align*}
          2 \sin x - 1 & = 0 \\
          x            & = \left\{ \frac{\pi}{6} + 2k \pi, \frac{5 \pi}{6} + 2k \pi \right\} \\
          \\
          \tan x + 1 & = 0 \\
          x          & = \frac{3 \pi}{4} + k \pi \\
          \\
          x & = \boxed{ \left\{ \frac{\pi}{6} + 2k \pi, \frac{5 \pi}{6} + 2k \pi, \frac{3 \pi}{4} + k \pi \right\} } \\
        \end{align*}

      \item[44] 
        \begin{align*}
          \sec x \tan x - \cos x \cot x                 & = \sin x \\
          \frac{\tan x}{\cos x} - \frac{\cos x}{\tan x} & = \sin x \\
          \frac{\tan^2 x - \cos^2 x}{\tan x \cos x}     & = \sin x \\
          \frac{\tan^2 x - \cos^2 x}{\sin x}            & = \sin x \\
          \tan^2 x - \cos^2 x                           & = \sin^2 x \\
          \tan^2 x                                      & = \sin^2 x + \cos^2 x \\
          \tan^2 x                                      & = 1 \\
          \tan x                                        & = \pm 1 \\
          \\
          x                                             & = \boxed{ \frac{\pi}{4} + \frac{k \pi}{2} } \\
        \end{align*}

      \item[61] 
        \begin{align*}
          \cos x \cos 3x - \sin x \sin 3x & = 0 \\
          \cos 4x                         & = 0 \\
          4x                              & = \frac{\pi}{2} + k \pi \\
          x                               & = \frac{\pi}{8} + \frac{k \pi}{4} \\
        \end{align*}
        \[
          x = \boxed{ \left\{ \frac{\pi}{8}, \frac{3 \pi}{8}, \frac{5 \pi}{8}, \frac{7 \pi}{8}, 
                                \frac{9 \pi}{8}, \frac{11 \pi}{8}, \frac{13 \pi}{8}, \frac{15 \pi}{8} \right\} } \\
        \]

      \item[62] 
        \begin{align*}
          \cos x \cos 2x + \sin x \sin 2x & = \frac{1}{2} \\
          \cos x                          & = \frac{1}{2} \\
          x                               & = \left\{ \frac{\pi}{3} + 2k \pi, \frac{5 \pi}{3} + 2k \pi \right\} \\
        \end{align*}
        \[
          x = \boxed{ \left\{ \frac{\pi}{8}, \frac{3 \pi}{8}, \frac{5 \pi}{8}, \frac{7 \pi}{8}, 
                              \frac{9 \pi}{8}, \frac{11 \pi}{8}, \frac{13 \pi}{8}, \frac{15 \pi}{8} \right\} } \\
        \]

      \item[65] 
        \begin{align*}
          \sin 2x + \cos x & = 0 \\
          2 \sin x \cos x + \cos x & = 0 \\
          \cos x (2 \sin x + 1) & = 0 \\
          \\
          \cos x & = 0 \\
          x      & = \left\{ \frac{\pi}{2} + 2k \pi, \frac{3 \pi}{2} + 2k \pi \right\} \\
          \\
          2 \sin x + 1 & = 0 \\
          x            & = \left\{ \frac{7 \pi}{6} + 2k \pi, \frac{11 \pi}{6} + 2k \pi \right\} \\
        \end{align*}

        \[
          x = \boxed{ \left\{ \frac{\pi}{2} + 2k \pi, \frac{3 \pi}{2} + 2k \pi, \frac{7 \pi}{6} + 2k \pi, \frac{11 \pi}{6} + 2k \pi \right\} } \\
        \]

      \item[66] 
        \begin{align*}
          \tan \frac{x}{2} - \sin x                      & = 0 \\
          \frac{\sin x}{1 + \cos x} - \sin x             & = 0 \\
          \sin x \left( \frac{1}{1 + \cos x} - 1 \right) & = 0 \\
          - \frac{\sin x \cos x}{1 + \cos x}             & = 0 \\
          x                                              & = \boxed{ \frac{k \pi}{2} } \\
        \end{align*}

      \item[69] 
        \begin{align*}
          \sin x + \sin 3x & = 0 \\
          2 \sin 2x \cos x & = 0 \\
          2 \sin 2x \cos x & = 0 \\
          \sin x \cos^2 x  & = 0 \\
          x                & = \boxed{ \frac{k \pi}{2} } \\
        \end{align*}

      \item[70] 
        \begin{align*}
          \cos 5x - \cos 7x     & = 0 \\
          - 2 \sin 6x \sin (-x) & = 0 \\
          \sin 6x \sin x        & = 0 \\
          \\
          6x & = k \pi \\
          x  & = \frac{k \pi}{6} \\
          \\
          x &= k \pi \\
          \\
          x & = \boxed{ \frac{k \pi}{6} } \\
        \end{align*}

      \item[79]
        \begin{align*}
          5000     & = \frac{2200^2 \sin 2 \theta}{32} \\
          2 \theta & = 1.8944 \dg \mbox{ or } 178.1056 \\
          \theta   & = \boxed{ 0.9472 \dg \mbox{ or } 89.0528 \dg } \\
        \end{align*}

      \item[80]
        \begin{align*}
          4 e^{-3t} \sin 2 \pi t & = 0 \\
          \sin 2 \pi t           & = 0 \\
          2 \pi t                & = k \pi \\
          t                      & = \boxed{ \frac{k}{2} } \\
        \end{align*}

      \item[84]
        \begin{parts}

          \part
            \begin{align*}
              \frac{1}{2} (1 - \cos \theta) & = 0 \\
              \theta                        & = \boxed{ 0 } \\
            \end{align*}

          \part
            \begin{align*}
              \frac{1}{2} (1 - \cos \theta) & = \frac{1}{4} \\
              \theta                        & = \boxed{ \pm \frac{\pi}{3} } \\
            \end{align*}

          \part
            \begin{align*}
              \frac{1}{2} (1 - \cos \theta) & = \frac{1}{2} \\
              \theta                        & = \boxed{ \pm \frac{\pi}{2} } \\
            \end{align*}

          \part
            \begin{align*}
              \frac{1}{2} (1 - \cos \theta) & = 1 \\
              \theta                        & = \boxed{ \pi } \\
            \end{align*}

        \end{parts}

    \end{description}

  \else
    \vspace{5 cm}

    \begin{quote}
      \begin{em}
      \end{em}
    \end{quote}
    \hspace{1 cm} --TO DO
  \fi

\end{document}

