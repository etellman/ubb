\documentclass[fleqn, onecolumn]{article}

\usepackage{fullpage}
\usepackage{graphicx}
\usepackage{float}
\usepackage{amsmath}
\usepackage{amssymb}
\usepackage{polynom}
\usepackage{caption}
\usepackage{mdwlist}
\usepackage{parskip}

\newcommand{\degree}{\ensuremath{^\circ}} 

\everymath{\displaystyle}
\setlength{\mathindent}{1 cm}

\author{Ed Tellman}
\title{Math 142---Precalculus II}
\date{\today}

\begin{document}

  \maketitle

  \section{Overview}
  This course will cover chapters 5-7 of {\em Precalculus}, by Stewart, Redlin, and Watson.  The entire course is about
  trigonometry---each chapter covers a different trigonometry topic.

  This course assumes that you are familiar with functions and the other material covered in math 141.  You don't have
  to have taken math 141, but ideally you will have taken something similar some time in the past.

  \section{Homework and Exams}

  You should expect to spend three or four hours each week doing homework.  Feel free to work together with other students
  on the homework.

  Each chapter will be followed by an in-class test.  

  \section{Credit}
  Anyone who completes the course and is interested in receiving credit can take the final exam from Edmonds Community
  College.  Unfortunately, the limited UBB budget doesn't cover the tuition, so you have to pay the fee for the exam
  yourself if you want to take it.

  \section{Topics}

  \subsection{Chapter 5---Trigonometric Functions of Real Numbers}

  One category of problem that can be solved with trigonometry is ``repetitive motion'' problems.  Examples of
  repetitive motion problems are turning wheels, pendulums, waves in water, sound, fluctuating populations of animals.

  In this chapter we'll define the trigonometric functions in a way that is useful for solving this type of problem and
  learn how to use trigonometry to solve repetitive motion problems.

  % The simplest way to define trigonometric functions is as points on a circle with a radius of one.

  % Topics include:
  % \begin{itemize*}
  %   \item Trigonometric functions (sine, cosine, tangent, etc.)
  %   \item Graphing 
  %   \item Applications 
  % \end{itemize*}

  \subsection{Chapter 6---Trigonometric Functions of Angles}

  The other category of problems that can be solved with trigonometry is calculating sizes, distances and directions.
  Examples of this type of problem are surveying, construction, and navigation.

  In this chapter we'll provide alternate but equivalent definitions for the trigonometric functions and show how to use
  them to solve distance/angle problems.

  % Topics include:
  % \begin{itemize*}
  %   \item Trigonometric functions of angles
  %   \item Law of Sines
  %   \item Law of Cosines
  % \end{itemize*}

  \subsection{Chapter 7---Analytic Trigonometry}

  There are numerous formulas for transforming trigonometric functions which allow you to convert an equation into a
  simpler form which is easier to work with.  In this chapter we'll learn about many of these transformations and how to
  use them to simplify and solve trigonometric equations.

  % Topics include:
  % \begin{itemize*}
  %   \item Addition and subtraction formulas
  %   \item Double-angle, half-angle, and sum-product formulas
  %   \item Inverse trigonometric functions
  %   \item Trigonometric equations
  % \end{itemize*}

\end{document}

