\documentclass[letterpaper, landscape]{exam}
\usepackage{2in1, lscape} 
\printanswers{}

\usepackage{units} 
\usepackage{parskip} 
\usepackage{xfrac} 
\usepackage[fleqn]{amsmath}
\usepackage{cancel}
\usepackage{float}
\usepackage{mdwlist}
\usepackage{booktabs}
\usepackage{cancel}
\usepackage{polynom}
\usepackage{caption}
\usepackage{fullpage}
\usepackage{comment}
\usepackage{enumerate}
\usepackage{graphicx}
\usepackage{mathtools} 
\usepackage{commath}

\everymath{\displaystyle}

\title{Statistics \\ Week Fifteen}
\date{\today}
\author{}

\begin{document}

  \maketitle
  \tableofcontents

  \section{Inference Cautions}
  You can't use inference when your sample is not an SRS or the outcome of a
  randomized controlled experiment.

  If your population doesn't have a Normal distribution, it's usually fine as
  long as you take a large enough sample. Because of the Central Limit Theorem,
  the values of $\bar{x}$ do have a Normal distribution if your sample is
  reasonably large.

  % \section{T-test}
  % Gosset was an employee of the Guinness Brewery published under name
  % ``Student'' so competitors might underestimate the value of the work.

  % If you don't have the standard deviation of the population, you can use the
  % standard deviation of the sample. You get a t-test instead of a z-test.

  % The probability density graph for a t-test is slightly different from a Normal
  % graph, with the differences diminishing as the sample size increases.

  % \section{Confidence Intervals}
  % A {\em Confidence Interval\/} is in the form $x \pm y$.

  % Associated with a confidence interval is:
  % \begin{description}
  %   \item[Confidence Level] is how confident you are that the interval
  %     captures the actual value you are looking for. Common values are 68\%,
  %     95\% and 99.7\%.

  %   \item[Margin of Error] the width of the interval
  % \end{description}

  % With a larger margin of error, you can be more confident that the interval
  % includes the actual value of the parameter.

  % \section{Significance}
  % ``Statistically significant'' means that you had a low P-value. You can be
  % confident that the result didn't happen by chance.

  % Statistically significant is not the same as practically significant. You may be
  % very confident in a difference of 1\% (statistically significant) but not care
  % because 1\% is such a small percentage (practical significant).

  % If the actual effect is small, you need a large sample size to get a
  % statistically significant result. This is because a large sample size results
  % in a small standard deviation, which allows you to pin down the effect you are
  % looking for.

  \section{Sample Size and Confidence Intervals}

  Calculate how many samples you need for a desired margin of error, $m$:

  \begin{align*}
    m        & = \frac{z^* \sigma}{\sqrt{n}} \\
    \sqrt{n} & = \frac{z^* \sigma}{m} \\
    n        & = \del{ \frac{z^* \sigma}{m} }^2 \\
  \end{align*}

  For example, suppose 
  \begin{align*}
    m      & = 4 \\
    \sigma & = 20 \\
  \end{align*}
  
  And you want to have 95\% confidence. 95\% is two standard deviations, so
  $z^* = 2$. 

  \begin{align*}
    n & = \del{ \frac{z^* \sigma}{m} }^2 \\
      & = \del{ \frac{2 \cdot 20}{4} }^2 \\
      & = 100 \\
  \end{align*}

  \section{Power}

  \subsection{Description} % (fold)
  
  The power of a test is the chance that it will correctly reject the null
  hypothesis when the null hypothesis is false by at least a specified amount.

  input to power calculation are:
  \begin{itemize*}
    \item population standard deviation ($\sigma$)
    \item sample size
    \item minimum effect size we want to detect
    \item significance level
  \end{itemize*}

  The power is roughly the chance that you will be in the ``significant'' part
  of the blue graph if the mean is actually the mean of the red graph.

  See Figures~\ref{fig:power1} through~\ref{fig:power3}. For all figures:
  \begin{align*}
    \mu_0 &= 0 \\
    \mu_a &= 3 \\
  \end{align*}

  The power is the chance of correctly detecting that the actual mean is at
  least 3 when the $H_0$ mean is 0.

  \begin{figure}[H]
    \centering
    \includegraphics[scale = 1.0]{power_sd_150}
    \caption{$sd = 1.5$, $z^* = 3$, $power = 0.5$}\label{fig:power1}
  \end{figure}

  \begin{figure}[H]
    \centering
    \includegraphics[scale = 1.0]{power_sd_100}
    \caption{$sd = 1.0$, $z^* = 2$, $power = 0.84$}
  \end{figure}
  
  \begin{figure}[H]
    \centering
    \includegraphics[scale = 1.0]{power_sd_075}
    \caption{$sd = 0.75$, $z^* = 1.5$, $power = 0.977$}\label{fig:power3}
  \end{figure}

  You can increase the power by either:
  \begin{itemize}
      
    \item looking for larger effect differences (move red graph to the right so
      there is less overlap)

    \item raising the significance level so that more of the blue graph is
      considered significant

    \item decreasing the sample distribution standard deviation by increasing
      the sample size (shrink the width of both graphs so there is less overlap)

  \end{itemize}

  When planning an experiment, use a computer to tell you what sample size you
  need for a desired power, significance level and minimum effect size.

  \pagebreak

  \subsection{Calculation} % (fold)
  
  exercise 48:

  \begin{enumerate}[(a)]
    \item 
      For this part, we use $\mu = 0$.

      Look up $z^*$ for a significance level of 0.05 in Table C. $z^* = 1.645$

    \item
      Still using $\mu = 0$.

      \begin{align*}
        \bar{x} & = \frac{1.645}{3.162} \\
                & = \boxed{ 0.5202 } \\
      \end{align*}

      Any $\bar{x}$ greater than \fbox{ 0.52 } will reject the hypothesis.

    \item 
      For this part, we use $\mu = 0.8$.

      $P(\bar{x} > 0.5202)$ when $\mu = 0.8$ is:
      \begin{align*}
        z & = \frac{0.5202 - 0.8}{1/\sqrt{10}} \\
          & \approx -0.8848 \\
      \end{align*}

      From Table A:\@ $P(z > -0.8848) \approx \boxed{ 0.8106 }$

      The power of the test is 0.8106. There is an 81\% chance of correctly
      rejecting the null hypothesis if the actual mean is at least 0.8.

  \end{enumerate}

  % The general formula for power is something like:
  % \begin{align*}
  %   z_{effect} & = \frac{effect}{\sigma/\sqrt{n}} \\
  %   z_{\alpha} & = qnorm(0.05) \\
  %   \\
  %   P(\text{reject } H_0 | H_0 \text{ false}) 
  %       & = pnorm(z_{\alpha} + z_{effect}) \\
  % \end{align*}

  You can get high power (high confidence in the outcome) by:
  \begin{itemize*}
    \item larger sample size
    \item null hypothesis very wrong so there is a large difference between the
      sample and the actual parameter (higher $z_{effect}$)
    \item lower desired significance level (higher $z_{\alpha}$)
  \end{itemize*}

  \pagebreak

  \section{Type I and II Errors}

  \begin{description*}
    \item[Type I] reject $H_0$ when it is actually true.
    \item[Type II] accept $H_0$ when it is actually false.
  \end{description*}

  \begin{tabular}[H]{lll}
    \toprule
    $H_0$ & reject          & accept \\
    \midrule
    true  & Type I Error    & correct outcome \\
    false & correct outcome & Type II error \\
    \bottomrule
  \end{tabular}

  Significance is probability of a Type I error. It's the chance that when
  $H_0$ is true you will incorrectly reject think it is false because you
  happened to have an unusual sample.
  
  With a significance of 0.05, there is a 5\% chance that when $H_0$ is true,
  the test will incorrectly think it is false.

  Power is $1 - P(\text{ Type II error })$. It's the chance that when $H_0$ is
  false, the test will incorrectly think it is true. 

  With a power of 0.95, there is a 5\% chance that when $H_0$ is false, the test
  will incorrectly think it is true.

  % \section{Check Your Skills}
  % \begin{description}
  %   \item[15.1] c---not a random sample

  %   \item[15.2] c---not a random sample
  %     \begin{enumerate}[(a)]
  %       \item 
  %         For a 95\% confidence interval, we need to within two standard
  %         deviations of the mean.

  %         \begin{align*}
  %           \sigma_{\bar{x}} & = \frac{1.83}{\sqrt{880}} \\
  %                            & \approx 0.06169 \\
  %           \\
  %           z^* &\approx 2 \\
  %           \\
  %           \mu & = 1.92 \pm 2 \cdot 0.06169 \\
  %               & \approx 1.92 \pm 0.1209 \\
  %               & \approx (1.80, 2.04) \\
  %         \end{align*}

  %       \item The distribution of the sample means is still normal since we have
  %         a large sample (Central Limit Theorem).

  %       \item 
  %         \begin{itemize*}
  %           \item it only includes people with phones
  %           \item it only includes people who answer their phone
  %         \end{itemize*}
  %     \end{enumerate}

  %   \item[15.3] Different times of day get different types of people.

  %   \item[15.4] 
  %     \begin{enumerate}[(a)]
  %       \item

  %         \begin{table}[ht]
  %         \centering
  %         \begin{tabular}{rrrrrrr}
  %           \toprule
  %             & p    & z    & s.z  & margin.of.error & min   & max \\
  %           \midrule
  %           1 & 0.90 & 1.64 & 0.29 & 0.48            & 26.32 & 27.28 \\
  %           2 & 0.95 & 1.96 & 0.29 & 0.57            & 26.23 & 27.37 \\
  %           3 & 0.99 & 2.58 & 0.29 & 0.76            & 26.04 & 27.56 \\
  %           \bottomrule
  %         \end{tabular}
  %         \end{table}
  %     \end{enumerate}

  % \end{description}
\end{document}

