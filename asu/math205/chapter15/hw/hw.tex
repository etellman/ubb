% no answer key
\documentclass[letterpaper]{exam}

% answer key
% \documentclass[letterpaper, landscape]{exam}
% \usepackage{2in1, lscape} 
% \printanswers{}

% for the cent symbol
\usepackage{textcomp}

% the textcent command eats the space following the symbol
\usepackage{xspace}
\newcommand{\cent}{\textcent\xspace}

\usepackage{units} 
\usepackage{xfrac} 
\usepackage[fleqn]{amsmath}
\usepackage{cancel}
\usepackage{float}
\usepackage{mdwlist}
\usepackage{booktabs}
\usepackage{cancel}
\usepackage{polynom}
\usepackage{caption}
\usepackage{fullpage}
\usepackage{comment}
\usepackage{enumerate}
\usepackage{graphicx}
\usepackage{parskip}

\everymath{\displaystyle}


\title{Statistics \\ Chapter 15 Homework}
\date{\today}
\author{}

\begin{document}

  \maketitle

  \section{Homework}
  \begin{itemize*}
    \item Chapter 15: 34--38, 41--43, 45--46
    \item Chapter 16: review exercises to prepare for test (don't hand in)
  \end{itemize*}

  \ifprintanswers{}
    \section{Solutions}
    \begin{description}

      \item[34] 

        The states are all different populations, so averaging the average for
        each state won't give an accurate global average. If you average the
        averages, you are counting Montana the same as California, for example,
        but there are more people in California than Montana.

      \item[35] If the sample is small, a single outlier will have a big effect.
        With a large sample, the outlier won't matter much. 

      \item[36] 
        \begin{enumerate}[(a)]
          \item 1999 was an outlier which will throw the average off.
          \item The values are generally decreasing over time.
        \end{enumerate}

      \item[37] B, A, C.

        \begin{itemize*}
          \item The strongest evidence is the well designed studies.
          \item The second strongest evidence is the poorly designed studies.
          \item The least strong evidence is expert opinion.
        \end{itemize*}

      \item[38]
        \begin{enumerate}[(a)]
          \item A poorly-designed study could still have a statistically
            significant result.

          \item Significance will help here. This is what significance is
            intended to signify.

          \item A very small and unimportant effect could still be statistically
            significant.

        \end{enumerate}

      \item[41] 
        \begin{enumerate}[(a)]
          \item 
            \begin{align*}
              \bar{x}          & = 9.524 \\
              \sigma_{\bar{x}} & \approx 0.8944 \\
              z                & \approx \frac{9.524 - 8}{0.8944} \\
                               & \approx 1.7039 \\
              \\
              p-value          & = 0.9743 \\
            \end{align*}

          \item The sample size was too small.
        \end{enumerate}

      \item[42] 
        \begin{enumerate}[(a)]
          \item It is unlikely that the difference can be explained by chance.

          \item If we did the same study 100 times, 95 of them would probably
            provide similar results.

          \item No. The mothers self-selected into each group, so the difference
            in the results may be because the mothers who signed up for the
            program are different from the mothers who didn't sign up.
        \end{enumerate}
        
      \item[43]
        \begin{enumerate}[(a)]
          \item Any variation might easily have been the result of chance in the
            selection of the sample rather than actual differences.

          \item Not really. If the results may have been the result of chance
            then it doesn't really matter how big they were.
        \end{enumerate}

      \item[45]
        \begin{enumerate}[(a)]
          \item There is only a 20\% chance that the test will correctly reject the
            null hypothesis when the null hypothesis is in fact false.

          \item With low power, there is only a small chance of detecting that
            the null hypothesis is false when it is in fact false.

          \item There weren't enough mice in each group and each mouse had very
            different sleep patterns from the others so the standard deviation
            was large.

        \end{enumerate}

      \item[46]
        \begin{enumerate}[(a)]
          \item 96\%

          \item The test has a 90\% chance of detecting the difference if it is
            present.
        \end{enumerate}

  \end{description}

  \else
    \vspace{12 cm}
    \begin{quote}
      \begin{em}
        Everything goes, everything returns; eternally rolls the wheel of
        being.
      \end{em}
    \end{quote}
    \hspace{1 cm}--Frederich Nietzsche
  \fi

\end{document}

