% no answer key
% \documentclass[letterpaper]{exam}

% answer key
\documentclass[letterpaper, landscape]{exam}
\usepackage{2in1, lscape} 
\printanswers{}

\usepackage{units} 
\usepackage{xfrac} 
\usepackage[fleqn]{amsmath}
\usepackage{cancel}
\usepackage{float}
\usepackage{booktabs}
\usepackage{polynom}
\usepackage{caption}
\usepackage{fullpage}
\usepackage{comment}
\usepackage{enumitem}
\usepackage{graphicx}
\usepackage{parskip}

\everymath{\displaystyle}

\title{Statistics \\ Chapter 18 Homework}
\date{\today}
\author{}

\begin{document}

  \maketitle

  \section{Homework}
  Chapter 18: 25--29, 31--34, 38--41

  \ifprintanswers{}
    \section{Solutions}
    \begin{description}

      \item[25] 
        \begin{enumerate}[label = {(\alph*)}]
          \item The hypotheses are:
            \begin{itemize}[itemsep = 0pt, label = {}]
              \item $H_0$: $\mu_m = \mu_w$
              \item $H_a$: $\mu_w > \mu_m$
            \end{itemize}

          \item
            \begin{align*}
              t_1 & \approx -0.2484 \\
              t_2 & \approx 1.5069 \\
            \end{align*}

          \item 55 and 19

          \item
            In the first case, the theory was that women talked more than men
            but the experiment showed that the men talked more than the women. 

            In the second case: $0.05 < P < 0.1$. This is weak evidence against
            $H_0$.

          \item There's little reason to think women talk more than men, based
            on this experiment.

        \end{enumerate}

      \item[26]
        \begin{enumerate}[label = {(\alph*)}]
          \item 
            \begin{align*}
              s_u & = 21 \\
              s_r & \approx 33.17 \\
            \end{align*}

          \item $9 - 1 = 8$

          \item 
            \begin{align*}
              \mu_1 - \mu_2 & \approx 59 - 32 \pm 1.860 \cdot 12.2066 \\
                            & = \boxed{ 27 \pm 22.7 } \\
                            & = \boxed{ (4.301, 49.699) } \\
            \end{align*}

        \end{enumerate}

      \item[27]
        \begin{enumerate}[label = {(\alph*)}]
          \item 
            \begin{align*}
              s_1 &= 1.56 \cdot \sqrt{6} \approx 3.8212 \\
              s_2 &= 2.68 \cdot \sqrt{7} \approx 7.0906 \\
            \end{align*}

            \begin{tabular}[ht]{clrrr}
              \toprule
              Group & Location   & $n$ & $\bar{x}$ & $s$ \\
              \midrule
              1     & Oregon     & 6   & 26.9      & 3.8212 \\
              2     & California & 7   & 11.9      & 7.0906 \\
              \bottomrule
            \end{tabular}

          \item 5 degrees of freedom

          \item $t = 4.8372$
            
          \item $0.002 < P < 0.005$. This doesn't exactly agree with the report.

        \end{enumerate}

      \item[28]
        \begin{enumerate}[label = {(\alph*)}]
          \item If the parents selected the school, we wouldn't have a random
            sample. There's probably a difference between parents who choose
            Montessori schools and other parents.

          \item There might be some difference between the parents who could be
            found and the other parents.

          \item $t = 1.9758$. With a 2-sided test: $0.05 < P < 0.1$.

            This isn't a statistically significant difference.

        \end{enumerate}

      \item[29]
        \begin{enumerate}[label = {(\alph*)}]
          \item If you don't give a placebo, one group or the other might
            perform better or worse because they know whether they were given
            the ginkgo, even if the ginkgo has no actual effect.

          \item Neither the experimenters nor the subjects should know which
            group they are in to avoid subconscious bias.

          \item 
            Use a 2-sided test since we don't know what kind of effect ginkgo will
            have.

            $t = 2.147$ and $0.04  < P < 0.05$. 

            There is statistically significant evidence, $P < 0.05$, that the
            ginkgo makes people's memory worse, not better.

        \end{enumerate}

      \item[31] $t = 0.7001$ and $P > 0.5$. Everybody seems to be equally
        stressed.

      \item[32]
        \begin{enumerate}[label = {(\alph*)}]
          \item The matched pairs test is correct since the same student is
            taking the test twice. For the two-sample test, you need two
            independent random samples.

          \item 
            \begin{enumerate}[label = (\roman*)]
              \item If you answered ``matched pair'' for part a:

                The hypotheses are:
                \begin{itemize}[itemsep = 0pt, label = {}]
                  \item $H_0$: $\mu_{gain} = 0$
                  \item $H_a$: $\mu_{gain} > 0$
                \end{itemize}

                $t = 10.16$ and $P < 0.001$.  
                
                There is a significant difference: $P < 0.0005$.

              \item If you answered ``two sample'' for part a:

                The hypotheses are:
                \begin{itemize}[itemsep = 0pt, label = {}]
                  \item $H_0$: $\mu_{1} = \mu_{2}$
                  \item $H_a$: $\mu_{2} > \mu_{1}$
                \end{itemize}

                $t = 4.4824$ and $P < 0.0005$.  
                
                There is a significant difference: $P < 0.0005$.

            \end{enumerate}
          \item 
            \begin{align*}
              t^* & = 2.626 \\
              se  & = \frac{59}{\sqrt{427}} \approx 1.4034 \\
              \\
              \mu & = 29 \pm 2.626 \cdot 1.4034 \\
                  & = \boxed{ 29 \pm 7.5 } \\
                  & = \boxed{ (21.50, 36.5) } \\
            \end{align*}
        \end{enumerate}

      \item[33]
        \begin{enumerate}[label = {(\alph*)}]
          \item To find out if the coaching works, we need to do a 2-sample
            T-test. 
            
            $t = 2.6459$ and $0.0025 < P < 0.005$. 

            There is a significant difference, $P < 0.005$, between the
            coaching and no-coaching groups.

          \item 
            \begin{align*}
              \mu_{c} - \mu_{uc} & \approx 8 \pm 2.626 \cdot 3.0235 \\
                                 & \approx \boxed{ 8 \pm 7.9397 } \\
                                 & \approx \boxed{ (0.0603, 15.9397) } \\
            \end{align*}

          \item You are going to gain at most 16 points, and some of that you'd
            probably gain anyway (see problem 34), so probably not.

        \end{enumerate}

      \item[34] If you're willing and able to pay for coaching, you're likely to
        be the kind of student who would do a lot of practicing anyway. It may be
        the practicing rather than the coaching which helped you.

      \item[38]
        \begin{enumerate}[label = {(\alph*)}]
          \item 
            The Permafresh process seems to be stronger: 
            
            $\bar{x}_{pf} = 29.54$ and $\bar{x}_{hl} = 25.2$ 

          \item 27.6 is the outlier

          \item 

            \begin{enumerate}[label = {(\roman*)}]
              \item With all the data:

                \begin{tabular}[H]{lrr}
                  \toprule
                             & $\bar{x}$ & $s$ \\
                  \midrule
                  Permafresh & 29.54     & 1.675 \\
                  Hylite     & 25.20     & 2.6693 \\
                  \bottomrule
                \end{tabular}

                $t = 3.3310$ and $0.02 < P < 0.04$

              \item Without the outlier:

                \begin{tabular}[ht]{lrr}
                  \toprule
                              & $\bar{x}$ & $s$ \\
                  \midrule
                  Permafresh  & 30.025    & 0.4992 \\
                  Hylite      & 25.20     & 2.6693 \\
                  \bottomrule
                \end{tabular}

                $t = 3.9564$. The degrees of freedom goes down to three, so the
                P-value doesn't change: $0.02 < P < 0.04$

            \end{enumerate}

            Either way, there is a statistically significant difference and
            Permafresh is stronger.
        \end{enumerate}

        \item[39]
          \begin{enumerate}[label = {(\alph*)}]
            \item The two means are 134.8 (Permafresh) and 143.2 (Hylite). It
              looks like Hylite does a better job with the wrinkles.

            \item $t = 6.2961$ and $0.002 < P < 0.005$. There is a
              statistically significant difference, $P < 0.005$.

          \end{enumerate}

        \item[40]
          $t^* = 2.132$ and $s \approx 1.3029$.
          \begin{align*}
            \mu_{pf} - \mu_{hy} & \approx 4.34 \pm 2.132 \cdot 1.3029 \\
                                & \approx \boxed{ 4.34 \pm 2.7778 } \\
          \end{align*}

        \item[41]
          $t^* = 2.132$ and $s \approx 1.3342$.
          \begin{align*}
            \mu_{hy} - \mu_{pf} & \approx 8.34 \pm 2.132 \cdot 1.3342 \\
                                & \approx \boxed{ 8.34 \pm 2.8445 } \\
          \end{align*}


  \end{description}

  \else
    \vspace{11 cm}
    \begin{quote}
      \begin{em}
        I don't usually deal with those big words because I don't usually deal
        with big people. I deal with small people. I find you can get a whole
        lot of small people and whip hell out of a whole lot of big people.
        They haven't got anything to lose, and they've got everything to gain.
      \end{em}
    \end{quote}
    \hspace{1 cm}--Malcolm X
  \fi

\end{document}

