% no answer key
\documentclass[landscape]{exam}
\usepackage{2in1, lscape} 

% \documentclass[letterpaper, landscape]{exam}

% answer key
\printanswers{}

\usepackage{units} 
\usepackage{graphicx}
\usepackage[fleqn]{amsmath}
\usepackage{mathtools}
\usepackage{cancel}
\usepackage{float}
\usepackage{mdwlist}
\usepackage{booktabs}
\usepackage{polynom}
\usepackage{caption}
\usepackage{fullpage}
\usepackage{comment}
\usepackage{enumerate}
\usepackage{parskip}
\usepackage{xfrac}
\usepackage{commath}
\usepackage{todonotes}

\DeclarePairedDelimiter{\ceil}{\lceil}{\rceil}

\addpoints{}

\title{Statistics \\ Part Two Exam}
\date{\today}
\author{}

\begin{document}

  \maketitle

  % \begin{center}
  %   \gradetable[h][pages]
  % \end{center}

  \ifprintanswers{}
  \else
    \section{Instructions}
    Show all your work and explain your answers---a correct guess won't receive
    credit without work/explanation. 
    
    An incorrect answer will receive partial credit if you understand the
    concept but made a mistake or two while solving the problem.
  \fi

  \section{Questions}


  \begin{questions}

    \question[5] A fair dice is rolled three times.
      \begin{parts}
        \part[5] What is the probability of all three rolls coming up 6?
        \begin{solution}
          \begin{align*}
            \del{ \frac{1}{6} }^3 \approx \boxed{ 0.004630 }
          \end{align*}
        \end{solution}

        \part[5] What is the probability of at least one 6 in the three rolls?

        \begin{solution}

          The probability of no sixes coming up is:
          \[
            P(\text{no sixes}) = \del{ \frac{5}{6} }^3 
          \]

          The probability of at least one six coming up is the complement of no
          sixes:
          \[
            1 - \del{ \frac{5}{6} }^3 \approx \boxed{ 0.4213 }
          \]

        \end{solution}

        \part[5] What is the probability of exactly two out of the three rolls
        coming up 6?

        \begin{solution}
          The probability of the first two rolls coming up six and the remaining
          two coming up not six is:

          \[
            P(\text{first two six}) = \cdot \frac{1}{6} \cdot \frac{1}{6} \cdot \frac{5}{6} 
          \]

          The probability of the second or third roll being the single non-six
          die is the same. Since these are disjoint events, you can add them to
          get the overall probability:

          \[
            3 \cdot \frac{1}{6} \cdot \frac{1}{6} \cdot \frac{5}{6} 
                \approx \boxed{ 0.06944 }
          \]
        \end{solution}
      \end{parts}

    % \question[5] A lottery has these rules: You buy a ticket and select three
    % different numbers between 1 and 100 which you write on your ticket and hand
    % in. Three balls are drawn at random without replacement. If the numbers on
    % the balls match the numbers on your ticket, you win (order doesn't matter).
    % What is your chance of winning?

    % \begin{solution}
    %   \begin{align*}
    %     \frac{3}{100} \cdot \frac{2}{100} \cdot \frac{1}{100} = 0.06
    %   \end{align*}
    % \end{solution}

    \question[8] If boys and girls are equally likely, and the gender of each
    child is independent of the gender of the other children, what is the probability
    of a four child family having more girls than boys?

    \begin{solution}
      There are 16 equally likely possibilities. 1 of these is ``four girls''
      and four of them are ``three girls'' (the single boy could be any one of
      the kids).  So the probability of having more girls than boys is
      \sfrac{5}{16}.

      Another way to look at it is that the probability of any particular
      combination of girls and boys is:
      \[
        \frac{1}{2} \cdot \frac{1}{2} \cdot \frac{1}{2} \cdot \frac{1}{2} = \frac{1}{16}
      \]

      Since there are 5 combinations that work, the probability of having more
      girls than boys is:
      \[
        P(g > b) = 5 \cdot \frac{1}{16} = \boxed{ \frac{5}{16} }
      \]

    \end{solution}

    \question{}
    Twins come in two flavors: identical and fraternal. Table~\ref{tab:twins}
    shows the probability of each boy/girl fraternal/identical combination.

    \begin{table}[H]
      \centering
      \begin{tabular}{lrrrr}
        \toprule
                  & boy/boy      & boy/girl     & girl/boy     & girl/girl \\
        \midrule
        Identical & \sfrac{1}{6} & 0            & 0            & \sfrac{1}{6} \\
        Fraternal & \sfrac{1}{6} & \sfrac{1}{6} & \sfrac{1}{6} & \sfrac{1}{6} \\
        \bottomrule
      \end{tabular}\caption{Twins}\label{tab:twins}
    \end{table}

    \begin{parts}
      % \part[5] If after having a sonogram you learn you are having twins, what
      % is the probability of having identical twins?

      % \begin{solution}
      %   \sfrac{1}{3} of the twins are identical, so the probability is 0.3333
      % \end{solution}

      \part[5] What is: $P(\text{Two Girls } | \text{ Fraternal})$ 

      \begin{solution}
        % One out of the four fraternal choices is girl/girl, so the probability
        % is 0.25.

        \begin{align*}
          P(\text{2G } | \text{ F}) & = \frac{P(\text{2G and F})}{P(\text{F})} \\
                                    & = \frac{ \sfrac{1}{6} }{ \sfrac{2}{3} } \\
                                    & = \boxed{ 0.25 } \\
        \end{align*}
      \end{solution}

      \part[5] What is: $P(\text{Fraternal } | \text{ Two Girls})$ 

      \begin{solution}
        % One out of the two girl/girl choices is fraternal, so the probability
        % is 0.5.

        \begin{align*}
          P(\text{F } | \text{ 2G}) & = \frac{P(\text{2G and F})}{P(\text{2G})} \\
                                    & = \frac{ \sfrac{1}{6} }{ \sfrac{1}{3} } \\
                                    & = \boxed{ 0.5 } \\
        \end{align*}
      \end{solution}
    \end{parts}

    \question[5] A widget manufacturer has two plants. 
    
    \begin{itemize*}
      \item Plant A makes 400 widgets per day, with 3 defective widgets on
        average, each day.
      \item Plant B makes 800 widgets per day, with 2 defective widgets on
        average, each day.
    \end{itemize*}

    If one of the 1200 widgets manufactured in a particular day is selected and
    found to be defective, what is the probability it was manufactured in Plant
    A\@?

    \begin{solution}
      \begin{align*}
        P(\text{plant A} | \text{ defective}) & = \frac{P(\text{plant A and defective})}{P(\text{defective})} \\
                                              & = \frac{ \sfrac{3}{1200} }{ \sfrac{5}{1200} } \\
                                              & = \boxed{ 0.6 } \\
      \end{align*}

      Or you could ignore the volume and notice that 3 out of 5 of the defective
      widgets made every day come from Plant A which also gives you a
      probability of 0.6 for a defective widget coming from plant A.
    \end{solution}

    \ifprintanswers{}
    \else
      \newpage
    \fi

    \question[5]
      When planning for Halloween, you want to keep all of your
      trick-or-treating customers satisfied. You theorize that candy preferences
      differ depending on the age and gender of the consumer.  You decide to
      conduct a poll of likely trick-or-treaters to help you decide what candy
      to buy. 
      
      Describe an experiment design which is likely to lead to your purchasing
      the correct types of candy, assuming you know from past years about how
      many people from each age/gender group are likely to show up. 

      \begin{solution}
        A block design makes sense for this problem. The procedure would be:

        \begin{itemize}
          \item Find a large group of likely trick-or-treaters.

          \item Divide them up into groups based on gender and age.

          \item Randomly select a sample from each group, where the size of the
            sample matches the expected proportion of kids of that type

          \item Poll each sample for candy preferences.

          \item Go to the store and buy candy. Make sure you also get something
            you like. Don't hand the ones you like out to the customers
            except as a last resort when you run out of everything else that
            they probably like better after all the effort you put into keeping
            them happy. 

        \end{itemize}

      \end{solution}

    \question[12]
      You want to determine if a coin is biased towards heads or tails, so you
      flip it 100 times and discover that it comes up heads 57 times. Is this
      evidence at the 0.1 significance level that the coin is biased?

     Count heads as 1 and tails as 0, so for a fair coin
     \begin{align*}
       \mu    & = 0.5 \\
       \sigma & = 0.5 \\
     \end{align*}

      \begin{solution}
        Since you don't know in advance which direction the coin might be
        biased towards, you need to use a two sided test.

        The hypotheses are:
        \begin{itemize*}
          \item $H_0$: $\mu = 0.5$
          \item $H_a$: $\mu \ne 0.5$
        \end{itemize*}

        \begin{align*}
          s & = \frac{0.5}{\sqrt{100}} = 0.05 \\
          z & = \frac{0.57 - 0.5}{0.05} = 1.4 \\
          \\
          P(z > 1.4) & \approx 0.08 \\
        \end{align*}

        Since this is a two-sided test, the P-level is 0.16. This doesn't
        rule out the null hypothesis at the 0.1 significance level. A result
        like this has a 16\% chance of happening by chance, even if the coin is
        fair.

      \end{solution}


    \uplevel{
      \hrulefill{}

      Questions~\ref{q:bats_first} through~\ref{q:bats_last} refer to
      Table~\ref{tab:bats}.

      Bats detect insects through echolocation. Scientists have determined that
      detection distances are approximately Normal with standard deviation
      of $\sigma = \unit[18]{cm}$.

      \begin{table}[H]
        \centering
        \begin{tabular}{lrrrrrrrrrrr}
          \toprule
          Sample    & 1  & 2  & 3  & 4  & 5  & 6  & 7  & 8  & 9  & 10 & 11 \\
          \midrule
          Dist (cm) & 62 & 52 & 68 & 23 & 34 & 45 & 27 & 42 & 83 & 56 & 40 \\
          \bottomrule
        \end{tabular}
        \caption{Bat/insect detection distances}\label{tab:bats}
      \end{table}
    }

    \question[10]\label{q:bats_first}
      Table~\ref{tab:bats} shows a simple random sample of bats and bug
      detection distances.  Use this sample to construct a 95\% confidence
      interval for the mean bat/insect detection distance.

      \begin{solution}
        \begin{align*}
          z^*     & = 1.960 \\
          s       & = \frac{18}{\sqrt{11}} \approx 5.4277 \\
          \bar{x} & \approx 48.36 \\
        \end{align*}

        The 95\% confidence interval is: 
        
        \fbox{ $\mu = \unit[48.36]{cm} \pm \unit[10.64]{cm}$ } or 
        \fbox{ 37.72 cm to 59 cm }

      \end{solution}
    \question{}
      A team of scientists has theorized that the mean insect detection distance
      is more than 40 cm. 

      \begin{parts}
        \part[5] What are the null and alternative hypotheses?

        \begin{solution}
          \begin{itemize*}
            \item $H_0$: $\mu = \unit[40]{cm}$
            \item $H_a$: $\mu > \unit[40]{cm}$
          \end{itemize*}

        \end{solution}

        \part[10]
          Does the sample in Table~\ref{tab:bats} provide statistically significant
          support for this theory at a significance level of 0.05?

          \begin{solution}
            \begin{align*}
              \bar{x} & \approx 48.36 \\
              s &= \frac{18}{\sqrt{11}} \approx 5.427 \\
              z       & = \frac{48.36 - 40}{5.417} \\
                      & \approx 1.54 \\
            \end{align*}

            This is a P-value of 0.0617, so the data don't support the
            hypothesis at the 0.05 significance level.
            
          \end{solution}

      \end{parts}
          
    \question[5]\label{q:bats_last} How large a sample of bats/insects would you
    need in order to provide a 95\% confidence interval of 
    $\mu \pm \unit[2]{cm}$?

    \begin{solution}
      For a 95\% confidence interval, $z^* = 1.6449$

      \[
        n = \del{ \frac{1.960 \cdot 18}{ 2 } }^2 \approx 311.16  
      \]

      Since you can't have a fractional number of samples, you'll need 
      \fbox{ 312 samples } to get the desired precision.

    \end{solution}

    \question{} The divination (predicting the future, making prophecies,
    reading crystal balls, etc.) professor at Hogwarts School of Witchcraft
    and Wizardry has a difficult job constructing an exam. Predicting
    the future is an uncertain science, even for skilled practitioners.

    The exam she came up with allows students to demonstrate proficiency by
    making a sequence of predictions of the future. They pass the test by
    showing a statistically significant result at significance level 0.05 
    in an experiment with a power 0.98 to detect a passing level of magic.

    \begin{solution}
      The hypotheses are:
      \begin{itemize*}
        \item $H_0$ the student can't predict the future
        \item $H_a$ the student can predict the future
      \end{itemize*}
    \end{solution}

    \begin{parts}
      \part[5] What is the probability of a student who can't predict
      the future passing the exam through a sequence of lucky guesses?

      \begin{solution}
        in this case $H_0$ is true but we incorrectly conclude it is false. The
        probability of this happening is the significance level: \fbox{ 0.05 }.
      \end{solution}

      \part[5] What is the probability of a student who can predict the future
      failing the exam through an unfortunate sequence of slightly inaccurate
      predictions? 

      \begin{solution}
        in this case $H_a$ is true but we are unable to detect this. The
        probability of this happening is the complement of the power: 
        \fbox{ 0.02 }.
      \end{solution}

    \end{parts}

  \end{questions}

\end{document}

