% no answer key
\documentclass[letterpaper, landscape]{exam}
\usepackage{2in1, lscape} 

% answer key
% \printanswers{}

\usepackage{units} 
\usepackage{graphicx}
\usepackage[fleqn]{amsmath}
\usepackage{cancel}
\usepackage{float}
\usepackage{mdwlist}
\usepackage{booktabs}
\usepackage{polynom}
\usepackage{caption}
\usepackage{fullpage}
\usepackage{comment}
\usepackage{enumerate}
\usepackage{parskip}
\usepackage{xfrac}
\usepackage{commath}

\newcommand{\degree}{\ensuremath{^\circ}} 
\everymath{\displaystyle}

% \printanswers
\excludecomment{comment}
\addpoints{}

\title{Statistics \\ Part Two Exam}
\date{\today}
\author{}

\begin{document}

  \maketitle

  \begin{center}
    \gradetable[h][pages]
  \end{center}

  \ifprintanswers{}
  \else
  \section{Instructions}
  Show all your work and explain your answers---a correct guess won't receive
  credit without work/explanation. 
  
  An incorrect answer will receive partial credit if you understand the
  concept but made an arithmetic mistake.

  \fi

  \section{Questions}

  \begin{questions}

    \question[5] A fair dice is rolled three times.
      \begin{parts}
        \part[5] What is the probability of all three rolls coming up 6?
        \begin{solution}
          \begin{align*}
            \del{ \frac{1}{6} }^3 \approx 0.004630
          \end{align*}
        \end{solution}

        \part[5] What is the probability of at least one 6 in the three rolls?
        \begin{solution}
          \begin{align*}
            1 - \del{ \frac{5}{6} }^3 \approx 0.4213
          \end{align*}
        \end{solution}

        \part[5] What is the probability of exactly two out of the three rolls
        coming up 6?

        \begin{solution}
          There are three choices for the non-six die:
          \begin{align*}
            3 \cdot \frac{1}{6} \cdot \frac{1}{6} \cdot \frac{5}{6} 
              \approx 0.06944
          \end{align*}
        \end{solution}
      \end{parts}

    % \question[5] A lottery has these rules: You buy a ticket and select three
    % different numbers between 1 and 100 which you write on your ticket and hand
    % in. Three balls are drawn at random without replacement. If the numbers on
    % the balls match the numbers on your ticket, you win (order doesn't matter).
    % What is your chance of winning?

    % \begin{solution}
    %   \begin{align*}
    %     \frac{3}{100} \cdot \frac{2}{100} \cdot \frac{1}{100} = 0.06
    %   \end{align*}
    % \end{solution}

    \question[5] If boys and girls are equally likely, what is the probability
    of a four child family having more girls than boys?

    \begin{solution}
      The easy way of doing this one is to notice that none of the families have
      the same number of girls and boys since there is an odd number of
      children. Since boys and girls are equally likely, half the families have
      more girls than boys and the other half has more boys than girls. So the
      probability of more girls than boys is 0.5.

      Another way to do it is to enumerate the eight possibilities: GGG, GGB,
      GBG, GBB, BGG, BGB, BBG, BBB

      Four of the eight have more girls than boys.

      Another way is\dots

      Since any of the three children might be the boy in a one-boy family, the
      probability of exactly one boy is:
      \[
        3 \cdot 0.5^3 = 0.375
      \]

      The probability of all three boys is:
      \[
        0.5^3 = 0.125
      \]

      The probability of either of these things happening is:
      \[
        0.375 + 0.125 = 0.5
      \]

    \end{solution}

    \question{}
    Twins come in two flavors: identical and fraternal. Table~\ref{tab:twins}
    shows the probability of each boy/girl fraternal/identical combination.

    \begin{table}[H]
      \centering
      \begin{tabular}{lrrrr}
        \toprule
                  & boy/boy      & boy/girl     & girl/boy     & girl/girl \\
        \midrule
        Identical & \sfrac{1}{6} & 0            & 0            & \sfrac{1}{6} \\
        Fraternal & \sfrac{1}{6} & \sfrac{1}{6} & \sfrac{1}{6} & \sfrac{1}{6} \\
        \bottomrule
      \end{tabular}\caption{Twins}\label{tab:twins}
    \end{table}

    \begin{parts}
      \part[5] If after having a sonogram you learn you are having twins, what
      is the probability of having identical twins?

      \begin{solution}
        \sfrac{1}{3} of the twins are identical, so the probability is 0.3333
      \end{solution}

      \part[5] What is: $P(\text{Two Girls } | \text{ Fraternal})$ 

      \begin{solution}
        One out of the four fraternal choices is girl/girl, so this probability
        is 0.25.
      \end{solution}

      \part[5] What is: $P(\text{Fraternal } | \text{ Two Girls})$ 

      \begin{solution}
        The two girl/girl choices are equally likely, so this is 0.5
      \end{solution}
    \end{parts}

    \question[5] A widget manufacturer has two plants. 
    
    \begin{itemize*}
      \item Plant A makes 400 widgets per day, with 3 defective widgets on
        average, each day.
      \item Plant B makes 800 widgets per day, with 2 defective widgets on
        average, each day.
    \end{itemize*}

    If one of the 1200 widgets manufactured in a particular day is selected and
    found to be defective, what is the probability it was manufactured in Plant
    A\@?

    \begin{solution}
      3 out of 5 of the defective widgets made every day come from Plant A, so
      the probability is 0.6. The volume doesn't matter because you already know
      you have a defective widget.
    \end{solution}

    \newpage

    \question{}
      Bats detect insects through echolocation. Scientists have determined that
      detection distances are approximately Normal with standard deviation
      of $\sigma = \unit[18]{cm}$.

      \begin{table}[H]
        \centering
        \begin{tabular}{lrrrrrrrrrrr}
          \toprule
          Sample    & 1  & 2  & 3  & 4  & 5  & 6  & 7  & 8  & 9  & 10 & 11 \\
          \midrule
          Dist (cm) & 62 & 52 & 68 & 23 & 34 & 45 & 27 & 42 & 83 & 56 & 40 \\
          \bottomrule
        \end{tabular}
        \caption{Bat/insect detection distances}\label{tab:bats}
      \end{table}

      \begin{parts}
        \part[10]
          Table~\ref{tab:bats} shows a simple random sample of bats and bugs detection distances. 

          Use this sample to construct a 95\% confidence interval for the mean
          bat/insect detection distance.

        \part[10]
          A team of scientists has theorized that the mean insect detection distance
          is more than 40 cm. 
          
          Does the sample in Table~\ref{tab:bats} provide statistically significant
          support for this theory at a significance level of 0.05?

        \part[10] How large a sample would you need in order to provide a 95\%
          confidence interval of $\mu \pm \unit[2]{cm}$?

      \end{parts}

    \question{}

    \question{} The divination (predicting the future, making prophecies,
    reading crystal balls, etc.) professor at Hogwarts School of Witchcraft
    and Wizardry has a difficult job constructing an exam, since predicting
    the future is an uncertain science, even for skilled practitioners.

    The exam she came up with allows students to demonstrate proficiency by
    making a sequence of predictions of the future. They pass the test by
    showing a statistically significant result at significance level 0.05 
    in an experiment with power 0.98.

    \begin{parts}
      \part[5] What is the probability of a student who can't predict
      the future passing the exam through a sequence of lucky guesses?

      \part[5] What is the probability of a student who can predict the future
      failing the exam through an unfortunate sequence of slightly inaccurate
      predictions? 

    \end{parts}

  \end{questions}

\end{document}

