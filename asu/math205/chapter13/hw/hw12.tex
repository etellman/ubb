% no answer key
% \documentclass[letterpaper]{exam}

% answer key
\documentclass[letterpaper, landscape]{exam}
\usepackage{2in1, lscape} 
\printanswers{}

\usepackage{units} 
\usepackage{xfrac} 
\usepackage[fleqn]{amsmath}
\usepackage{cancel}
\usepackage{float}
\usepackage{mdwlist}
\usepackage{booktabs}
\usepackage{cancel}
\usepackage{polynom}
\usepackage{caption}
\usepackage{fullpage}
\usepackage{comment}
\usepackage{enumerate}
\usepackage{graphicx}
\usepackage{parskip}

\everymath{\displaystyle}

% the textcent command eats the space following the symbol
\usepackage{xspace}
\newcommand{\cent}{\textcent\xspace}


\title{Statistics \\ Homework Twelve}
\date{\today}
\author{}

\begin{document}

  \maketitle

  \section{Homework}

  Chapter 13: 22--26, 28, 31--32, 34--36, 38--39

  \ifprintanswers{}
    \section{Solutions}
    \begin{description}

      \item[22] 
        \begin{enumerate}[(a)]
          \item \fbox{ not binomial }. This is a count not a binary
            success/failure.

          \item \fbox{ binomial }

          \item \fbox{ binomial }
        \end{enumerate}

      \item[23]
        \begin{enumerate}[(a)]
          \item This one isn't appropriate because the sample size is small and
            the probability of success varies widely from attempt to attempt.

          \item This one is appropriate for the binomial distribution.
        \end{enumerate}

      \item[24]
        \begin{enumerate}[(a)]
          \item \fbox{ $n = 20$; $p = 0.25$ }

          \item $\mu = 20 \cdot 0.5 = \boxed{ 5 }$

          \item $ \binom{20}{5} 0.25^5 \cdot 0.75^{15} \approx \boxed{ 0.2023 } $
        \end{enumerate}

      \item[25]
        \begin{enumerate}[(a)]
          \item $n = 5$; $p = 0.65$

          \item 0 to 5

          \item 
            \begin{table}[ht]
              \centering
              \begin{tabular}{rr}
                \toprule
                years & probability \\
                \midrule
                0     & 0.0053 \\
                1     & 0.0488 \\
                2     & 0.1811 \\
                3     & 0.3364 \\
                4     & 0.3124 \\
                5     & 0.1160 \\
                \bottomrule
              \end{tabular}
            \end{table}

            see Figure~\ref{fig:ex25}

            \begin{figure}[H]
              \centering
              \includegraphics{ex25.pdf}
              \caption{Exercise 25}\label{fig:ex25}
            \end{figure}

          \item 
            \begin{align*}
              \mu    & = 0.65 \cdot 5 \\
                     & = \boxed{ 3.25 } \\
              \\
              \sigma & = \sqrt{5 \cdot 0.65 \cdot 0.35} \\
                     & \approx \boxed{ 1.0665 } \\
            \end{align*}
        \end{enumerate}

      \item[26]
        \begin{enumerate}[(a)]
          \item 
            Binomial distribution with $n = 12$ and $p = 0.5$
            % \begin{align*}
            %   \mu    & = $12 \cdot 0.5 \\
            %   = 6$ \\
            %   \sigma & = \sqrt{12 \cdot 0.5 \cdot 0.5} \\
            %   &\approx \boxed{ 1.7321 } \\
            % \end{align*}

          \item The probability of exactly 8 is:
            \[
              \binom{12}{8} 0.5^8 \cdot 0.5^4 \approx \boxed{ 0.1208 } 
            \]

            According to my computer, the probability of at least 8 is
            approximately 0.1938.

        \end{enumerate}

      % \item[27]
      %   \begin{enumerate}[(a)]
      %     \item There is a probability of success and a fixed sample size.

      %     \item 
      %       The probability of nobody getting pregnant under ideal
      %       conditions is:
      %       \[
      %         0.99^{20} \approx \boxed{ 0.8179 } 
      %       \]

      %       The probability of nobody getting pregnant under ideal
      %       conditions is:
      %       \[
      %         0.95^{20} \approx \boxed{ 0.3585 }
      %       \]
      %   \end{enumerate}

      \item[28]
        \begin{align*}
          \mu     & = 0.5 \cdot 500 \\
                  & = 250 \\
          \sigma  & = \sqrt{500 \cdot 0.5 \cdot 0.5} \\
                  & \approx 11.18 \\
        \end{align*}

        Convert 235 to a z-score:
        \[
          z_{235} = \frac{235 - 250}{11.18} \approx -1.3416
        \]

        Use table A\@:
        \begin{align*}
          P(X < 235)    & \approx 0.0899 \\
          P(X \geq 235) & \approx 1 - 0.0899 \\
                        & = \boxed{ 0.9101 }
        \end{align*}

        According to my computer, the exact binomial probability is
        approximately 0.9172

      \item[31]
        \begin{enumerate}[(a)]
          \item 
            \[
              \binom{8}{6} 0.75^6 \cdot 0.25^2 \approx \boxed{ 0.3115 } 
            \]

          \item \fbox{ $\mu = 0.75 \cdot 80 = 60$ }

          \item Since 60 is the mean, there is a \fbox{ 0.5 } probability of a
            result of at least 60 using the Normal distribution.

            According to my computer, the exact binomial probability is
            approximately 0.5597

        \end{enumerate}

      \item[32]
        \begin{enumerate}[(a)]
          \item This is a binomial distribution with 
            \fbox{ $n = 1000$ and $p = 0.004$ }

          \item $\mu = 1000 \cdot 0.004 = \boxed{ 4 }$

          \item $np = 4$. The Normal approximation should only be used when 
            $np \geq 10$.
        \end{enumerate}

      \item[34]
        \begin{enumerate}[(a)]
          \item
            \begin{align*}
              \mu    & = 0.13 \cdot 1200 \\
                     & = \boxed{ 156 } \\
              \sigma & = \sqrt{1200 \cdot 0.13 \cdot 0.87} \\
                     & \approx \boxed{ 11.65 } \\
            \end{align*}

          \item 95\% of the time the sample will be with 2 standard deviations
            of the mean. So 95\% of the time, the result will contain somewhere
            \fbox{ between 144 and 168 } Hispanics.

          \item 
            \begin{align*}
              0.13 n & = 200 \\
              n      & = \boxed{ 1539 } \\
            \end{align*}
        
        \end{enumerate}

      \item[35]
        \begin{enumerate}[(a)]
          \item
            With 100 questions:

            \begin{align*}
              \mu    & = 0.75 \cdot 100 \\
                     & = 75 \\
              \sigma & = \sqrt{100 \cdot 0.75 \cdot 0.25} \\
                     & \approx 4.3301 \\
            \end{align*}

            convert to z-scores:
            \begin{align*}
              z_{70} & = \frac{70 - 75}{4.3301} \\
                     & \approx -1.1547 \\
              z_{80} & = \frac{80 - 75}{4.3301} \\
                     & \approx 1.1547 \\
            \end{align*}

            From Table A\@:
            \begin{align*}
              P(X < 70) & \approx 0.1241 \\
              P(X < 80) & \approx 0.8759 \\
              \\
              P(70 < X < 80) & \approx 0.8759 - 0.1241 \\
                             & = \boxed{ 0.7518 } \\
            \end{align*}

          \item
            With 250 questions:

            \begin{align*}
              \mu    & = 0.75 \cdot 250 \\
                     & = 187.5 \\
              \sigma & = \sqrt{250 \cdot 0.75 \cdot 0.25} \\
                     & \approx 6.8465 \\
            \end{align*}

            convert to z-scores:
            \begin{align*}
              z_{175} & = \frac{175 - 187.5}{6.8465} \\
                      & \approx -1.8257 \\
              z_{200} & = \frac{200 - 187.5}{6.1237} \\
                      & \approx 1.8257 \\
            \end{align*}

            From Table A\@:
            \begin{align*}
              P(X < 175) & \approx 0.0339 \\
              P(X < 200) & \approx 0.9661 \\
              \\
              P(175 < X < 200) & \approx 0.9661 - 0.0339 \\
                               & = \boxed{ 0.9321 } \\
            \end{align*}
        
        \end{enumerate}

      \item[36]
        \begin{align*}
          \mu    & = 0.5 \cdot 10,000 = 5,000 \\
          \sigma & = \sqrt{10,000 \cdot 0.5 \cdot 0.5} = 50 \\
        \end{align*}

        convert the actual result to a z-score. Since we're interested in at
        least 67 more than the mean, the interesting values are $\pm 66$ around
        the mean.

        \begin{align*}
          z_{\max} &= \frac{5066 - 5000}{50} \approx 1.32 \\
          z_{\min} &= \frac{4934 - 5000}{50} \approx -1.32 \\
        \end{align*}

        From Table A\@:
        \begin{align*}
          P(X \leq 5066) & \approx 0.9066 \\
          P(X \leq 4934) & \approx 0.0934 \\
          \\
          P(4933 \leq X \leq 5066) & \approx 0.9066 - 0.0934 \\
                                   & =0.8132 \\
          \\
          P(X < 4933 \text{ or } X > 5066) & \approx 1 - 0.8132 \\
                                           & = 0.1869 \\
        \end{align*}

        There's about a 19\% chance of a number this far from the mean, so it's
        a reasonable value to expect for an unbiased coin.

      \item[38]
        You can't use the Normal approximation for this problem because the
        sample size is too small.

        \begin{enumerate}[(a)]
          \item 
            $\mu = 0.05 \cdot 20 = \boxed{ 1 }$

            \begin{align*}
              P(X = 0) & = 0.95^{20} \\
                       & \approx 0.3585 \\
              P(X = 1) & = \binom{20}{1} 0.05 \cdot 0.95^{19} \\
                       & \approx 0.3774 \\
              P(X = 2) & = \binom{20}{2} 0.05^2 \cdot 0.95^{18} \\
                       & \approx 0.1887 \\
                       \\
              P(X \leq 2) & \approx 0.3585 + 0.3774 + 0.1887 \\
                          & = \boxed{ 0.9245 }
            \end{align*}

          \item 
            $\mu = 0.8 \cdot 20 = \boxed{ 16 }$

            \begin{align*}
              P(X = 20) & = 0.80^{20} \\
                        & \approx 0.0115 \\
              P(X = 19) & = \binom{20}{1} 0.20 \cdot 0.80^{19} \\
                        & \approx 0.0576 \\
              P(X = 18) & = \binom{20}{2} 0.20^2 \cdot 0.80^{18} \\
                        & \approx 0.1369 \\
                       \\
              P(X \geq 18) & \approx 0.0115 + 0.0576 + 0.1369 \\
                           & = \boxed{ 0.2061 }
            \end{align*}

        \end{enumerate}

      \item[39]
        \begin{enumerate}[(a)]
          \begin{align*}
            \mu    & = 0.95 \cdot 1400 \\
                   & = 1,120 \\
            \sigma & = \sqrt{1400 \cdot 0.95 \cdot 0.05} \\
                   & \approx 8.15 \\
          \end{align*}

          75\% of the students is 1050 students. Convert to z-score:
          \[
            z = \frac{1,050 - 1,330}{8.15} \approx -34
          \]

          There is essentially a 100\% chance that at least 75\% of the
          students will contract the disease.

        \end{enumerate}
  \end{description}

  \else
    \vspace{7 cm}
    \begin{quote}
      \begin{em}
        The smart way to keep people passive and obedient is to strictly limit
        the spectrum of acceptable opinion, but allow very lively debate within
        that spectrum---even encourage the more critical and dissident views.
        That gives people the sense that there's free thinking going on, while
        all the time the presuppositions of the system are being reinforced by
        the limits put on the range of the debate.
      \end{em}
    \end{quote}
    \hspace{1 cm}--Noam Chomsky
  \fi

\end{document}

