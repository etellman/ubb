\documentclass[letterpaper,landscape]{exam}
\usepackage{2in1, lscape} 
\printanswers{}

\usepackage{units} 
\usepackage{parskip} 
\usepackage{xfrac} 
\usepackage[fleqn]{amsmath}
\usepackage{cancel}
\usepackage{float}
\usepackage{mdwlist}
\usepackage{booktabs}
\usepackage{cancel}
\usepackage{polynom}
\usepackage{caption}
\usepackage{fullpage}
\usepackage{comment}
\usepackage{enumerate}
\usepackage{graphicx}
\usepackage{mathtools} 
\usepackage{commath}

\everymath{\displaystyle}

\title{Statistics \\ Week Eleven}
\date{\today}
\author{}

\begin{document}

  \maketitle
  \tableofcontents

  \newpage

  \section{Parameter vs. Statistics}

  parameter:
  \begin{itemize*} 
    \item fact about the population you are trying to find out (average
      weight, average income, etc.). 
    \item uses symbols $\mu$ and $\sigma$.
  \end{itemize*}

  statistic
  \begin{itemize*} 
    \item fact about the sample that you can calculate. With a
      correctly chosen sample, the statistic is likely to match the
      parameter.
    \item uses symbols $\bar{x}$ and $\sigma$.
  \end{itemize*}

  \section{Law of Large Numbers}

  \subsection{Definition}
  With a population with a defined $\mu$ (parameter), when you take a random sample,
  $\bar{x}$ (statistic) will get closer to $\mu$ as the sample size increases.

  Talk about coin handout.

  As the number of flips increased, the last column got closer and closer to 50\%.

  Notice that the absolute error is increasing. Law of Large numbers doesn't say that the number of
  heads gets closer to the number of tails as you flip more times. It says the percentage of heads
  gets closer to 50\% as you flip more times.

  If you start out with 4 heads in a row, you won't necessarily get 4 tails in a row to cancel out.
  But as you have more and more flips, those 4 heads in a row matter less and less to the
  percentage.

  \section{Central Limit Theorem}

  \subsection{Expected Value}
  If head is worth 1 and tails 0, what is the expected value for 100 flips (how many heads)?

  \begin{itemize*}
    \item $\mu = 0.5$; $0.5 \cdot 100 = 50$
    \item $50 \pm SE$
  \end{itemize*}

  \subsection{Standard Error}
  The {\em Standard Error\/} is the expected deviation from the mean for a sample of size $n$.
  \[
    SE = \sigma \cdot \sqrt{n}
  \]

  \begin{itemize*}
    \item larger $\sigma$ (greater variation in the population) leads to larger SE
    \item larger number of samples leads to larger SE.\ You expect a larger
      absolute deviation from the mean after 1000 samples than after 100
      samples.
    \item SE is just like SD.\ Observed values are usually within 2 SEs of $\mu$.
      same rules as SD (68\%, 95\%, etc.)
  \end{itemize*}

  Standard error is total error to expect. It goes up as the number of trials increases.

  If you take multiple samples size $n$:

  \begin{align*}
    x_{total} & = n \mu \pm SE \\
    \bar{x}   & = \mu \pm \frac{SE}{n} \\
              & = \mu \pm \frac{\sigma \sqrt{n}}{n} \\
              & = \mu \pm \frac{\sigma}{\sqrt{n}} \\
    \\
    sd_{\bar{x}} & = \frac{\sigma}{\sqrt{n}} \\
  \end{align*}

  For a reasonable sized sample (30 or more), the distribution of $\bar{x}$ is
  $N\del{ \mu, \frac{\sigma}{\sqrt{n}} }$

  Draw Normal graph.

  \begin{itemize}
    \item $\bar{x} \approx \mu$

    \item Doesn't matter if population distribution is normal 

    \item SD of sampling distribution always smaller than $\sigma$

    \item sampling distribution generally more normal than population distribution

    \item sampling distribution won't, in general, look anything like population
      distribution. 

  \end{itemize}

  The population size doesn't matter---only sample size matters. For example, if
  you were trying to find the average weight of people in TX vs people in WA,
  you would be just as accurate in either case with a sample of 500. The number
  of people in the population doesn't appear in the formula.
  Example of 100 coin tosses:
  \begin{align*}
    \sigma        & = \frac{1}{2} \\
    \mu           & = \frac{1}{2} \\
    SE            & = 10 \cdot 0.5 \\
                  & = 5 \\
     sd_{\bar{x}} & = \frac{5}{\sqrt{100}} \\
                  & = 0.5 \\
  \end{align*}

  As the sample size increases, the standard deviation for sample decreases
  
  \section{Law of Large Numbers vs. Central Limit}

  \begin{description}
    \item[Law of Large Numbers]---as sample size increases, $\bar{x}$ tends to
      approach $\mu$. 

    \item[Central Limit Theorem]---if you take a bunch of samples of size
      $n$, the distribution of $\bar{x}$ is approximately 
      $N \del{ \mu, \frac{\sigma}{ \sqrt{n} } }$. 
      
  \end{description}

  \section{Box Model}
  A good model for many problems is drawing tickets from a box.
  \begin{itemize*}
    \item each ticket labeled with the value
    \item more tickets for more likely outcomes
    \item compute $\mu$ and $\sigma$ for the tickets
  \end{itemize*}

  A shortcut for finding $\sigma$ when there are only two kinds of tickets is:
  \[
    (\max - \min) \cdot \sqrt{proportion_{\min} \cdot proportion_{\max}}
  \]

  \section{Examples}

  \begin{enumerate}
    \item A boat has a weight limit of 20 people and 3500 lbs.
      20 175 lb men.
      
      Men have a weight distribution of $N(172, 29)$.

      \begin{enumerate}[(a)]
        \item What's the probability of an individual being 175 lb or more?

          \begin{align*}
            z & = \frac{175 - 172}{29} \\
              & \approx 0.1034 \\
          \end{align*}

          The probability of 1 person being this heavy is 0.4588

        \item What's the probability of 20 men being at least this heavy?

          \[
            0.4588^{20} \approx 0
          \]

        \item What's the probability of the average weight of a boat full of 20
          men is greater than 175 lb?

          \begin{align*}
            s & = \frac{29}{\sqrt{20}} \\
              & \approx 6.4846 \\
            \\
            z & = \frac{175 - 172}{6.4846} \\
              & \approx 0.4626 \\
          \end{align*}

          The probability of the average being this high is 0.3218. There is a
          32\% chance that a boat full of men will sink because it's over
          weight.

          Draw normal distribution graphs with $\sigma$ and $s$.

      \end{enumerate}

    \item Which is the better bet: 10 tosses or 100 tosses?
      According to Central Limit Theorem:

      \begin{align*}
        \sigma  & = 0.5 \\
        s_{10}  & = \frac{0.5}{\sqrt{10}} \\
                & \approx 0.1581 \\
        s_{100} & = \frac{0.5}{\sqrt{100}} \\
                & = 0.05 \\
                \\
        \bar{x}_{10}  & \approx 0.5 \pm 0.1581 \\
        \bar{x}_{100} & = 0.5 \pm 0.05 \\
      \end{align*}

      \begin{enumerate}
        \item win if more than 60\% heads (10 tosses)
        \item win if more than 40\% heads (100 tosses)
        \item win if between 40\% and 60\% heads (100 tosses)
        \item win if exactly 50\% heads (10 tosses)
      \end{enumerate}

    \item 100 Micholob drinkers tried Schlitz live at half time of super bowl in
      a live taste test (actual Super Bowl commercial from the 80s). 
      
      Assuming the two beers are indistinguishable, what's the chance that at
      least 40 of the Micholob drinkers will prefer Schlitz?

      \begin{solution}
        If nobody can tell the difference, with 0 for Micholob and 1 for
        Schlitz, the expected number of people who prefer Schlitz is 50.

        According to the Central Limit Theorem:
        \begin{align*}
          \mu     & = 0.5 \\
          \sigma  & = 0.5 \\
          \\
          s & = \frac{0.5}{\sqrt{100}} \\
            & = 0.05 \\
          \\
          \bar{x} &= 0.5 \pm 0.05 \\
        \end{align*}

        The chance of less than 40\% of the people preferring Schlitz is the
        chance of being more than 2 SDs less than the mean, or around 1\%

      \end{solution}

    \item Play roulette 50 times, betting \$1 on one number to win. Winning
      bet pays \$35. How much should you expect to lose on each bet? 

      \begin{solution}
        box model: 37 $-1$ tickets and one \$35 ticket

        \begin{align*}
          \mu     & = \frac{35 + 37 (-1)}{38} \\
                  & = -0.05 \\
          \sigma  & = 36 \cdot \sqrt{\frac{1}{38} \cdot \frac{37}{38}} \\
                  & \approx 5.76 \\
          \\
          s &= \frac{5.76}{\sqrt{50}} \\
          \approx 0.8146 \\
          \\
          \bar{x} &= -0.05 \pm 0.8146 \\
        \end{align*}

      \end{solution}

      How much should he expect to lose total?
      \begin{solution}
        \begin{align*}
          x  & = -0.05 \cdot 50 \\
             & = -2.5 \\
          SE & = 50 \cdot 0.8146 \\
             & = 40.73 \\
          \\
          x &= -2.5 \pm 40.73 \\
        \end{align*}
      \end{solution}

    \item Play roulette 50 times, with section bet with 2 to 1 payoff and 12
      chances in 38 to win. How much should he expect to lose per bet and total?

      \begin{solution}
        box model: 26 -\$1 tickets and 12 \$2 ticket

        \begin{align*}
          \mu     & = \frac{24 - 26}{38} \\
                  & = -0.05 \\
          \sigma  & = 3 \cdot \sqrt{\frac{12}{38} \cdot \frac{26}{38}} \\
                  & \approx 1.39 \\
          \\
          s       & = \frac{\sigma}{\sqrt{50}} \\
                  & \approx 0.1923 \\
          \bar{x} & = -0.05 \pm 0.1923 \\
          \\
          SE & \approx 50 \cdot 0.1923 \\
             & \approx 9.62 \\
          x  & = -2.5 \pm 9.62 \\
        \end{align*}

      \end{solution}

    \item Tickets are drawn at random with replacement from a box of numbered
      tickets. The sum of 25 draws has expected value equal to 50 with SE of 10.

      Find $\bar{x}$ and $s$ for 100 draws.

      \begin{solution}
        \begin{align*}
          \mu      & = \frac{50}{25} \\
          &= 2 \\
          \\
          s_{25}       & = \frac{10}{25} \\
                       & = 0.4 \\
          \bar{x}_{25} & = 2 \pm 0.4 \\
          \\
          0.4    & = \frac{\sigma}{\sqrt{25}} \\
          \sigma & = 2 \\
          \\
          s_{100} & = \frac{2}{\sqrt{100}} \\
                  & = 0.2 \\
          \\
          \bar{x}_{100} & = 2 \pm 0.2 \\
          x_{100}       & = 200 \pm 20 \\
        \end{align*}
        
      \end{solution}

    \item A box contains 10 tickets. Each ticket is marked with a number between
      -5 and 5. The numbers are not all the same but their average is 0. There
      are two choices:
      \begin{itemize*}
        \item 100 draws and you win if the sum is between -15 and 15
        \item 200 draws and you win if the sum is between -30 and 30
      \end{itemize*}

      Which choice gives you a better chance of winning?

      \begin{solution}
        Re-phrase this question as a question of averages:

        Which is more likely:
        \begin{align*}
          \bar{x}_{100} &= 0 \pm 0.15 \\
          \bar{x}_{200} &= 0 \pm 0.15 \\
        \end{align*}

        According to the Central Limit Theorem, the larger sample has a smaller
        standard deviation, so the second equation is more likely.

        % Here's the math:
        % \begin{align*}
        %   s_{200} & \approx \frac{\sigma}{\sqrt{200}} \\
        %   \sigma  & = 14.14 s_{200} \\
        %   \\
        %   s_{100} & = \frac{\sigma}{\sqrt{100}} \\
        %           & = 1.41 s_{200} \\
        %   \\
        %   z_{15} & = \frac{0.15}{1.41 s_{200}} \\
        %          & = \frac{0.0106}{s_{200}} \\
        %   z_{30} & = \frac{0.15}{s_{200}} \\
        % \end{align*}

        % $z_{15}$ is smaller so there's a higher chance of the sample being
        % outside of that value. 

      \end{solution}
  \end{enumerate}


  % Use example of buses full of:
  % \begin{itemize*}
  %   \item marathon runners
  %   \item ``Overeaters Anonymous'' convention participants
  % \end{itemize*}

  % There is some overlap between the populations because there are some heavy
  % marathoners and some light (short, long-term dieters, etc.) overeaters. But
  % you could examine the average for a bus full of people and tell whether the
  % bus contains a sample of marathoners or a sample of overeaters.

  \begin{enumerate}

    \item You want to do a poll to find out how many people have seen the latest
      episode of ``Walking Dead'' with an accuracy of $\pm 5\%$, with a 68\%
      confidence that you are right. How big should the sample be? 

      \begin{solution}
        The largest the standard deviation might be is (min = 0, max = 1):
        \[
          \sigma = \sqrt{\frac{1}{2} \cdot \frac{1}{2}} = 0.5
        \]

        If you want to be within 5 percentage points of the correct value 68\%
        of the time, you'll want a standard deviation of $0.05$.

        \begin{align*}
          s & = \frac{\sigma}{\sqrt{n}} \\
          n & = \del{ \frac{\sigma}{s} }^2 \\
          n & = \del{ \frac{0.5}{0.05} }^2 \\
            & = 100 \\
        \end{align*}

        With a sample of 100 people, you will be within 5 percentage points of
        the actual value 68\% of the time.

        If you want to be 95\% confident in the result, you'll want to be within
        two standard deviations of the mean:
        
        \begin{align*}
          n & = \del{ \frac{0.5}{0.025} }^2 \\
            & = 400 \\
        \end{align*}

        To double the confidence, you need to multiply the sample size by 4.

        With If more or less than half have seen the show, the survey will be
        even more accurate, since this will make the standard deviation smaller
        than 0.5.

        % If you take the sample and find that only about 10\% of the population
        % has seen the show, you can be even more confident in your estimate:

        % \begin{align*}
        %   \sigma & = \sqrt{\frac{1}{10} \cdot \frac{9}{10}} = 0.3 \\
        %   \\
        %   sd_{\bar{x}} & = \frac{0.3}{\sqrt{2500}} = 0.006 \\
        % \end{align*}

        % With the new standard deviation of 0.006, 1\% is now 1.6 SDs from the
        % mean: 
        
        % \begin{align*}
        %   0.01 & = x \cdot 0.006 \\
        %   x   & \approx 1.6 \\
        % \end{align*}
        % From Table A, the probability of being more than 1.6 SDs from the mean
        % is only 10\%.

      \end{solution}

    \item You observe 1000 spins of a roulette wheel and find that 550 came out
      red. Is the wheel fair?

      \begin{solution}
        Find the standard deviation for the sample distribution, assuming the
        wheel is fair:
        \begin{align*}
          \sigma & = \sqrt{\frac{18}{38} \cdot \frac{20}{38}} \\
                 & \approx 0.5 \\
          \\
          s & = \frac{0.5}{\sqrt{1000}} \\
            & \approx 0.016 \\
        \end{align*}

        You expect:
        \[
          \bar{x} = 0.5 \pm 0.016
        \]
        for 1000 samples.

        For our particular sample:
        \begin{align*}
          \bar{x} & = \frac{550}{1000} \\
                  & = 0.55 \\
          \\
          z_{0.55} & = \frac{0.55 - 0.5}{0.016} = 3.125 \\
        \end{align*}

        The value is more than three standard deviations away from the mean. The
        chance of this happening by chance is essentially zero.

        68\% of the time, the percentage of red would be between 0.484 and
        0.516. Converted back to rolls, you would expect that 68\% of the time
        you would see between 484 and 516 reds in 1000 spins.

      \end{solution}
  \end{enumerate}
\end{document}

