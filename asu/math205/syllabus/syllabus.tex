\documentclass[letterpaper, landscape]{article}
\usepackage{2in1, lscape} 
% \printanswers{}

\usepackage{fullpage}
\usepackage{graphicx}
\usepackage{float}
\usepackage{amsmath}
\usepackage{amssymb}
\usepackage{polynom}
\usepackage{caption}
\usepackage{mdwlist}
\usepackage{parskip}
\usepackage{booktabs}

\newcommand{\degree}{\ensuremath{^\circ}} 

\everymath{\displaystyle}
% \setlength{\mathindent}{1 cm}

\title{Math 205 \\ Statistical Methods}
\author{University Beyond Bars/Adams State University}
\date{\today}

\begin{document}

  \maketitle

  % \tableofcontents

  % Everything HIGHLIGHTED should be taken from the ASU Institutional Syllabus.

  \section{Basic Information} % (fold)
 
  \begin{description*}
    \item[Instructor] Ed Tellman 
    \item[Course Title] Statistical Methods
    \item[Course Prefix] Math 205
    \item[Credit Hours] Three undergraduate semester hours 
    \item[Textbook] {\em The Basic Practice of Statistics}, 5th edition, by
      Moore, Notz, and Fligner;  ISBN 978--1429224253.
    \item[Prerequisites] At least a $C-$ or $T$ in Math 104.
  \end{description*}

  \section{Catalog Description}
  This course covers basic techniques of applied statistics. This includes data
  organization and presentation, experiment design, calculating statistical
  measures, choosing, applying, and interpreting statistical tests, correlation
  and regression, and software utilization.

  % \section{Curricular Relationships} 
  % This is an introductory course in music literature, musical styles, major
  % composers, and principal historical periods in art music of Western
  % civilization, including a brief introduction to non-western music.  The
  % course is designed for the general college student, and is proposed to
  % satisfy one Arts and Humanities requirement of the new general education
  % curriculum.

  \section{Course Content} 

  \begin{itemize}

    \item Exploring Data: Variables, Distributions, and Visualizing Data with
      scatter plots, histograms, two-way tables. The normal distribution is
      studied, along with correlation, and linear regression.

    \item Introduction to Inference: Sampling techniques and data production
      through experimentation. Introductory probability and sampling
      distributions. Confidence intervals and tests of significance are covered
      in this area.

    \item Inference about Variables: We distinguish between a quantitative
      response variable, learning about population means and two-sample
      problems, and a categorical response variable, learning about population
      proportions and comparing them.

    \item Inference about Relationships: We learn about the chi-square test,
      inference for regression, and one-way analysis of variance.

  \end{itemize}

  \section{Student Learning Outcomes}
  \begin{itemize}
    \item Students will be able to explain the uses of elementary applied
      statistics and describe and recognize its potential misuses.

    \item Students will be able to design an experiment and collect, organize,
      and present data.

    \item Students will be able to apply the basic techniques of statistical
      analysis and decision making to various data sets.

    % \item Students will be able to make productive use of Microsoft Excel for
    %   statistical calculations (depending on availability of computer lab).
  \end{itemize}


  % There are four exams, including a comprehensive final, that are worth 20\% of
  % the course grade each. The remaining 20\% of the grade is homework.

  % Syllabus Assignment: Students will be expected to read the syllabus and
  % understand all course requirements and expectations.  
  
  % It is HIGHLY RECOMMENDED that you include some sort of Syllabus Assignment in
  % your course, graded or not graded.

  % \section{Course Description} 

  % The class will meet on Thursday nights from 6:00--7:45 PM\@. 

  \section{Course Outline}

    \begin{tabular}[H]{rcl}
    \toprule
    Week & Date     & Topic \\
    \midrule
    1      & 07/01/15 & Chapter 1: Graphing \\
    2      & 07/08/15 & Chapter 2: Central Tendencies \\
    3      & 07/15/15 & No Class \\
    4      & 07/22/15 & No Class \\
    5      & 07/29/15 & Chapter 3: Normal distributions \\
    6      & 08/05/15 & Chapter 4: Scatterplots and Correlation \\
    7      & 08/12/15 & Chapter 5: Regression \\
    8      & 08/19/05 & Chapter 6: Two-Way Tables \\
    9      & 08/26/15 & Exam One \\
    10      & 09/02/15 & Chapter 8: Sampling \\
    11      & 09/09/15 & Chapter 9: Experiments \\
    12     & 09/16/15 & Chapter 10: Probability \\
    13     & 09/23/15 & Chapter 11: Sampling Distributions \\
    14     & 09/30/15 & Chapter 14: Confidence Intervals \\
    15     & 10/07/15 & Chapter 15: Tests of Significance \\
    16     & 10/14/15 & Chapter 16: Inference \\
    17     & 10/21/15 & Exam Two \\
    18     & 10/28/15 & Chapter 18: Inference about a Pop. Mean \\
    19     & 11/04/15 & Chapter 19: Two-Sample Problems \\
    20     & 11/11/15 & Chapter 20: Inference about a Pop. Proportion \\
    21     & 11/18/15 & Chapter 21: Comparing Two Proportions \\
    22     & 11/25/15 & Exam Three \\
    23     & 12/02/15 & Chapter 23: Chi-Square Test \\
    24     & 12/09/15 & Chapter 24: Inference for Regression \\
    25     & 12/16/15 & Chapter 25: Analysis of Variance \\
    26     & 12/23/15 & Final Exam \\
    \bottomrule
  \end{tabular}


  \section{Course Requirements}
  % In this section you will need to list and describe anything graded – everything
  % that shows up in the Grade Distribution also needs to be listed here.

  Students should complete all weekly homework assignments and attend lecture each week.
  \section{Grade Distribution} % (fold)

  \begin{tabular}{lr}
    \toprule
    Homework                             & 20\% \\
    Exam One, covering Chapters 1--7     & 20\% \\
    Exam Two, covering Chapters 8--17    & 20\% \\
    Exam Three, covering Chapters 18--22 & 20\% \\
    Comprehensive Final Exam             & 20\% \\
    \midrule
    Total                                                    & 100\% \\
    \bottomrule
  \end{tabular}

  \section{Grade Scale} % (fold)
  
  \begin{tabular}{lr}
    \toprule
    90--100\% & A \\
    80--89\%  & B \\
    70--79\%  & C \\
    60--69\%  & D \\
    0--59\%   & F \\
    \bottomrule
  \end{tabular}

  In alignment with ASU academic policies, no D may apply to a major or minor field.

  % Everything listed here should also be listed and described in the Course Requirements section.  

  % \section{ADA Statement}
  % Students who need special accommodation to complete this class should contact
  % the instructor.

  \section{Academic Integrity}

  Cheating, plagiarism, unauthorized possession or disposition of academic
  materials, or the falsification or fabrication of one’s academic work will
  not be tolerated. Any offense may result in a zero for the exam or exercise
  in question and may result in failure of the course. 
  

  % Please refer to the ASU Student Handbook for more information.
  
  % http://www.adams.edu/pubs

  % The two statements above are standard for ASU – you can add or change this information as needed.
  

\end{document}

