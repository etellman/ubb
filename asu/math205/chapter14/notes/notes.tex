% % no answer key
% \documentclass[letterpaper]{exam}

% answer key
\documentclass[letterpaper, landscape]{exam}
\usepackage{2in1, lscape} 
\printanswers{}

\usepackage{units} 
\usepackage{parskip} 
\usepackage{xfrac} 
\usepackage[fleqn]{amsmath}
\usepackage{cancel}
\usepackage{float}
\usepackage{mdwlist}
\usepackage{booktabs}
\usepackage{cancel}
\usepackage{polynom}
\usepackage{caption}
\usepackage{fullpage}
\usepackage{comment}
\usepackage{enumerate}
\usepackage{graphicx}
\usepackage{mathtools} 
\usepackage{commath}

\usepackage[group-separator={,}]{siunitx}

\everymath{\displaystyle}

\title{Statistics \\ Chapter Fourteen Notes}
\date{\today}
\author{}

\begin{document}

  \maketitle
  \tableofcontents

  \section{Confidence Intervals}

  \subsection{Election}

  500 person exit poll showing:
  \begin{itemize*}
    \item Alice: 265 (0.53)
    \item Bob: 235 (0.47)
  \end{itemize*}

  If the race is fairly evenly split $\sigma \approx 0.5$. 

  With a sample size of 500:
  \begin{align*}
    s & \approx \frac{0.5}{\sqrt{500}} \\
      & \approx 0.02 \\
  \end{align*}

  68\% of the time, the sample value will be within one standard deviation of
  the actual value. 
  
  You can say that you are 68\% confident that:
  \begin{itemize*}
    \item Alice: between 0.51 and 0.55
    \item Bob: between 0.45 and 0.49
  \end{itemize*}

  You can declare Alice the winner with 68\% confidence.

  95\% of the time, the sample value will be within two standard deviations of
  the actual value. 
  
  You can say that you are 95\% confident that:
  \begin{itemize*}
    \item Alice: between 0.49 and 0.57
    \item Bob: between 0.43 and 0.51
  \end{itemize*}

  Now there is overlap, so you can't declare a winner. To get more confident,
  you have to broaden the range.

  Later in the day you get a bigger poll with 2000 people, and get the same
  results.
  
  the standard deviation goes down to:
  \begin{align*}
    s & \approx \frac{0.5}{\sqrt{2000}} \\
      & \approx 0.01 \\
  \end{align*}

  Now you can say that you are 68\% confident that:
  \begin{itemize*}
    \item Alice: between 0.52 and 0.54
    \item Bob: between 0.46 and 0.48
  \end{itemize*}

  Now you can say that you are 95\% confident that:
  \begin{itemize*}
    \item Alice: between 0.51 and 0.55
    \item Bob: between 0.45 and 0.49
  \end{itemize*}

  You can also say that you are 99.7\% confident (3 standard deviations) that:
  \begin{itemize*}
    \item Alice: between 0.50 and 0.56
    \item Bob: between 0.44 and 0.50
  \end{itemize*}

  Now you can be confident in declaring a winner. Even if you are wrong, there
  is a chance that you have underestimated Bob's total, even slimmer chance of
  being wrong in the direction that makes a difference to the outcome.

  \subsection{NFL Rushing}

  Suppose you want to determine the average yards per rush in the NFL by taking
  an SRS of 20 rushes.

  The actual value is:
  \begin{align*}
    \mu    & \approx 4.12 \\
    \sigma & \approx 6.32 \\
  \end{align*}

  From the Central Limit Theorem, $\bar{x}$ has a Normal distribution with:
  \begin{align*}
    \mu         & = 4.12 \\
    s_{\bar{x}} & = \frac{\sigma}{\sqrt{20}} \\
                & \approx 1.41 \\
  \end{align*}

  \begin{itemize*}
    \item 68\% of the time $\bar{x}$ is within 1.41 of $\mu$.
    \item 95\% of the time $\bar{x}$ is within 2.82 of $\mu$.
  \end{itemize*}

  Turning this around:
  \begin{itemize*}
    \item 68\% of the time $\mu$ is within 1.41 of $\bar{x}$.
    \item 95\% of the time $\mu$ is within 2.82 of $\bar{x}$.
  \end{itemize*}

  If we take a random sample of size 20, we can calculate its mean and conclude
  that:
  \begin{itemize*}
    \item We can be 68\% confident that the actual value is in the range
      $\bar{x} \pm 1.41$
    \item We can be 95\% confident that the actual value is in the range
      $\bar{x} \pm 2.82$
  \end{itemize*}

  ``x\% confident'' means that x\% of the time when we take a sample of this
  size, the actual value of the parameter will be in this range.

  I did this for 10 different samples and got:

  \[
    \bar{x} = \{ 3.05, 4.20, 3.50, 3.70, 2.25, 4.25, 3.50, 3.90, 7.50, 2.50 \}
  \]

  Out of these trials, 2.25, 7.50, and 2.50 are more than 1 standard deviation
  from the actual mean and only 7.5 is more than 2 standard deviations from the
  actual mean.

  So it is safe to say that for example:
  \begin{itemize*}
    \item the actual value is $3.05 \pm 1.41$ (68\% confidence)
    \item the actual value is $3.05 \pm 2.82$ (95\% confidence)
  \end{itemize*}

  Note that the larger the range, the more confident you can be. 
  
  68\% confidence means that about 32\% of the time, you'll be wrong. For
  instance: $2.25 \pm 1.41$ is actually wrong.

  \subsection{Procedure}

  \begin{itemize}
    \item There are two parts to an estimate:
      \begin{itemize*}
        \item interval: $estimate \pm error$
        \item confidence level: probability that the interval captures the actual
          parameter
      \end{itemize*}

    \item A bigger margin of error goes with higher confidence.

  \end{itemize}

  The general procedure is:
  \begin{enumerate*}
    \item decide how much you care about the result being correct (desired
      confidence level 65\%, 90\%, etc.)

    \item use Table C to find $z^*$

    \item calculate $\bar{x}$

    \item $s = \frac{\sigma}{\sqrt{n}}$

    \item The actual value is: $\bar{x} \pm z^* \cdot s$ with the desired
      confidence

  \end{enumerate*}

  notes:
  \begin{itemize}
    \item Assumes you know the standard deviation of the population. If you
      don't, you can use the standard deviation of the sample and get pretty
      close (student's T-test).

    \item Larger samples result in more precision.

    \item You can never be 100\% confident you have the right value.

    \item Smaller margin of error gives you less confidence.
  \end{itemize}

  \subsection{Check Your Skills}
  \begin{description}
    \item[14.1] 
      \begin{enumerate}[(a)]

        \item $s = \frac{60}{\sqrt{840}} \approx 2.07$

        \item 4.14

        \item 267.86 to 276.14

      \end{enumerate}

    \item[14.3] 97.5\% is 2.5\% less than 100\%. The value might be either very low
      or very high, so we're looking for the $z^*$ value that corresponds to
      1.25\%. From Table A, this is $z^* \approx 2.24$. A z-score less than
      -2.25 or greater than 2.25 will satisfy the requirements.

    \item[14.4]
      For a 90\% confidence interval and Table A, $z^* \approx 1.645$. None of
      the measurements should be more than 1.645 standard deviations from the
      mean.

      In conductivity units, this corresponds to $\pm 1.645 \cdot 0.2 = \pm
      0.329$. All of the measurements should be between 4.67 and 5.33.

      They all are, so they can say with 95\% confidence that everything is
      fine with what they are providing to the customers.

      If any of the samples had been outside this range, we would not have had
      95\% confidence that the liquid actually had a mean of 5.

    \item[14.5]
      The mean for the sample is 105.84. 

      For a 99\% confidence interval, $z^* = 2.576$. Converting this to IQ
      scores gives:
      \[
        iq = 2.576 \frac{15}{\sqrt{31}} \approx 6.94
      \]

      We can be 99\% confident that the actual mean is between and 98.90 and 112.78

    \end{description}

  \section{Hypothesis Testing}

  \subsection{General Idea}
  \begin{itemize}
    \item have some theory about a parameter ($\mu = 17$) (null hypothesis)
    \item measure the relevant statistic ($\bar{x} = 25$)
    \item if $\bar{x}$ is very unlikely to occur if null hypothesis is true, you
      can reasonably conclude that null hypothesis is false
    \item Draw normal curve with shaded regions for one and two sided
      hypothesis.
  \end{itemize}

  \subsection{Rushing Example}

  \subsubsection{Hypothesis One}
  For example suppose our hypothesis is that the average rush in the NFL is 5
  yards and we sampled 20 rushes and found that their average was 3.70. 3.70 is
  pretty far from 5.

  We know from past seasons that $\sigma \approx \unit[6.3]{yd}$

  With 20 rushes, the standard deviation is:
  \begin{align*}
    s & = \frac{6.3}{\sqrt{20}} \\
      & \approx \unit[1.41]{yd} \\
  \end{align*}

  Calculate the z-score for the observed $\bar{x}$:
  \begin{align*}
    z_{\bar{x}} & = \frac{3.70 - 5.0}{1.41} \\
                & \approx -0.92 \\
  \end{align*}

  From table A, the chance of getting a result at least $0.92$ standard
  deviations below the mean is approximately 18\%. Since this is a pretty big
  number, we can't be very confident in ruling out the possibility that the
  actual mean is 5.

  \subsubsection{Hypothesis Two}
  Suppose we instead theorized that the actual mean was 7. With this mean, the
  z-score is:
  \begin{align*}
    z_{\bar{x}} & = \frac{3.70 - 7.0}{1.41} \\
                & \approx -2.34 \\
  \end{align*}

  From table A, the chance of getting a result at least $2.34$ standard
  deviations below the mean is approximately 1\%. Since this is a pretty small
  number, we can be fairly confident in ruling out the possibility that the
  actual mean is 7. If the mean was really 7, then only 1 out of 100 times we
  took a sample of 20 would we get an $\bar{x}$ as low as 3.7.

  \subsection{Procedure and Notes}

  \begin{itemize}
    \item With hypothesis testing, the goal is to try to rule out the null
      hypothesis. If we get a result that is very unlikely if the hypothesis is
      true, then we can assume with high confidence that the hypothesis is
      false.

    \item The probability of the observed result, assuming the hypothesis is
      true is the P-value. A low P-value means we can be confident in ruling out
      the hypothesis. A high P-value means that the hypothesis might be true and
      we can't rule it out.

    \item P-value from first example was 0.18. P-value from second example was
      0.01.

    \item you can never be sure the null hypothesis is incorrect

    \item A ``statistically significant at X'' result means that you got a P-value
      less than X. Common choices for $X$ are 0.05 and 0.01. You can pick one
      that is appropriate for your domain.

    \item ``Statistically Significant'' doesn't mean ``Actually Significant.'' A
      very small difference might be statistically significant but
      uninteresting.

    \item With a two-sided test, the hypothesis is something like ``the average
      yards rushing is 7.'' You rule out the null hypothesis if your $\bar{x}$
      is much larger or smaller than 7.

    \item With a one-sided test, the hypothesis is something like ``the average
      yards rushing is more than 7.'' You can rule out the null hypothesis if
      your sample is much less than 7.

    \item For now, assume you have standard deviation from population and
      population is fairly Normally distributed

  \end{itemize}

  \subsection{Examples}

  \subsubsection{Education and Salary}

  From the 2012 WA population survey (week 2), the mean income from wages for
  people that worked (wage greater than 0) was:
  \begin{align*}
    \mu    & = 46,122 \\
    \sigma & = 52,071 \\
  \end{align*}

  The null hypothesis is that people without a degree make as much as people
  with a degree.

  This example doesn't start with a normal distribution, so it doesn't quite
  follow the rules. It doesn't matter if you use a large sample size. With a
  sample size less than 100 there was quite a bit of variation in the p-value
  from 0.01 to 0.000001, depending on the sample.

  With a sample size of 500:
  \begin{align*}
    s_{500} & = \frac{\num{ 52 071 }}{\sqrt{500}} \\
            & \approx \num{ 2 329 } \\
  \end{align*}

  Taking a random sample of 500 people with at least an associate's degree, 
  $\bar{x} = \$\num{64 792}$

  Calculate z-value:
  \begin{align*}
    z & = \frac{\num{ 64 792 } - \num{ 46 122 }}{2329} \\
      & \approx 8.02 \\
  \end{align*}

  This is a P-value of approximately $5 \times 10^{-16}$. There is essentially
  zero chance of being this many standard deviations away from the mean for all
  people just by chance. We can safely conclude that getting a degree helps you
  make more money.

  \subsubsection{NBA Field Goal Percentage}

  For all players:
  \begin{align*}
    \mu    & \approx 0.43 \\
    \sigma & \approx 0.10 \\
  \end{align*}

  With a sample of 30 centers:
  \begin{align*}
    \bar{x} & \approx 0.47 \\
    s       & = \frac{\sigma}{\sqrt{30}} \\
            & \approx 0.023 \\
    z       & \approx 1.84 \\
    p       & \approx 0.033 \\
  \end{align*}

  For a 2-sided hypothesis 
  \begin{itemize*}
    \item $H_a$: centers have a different field goal percentage from everyone
      else
    \item $H_0$ centers have the same field goal percentage as everyone else
  \end{itemize*}
  
  We can reject the null hypothesis at a P-value of 0.066.  There is only a
  0.066 probability of centers being $\pm 1.84$ standard deviations from the
  mean.

  For a 1-sided null hypothesis: 
  \begin{itemize*}
    \item $H_a$ centers better field goal percentage as everyone else
    \item $H_0$: centers are no better than anyone else
  \end{itemize*}
  
  We can reject the null hypothesis at a P-value of 0.033.  There is only a
  0.033 probability of a value 1.84 standard deviations above the mean if the
  actual center percentage is equal to or lower than the mean.

  With a sample of 50 centers:
  \begin{align*}
    \bar{x} & \approx 0.49 \\
    s       & \approx 0.018 \\
    z       & \approx 2.19 \\
    p       & \approx 0.014 \\
  \end{align*}

  With a larger sample, we can be more confident of the result.

  % \subsubsection{Adrian Peterson}
  % All rushers:
  % \begin{align*}
  %   \mu    & \approx 4.12 \\
  %   \sigma & \approx 6.32 \\
  % \end{align*}

  % Adrian Peterson, sample 500 rushes
  % \begin{align*}
  %   \bar{x} & \approx 4.64 \\
  %   s       & \approx 0.184 \\
  %   z       & \approx 2.85 \\
  %   p       & \approx 0.002 \\
  % \end{align*}

  % Null hypothesis: Adrian Peterson is no better than an average running back.

  % We can rule out the null hypothesis with a p-value of 0.002

  % notes:
  % \begin{itemize*}
  %   \item actual mean for Adrian Peterson is 4.95
  %   \item need large sample because standard deviation is so large
  %   \item data is approximately normal
  % \end{itemize*}

  \subsection{Check Your Skills}
  \begin{description}
    \item[14.6]

      part b:

      \begin{align*}
        s          & = \frac{30}{\sqrt{25}} \\
                   & = 6 \\
        \\
        z_{118.6} & = \frac{118.6 - 115}{6} \\
                  & \approx 0.58 \\
        P_{118.6} & \approx 0.72 \\
        \\
        z_{125.8} & = \frac{125.8 - 115}{6} \\
                  & \approx 1.8 \\
        P_{125.8} & \approx 0.036 \\
      \end{align*}

      About 72\% of the time a value as large as 118.6 will come up by random
      chance for the sample. A value as large as 125.8 will only come up about
      3.6\% of the time by chance.

    \item[14.7]
      \begin{align*}
        s_6 &= \frac{0.2}{\sqrt{6}} \\
        &\approx 0.082 \\
        \\
        z_{4.98} & = \frac{4.98 - 5}{0.082} \\
                 & \approx -0.24 \\
        \\
        z_{4.7} & = \frac{4.7 - 5}{0.082} \\
                & \approx -3.7 \\
      \end{align*}
  \end{description}

  \section{Lucia De Berk}

  \subsection{Summary} % (fold)
  
  After nurse was present at death of baby, someone noticed that she was often
  present at deaths. The director of the hospital looked through the files for
  other similar unexplained deaths. 

  \begin{table}[H]
    \centering
    \begin{tabular}{lrrr}
      \toprule
      shifts        & without incident & with incident & total \\
      \midrule
      without Lucia & 887              & 0             & 887 \\
      with Lucia    & 134              & 8             & 142 \\
      Total         & 1,021            & 8             & 1,029 \\
      \bottomrule
    \end{tabular}
    \caption{Lucia de Berk}\label{tab:ldb1}
  \end{table}

  Amateur statistician testified at trial that the P-value for this hospital was
  0.00000029854. He did this for the other two hospitals where Lucia worked and
  got P-values of 0.0715 and 0.0136, both within the unusual but still plausible
  range.

  Multiplying all three P-values together gives: 1 in 342 million and Lucia was
  convicted.

  She spent 13 years in prison, and finally was released based on:
  \begin{itemize*}
    \item ``chain link'' argument---first two were considered proved by medical
      evidence, so barrier of proof for remaining 5 was lowered.

    \item there were other similar cases ``with incident'' which the prosecution
      didn't mention. Part of the definition of ``with incident'' was ``Lucia
      present.''

    \item some of the cases used in the trial had actually occurred when Lucia
      had been out sick.

    \item the prosecution didn't compute new P-values taking this stuff into
      consideration

    \item no evidence of any poison, etc.\ was ever found. The theory of Lucia
      administering an OD was disproved with medical evidence. 
  \end{itemize*}
  
  \subsection{Nature Article} % (fold)
  
  The court needs to weigh up two different explanations: murder or
  coincidence. The argument that the deaths were unlikely to have occurred by
  chance (whether 1 in 48 or 1 in 342 million) is not that meaningful on its
  own---for instance, the probability that ten murders would occur in the same
  hospital might be even more unlikely. What matters is the relative
  likelihood of the two explanations. However, the court was given an estimate
  for only the first scenario.

  \subsection{Guardian Article} % (fold)

  The case against Lucia was built on a suspicious pattern: there were nine
  incidents on a ward where she worked and Lucia was present during all of them.
  This could be suspicious but it could be a random cluster, best illustrated by
  the ``Texas sharpshooter'' phenomenon: imagine I am firing a thousand machine gun
  bullets into the side of a barn. I remove my blindfold, find three bullets
  very close together and paint a target around them. Then I announce that I am
  an Olympic standard rifleman.

  This is plainly foolish. All across the world, nurses are working on wards
  where patients die, and it is inevitable that on one ward, in one hospital, in
  one town, in one country, somewhere in the world, you will find one nurse who
  seems to be on a lot when patients die. It's very unlikely that one particular
  prespecified person will win the lottery but inevitable someone will win: we
  don't suspect the winner of rigging the balls.

  \dots

  If you multiply p-values together, then chance incidents will rapidly appear
  to be vanishingly unlikely. Let's say you worked in 20 hospitals, each with
  a pattern of incidents that is purely random noise: let's say p=0.5. If you
  multiply those harmless p-values, of entirely chance findings, you end up
  with a final p-value of p < 0.000001, falsely implying that the outcome is
  extremely highly statistically significant. By this reasoning, if you change
  hospitals a lot, you automatically become a suspect.

  \dots

  And did the idea that there was a killer on the loose make any sense,
  statistically, for the hospital as a whole? There were six deaths over three
  years on one ward where Lucia supposedly did her murdering. In the three
  preceding years, before she arrived, there were seven deaths. So the death
  rate on this ward went down at the precise moment that a serial killer moved
  in.


  \subsection{Gill Paper} % (fold)
  
  Prosecutor restricted model to ward where something unusual happened and tried
  to compensate by multiplying the p-value by the number of nurses in the ward
  (27, in this case). He was trying to figure out what was the probability that
  something this unusual happened in this ward. He could just as well have
  multiplied by the number of nurses in the country to figure out the
  probability that something this unusual happened in this country.

  \dots

  An analogy might clarify this point. Consider a lottery with tickets
  numbered 1 to 1,000,000. The jackpot falls on the ticket with number
  223,478, and the ticket has been bought by John Smith. John Smith lives in
  the Da Costastraat in the city of Leiden. Given these facts, we may compute
  the chance that John Smith wins the jackpot; a simple and uncontroversial
  model shows that this probability will be extremely small. Do we conclude
  from this that the lottery was not fair since an event with very small
  probability has happened? Of course not. We can also compute the probability
  that someone in the Da Costastraat wins the jackpot, but it should be clear
  that the choice of the Da Costastraat as reference point is completely
  arbitrary. We might similarly compute the probability that someone in Leiden
  wins the jackpot or someone living in Zuid-Holland (the state in which
  Leiden is situated). With these data-dependent hypotheses, there simply is
  no uniquely defined scale of the model at which the problem must be studied.

  The analogy with the case of Lucia will be clear: the winner of the jackpot
  represents the suspect being present at eight out of eight incidents, the
  street represents the ward. Elffers restricts his model to the ward in which
  something unusual has happened. With perhaps equal justification, another
  statistician might have considered the entire JKZ (Leiden, in the analogy)
  instead of the ward as a basis for her computations--—with vastly higher
  probability for the relevant event to happen somewhere. Still another
  statistician might have taken the Netherlands as the basis for the
  computation, which yields again a higher probability. The important point to
  note is that subjective choices are unavoidable here, and it is rather
  doubtful whether a court’s judgement should be based on such choices. If one
  wants to avoid this kind of subjective choice, one should adopt an approach
  where the data are not used twice. In Section 3.3, we discuss such an
  approach.

\end{document}



