\documentclass[letterpaper, landscape]{exam}
\usepackage{2in1, lscape} 
\printanswers{}

\usepackage{units} 
\usepackage{parskip} 
\usepackage{xfrac} 
\usepackage[fleqn]{amsmath}
\usepackage{cancel}
\usepackage{float}
\usepackage{mdwlist}
\usepackage{booktabs}
\usepackage{cancel}
\usepackage{polynom}
\usepackage{caption}
\usepackage{fullpage}
\usepackage{comment}
\usepackage{enumerate}
\usepackage{graphicx}
\usepackage{mathtools} 
\usepackage{commath}

\everymath{\displaystyle}

\title{Statistics \\ Chapter 17}
\date{\today}
\author{}

\begin{document}

  \maketitle
  \setcounter{tocdepth}{2}
  \tableofcontents

  \section{Exam} % (fold)
  
  \begin{itemize}
    \item Next time I'll spend more time on probability. 

    \item Everyone's doing fine and getting C or better, with A still the most
      common grade. I can't give everyone an A or Adams State will reject the
      course as not being a real course with acceptable standards.

    \item Skip ahead if you don't know how to do something.

    \item Take a guess if you don't know the answer.

    \item Use common sense
      \begin{itemize*}

        \item more girls than boys is just asking how many 1 or 0 boy families
          there are. Don't panic.

        \item With 5 defective parts per day and 3 from factory A and 2 from
          factory B, $P(\text{Plant A } | \text{ defective}) = 0.6$. 
          
          Work backwards to figure out how to get 0.6 from other numbers using
          conditional probability formula.

        \item 57 heads isn't that unlikely in 100 coin flips. Double check your
          work if you get a z-value of 140 for something that seems possible.

      \end{itemize*}

    \item Dice/card problems are common and there aren't that many of them. Practice
      so you know how to solve any kind of dice/card problem you see.

    \item Probability can be reduced to counting. Have a systematic way of
      enumerating all possibilities (count in binary, or draw a tree).

    \item clarify Type-I and Type-II errors. Power isn't just the probability of
      getting something right.

    \item Don't round for significance. If the P-value is 0.6, it's not
      significant, even though 0.6 is pretty close to 0.5. 
      
      You can get the effect of rounding by setting the significance level to
      0.075.

    \item Goal of class is not that you can exactly duplicate problems you've
      seen before on homework. Goal is that you understand the ideas and can
      apply them to other related things you haven't seen before on the
      homework.

  \end{itemize}

  \section{Lucia De Berk}

  \subsection{Summary} % (fold)
  
  After nurse was present at death of baby, someone noticed that she was often
  present at deaths. The director of the hospital looked through the files for
  other similar unexplained deaths. 

  \begin{table}[H]
    \centering
    \begin{tabular}{lrrr}
      \toprule
      shifts        & without incident & with incident & total \\
      \midrule
      without Lucia & 887              & 0             & 887 \\
      with Lucia    & 134              & 8             & 142 \\
      Total         & 1,021            & 8             & 1,029 \\
      \bottomrule
    \end{tabular}
    \caption{Lucia de Berk}\label{tab:ldb1}
  \end{table}

  Amateur statistician testified at trial that the P-value for this hospital was
  0.00000029854. He did this for the other two hospitals where Lucia worked and
  got P-values of 0.0715 and 0.0136, both within the unusual but still plausible
  range.

  Multiplying all three P-values together gives: 1 in 342 million and Lucia was
  convicted.

  She spent 13 years in prison, and finally was released based on:
  \begin{itemize*}
    \item ``chain link'' argument---first two were considered proved by medical
      evidence, so barrier of proof for remaining 5 was lowered.

    \item there were other similar cases ``with incident'' which the prosecution
      didn't mention. Part of the definition of ``with incident'' was ``Lucia
      present.''

    \item some of the cases used in the trial had actually occurred when Lucia
      had been out sick.

    \item the prosecution didn't compute new P-values taking this stuff into
      consideration

    \item no evidence of any poison, etc.\ was ever found. The theory of Lucia
      administering an OD was disproved with medical evidence. 
  \end{itemize*}
  
  \subsection{Nature Article} % (fold)
  
  The court needs to weigh up two different explanations: murder or
  coincidence. The argument that the deaths were unlikely to have occurred by
  chance (whether 1 in 48 or 1 in 342 million) is not that meaningful on its
  own---for instance, the probability that ten murders would occur in the same
  hospital might be even more unlikely. What matters is the relative
  likelihood of the two explanations. However, the court was given an estimate
  for only the first scenario.

  (Prosecutor's fallacy)

  \subsection{Guardian Article} % (fold)

  The case against Lucia was built on a suspicious pattern: there were nine
  incidents on a ward where she worked and Lucia was present during all of them.
  This could be suspicious but it could be a random cluster, best illustrated by
  the ``Texas sharpshooter'' phenomenon: imagine I am firing a thousand machine gun
  bullets into the side of a barn. I remove my blindfold, find three bullets
  very close together and paint a target around them. Then I announce that I am
  an Olympic standard rifleman.

  This is plainly foolish. All across the world, nurses are working on wards
  where patients die, and it is inevitable that on one ward, in one hospital, in
  one town, in one country, somewhere in the world, you will find one nurse who
  seems to be on a lot when patients die. It's very unlikely that one particular
  prespecified person will win the lottery but inevitable someone will win: we
  don't suspect the winner of rigging the balls.

  \dots

  If you multiply p-values together, then chance incidents will rapidly appear
  to be vanishingly unlikely. Let's say you worked in 20 hospitals, each with a
  pattern of incidents that is purely random noise: let's say $p = 0.5$. If you
  multiply those harmless p-values, of entirely chance findings, you end up with
  a final p-value of $p < 0.000001$, falsely implying that the outcome is
  extremely highly statistically significant. By this reasoning, if you change
  hospitals a lot, you automatically become a suspect.

  \dots

  And did the idea that there was a killer on the loose make any sense,
  statistically, for the hospital as a whole? There were six deaths over three
  years on one ward where Lucia supposedly did her murdering. In the three
  preceding years, before she arrived, there were seven deaths. So the death
  rate on this ward went down at the precise moment that a serial killer moved
  in.


  \subsection{Gill Paper} % (fold)
  
  \subsubsection{Notes} % (fold)
  
  Prosecutor restricted model to ward where something unusual happened and tried
  to compensate by multiplying the p-value by the number of nurses in the ward
  (27, in this case). He was trying to figure out what was the probability that
  something this unusual would happen in this ward. He could just as well have
  multiplied by the number of nurses in the country to figure out the
  probability that something this unusual would happen in the country.

  \subsubsection{From Paper} % (fold)

  An analogy might clarify this point. Consider a lottery with tickets
  numbered 1 to 1,000,000. The jackpot falls on the ticket with number
  223,478, and the ticket has been bought by John Smith. John Smith lives in
  the Da Costastraat in the city of Leiden. Given these facts, we may compute
  the chance that John Smith wins the jackpot; a simple and uncontroversial
  model shows that this probability will be extremely small. Do we conclude
  from this that the lottery was not fair since an event with very small
  probability has happened? Of course not. We can also compute the probability
  that someone in the Da Costastraat wins the jackpot, but it should be clear
  that the choice of the Da Costastraat as reference point is completely
  arbitrary. We might similarly compute the probability that someone in Leiden
  wins the jackpot or someone living in Zuid-Holland (the state in which
  Leiden is situated). With these data-dependent hypotheses, there simply is
  no uniquely defined scale of the model at which the problem must be studied.

  The analogy with the case of Lucia will be clear: the winner of the jackpot
  represents the suspect being present at eight out of eight incidents, the
  street represents the ward. Elffers restricts his model to the ward in which
  something unusual has happened. With perhaps equal justification, another
  statistician might have considered the entire JKZ (Leiden, in the analogy)
  instead of the ward as a basis for her computations--—with vastly higher
  probability for the relevant event to happen somewhere. Still another
  statistician might have taken the Netherlands as the basis for the
  computation, which yields again a higher probability. The important point to
  note is that subjective choices are unavoidable here, and it is rather
  doubtful whether a court’s judgement should be based on such choices. If one
  wants to avoid this kind of subjective choice, one should adopt an approach
  where the data are not used twice. In Section 3.3, we discuss such an
  approach.
  
  \section{P-value Reasoning} % (fold)

  \subsection{Statistics} % (fold)
  
  It's hard to demonstrate something is true using statistics. Anything is
  possible, so any $\bar{x}$ is consistent with any hypothesized $\mu$.
  
  Although anything is possible, some things are incredibly unlikely, and you
  can quantify this. Instead of trying to prove that $H_a$ is true you
  demonstrate that $H_0$ is extremely unlikely, though consistent with the
  observed result.

  \subsection{Science} % (fold)
  
  The scientific method is about trying to prove something is false, and if
  unsuccessful, assuming it's probably true.

  Newton: $p = mv$

  Einstein: $p = \frac{mv}{\sqrt{1 - \sfrac{v^2}{c^2}}}$
  
  In science, you try to prove the thing you think is true is actually false. If
  you don't succeed, it's probably true. 

  In statistics, you try to show the opposite of the thing you think is true is
  probably false. If you succeed, you've demonstrated the thing you think is true
  is probably actually true.

  \section{T-test} % (fold)

  \begin{itemize}
    \item Same as z-test but use sample standard deviation and T-distribution
      instead of $\sigma$ and Normal distribution.

    \item Degrees of freedom is number of samples minus 1. More degrees of
      freedom approaches Normal distribution.

    \item Otherwise, exactly the same as z-procedures for confidence intervals
      and significance testing.

  \end{itemize}

  \section{Matched Pairs T Procedures} % (fold)

  \begin{table}[ht]
  \centering
    \begin{tabular}{rrr}
      \toprule
      Judge 1 & Judge 2 & Difference \\
      \midrule
      9.8     & 7.5     & 2.3 \\
      9.5     & 7.4     & 2.1 \\
      9.1     & 7.4     & 1.7 \\
      8.9     & 7.0     & 1.9 \\
      8.9     & 7.0     & 1.9 \\
      8.5     & 7.0     & 1.5 \\
      8.3     & 6.9     & 1.4 \\
      8.2     & 6.8     & 1.4 \\
      8.0     & 6.7     & 1.3 \\
      8.0     & 6.5     & 1.5 \\
      8.0     & 6.4     & 1.6 \\
      8.0     & 6.3     & 1.7 \\
      7.9     & 6.0     & 1.9 \\
      7.7     & 5.8     & 1.9 \\
      7.6     & 5.6     & 2.0 \\
      7.5     & 5.6     & 1.9 \\
      6.9     & 5.5     & 1.4 \\
      6.8     & 5.4     & 1.4 \\
      6.8     & 5.3     & 1.5 \\
      6.7     & 4.7     & 2.0 \\
      \bottomrule
    \end{tabular}
  \end{table}
  
  t.test (95\% confidence interval):
  \begin{description}
    \item[judge 1] 7.64 to 8.47, $\bar{x} = 8.06$
    \item[judge 2] 5.96 to 6.72, $\bar{x} = 6.34$
  \end{description}

  \subsection{Difference of Original T-tests (hypothetical)} % (fold)
  
  Difference of means:
  \[
    \bar{x}_{diff} = 8.06 - 6.34 = 1.72
  \]
  Aggregated 95\% confidence interval for difference by subtracting min1/max2
  and max1/min2 (not actually correct procedure): 0.92 to 2.51
  
  \subsection{T-test of Differences} % (fold)

  95\% confidence interval: 1.20 to 2.23.

  We can adjust one judge's scores by adding 1.72 and be 95\% confident that the
  new score is within $\pm 0.5$ of the correct score for the other judge.

\end{document}

