\documentclass[letterpaper, landscape]{exam}
\usepackage{2in1, lscape} 
\printanswers{}

\usepackage{units} 
\usepackage{parskip} 
\usepackage{xfrac} 
\usepackage[fleqn]{amsmath}
\usepackage{cancel}
\usepackage{float}
\usepackage{booktabs}
\usepackage{cancel}
\usepackage{polynom}
\usepackage{caption}
\usepackage{fullpage}
\usepackage{comment}
\usepackage{enumitem}
\usepackage{graphicx}
\usepackage{mathtools} 
\usepackage{commath}

\everymath{\displaystyle}

\title{Statistics \\ Chapter 19}
\date{\today}
\author{}

\begin{document}

  \maketitle
  \setcounter{tocdepth}{2}
  \tableofcontents

  \section{Table Review} % (fold)

  Data in table is t-value which comes from calculation. Once you know the value, look
  up the probability of that value at the bottom of the table.

  
  \section{$\sigma$ for population proportion} % (fold)

  What is the standard deviation of a collection of 1s and 0s?

  definitions:
  \begin{itemize}[nosep, label={}]
    \item s: number of 1s
    \item n: number of items in the collection
    \item p: probability of getting a one. Note that $p =
      \frac{s}{n} = \bar{x}$
  \end{itemize}

  \begin{align*}
    \sigma & = \sqrt{ \frac{s \del{1 - p}^2 + (n - s)p^2}{n} } \\
           & = \sqrt{ \frac{s \del{ 1 - 2p + p^2} + np^2 - sp^2}{n} } \\
           & = \sqrt{ \frac{s - 2 s p + s p^2 + np^2 - sp^2}{n} } \\
           & = \sqrt{ \frac{s - 2 s p + np^2 }{n} } \\
           & = \sqrt{ \frac{s}{n} - 2 p \cdot \frac{s}{n}  + p^2 } \\
           & = \sqrt{ p - 2 p^2 + p^2 } \\
           & = \sqrt{ p - p^2 } \\
           & = \sqrt{ p (1 - p) } \\
  \end{align*}

  You can generalize this so that $p$ is the probability of ``success''. This is
  the standard deviation for a single random draw from a box.
  
  Notice that $\sigma$ increases the closer $p$ is to 0.5. The maximum variation you
  can have is half ones and half zeros.

  Instead of thinking of drawing from a box of zeros and ones, you can think of a box
  of marbles where some are green and others are blue, population of people where
  some are Democrats and some are Republicans, etc. In all these cases:

  $\sigma = \sqrt{p(1 - p)}$.

  \section{Guessing Contents of Box} % (fold)

  Suppose you have a box full of green and blue marbles and you want to figure out the
  percentage of each in the box.

  Since $\sigma = \sqrt{ p(1 - p) }$, the standard deviation of the sample
  distribution is:
  \begin{align*}
    sd & = \frac{\sigma}{\sqrt{n}} \\
       & = \frac{\sqrt{ p(1 - p) }}{\sqrt{n}} \\
       & = \sqrt{ \frac{p (1 - p)}{n} } \\
  \end{align*}

  If you have enough samples, the sample distribution will be approximately:
  \[
    N \del{ p, \sqrt{ \frac{p (1 - p)}{n} } }
  \]
  
  Since $p$ is what we are trying to figure out, we approximate it with $\hat{p}$.

  \[
    SE = \sqrt{ \frac{ \hat{p}(1 - \hat{p}) }{n} }
  \]

  and

  \[
    p = \hat{p} \pm z^* \sqrt{ \frac{ \hat{p}(1 - \hat{p}) }{n} }
  \]

  This only works when the number of successes and the number of failures are both at
  least 15. Otherwise the sample size is too small.

  If you don't have many samples, you can improve the accuracy of the estimate by
  pretending like you have 4 more samples than you do and adding 2 to the number of
  successes:

  \[
    \tilde{p} = \frac{s + 2}{n + 4}
  \]

  Procedure for figuring out the proportion of marbles is:
  \begin{enumerate}[nosep]
    \item take a sample
    \item count number of successes and add 2
    \item look up $z^*$ in Table C.
    \item compute confidence interval
  \end{enumerate}

  \section{Sample Size} % (fold)

  \begin{enumerate}[nosep, label={}]
    \item $p^*$: estimate of $p$
    \item $m$: desired margin of error
    \item $z*$: $z^*$ for desired confidence level
  \end{enumerate}

  \begin{align*}
    m &= z^* \sqrt{ \frac{p^* (1 - p^*)}{n} } \\
    n &= \del{ \frac{z^*}{m} } p^* \del{ 1 - p^* } \\
  \end{align*}
  
  \section{Significance Test} % (fold)

  \begin{align*}
    s & = \sqrt{ \frac{p_0 ( 1 - p_0 )}{n} } \\
    z & = \frac{ \hat{p} - p_0 }{ s }  \\
  \end{align*}
  
  Since we want to find out how unlikely $\hat{p}$ is if $p_0$ is true, we use
  $p_0$ to calculate the standard error.

\end{document}

