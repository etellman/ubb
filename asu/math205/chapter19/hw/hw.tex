% no answer kef
\documentclass[letterpaper]{exam}

% answer key
% \documentclass[letterpaper, landscape]{exam}
% \usepackage{2in1, lscape} 
% \printanswers{}

% for the cent symbol
\usepackage{textcomp}

% the textcent command eats the space following the symbol
\usepackage{xspace}
\newcommand{\cent}{\textcent\xspace}

\usepackage{units} 
\usepackage{xfrac} 
\usepackage[fleqn]{amsmath}
\usepackage{cancel}
\usepackage{float}
\usepackage{mdwlist}
\usepackage{booktabs}
\usepackage{cancel}
\usepackage{polynom}
\usepackage{caption}
\usepackage{fullpage}
\usepackage{comment}
\usepackage{todonotes}
\usepackage{enumerate}
\usepackage{graphicx}
\usepackage{parskip}

\everymath{\displaystyle}

\title{Statistics \\ Chapter 19 Homework}
\date{\today}
\author{}

\begin{document}

  \maketitle

  \section{Homework}
  Chapter 19: 25--28, 30--31, 33--34, 36--44

  \ifprintanswers{}
    \section{Solutions}
    \begin{description}

      \item[25] 
        \begin{enumerate}[(a)]
          \item We don't have an SRS due to non-response.

          \item $0.8396 \pm 0.0226$ or 0.8170 to 0.8622
        \end{enumerate}

      \item[26] 
        Using the ``plus four'' approach: 

        $0.1193 \pm 0.0479$ or 0.0714 to 0.1672

      \item[27]
        \begin{enumerate}[(a)]
          \item 
            \begin{align*}
              p & = \frac{848}{1010} \approx 0.8396 \\
              s & = \sqrt{ \frac{0.8396 (1 - 0.8396)}{1010} } \approx 0.01155 \\
              m & = 1.96 \cdot s \approx 0.0226 = \boxed{ 2.26\% } \\
            \end{align*}

          \item 
            \begin{align*}
              s & = \sqrt{ \frac{0.5 (1 - 0.5)}{1010} } \approx 0.01573 \\
              m & = 1.96 \cdot s \approx 0.0308 = \boxed{ 3.08\% } \\
            \end{align*}

          \item 3\% is the worst case margin of error, regardless of how the poll
            turns out.

        \end{enumerate}

      \item[28]
        \begin{enumerate}[(a)]
          \item Psychology and communications students probably aren't that similar
            to science and engineering students (for example) in terms of their
            religious behavior.

          \item $0.8425 \pm 0.0833$ or 0.7593 to 0.9258

          \item $0.8321 \pm 0.0841$ or 0.7479 to 0.9162

            \begin{itemize*}
              \item The first interval is 75.9\% to 92.6\%.
              \item The second interval is 74.8\% to 91.6\%
              \item The second interval is about 0.001 percentage points wider and
                the min and max are both about 0.01 percentage points smaller.
            \end{itemize*}
        \end{enumerate}

      \item[30]
        \begin{enumerate}[(a)]
          \item If we count ``liked the font'' as a success, the number of failures
            in each case is only 8 and 5. The number of successes and failures both
            need to be at least 15 to use the large-sample confidence interval.

            For the plus four approach, we only need a sample size of at least 10 and
            a confidence interval of at least 90\%.

          \item 
            \begin{enumerate}[(a)]
              \item prefer Times New Roman: $0.6552 \pm 0.1730$ or
                0.4822 to 0.8282

              \item find Gigi more attractive: $0.7586 \pm 0.1557$ or
                0.6029 to 0.9144

            \end{enumerate}
        \end{enumerate}

      \item[31]
        \begin{enumerate}[(a)]
          \item Only 5 facilities didn't detect the beans, and the number of failures
            needs to be at least 15 for the large-sample confidence interval.

            We have at least 10 samples and a confidence interval greater than 90\%, so
            the experiment is fine for the plus four approach.

          \item $0.7407 \pm 0.1387$ or 0.6020 to 0.8795
        \end{enumerate}

      \item[33]
        \begin{enumerate}[(a)]
          \item No. The sample size is the same for all states and the population is
            much larger than the sample, even for the small states.

          \item Yes. The sample size will be larger for larger states.This will
            reduce the margin of error for the large states.
        \end{enumerate}

      \item[34]
        \begin{enumerate}[(a)]
          \item 
            \begin{align*}
              z^* & = 2.576 \\
              m   & = 0.015 \\
              p^* & = 0.2 \\
              \\
              n &= \boxed{ 1832 } \\
            \end{align*}

          \item 10\% of the sample is 184 people (rounding up). With 184 successes
            (if you count customer complaints as a success)
            and a sample size of 1832, the confidence interval is:

            $0.1004 \pm 0.0070$ or 0.0953 to 0.1074

            The actual margin of error was only 0.007 instead of the target 0.015.

            Because the number of successes was smaller than expected, the margin of
            error was smaller than expected.
        \end{enumerate}

      \item[36] This experiment meets the criteria for the large sample approach.
        
        The confidence interval is: $0.67 \pm 0.01$ or 0.66 to 0.68

      \item[37]
        For this sample we need to use the plus four approach because of the small
        sample size.

        The confidence interval is: 

        $0.4375 \pm 0.2040$ or 0.2335 to 0.6415

      \item[38]
        The hypotheses are:
        \begin{itemize*}
          \item $H_0$: $p = 0.3333$
          \item $H_a$: $p > 0.3333$
        \end{itemize*}

        \begin{align*}
          \hat{p} & = \frac{304}{803} \approx 0.3786 \\
          p_0     & = 0.3333 \\
          \\
          z       & \approx 2.6454 \\
        \end{align*}

        With a one-sided test, $P < 0.004$, so this is strong evidence against the
        null hypothesis and we can conclude that more than 33\% of this population
        doesn't use condoms.

      \item[39]
       ``False'' and ``Not sure'' are equivalent given how the question is phrased.

       There is a large enough sample to use the large sample approach. The 95\%
       confidence interval is:

       $0.40 \pm 0.025$ or 0.3753 to 0.4252.

       We can say with 95\% confidence that somewhere between 37.5\% and 42.5\% of
       American adults think that humans developed from earlier species of animals.

     \item[40]
       With this sample size, we should use the plus four approach.
       The 95\% confidence interval is:

       $0.5785 \pm 0.0880$ or 0.4905 to 0.6665

       49\% to 67\% of the drivers wear seat belts.

     \item[41]
       The hypotheses are:
       \begin{itemize*}
         \item $H_0$: $p = 0.5$
         \item $H_a$: $p < 0.5$
       \end{itemize*}

       \begin{align*}
          \hat{p} & = \frac{594}{1484} \approx 0.4003 \\
          p_0     & = 0.5 \\
          \\
          z       & \approx -7.8413 \\
        \end{align*}

        The p-value is approximately 0, so we can safely conclude that less than half
        the public believes in evolution.


     \item[42]
       The hypotheses are:
       \begin{itemize*}
         \item $H_0$: $p = 0.5$
         \item $H_a$: $p > 0.5$
       \end{itemize*}

       \begin{align*}
          \hat{p} & = \frac{68}{117} \approx 0.5812 \\
          p_0     & = 0.5 \\
          \\
          z       & \approx 1.7802 \\
        \end{align*}

        With a one sided test, there is statistically significant evidence, 
        $P < 0.04$ that more than half of the drivers wear seat belts.

      \item[43]
        With this sample size, we should to use the plus four approach. 

        The confidence interval is: 

        $0.0891 \pm 0.0556$ or 0.0335 to 0.1447

      \item[44]
        Assuming we still want a 95\% confidence interval:
        \begin{align*}
          \hat{p} & \approx 0.4003 \\
          m       & = 0.05 \\
          \\
          n &= \boxed{ 192 } \\
        \end{align*}


  \end{description}

  \else
    \vspace{12 cm}
    \begin{quote}
      \begin{em}
        The scientist has a lot of experience with ignorance and doubt and
        uncertainty, and this experience is of very great importance, I think.
      \end{em}
    \end{quote}
    \hspace{1 cm}--Richard Feynman
  \fi

\end{document}

