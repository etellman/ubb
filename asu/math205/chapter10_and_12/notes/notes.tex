% % no answer key
% \documentclass[letterpaper]{exam}

% answer key
\documentclass[letterpaper, landscape]{exam}
\usepackage{2in1, lscape} 
\printanswers{}

\usepackage{units} 
\usepackage{parskip} 
\usepackage{xfrac} 
\usepackage[fleqn]{amsmath}
\usepackage{cancel}
\usepackage{float}
\usepackage{mdwlist}
\usepackage{booktabs}
\usepackage{cancel}
\usepackage{polynom}
\usepackage{caption}
\usepackage{fullpage}
\usepackage{comment}
\usepackage{enumerate}
\usepackage{graphicx}
\usepackage{mathtools} 
\usepackage{commath}
\usepackage[group-separator={,}]{siunitx}

\everymath{\displaystyle}


\title{Statistics \\ Chapters 10 and 12}
\date{\today}
\author{}

\begin{document}

  \maketitle
  \tableofcontents

  \section{Collins Case}

  \subsection{Case}
  \begin{description*}
    \item[Juanita Brooks] elderly victim
    \item[John Bass] witness
    \item[Malcolm and Janet Collins] suspects
  \end{description*}

  Brooks' son visited nearby gas stations until he found one that mentioned
  interracial couple.

  Eyewitness descriptions of suspects:
  \begin{itemize*}
    \item woman with hair ``between light and dark blond'' and ponytail
    \item yellow car
    \item bearded and mustached black man
    \item nobody could positively identify suspects in lineups
    \item In trial, Bass claimed that Collins was person he saw in car, despite
      failure to identify him earlier.
  \end{itemize*}

  Actual appearance of suspects:
  \begin{itemize*}
    \item woman light blond hair sometimes in ponytail
    \item yellow car
    \item clean-shaven black man
  \end{itemize*}

  The police arrested, interrogated, harassed suspects in unsuccessful attempt
  to extract confession.

  alibi:

  \begin{itemize*}
    \item Janet was picked up by Malcolm from work about time crime had been committed but no one
      was too sure about either time.

    \item Both claimed to have gone to visit a friend, but exact time couldn't be established.

    \item Hair color, ponytail, and beard evidence all based on things easily changed.
  \end{itemize*}

  \subsection{Trial}
  Case based on probabilities:
  \begin{itemize*}
    \item Black man with a beard: 1 in 10
    \item Man with a mustache: 1 in 4
    \item White woman with blond hair: 1 in 3
    \item Woman with ponytail: 1 in 10
    \item Interracial couple in car: 1 in 1000
    \item Yellow car: 1 in 10
  \end{itemize*}

  Draw probability tree.

  \begin{description*}
    \item[sample space] all possible outcomes (all possible couples in this case)
    \item[event] a particular outcome (a particular couple in this case)
    \item[probability] proportion of times something happens over many repetitions
    \item[probability model] sample space and way of assigning probabilities to events. 
  \end{description*}


  Probability of event (black man with a beard, white woman with blond hair,
  woman with a ponytail, interracial couple in a car, yellow car): 1 in 12,000,000.

  Guilty verdict after 8 hours and 5 ballots.

  \subsection{Appeals}

  \subsubsection{Notes}
  Malcolm lost first appeal.

  In second appeal, 25 year old law clerk Lawrence Tribe pointed out flaws in
  arguments.

  \subsubsection{Fabricated Statistics}
  \begin{itemize*}
    \item all statistics made up
    \item hair color, ponytail, beard, and mustache all easily altered
    \item no reason to assume actual perpetrators were married 
  \end{itemize*}

  From judgement:
  \begin{quote}
    \begin{em}
      The prosecution produced no evidence whatsoever showing, or from which it
      could be in any way inferred, that only one out of every ten cars which
      might have been at the scene of the robbery was partly yellow, that only
      one out of every four men who might have been there wore a mustache, that
      only one out of every ten girls who might have been there wore a
      ponytail, or that any of the other individual probability factors listed
      were even roughly accurate.
    \end{em}
  \end{quote}

  \subsubsection{Independence}
  Product rule doesn't apply when probabilities aren't independent

  From Tribe:
  \begin{quote}
    \begin{em}
      There was another glaring defect in the prosecution’s technique, namely an
      inadequate proof of the statistical independence of the six factors. No
      proof was presented that the characteristics selected were mutually
      independent, even though the witness himself acknowledged that such
      condition was essential to the proper application of the ``product rule''
      or ``multiplication rule.'' \dots To the extent that the traits or
      characteristics were not mutually independent (e.g., Negroes with beards
      and men with mustaches obviously represent overlapping categories) the
      ``product rule'' would inevitably yield a wholly erroneous and exaggerated
      result even if all of the individual components had been determined with
      precision.
    \end{em}
  \end{quote}

  independence:
  \begin{itemize*}
    \item outcome of A doesn't influence B
    \item dice and roulette are independent
    \item drawing cards without replacement isn't---cards already drawn are gone
      from the deck
    \item yellow car and mustache are probably independent
    \item mustache and beard probably aren't independent
  \end{itemize*}

  For independent events:
  \[
    P(\text{A and B}) = P(A) P(B)
  \]

  For dice, the probability of getting a pair of 6s is:
  \begin{itemize*}
    \item 36 possible outcomes
    \item only 1 of these outcomes is a pair of 6s, so $\sfrac{1}{36}$
  \end{itemize*}

  Or, probability of each die being 6 is $\sfrac{1}{6}$ and product is
  $\sfrac{1}{36}$ since the events are independent.

  \subsubsection{Conditional Probability}

  The probability that both A and B happens is either:
  \begin{itemize*}
    \item the probability that A happens and that B happens given that A already
      happened.
    \item the probability that B happens and that A happens given that B already 
      happened.
  \end{itemize*}

  \begin{align*}
    P(\text{A and B}) & = P(A) P(B | A) \\
                      & = P(B) P(A | B) \\
    \\
    P(B) P(A | B) & = P(\text{A and B}) \\
    P(A | B)      & = \frac{ P(\text{A and B}) }{P(B)} \\
    \\
    P(\text{ mustache and beard }) &= P(\text{mustache}) P(\text{beard } | \text{ mustache}) \\
  \end{align*}

  When A and B are independent, $P(A | B) = P(A)$

  Draw new probability tree with a few corrected probabilities.

  \subsubsection{Wrong Probability to Investigate}
    40\% probability of at least one other couple in LA area with similar
    characteristics, even if you accept the original made-up prosecution
    figures. 

    $\sfrac{1}{\num{ 12 000 000 }}$ represents supposed probability randomly
    selected couple will have this particular set of characteristics. Correct
    probability to look at is probability that the suspects are the correct pair
    of people with this set of statistics, given there are probably multiple
    people like this in the LA area.

    Probability given at trial was trying to be probability that if you grabbed
    a random couple off the street, they would look like this, which isn't
    relevant.

  \subsection{Aftermath}

  Tribe went on to:
  \begin{itemize*}
    \item Defend Gore in Gore vs. Bush
    \item Become one of Obama's instructors
    \item Serve in the Obama administration
    \item Write numerous papers on use of law in trials
  \end{itemize*}

  % \section{Monty Hall Game}
  % Select one of 3 doors, etc., etc.

  \section{Probability Rules}

  \begin{itemize}
    \item probability between 0 and 1
    \item all possible outcomes add up to probability one
    \item if events are disjoint (no outcomes in common, can never occur
      together) probability of either one is sum of individual probabilities
    \item ``mutually exclusive'' is another term for disjoint. If one event
      occurs, the other event is excluded from occurring.
    \item if events are independent, probability of both is product of
      individual probabilities
  \end{itemize}

  % Independence vs. Disjoint:
  % \begin{itemize*}
  %   \item two independent events may both happen. Event one has no effect on
  %     event two.

  %   \item to disjoint events can never both happen. If event one happens, event
  %     two can't happen. 
  % \end{itemize*}

  \section{Sally Clark}

  \subsection{Trial}

  Prosecution case:
  \begin{itemize*}
    \item two children died unexplained deaths (SIDS)
    \item probability of SIDS 1 in 1300
    \item for Sally Clark's family (affluent non-smokers) probability is 1 in
      8543.
    \item overall probability is:
      \[
        \frac{1}{8543^2} \approx \frac{1}{\num{73 000 000}}
      \]

    \item with 700,000 births per year, the chance of a double SIDS happening at
      least once in 100 years is:
      \[
        1 - \del{ 1 - \frac{1}{\num{73 000 000}} }^{\num{700 000} \cdot 100} \approx 0.62
      \]

      This sort of event will only happen about once in every hundred years.

  \end{itemize*}

  \subsection{Problems}

  The prosecution ignored factors that make Sally's family more likely for
  SIDS\@. Both of Sally's kids were boys and boys are more likely to die from
  SIDS\@.

  SIDS depends on genetic and environmental factors. If a family has one SIDS
  death, the chance of a second one is approximately 1 in 130, not 1 in 1300.

  The ``once in a hundred years'' calculation used the high number (8543)
  which was for a small subset of a population, but then used the whole
  population (73,000,000).

  With these corrections, the probability of double SIDS would be about:
  \[
    \frac{1}{1300} \cdot \frac{1}{130} \approx \frac{1}{\num{169 000}} 
  \]

  With 700,000 births per year, the chance of a double-SIDS event in one year is
  about:
  \[
    1 - \del{ 1 - \frac{1}{\num{169 000}} }^{\num{ 700 000 }} \approx 0.98
  \]

  \subsection{Appeal}
  The probability of a double SIDS event is irrelevant. The alternative
  explanation for the evidence is double infant murder, which is even less
  likely that double SIDS\@. 
  
  The interesting probability is the conditional probability of Clark being
  innocent given the evidence.

  Since double infant murder is so rare, it's hard to get a good estimate of its
  frequency. The best estimate is that double murder is about 9 times less
  likely than double SIDS\@.

  \begin{align*}
    P(\text{double SIDS}) & = \frac{1}{8543 \cdot 132} \\
                            & \approx \frac{1}{\num{ 125 000 }} \\
    P(\text{double murder}) & = \frac{1}{9 \cdot 8543 \cdot 132} \\
                            & \approx \frac{1}{\num{ 1 128 000 }} \\
  \end{align*}

  \begin{align*}
    P(\text{Innocent } | \text{ Evidence}) & = \frac{P(\text{I and E})}{P(E)} \\
       & = \frac{P(\text{double murder})}{P(\text{double murder or double SIDS})} \\
       & = \frac{\frac{1}{\num{ 125 000 }}}{\frac{1}{\num{ 125 000 }} + \frac{1}{\num{ 1 128 000 }}} \\
       & \approx 0.9 \\
  \end{align*}

  Even this didn't get Sally out of prison. Her third appeal succeeded because
  they reviewed the medical records more carefully and found the actual cause of
  death for the second baby was an infection that had gone undetected.

  \section{Data Dredging}
  Suppose 
  \begin{itemize*}
    \item database of 20,000 men.
    \item 1 out of 10,000 men will match DNA by chance
    \item search DB and find match
  \end{itemize*}

  What is the probability that somebody will match?

  \begin{solution}
    The probability that any one person matches is independent of the
    probability of anybody else matching.

    The probability that nobody matches is
    \[
      0.9999^{20,000} \approx 0.14
    \]

    The probability that somebody matches:
    \[
      1 - 0.1353 \approx 0.86
    \]
  \end{solution}

  \section{OJ Simpson}
  Prosecutors said that there is a natural progression from abuse to murder.

  Defense countered with 
  \begin{itemize*}
    \item 14,000,000 women battered annually by spouse or boyfriend
    \item 1,432 of these, or 1 in 2,500 murdered by spouse
    \item chance of OJ having murdered Nicole only 1 in 2,500
    \item what is this ``natural progression'' crap?
  \end{itemize*}

  Prosecution should have countered with conditional probability. Murder by
  spouse is very rare, but murder by someone other than spouse is even more
  rare. Only about 160 battered spouses were murdered by someone other than
  the batterer.

  Probability of non-spouse murder of abused spouse:
  \[
    \frac{160}{\num{ 14 000 000 }} \approx \frac{1}{\num{ 87 500 }}
  \]

  Probability of abused spouse getting murdered:
  \[
    \frac{160 + 1432}{\num{ 14 000 000 }} \approx \frac{1}{8800}
  \]

  \begin{align*}
    P(\text{Innocent} | \text{Evidence}) & = \frac{P(\text{I and E})}{P(E)} \\
                                         & = \frac{P(\text{someone else guilty})}{P(\text{murdered})} \\
                                         & = \frac{\frac{1}{\num{ 87 500 }}}{\frac{1}{8800}} \\
                                         & \approx 0.1 \\
  \end{align*}


  \section{Continuous Probability Models}

  \subsection{Notes}
  \begin{itemize*}
    \item Pick a random number between 0 and 1, etc.

    \item Probability density curve from previous chapter. 
  
    \item Area under curve is chance of getting outcome from in range.

    \item {\em Uniform Distribution\/} is flat

    \item {\em Normal Distribution\/}

    \item $P(X < 2)$, etc.\ notation

  \end{itemize*}

  \subsection{Packaging Laws}
  \begin{itemize*}
    \item too heavy expensive for manufacturer
    \item too light expensive for consumer
    \item impossible to be 100\% accurate
    \item normal distribution around target weight
  \end{itemize*}


  \begin{enumerate}
    \item If target weight is 1 kg and machine with standard deviation of 40 g,
      what is the probability the actual weight is less than 950 g?

    \begin{solution}
      Convert to z-score:
      \[
        z = \frac{950 - 1000}{40} = -1.25
      \]

      Use table A:\@
      \[
        P(X < 980) \approx 0.1056
      \]
    \end{solution}

    \item If the same machine is set at a target weight of 1020 g instead, what
      is the probability the actual weight is less than 1000 g?

    \begin{solution}
      Convert to z-score:
      \[
        z = \frac{1000 - 1020}{40} = -0.5
      \]

      Use table A:\@
      \[
        P(X < 1000) \approx 0.3085
      \]
    \end{solution}

    In Australia, law is that if a package is found to be underweight,
    everything is cool as long as the average of 12 similar items is not
    underweight.

    \begin{itemize*}
      \item Average also normal curve
      \item Same area (1, of course) but steeper and narrower graph
      \item same mean, $sd = \frac{ sd_{original} }{\sqrt{n}}$
    \end{itemize*}

    Strategy is to set machine slightly high, but chance of violating law is
    much lower with this strategy.

    What is the probability of violating the law when setting the target weight
    to 1020 with the same machine?

    \begin{solution}
      Calculate new sd:
      \[
        sd = \frac{40}{\sqrt{12}} \approx 11.55
      \]
      Convert to z-score:
      \[
        z = \frac{1000 - 1020}{11.55} = -1.73
      \]

      Use table A:\@
      \[
        P(X < 1000) \approx 0.0417
      \]
    \end{solution}

  \end{enumerate}

  \section{Test for Disease}

  \begin{itemize*}
    \item test 99.9999\% accurate
    \item no false negatives (everyone with disease tests positive)
    \item 1 out of 100,000 people have disease
    \item 175,000,000 people in US
  \end{itemize*}

  If test is administered to all people in US, what percentage of the people
  with positive results on test actually have disease?


  \begin{solution}
    Draw a tree diagram.

    level one:
    \begin{itemize*}
      \item 0.0001 have disease
      \item 0.99999 don't have disease
    \end{itemize*}

    level two: test positive/negative. 
    \begin{itemize*}
      \item Have the disease and test positive: 1,750
      \item Have the disease and test negative: 0
      \item Don't have the disease and test positive: 17,500
      \item Don't the disease and test negative: 174,980,750
    \end{itemize*}

    percentage of people who tested positive who have the disease:
    \[
      \frac{1,750}{19,250} \approx 9\%
    \]

  \end{solution}
  \section{More Examples}

  \subsection{Paradox of Chevalier de Moivre}

  Which is more likely:
  \begin{itemize}
    \item Getting at least one one with four rolls of a single die.
    \item Getting at least one pair of ones with 24 rolls of a pair of dice.
  \end{itemize}

  The naive approach 

  \begin{itemize*}
    \item at least one one: $4 \cdot \frac{1}{6} = \frac{2}{3}$
    \item at least one pair of ones: $24 \cdot \frac{1}{36} = \frac{2}{3}$
  \end{itemize*}

  This is incorrect because the events are not mutually exclusive. 1 on the
  first two rolls, for example, is counted twice.

  actual solution is 1 minus the probability of not getting the desired result:

  \begin{itemize*}
    \item at least one one: $1 - \del{ \frac{5}{6} }^4 \approx 0.5177$
    \item at least one pair of ones: $1 - \del{ \frac{35}{36} }^{24} \approx 0.4914$
  \end{itemize*}
  \subsection{Conditional Probability}

  If you turn over two cards in a deck, what is the probability of the second
  card being the queen of hearts?

  \begin{solution}
    \[
      P(\text{2nd card QH}) = \frac{1}{52}
    \]
  \end{solution}

  If you turn over two cards in a deck and the first card is the seven of clubs,
  what is the probability of the second card being the queen of hearts?

  \begin{solution}
    \[
      P(\text{QH } | \text{ 7C}) = \frac{1}{51}
    \]
  \end{solution}

  \begin{solution}
    $\frac{1}{52}$
  \end{solution}

  another way:
  \begin{solution}
    \begin{align*}
    P(\text{QH | 7C}) & = \frac{P(\text{7C and QH})}{P(7C)} \\
                      & = \frac{\cfrac{1}{52 \cdot 51}}{\cfrac{1}{52}} \\
                      & = \frac{1}{52} \\
    \end{align*}
  \end{solution}

  What is probability of getting tails given the previous three flips were heads?
  \begin{solution}
    \[
      P(T | 3H) = 0.5
    \]
  \end{solution}
  \[
    P(\text{A or B}) = P(A) + P(B) - P(\text{A and B})
  \]

  \begin{itemize*}
    \item if $P(\text{A and B})$, events are {\em disjoint\/} or {\em mutually exclusive\/}
    \item for disjoint events, if A occurs, B can't occur
    \item draw Venn diagram to explain why subtracting $P(\text{A and B})$ is
      necessary for non-disjoint events
  \end{itemize*}

  \begin{enumerate}
    \item red cars, compact cars, red compact cars

    \item people with beards, people with mustaches, people with both

    \item If one of your first two cards in a poker hand is an ace, what is the
      probability of getting at least one additional ace?

      \begin{solution}
        If you have one of the aces, there are 3 left in the deck. The
        probability of getting at least one of them is one minus the probability
        of not getting any of them:

        \begin{align*}
          1 - \frac{47}{50} \cdot \frac{46}{49} \cdot \frac{45}{48} \approx 0.173 \\
        \end{align*}
      \end{solution}

    \item If one of your first two cards in a poker hand is an ace, what is the
      probability of getting exactly one additional ace?

      \begin{solution}
        Exactly one of your three remaining cards needs to be an ace:
        \begin{align*}
          P(1st) & = \frac{3}{50} \cdot \frac{47}{49} \cdot \frac{46}{48} \\
          P(2nd) & = \frac{47}{50} \cdot \frac{3}{49} \cdot \frac{46}{48} \\
          P(3rd) & = \frac{47}{50} \cdot \frac{46}{49} \cdot \frac{3}{48} \\
                 & \approx 0.055
        \end{align*}

        The probabilities are all the same: $0.055$. Since there are three of
        them, the probability of getting exactly one ace is: 
        \[
          3 \cdot 0.055 = 0.165 \\
        \]
      \end{solution}

    \item If one of your first two cards in a poker hand is an ace, what is the
      probability of getting exactly two additional aces?

      \begin{solution}
        Exactly one of your three remaining cards needs to not be an ace:
        \begin{align*}
          P(1st) & = \frac{47}{50} \cdot \frac{3}{49} \cdot \frac{2}{48} \\
          P(2nd) & = \frac{3}{50} \cdot \frac{47}{49} \cdot \frac{2}{48} \\
          P(3rd) & = \frac{3}{50} \cdot \frac{2}{49} \cdot \frac{47}{48} \\
                 & \approx 0.0024 \\
        \end{align*}

        The probabilities are all the same: $0.055$. Since there are three of
        them, the probability of getting exactly one ace is: 
        \[
          3 \cdot 0.0024 = 0.0072 \\
        \]
      \end{solution}

    \item If one of your first two cards in a poker hand is an ace, what is the
      probability of getting 4 aces?

      \begin{solution}
        All of your three remaining cards need to be an aces:
        \begin{align*}
          P(\text{4 aces}) & = \frac{3}{50} \cdot \frac{2}{49} \cdot \frac{1}{48} \\
                           & \approx \boxed{ 0.000051 } \\
        \end{align*}
      \end{solution}
  \end{enumerate}


  % \subsection{Roulette}
  % In a roulette style game from the 1700's, with a wheel with 32 slots, the
  % players win 27 pounds plus the original pound on a 1 pound bet if their number
  % comes up and they lose otherwise. The players complained that the payoff
  % should have been closer to 31. The casino ``proved'' them wrong by betting
  % even money that any number the player picked would come up within 21 rolls.
  % What is the probability of the casino winning this second bet:

  % \begin{solution}
  %   \[
  %     1 - \del{ \frac{1}{32} }^{22} \approx 0.5027
  %   \]
  % \end{solution}

  \subsection{WW II Pilots}
  If a pilot has a 2\% chance of getting shot down, what's the chance of him
  surviving 50 missions?

  \begin{solution}
    \[
      0.98^{50} \approx 0.36
    \]
  \end{solution}

  % \subsection{Dice}

  % \begin{enumerate}
  %   \item What is the probability of rolling three 6s in a row?
  %     \begin{solution}
  %       \begin{align*}
  %         \frac{1}{6^3} & = \frac{1}{216} \\
  %                       & \approx \boxed{ 0.0046 } \\
  %       \end{align*}
  %     \end{solution}

  %   \item What is the probability of rolling at least one 6?
  %     \begin{solution}
  %       This one minus the probability of not rolling a 6 three times in a row:

  %       \begin{align*}
  %         1 - \del{ \frac{5}{6} }^3 & = 1 - \frac{125}{216} \\
  %                                   & = \frac{91}{125} \\
  %                                   & \approx \boxed{ 0.728 } \\
  %       \end{align*}

  %     \end{solution}

  % \end{enumerate}

  % \subsection{Pairs}
  % Draw table with 1--6 labeling rows and 1--6 labeling columns and cells sum of
  % row/column labels. 36 ways for dice to land. Show getting a particular number
  % is diagonal going up and to the right.

  % \begin{itemize*}
  %   \item what's the most common roll?
  %   \item how many ways are there to get 5?
  %   \item how many ways are there to get 12?
  % \end{itemize*}

  \subsection{Three Dice/Sum 9 or 10}

  It looks like 9 and 10 are equally likely:
  \begin{itemize*}
    \item combinations for 9: 126, 135, 144, 234, 225, 333
    \item combinations for 10: 145, 136, 226, 235, 244, 334
    \item $6^3 = 216$ ways for dice to fall
  \end{itemize*}

  Galileo analysis:
  \begin{itemize*}
    \item combinations with all different numbers can come up 
      $3 \times 2 \times 1 = 6$ ways
    \item combinations with two different numbers can come up in 3 different
      ways (odd number first, second or third)
    \item 333 can only come up one way
    \item actually 25 ways to get 9 and 27 ways to get 10 
  \end{itemize*}

  \subsection{Dice vs. Cards}
  A die is rolled twice; a deck of cards is shuffled.

  \begin{enumerate}[(a)]
    \item What's the probability that either the first roll is a one or the
      second roll is a one?

      \begin{solution}
      The first roll is a one or the second roll is a 1, subtracting the time
      when they are both one to avoid counting it twice:
        \[
          \frac{1}{6} + \frac{1}{6} - \frac{1}{36} = \frac{11}{36}
        \]

        The first roll is a one and the second one isn't, or the first roll
        isn't a one and the second one is, or both are ones:
        \[
          \frac{1}{6} \cdot \frac{5}{6} + \frac{5}{6} \cdot \frac{1}{6} +
            \frac{1}{6} \cdot \frac{1}{6} = \frac{11}{36} 
        \]

        One minus the probability that neither is a one:
        \[
          1 - \frac{5}{6} \cdot \frac{5}{6} = \frac{11}{36} 
        \]

      \end{solution}

    \item What's the probability that the first card is the ace of spades or the
      last card is the ace of spades?
      \begin{solution}
        Mutually exclusive events: $\frac{1}{52} + \frac{1}{52} = \frac{1}{26}$
      \end{solution}

    \item What's the probability that the first roll is a 1 and the second roll
      is a 1?
      \begin{solution}
        Independent events: $\frac{1}{36}$
      \end{solution}

    \item What's the probability that the first card is the ace of spades and
    the last card is the ace of spades?
      \begin{solution}
        Mutually exclusive events: $0$
      \end{solution}
  \end{enumerate}

  \subsection{Card Drawing}
  Contestants are given a shuffled deck of cards. They win a prize if the first
  card is the ace of hearts or the second card is the king of hearts.

  
  \begin{enumerate}[(a)]
    \item What's the probability of winning with the king of hearts?: $1/52$
    \item What's the probability of winning with the ace of hearts?: $1/52$

    \item What's the overall probability of winning? 
    
    \begin{solution}
      \begin{align*}
        \frac{1}{52} + \frac{1}{52} - \frac{1}{52^2} & = \frac{103}{51^2} \\
                                                     & \approx 0.0381
      \end{align*}
      
      If you don't account for the possibility of getting both winning cards,
      your odds would be about 0.0384.
      
      You need to subtract the possibility of getting both cards. The events are
      not mutually exclusive.

      Win only on the first card or only on the second card or on both cards:
      \begin{align*}
        \frac{1}{52} \cdot \frac{51}{52} + \frac{1}{52} \cdot \frac{51}{52} 
            + \frac{1}{52^2} & = \frac{103}{51^2} \\
                             & \approx 0.0381
      \end{align*}

      1 minus chance of losing on both attempts:
      \begin{align*}
            1 - \frac{51^2}{52^2} & = \frac{103}{51^2} \\
                                  & \approx 0.0381
      \end{align*}

    \end{solution}
  \end{enumerate}


  % \subsection{Exercise 10.16}
  % \begin{enumerate}[(a)]
  %     \item $P(X \geq 10)$

  %     \item 
  %       Convert 10 to a z-score:
  %       \[
  %         z = \frac{10 - 6.8}{1.6} = 2
  %       \]

  %       $P(Z < 2) = 0.9772$

  % \end{enumerate}

\end{document}

