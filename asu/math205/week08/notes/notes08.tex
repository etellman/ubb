\documentclass[landscape]{exam}

\usepackage{2in1, lscape} 
\usepackage{units} 
\usepackage[fleqn]{amsmath}
\usepackage{float}
\usepackage{mdwlist}
\usepackage{booktabs}
\usepackage{caption}
\usepackage{fullpage}
\usepackage{enumerate}
\usepackage{graphicx}
\usepackage{parskip}

\printanswers{}

\everymath{\displaystyle}

\title{Statistics \\ Week Nine}
\date{\today}
\author{}

\begin{document}

  \maketitle
  \tableofcontents

  \part{Overview}


  \section{Sample vs. Population}

  \begin{description}
    \item[population] entire group
    \item[sample] group you actually get data from
  \end{description}

  \section{Statistical Inference vs. Exploratory Data Analysis}
  
  Exploratory:
  \begin{itemize*}
    \item given data, draw conclusions
    \item conclusions only apply to data you actually have
  \end{itemize*}

  Statistical Inference:
  \begin{itemize*}
    \item try to gather data to answer a question

    \item conclusions apply to more than just data actually gathered. You gather
      data on a subset of the population and make a conclusion about the entire
      population.

    \item answers come with a statement of how confident we are that they are
      accurate and a range
  \end{itemize*}

  Compare to:
  \begin{itemize*}
    \item Tasting soup---one spoonful, or better, one bowl, will tell you what
      the whole pot tastes like, as long as the pot is well mixed.

    \item bucket with 600 red balls and 400 blue balls evenly mixed. Any random
      selection of 100 balls will be around 60/40.

  \end{itemize*}

  Types of inference:
  \begin{itemize*}

    % \item If you know a lot about the population, you know something about any
    %   randomly-selected reasonably-large sample of the population. If I know
    %   that the average SAT score for my school is 750, then any random sample of
    %   students will have an average SAT score of around 750.

    \item Any randomly-selected reasonably-large sample of a population will the
      similar to the entire population.
      
      If I want to know how effective a school is, I can give a standardized
      test to 100 students. Their mean score will be about the same as the mean
      score for the entire school would have been, had I given the test to the
      entire school.

      Average height of students in UBB probably representative of average
      height of prisoners in WSR\@.

    \item If you have data on a sample and data on a population, you can infer
      whether sample probably came from population. 

      If you know the mean standardized test score for a school is 75 but the
      students in one class all got 100, perhaps there is something fishy going
      on.

      The sample didn't come from the population: ``students at that school who
      didn't cheat.''

    \item If you have data for two samples, you can infer whether they were
      likely to have come from the same population. 

      For example, if you have data on:
      \begin{itemize*}
        \item cancer rate for smokers
        \item cancer rate for non-smokers
      \end{itemize*}

      You can infer whether they both came from the population of ``people with
      cancer who engaged in cancer-causing behavior''.

  \end{itemize*}

  \section{Sample Design}

  \subsection{Notes}

  sample design---plan for getting a sample that represents population

  Planning sample design:
  \begin{itemize*}
    \item What is the population we want to find out about? 
    \item What do we want to find out from each observation? (What are the
      variables?)
  \end{itemize*}

  \subsection{Examples}

  \subsubsection{US Census}
  \begin{itemize*}
   \item population: US
   \item variables: age, income, race, marital, status, number of children, etc.
   \item sample design: make everyone fill out a questionnaire every 10 years 
  \end{itemize*}

  \subsubsection{Election Polls}
  \begin{itemize*}
   \item variables: who are you going to vote for?
   \item population: probable voters (not just registered voters)
   \item sample design: difficult---random calls, etc.
  \end{itemize*}

  % \section{Control/Treatment}
  % If you're trying to find out if something has an effect (new drug, drinking,
  % smoking) the subjects are in two groups.  The ``control'' group doesn't
  % drink, smoke, take the new medicine, etc., while the ``treatment'' group does
  % drink, smoke, medicate, etc.

  % Subject of Chapter 9.

  \section{Longitudinal Studies} 
  Observe changes in the same sample over a long period of time.

  Framingham Heart Study:
  \begin{itemize*}
    \item 5209 adult residents of Framingham, MA from 1948 to now
    \item height, weight, blood pressure, education, etc.
    \item information on kids 
    \item 2000 academic articles since 1950, 1000 between 2000 and 2009 on
      smoking, cholesterol, heart disease, blood pressure, 
  \end{itemize*}

  With a longitudinal study you don't have to reconstruct the past history of
  the subjects. You don't need to find someone today and ask him what his blood
  pressure was when he was 19, for example. This is good, because he probably
  doesn't remember.

  \part{Frequent Problems}

  \section{Credits}
  Most of this stuff is from {\em Naked Statistics}.

  \section{Selection Bias}

  \subsection{Notes}
  Sample is not representative of population.  Every member of the population
  doesn't have an equal chance of appearing in the sample.

  \subsection{Examples}

  Republican Iowa straw poll:
  \begin{itemize*}
    \item August of year before presidential election
    \item Iowa only---not representative of US republican voters
    \item \$30 to vote---not representative of US republican voters
  \end{itemize*}

  Other examples:
  \begin{itemize*}
    \item New York Times reporters only make attempt to interview disgruntled
      ex-Amazonians when doing an article on Amazon. 

      Sample is not representative of opinions of all Amazon employees.

    \item ``Did you find what you were looking for'' survey only seen by people
      who didn't find what they were looking for.

      Sample is not representative of all Amazon shoppers.

    \item UBB students as sample of US prison population (WA may not be typical.
      UBB students may not be typical)

  \end{itemize*}

  \subsubsection{Convenience Sample}

  Sample that is easiest to gather:

  \begin{itemize*}

    \item ask your friends about their views on something to estimate US
      population's opinion.

    \item observe cars in your neighborhood to find most common car in US (my
      neighborhood has disproportionate number of SUVs)


  \end{itemize*}

  \section{Self-Selection Bias}

  \subsection{Notes}
  If people assign themselves to the sample group, the sample may not be
  representative of the population.
  
  \subsection{Examples}

  \begin{itemize*}
    \item web poll
    \item mail poll
    \item AA members---probably not representative of drinkers, so it's hard to
      determine effectiveness of AA
    \item UBB students
  \end{itemize*}

  For any poll, if the population is the US population:
  \begin{itemize*}
    \item People who know about the poll may not be typical (listeners to the
      show, visitors to the web site)
    \item People who actually respond may be even less typical (too much time on
      their hands, feel very strongly)
  \end{itemize*}

  \subsubsection{Clofibrate}

  From {\em Statistics}, by Freedman, Pasani and Purves, pp 13--14

  \begin{table}
    \centering
    \begin{tabular}{lrrrr}
      \toprule
      & \multicolumn{2}{c}{Clofibrate} & \multicolumn{2}{c}{Placebo} \\
                   & number & deaths & number & deaths \\
      \cmidrule(r){2-3} \cmidrule(r){4-5}    
      adherers     & 708    & 15\%   & 1813   & 15\% \\
      non-adherers & 357    & 25\%   & 882    & 28\% \\
      total group  & 1103   & 20\%   & 2789   & 21\% \\
      \bottomrule
    \end{tabular}
    \caption{Clofibrate study}\label{tab:clofibrate}
  \end{table}

  \begin{itemize*}
    \item Study to determine if Clofibrate prevents heart disease.

    \item Overall number showed about the same death rate with or without the
      Clofibrate.

    \item Perhaps explanation is people who didn't take their medicine.
      Clofibrate adherers had only 15\% death rate vs.\ non-adherers.

    \item However, control adherers had only 15\% death rate vs.\ non-adherers.

    \item Conclusions:
      \begin{itemize*}
        \item Clofibrate is ineffective.

        \item Adherers are different from non-adherers and take better care of
          themselves generally.

        \item Even though this was a randomly-selected double-blind sample,
          people self-selected into adherer/non-adherer groups.

      \end{itemize*}

  \end{itemize*}

  \subsubsection{Smoking Study}
  A smoking study found:
  \begin{itemize*}
    \item people who had never smoked were slightly healthier than current
      smokers
    \item people who had recently stopped smoking were much less healthy than
      current smokers
  \end{itemize*}

  True or false: you shouldn't start smoking, but if you start you shouldn't
  stop.  Explain.

  \begin{solution}
    The people who just stopped probably stopped because their health was shot.
  \end{solution}

  \section{Publication Bias}

  \subsection{Notes}
  Every study has some possibility of error.  With publication bias, the
  researcher throws away the studies that yield the ``wrong'' result and
  publishes the few studies that yield the desired result. 

  \subsection{Example}

  Suppose study is 95\% likely to be correct.  1 out of 20 times the study
  returns an unlikely and wrong result. 
  
  To demonstrate a new drug is effective, a drug
  company just needs to do 20 studies, discard the 19 that say the drug does
  nothing, and publish study number 20.

  For example, you might do a study to find out if Jelly Beans cure cancer and
  find that they didn't. Then you could do 20 studies to find out if a
  particular color of Jelly Bean cures cancer. One of these studies might happen
  to get a group of people who happen to get better, irrespective of the Jelly
  Bean effect.

  The press release says: ``With 95\% certainty, Green Jelly Beans cure Cancer.''

  With Prozac, 94\% of the positive studies were published and 14\% of the
  non-positive studies were published.

  \section{Recall Bias}

  \subsection{Notes}

  Current events may affect how people remember the past.  

  \subsection{Example}

  In 1993 study of diet/breast cancer, people with breast cancer reported high
  fat diets earlier in life.  Cause was people struggling to think of
  explanation for cancer rather than actual high-fat diets.

  \begin{em}
    The diagnosis of breast cancer had not just changed a woman’s present and the
    future; it had altered her past. Women with breast cancer had (unconsciously)
    decided that a higher-fat diet was a likely predisposition for their disease
    and (unconsciously) recalled a high-fat diet.
  \end{em}

  Longitudinal study wouldn't be susceptible to this problem.

  \section{Survivorship Bias}

  In a longitudinal study, the sample may change over time.

  A mutual fund company can start 16 mutual funds. Each mutual fund has a
  50\% chance of beating the index. By chance, 8 are above
  the index after year 1, 4 after year 2, and 2 after year 3.  Below average
  funds quietly shut down and above average funds heavily advertised.

  The success rate isn't a representative sample of the success
  rate for those particular mutual funds over many years.

  \section{Polling}

  \subsection{Non-Response}
  It's hard to get people to respond to a poll, particularly with caller ID\@.
  People who answer the phone and are willing to talk are probably different
  (retired, young, unemployed, bored) from people who don't.

  Ways to mitigate:
  \begin{itemize*}
    \item call repeatedly at all hours of the day to maximize chance of getting
      people who work at different hours

    \item get response from adult with most recent birthday rather than person
      who answered phone
  \end{itemize*}

  \subsection{Biased Questions}

  Different ways of phrasing the same question can get wildly different
  responses.

  examples:
  \begin{itemize*}
    \item tax relief (good) vs.\ tax cut (less good)
    \item welfare (bad) vs.\ assistance (good)
    \item ``reasonable measures to keep us safe from terrorists'' vs.
      ``governmental monitoring of every phone call.''
  \end{itemize*}

  \part{Sampling}

  \section{Simple Random Sample}
  \begin{itemize*}
    \item make list of subjects and randomly select subset
    \item use table of random numbers in back of book
    \item if you can really get a random sample, confounding factors you didn't
      think of are likely to be present the sample at about the same proportion
      as they are in the population, so they'll average out to having no effect.
  \end{itemize*}

  Every subset needs to be equally likely to be selected, not just every
  sample. An example would be a 4x4 grid of plots in a field where you randomly select
  one of the first 4 plots and then select every 4th plot. You end up with a
  single column. Every plot is equally likely to be selected, but you always end
  up with a single column and the columns may be different because they may have
  different light, water, soil, etc.

  \section{Stratified Sample}

  \subsection{Notes}
  \begin{itemize*}
    \item Divide population into groups (age, race, income, etc.) according to
      things you think may affect the outcome.
    \item Do SRS from each group.
    \item Helps to get appropriate representation from each group.
  \end{itemize*}

  \part{Miscellaneous Examples}

  \section{Breast Cancer Study}

  From {\em Statistics}, by Freedman, Pasani and Purves, p. 22

  Health Insurance Plan of NY (HIP) study in 1963.

  \paragraph{Conclusions}
  \begin{itemize*}
    \item screening didn't affect diseases other than breast cancer
    \item poor less likely to accept screening
    \item breast cancer affects rich people more than poor people
  \end{itemize*}

  \begin{table}[H]
    \centering
    \begin{tabular}{lrrrrr}
      & & \multicolumn{2}{c}{Breast Cancer} & \multicolumn{2}{c}{All Other} \\
                               \cmidrule(r){3-4} \cmidrule(r){5-6}    
                      &        & number & rate & number & rate \\
      Treatment group \\
      examined        & 20,200 & 23     & 1.1  & 428    & 21 \\
      refused         & 10,800 & 16     & 1.5  & 409    & 38 \\
      total           & 31,000 & 39     & 1.3  & 837    & 27 \\
    \midrule
      control group   & 31,000 & 63     & 2.0  & 879    & 28 \\
    \end{tabular}
    \caption{Breast cancer study: cause of death}\label{tab:breast.cancer}
  \end{table}

  \paragraph{Questions} % (fold)
  
  \begin{itemize*}

    \item Does screening save lives? (yes)

    \item Why is rate for ``all other'' same for treatment and control groups?

    \item why is rate for all other causes higher for ``refused'' than
      ``examined'' groups?

    \item Breast cancer affects rich more than poor.  Which numbers confirm
      this?

      \begin{solution}
        Rich people tend to accept.  Income of groups, in order, is 
        
        \begin{itemize*}
          \item examined (mostly rich)
          \item control (mixture of rich and poor)
          \item refused (mostly poor)
        \end{itemize*}

        Death rate among control group is higher than death rate among refused,
        and only thing different is income.

      \end{solution}

    \item death rate (all causes) among examined is about half death rate among
      refused.  Did screening cut the death rate in half?  What explains the
      difference in death rate?

      \begin{solution}
        The people in the examined group probably take better care of themselves
        generally.
      \end{solution}

    \item Is comparing accepted vs.\ refused a fair comparison?  Is it biased for
      or against screening?

      \begin{solution}
        Since the people who accepted the screening tend to have more money, and
        money is positively correlated with breast cancer incidence, this
        comparison would be biased against screening.
      \end{solution}

    \item Someone claims that encouraging women to come in for breast cancer
      screening encourages their health conciousness so they live longer for
      that reason.  Is the data consistent or inconsistent with that claim?

      \begin{solution}
        Inconsistent.  The only thing affected by the screening is the breast
        cancer rate.
      \end{solution}

    \item In the first year of the study, 67 breast cancers were found in the
      ``examined'' group, 12 in the refused group, and 58 in the control group.
      True or false: screening causes breast cancer.

      \begin{solution}
        Comparing raw numbers doesn't make sense, since there are different
        numbers in each group.  It makes more sense to look at the percentages
        with cancer detected in each group:
        \begin{itemize*}
          \item examined: 0.33\%
          \item refused: 0.11\%
          \item control: 0.19\%
        \end{itemize*}

        More cancer is detected among the examined because:
        \begin{itemize*}
          \item they have more money, so they are more likely to get breast
            cancer
          \item the regular exams detect cases that go undetected in the other
            groups
        \end{itemize*}

        Control has more than refused because control is slightly richer and
        rich people tend to get breast cancer more often.
      \end{solution}

  \end{itemize*}

  \section{Body Fat}
  Study found heavier kids tend to have more controlling mothers, leading to the
  headline ``Parents of Fat Kids Should Lighten Up.''

  (Freedman, p. 26)

  \begin{itemize*}
    \item is there an association between the mother's behavior and the kids'
      weight? (yes)
    \item Is the controlling behavior necessarily causing the overweight kids,
      or might there be a confounding variable?

      \begin{solution}
        Rich parents might both be more controlling and more likely to have kids
        who sit around and play video games all day instead of getting out and
        exercising.
      \end{solution}

    \item Will following the headline's advice necessarily help the kids lose
      weight?

      \begin{solution}
        Not if there is some confounding factor like income which is positively
        correlated with both weight and controlling moms.
      \end{solution}

  \end{itemize*}

  \section{Vitamin C}

  From {\em Statistics}, by Freedman, Pasani and Purves, p. 21

  \begin{table}
    \centering
    \begin{tabular}{cll}
      \toprule
      group & prevention & therapy \\
      \midrule
      1     & placebo    & placebo \\
      2     & vitamin C  & placebo \\
      3     & placebo    & vitamin C \\
      4     & vitamin C  & vitamin C \\
    \end{tabular}
    \caption{Vitamin C study}\label{tab:vitamin.c}
  \end{table}

  \begin{itemize*} \item dropout rate was higher for groups 1--3 than 4
    \item easy to distinguish vitamin from placebo by taste
    \item among subjects who remained blind vitamin C had no effect
    \item among subjects who broke blind groups 2 and 4 had fewest colds and
      groups 3 and 4 had shortest colds---everybody who knew which pill was
      Vitamin C thought it was doing them some good.
  \end{itemize*}

  % \section{Voting}
  % A hypothetical city is divided into 2 wards.  In each ward, the percentage of
  % registered democrats who vote is greater than the percentage of registered
  % republicans who vote.  True or false: the overall percentage of registered
  % democrats who vote is greater than the percentage of republicans who vote?

  % \begin{solution}
  %   False.  
    
  %   District 1 is democratic with low turnout and district 2 is republican with
  %   high turnout.  
    
  %   Specifically, district one has 200 out of 1000 democrats voting and 10 out
  %   of 100 republicans voting.  District two has 90 out of 100 democrats voting
  %   and 800 out of 1000 republicans voting.

  %   Overall, $290/1100 \approx 26\%$ of democrats vote and 
  %   $810/1100 \approx 74\%$ of republicans vote.

  % \end{solution}

  \section{Jury Selection}

  \subsection{Azania vs. State of Indiana}

  pg. 272 of {\em Statistics in the Law}, by Joseph Kadane

  \begin{itemize*}
    \item Program randomly selected 14,364 voters to be assigned a random
      number.
    \item 364 extra because of rounding error.
    \item Jurors were selected alphabetically by township.  Program stopped
      after juror 10,000 instead of going all the way to 14,364.    
    \item All 4,364 excluded jurors were in Wayne Township residents.  87\% of
      Wayne Township was excluded
    \item Black people are 8.5\% of Allen County, and 75\% of the 8.5\% live in
      Wayne Township
  \end{itemize*}

  \begin{itemize*}
    \item Pool of x\% of population of each town in pool until target reached
    \item Select juries from pool
    \item Worked OK until population grew and pool was filled before W's were
      reached.
    \item Winthrop (check?) was 74\% black with 0 jurors in pool.
    \item Sentence overturned because jury wasn't representative of state
      population
  \end{itemize*}

  \subsection{Lockhart vs. McCree}
  Are juries for death penalty cases which exclude anti-death penalty jurors
  fair?  Supreme court says yes. See pg. 238.
\end{document}

