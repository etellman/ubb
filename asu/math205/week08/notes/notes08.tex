\documentclass[landscape]{exam}

\usepackage{2in1, lscape} 
\usepackage{units} 
\usepackage[fleqn]{amsmath}
\usepackage{float}
\usepackage{mdwlist}
\usepackage{booktabs}
\usepackage{caption}
\usepackage{fullpage}
\usepackage{enumerate}
\usepackage{graphicx}
\usepackage{parskip}

\printanswers{}

\everymath{\displaystyle}

\title{Statistics \\ Week Nine}
\date{\today}
\author{}

\begin{document}

  \maketitle
  \tableofcontents

  \part{Overview}

  \section{Statistical Inference vs. Exploratory Data Analysis}
  
  Exploratory:
  \begin{itemize*}
    \item given data, draw conclusions
    \item conclusions only apply to data you actually have
  \end{itemize*}

  Statistical Inference:
  \begin{itemize*}
    \item try to gather data to answer a question
    \item conclusions apply to more than just data actually gathered
    \item answers come with a statement of how confident we are that they are
      accurate
  \end{itemize*}

  \section{Sample/Population}

  \subsection{Notes}
  \begin{description}
    \item[population] entire group
    \item[sample] group you actually get data from
    \item[sample design] plan for getting a sample that represents population
  \end{description}

  Planning sample survey:
  \begin{itemize*}
    \item What is the population we want to find out about? 
    \item What do we want to find out from each observation? (What are the variables?)
  \end{itemize*}

  \subsection{Examples}

  \subsubsection{US Census}
  \begin{itemize*}
   \item variables: age, income, race, marital, status, number of children, etc.
   \item population: US
  \end{itemize*}

  \subsubsection{Election Polls}
  \begin{itemize*}
   \item variables: who are you going to vote for?
   \item population: probable voters (not just registered voters)
  \end{itemize*}

  \section{Control/Treatment}
  If you're trying to find out if something has an effect (new drug, drinking,
  smoking) the subjects are in two groups.  The ``control'' group doesn't
  drink, smoke, take the new medicine, etc., while the ``treatment'' group does
  drink, smoke, medicate, etc.

  \section{Longitudinal Studies} 
  Observe changes in the same sample over a long period of time.

  Framingham Heart Study:
  \begin{itemize*}
    \item 5209 adult residents of Framingham, MA from 1948 to now
    \item height, weight, blood pressure, education, etc.
    \item information on kids 
    \item 2000 academic articles since 1950, 1000 between 2000 and 2009 on
      smoking, cholesterol, heart disease, blood pressure, 
  \end{itemize*}

  With a longitudinal study you don't have to reconstruct the past history of
  the subjects. You don't need to find someone today and ask him what his blood
  pressure was when he was 19, for example.

  \part{Frequent Problems}

  \section{Credits}
  Most of this stuff is from {\em Naked Statistics}.

  \section{Selection Bias}
  \subsection{Notes}
  Sample is not representative of population.  Every member of the population
  doesn't have an equal chance of appearing in the sample.

  \subsection{Examples}
  \subsubsection{Republican Iowa straw poll}

  \begin{itemize*}
    \item August of year before presidential election
    \item Iowa only
    \item \$30 to vote
  \end{itemize*}

  \subsubsection{Convenience Sample}
  Sample that is easiest to gather:

  \begin{itemize*}
    \item ask your friends about their views on something to estimate US population's opinion.
    \item observe cars in your neighborhood to find most common car in US (my
      neighborhood has disproportionate number of SUVs)
    \item UBB students as sample of US prison population (WA may not be typical.
      UBB students may not be typical)
  \end{itemize*}

  \section{Self-Selection Bias}

  \subsection{Notes}
  If people assign themselves to either the control or treatment group, there
  may be something different about the subjects other than control/treatment.
  
  \subsection{Examples}

  \subsubsection{Self-Selecting Sample}
  \begin{itemize*}
    \item web poll
    \item call in radio poll
    \item Amazon ``Did you find what you were looking for?''
  \end{itemize*}

  % problems:

  % For any poll, if the population is the US population:
  % \begin{itemize*}
  %   \item People who know about the poll may not be typical (listeners to the
  %     show, visitors to the web site)
  %   \item People who actually respond may be even less typical (too much time on
  %     their hands, feel very strongly)
  % \end{itemize*}

  % \subsubsection{AA}
  % People who sign up for AA may not be representative of drug users.  Did they
  % stop drinking because of AA or did they sign up for AA because they were
  % ready to stop drinking?

  \subsubsection{Clofibrate}
  \begin{table}
    \centering
    \begin{tabular}{lrrrr}
      \toprule
      & \multicolumn{2}{c}{Clofibrate} & \multicolumn{2}{c}{Placebo} \\
                   & number & deaths & number & deaths \\
      \cmidrule(r){2-3} \cmidrule(r){4-5}    
      adherers     & 708    & 15\%   & 1813   & 15\% \\
      non-adherers & 357    & 25\%   & 882    & 28\% \\
      total group  & 1103   & 20\%   & 2789   & 21\% \\
      \bottomrule
    \end{tabular}
    \caption{Clofibrate study}\label{tab:clofibrate}
  \end{table}

  \subsubsection{Vitamin C}
  \begin{table}
    \centering
    \begin{tabular}{cll}
      \toprule
      group & prevention & therapy \\
      \midrule
      1     & placebo    & placebo \\
      2     & vitamin C  & placebo \\
      3     & placebo    & vitamin C \\
      4     & vitamin C  & vitamin C \\
    \end{tabular}
    \caption{Vitamin C study}\label{tab:vitamin.c}
  \end{table}

  \begin{itemize*}
    \item dropout rate was higher for groups 1--3 than 4
    \item easy to distinguish vitamin from placebo by taste
    \item among subjects who remained blind vitamin C had no effect
    \item among subjects who broke blind groups 2 and 4 had fewest colds and
      groups 3 and 4 had shortest colds
  \end{itemize*}


  \subsubsection{Smoking Study}
  A smoking study found:
  \begin{itemize*}
    \item people who had never smoked were slightly healthier than current
      smokers
    \item people who had recently stopped smoking were much less healthy than
      current smokers
  \end{itemize*}

  True or false: you shouldn't start smoking, but if you start you shouldn't
  stop.  Explain.

  \begin{solution}
    The people who just stopped probably stopped because their health was shot.
  \end{solution}

  \section{Publication Bias}
  \subsection{Notes}
  Every study has some possibility of error.  With publication bias, the
  researcher throws away the studies that yield the ``wrong'' result and
  publishes the few studies that yield the desired result. 

  \subsection{Example}

  Suppose study is 95\% likely to be correct.  1 out of 20 times the study
  returns the wrong result. To demonstrate a new drug is effective, a drug
  company just needs to do 20 studies, discard the 19 that say the drug does
  nothing, and publish study number 20.

  The press release says: ``With 95\% certainty, New Miracle Drug cures Cancer.''

  With Prozac, 94\% of the positive studies were published and 14\% of the
  non-positive studies were published.

  \section{Recall Bias}
  \subsection{Notes}

  Current events may affect how people remember the past.  

  \subsection{Example}

  In 1993 study of diet/breast cancer, people with breast cancer reported high
  fat diets earlier in life.  Cause was people struggling to think of
  explanation for cancer rather than actual high-fat diets.

  \begin{em}
    The diagnosis of breast cancer had not just changed a woman’s present and the
    future; it had altered her past. Women with breast cancer had (unconsciously)
    decided that a higher-fat diet was a likely predisposition for their disease
    and (unconsciously) recalled a high-fat diet.
  \end{em}

  Longitudinal study wouldn't be susceptible to this problem.

  \section{Survivorship Bias}
  \subsection{Notes}
  In a longitudinal study, the sample may change over time.

  \subsection{Examples}
  \subsubsection{High School Test Scores}
  GPA of high school students who haven't dropped out goes up.  Sophomores do
  better than freshman, juniors to better than sophomores, etc., because the bad
  students are the ones that drop out.

  \subsubsection{Mutual Fund}
  A mutual fund company can start 16 mutual funds.  By chance, 8 are above
  the index after year 1, 4 after year 2, and 2 after year 3.  Below average
  funds quietly shut down and above average funds heavily advertised.

  \section{Polling}
  \subsection{Non-Response}
  It's hard to get people to respond to a poll, particularly with caller ID\@.
  People who answer the phone and are willing to talk are probably different
  (retired, young, unemployed, bored) from people who don't.

  Ways to mitigate:
  \begin{itemize*}
    \item call repeatedly at all hours of the day to maximize chance of getting
      people who work at different hours

    \item get response from adult with most recent birthday rather than person
      who answered phone
  \end{itemize*}

  \subsection{Non-Representative Sample}
  See ``Convenience Sample'' and ``Selection Bias''.

  Phone polls only have a chance to reach people with phones, etc.

  \subsection{Biased Questions}
  Different ways of phrasing the same question can get wildly different
  responses.

  examples:
  \begin{itemize*}
    \item tax relief (good) vs.\ tax cut (less good)
    \item welfare (bad) vs.\ assistance (good)
    \item ``reasonable measures to keep us safe from terrorists'' vs.
      ``governmental monitoring of every phone call.''
  \end{itemize*}

  \part{Sampling}

  \section{Simple Random Sample}
  \begin{itemize*}
    \item make list of subjects and randomly select subset
    \item use table of random numbers in back of book
    \item subjects who respond may not be representative of population
  \end{itemize*}

  \section{Stratified Sample}
  \subsection{Notes}
  \begin{itemize*}
    \item Divide population into groups (age, race, income, etc.).
    \item Do SRS from each group.
    \item Helps to get appropriate representation from each group.
  \end{itemize*}

  \part{Miscellaneous Examples}
  \section{Breast Cancer Study}
  Health Insurance Plan of NY (HIP) study in 1963.

  Conclusions:
  \begin{itemize*}
    \item screening didn't affect diseases other than breast cancer
    \item poor less likely to accept screening
    \item most diseases other than breast cancer affect poor more than rich
  \end{itemize*}

  \begin{table}
    \centering
    \begin{tabular}{lrrrrr}
      & & \multicolumn{2}{c}{Breast Cancer} & \multicolumn{2}{c}{All Other} \\
                               \cmidrule(r){3-4} \cmidrule(r){5-6}    
                      &        & number & rate & number & rate \\
      Treatment group \\
      examined        & 20,200 & 23     & 1.1  & 428    & 21 \\
      refused         & 10,800 & 16     & 1.5  & 409    & 38 \\
      total           & 31,000 & 39     & 1.3  & 837    & 27 \\
    \midrule
      control group   & 31,000 & 63     & 2.0  & 879    & 28 \\
    \end{tabular}
    \caption{Breast cancer study: cause of death}\label{tab:breast.cancer}
  \end{table}

  \begin{itemize*}
    \item does screening save lives?
    \item why is rate for ``all other'' same for treatment and control groups?
    \item why is rate for all other causes higher for ``refused'' than
      ``examined'' groups?

    \item Breast cancer affects rich more than poor.  Which numbers confirm
      this?

      \begin{solution}
        Rich people tend to accept.  Income of groups, in order, is examined,
        control, refused, since control has mix of rich and poor and
        examined/refused are segregated by income.  Death rate among control
        group is higher than death rate among refused, and only thing different
        is income.
      \end{solution}

    \item death rate (all causes) among accepted is about half death rate among
      refused.  Did screening cut the death rate in half?  What explains the
      difference in death rate?

    \item Is comparing accepted vs.\ refused a fair comparison?  Is it biased for
      or against screening?

      \begin{solution}
        Since the people who accepted the screening tend to have more money, and
        money is positively correlated with breast cancer incidence, this
        comparison would be biased against screening.
      \end{solution}

    \item Someone claims that encouraging women to come in for breast cancer
      screening encourages their health conciousness so they live longer for
      that reason.  Is the data consistent or inconsistent with that claim?

      \begin{solution}
        Inconsistent.  The only thing affected by the screening is the breast
        cancer rate.
      \end{solution}

    \item In the first year of the study, 67 breast cancers were found in the
      ``examined'' group, 12 in the refused group, and 58 in the control group.
      True or false: screening causes breast cancer.

      \begin{solution}
        Comparing raw numbers doesn't make sense, since there are different
        numbers in each group.  It makes more sense to look at the percentages
        with cancer detected in each group:
        \begin{itemize*}
          \item examined: 0.33\%
          \item refused: 0.11\%
          \item control: 0.19\%
        \end{itemize*}

        More cancer is detected among the examined because:
        \begin{itemize*}
          \item they have more money, so they are more likely to get breast
            cancer
          \item the regular exams detect cases that go undetected in the other
            groups
        \end{itemize*}

        Control has more than refused because control is slightly richer.
      \end{solution}

  \end{itemize*}

  \section{Body Fat}
  Study found heavier kids tend to have more controlling mothers, leading to the
  headline ``Parents of Fat Kids Should Lighten Up.''

  \begin{itemize*}
    \item is there an association between the mother's behavior and the kids'
      weight?
    \item Is the controlling behavior necessarily causing the overweight kids,
      or might there be a confounding variable?
    \item Will following the headline's advice necessarily help the kids lose
      weight?
  \end{itemize*}

  \section{Voting}
  A hypothetical city is divided into 10 wards.  In each ward, the percentage of
  registered democrats who vote is greater than the percentage of registered
  republicans who vote.  True or false: the overall percentage of registered
  democrats who vote is greater than the percentage of republicans who vote?

  \begin{solution}
    False.  
    
    District 1 is democratic with low turnout and district 2 is republican with
    high turnout.  
    
    Specifically, district one has 200 out of 1000 democrats voting and 10 out
    of 100 republicans voting.  District two has 90 out of 100 democrats voting
    and 800 out of 1000 republicans voting.

    Overall, $290/1100 \approx 26\%$ of democrats vote and 
    $810/1100 \approx 74\%$ of republicans vote.

  \end{solution}
  \section{Jury Selection}

  \subsection{Azania vs. State of Indiana}

  pg. 273

  \begin{itemize*}
    \item Program randomly selected 14,364 voters to be assigned a random
      number.
    \item 364 extra because of rounding error.
    \item Jurors were selected alphabetically by township.  Program stopped
      after juror 10,000 instead of going all the way to 14,364.    
    \item All 4,364excluded jurors were in Wayne Township residents.  87\% of
      Wayne Township was excluded
    \item Black people are 8.5\% of Allen County, and 75\% of the 8.5\% live in
      Wayne Township
  \end{itemize*}

    \begin{itemize*}
    \item Pool of x\% of population of each town in pool until target reached
    \item Select juries from pool
    \item Worked OK until population grew and pool was filled before W's were
      reached.
    \item Winthrop (check?) was 74\% black with 0 jurors in pool.
    \item Sentence overturned because jury wasn't representative of state
      population
  \end{itemize*}

  \subsection{Lockhart vs. McCree}
  Are juries for death penalty cases which exclude anti-death penalty jurors
  fair?  Supreme court says yes. See pg. 238.
\end{document}

