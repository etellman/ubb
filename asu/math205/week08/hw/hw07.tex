% no answer key
% \documentclass[letterpaper]{exam}

% answer key
\documentclass[letterpaper, landscape]{exam}
\usepackage{2in1, lscape} 
\printanswers{}

\usepackage{units} 
\usepackage{xfrac} 
\usepackage[fleqn]{amsmath}
\usepackage{cancel}
\usepackage{float}
\usepackage{mdwlist}
\usepackage{booktabs}
\usepackage{cancel}
\usepackage{polynom}
\usepackage{caption}
\usepackage{fullpage}
\usepackage{comment}
\usepackage{enumerate}
\usepackage{graphicx}

\everymath{\displaystyle}

\title{Statistics \\ Homework Seven}
\date{\today}
\author{}

\begin{document}

  \maketitle

  \section{Homework}
    \begin{itemize*}
      \item read Chapter 8 
      \item exercises: 25, 27--28, 32--36, 38, 41--43, 47
    \end{itemize*}

  \ifprintanswers{}
    \begin{description}

      \item[25] The population is adults 18 and over and the sample is the 1027
        people actually interviewed.

      \item[27] If you number the students from 01 to to 40 and use Table B
        starting at line 117, you get students: 

        \fbox{ 38, 16, 32, 18, 37, 06, 23, 19, 03 and 25. }

      \item[28] The first five codes would be: codes 
        \fbox{288, 229, 131, 303, and 007}.

      \item[32] 
        \begin{description}
          \item[false] There will typically be around four 0's in each row but
            some rows will have more and some less.  The mean number of zeros
            per row will be 4.

          \item[true] Since there are 100 different pairs of digits, any
            particular pair has a 1/100 chance of appearing, including the pair
            00.

          \item[false] This pattern will come up once in a while but not very
            frequently.
        \end{description}

      \item[33]
        \begin{enumerate}[(a)]
          \item The population is all the residential households.

          \item 1169 out of 2000 or \fbox{ 58\% } didn't respond.

          \item People are likely to telescope the events, thinking that they've
            seen more movies in the last 12 months than they actually have.
        \end{enumerate}

      \item[34] The online poll only included people who were watching Lou
        Dobb's show.  Lou's audience is primarily composed of people who think
        issuing driver's licenses to illegal immigrants is a bad idea, so they
        don't represent the population as a whole.

      \item[35] 40,927 out of 45,956 or \fbox{ 89\% } didn't respond.

      \item[36] 
        \begin{enumerate}[(a)]
          \item If you assign each driver a number between 0001 and 5024 and use
            line 104 of Table B, the first 5 drivers are: 
          
          \fbox{ 1388, 0746, 0227, 4001 and 1858 }

          \item People generally don't want to admit they ran a red light, so
            the actual number is probably higher.
        \end{enumerate}

      \item[38] Everybody has a 1 in 10 chance of being interviewed.  This is a
        stratified sample since the students are divided into two groups
        according to age.

      \item[41] 
        \begin{enumerate}[(a)]
          \item The only thing you need Table B for in this example is the
          starting number.  After that, you take every 40th number, since
          $200/5 = 40$. 
          
          The numbers you get are: $\boxed{ \{35, 75, 115, 155, 195 \} }$

          \item There could be some repeating pattern in the data which shows up
          when you use a systematic sample.  For example, suppose you are
          sampling houses and there are 5 houses per block.  If you start with
          house 1 and select every 10th house, you will always be selecting a
          corner house and corner houses might not be representative of all the
          houses.
        \end{enumerate}

      \item[42]
        \begin{enumerate}[(a)]
          \item Houses that don't have land-line telephones or have unlisted
          numbers are omitted.

          \item By selecting random numbers, you can include the people with
          unlisted numbers and people with cell phones.  You'll still miss the
          people without phones, of course.

        \end{enumerate}

      \item[43]
        \begin{enumerate}[(a)]
          \item As described in question 42, random digit dialing means that the
            computer dials digits at random instead of going through a phone
            directory.

          \item Selecting the adult with the most recent birthday takes a SRS of
            the adults in the household.  If you don't do this, your sample
            might be biased because you mostly get the people who are most
            likely to answer the phone and this group might not be
            representative.
        \end{enumerate}

      \item[47]
        \begin{enumerate}[(a)]
          \item The sampling error for Hispanics is higher because there are
          fewer of them in the sample.  The smaller your sample, the less likely
          it is that the sample is representative of the population.

          \item There are even fewer Cubans than Hispanics, so the sample is so
          small that you can't draw any reliable conclusions about the
          population from the sample.
        \end{enumerate}

  \end{description}

  \else
    \vspace{11 cm}
    \begin{quote}
      \begin{em}
        For one man cannot do right in one department of life whilst he is
        occupied in doing wrong in any other department. Life is one indivisible
        whole.
      \end{em}
    \end{quote}
    \hspace{1 cm} --Mahatma Gandhi
  \fi

\end{document}

