% no answer key
\documentclass[letterpaper]{exam}

% answer key
% \documentclass[letterpaper, landscape]{exam}
% \usepackage{2in1, lscape} 
% \printanswers{}

\usepackage{units} 
\usepackage{xfrac} 
\usepackage[fleqn]{amsmath}
\usepackage{cancel}
\usepackage{float}
\usepackage{booktabs}
\usepackage{cancel}
\usepackage{polynom}
\usepackage{caption}
\usepackage{fullpage}
\usepackage{comment}
\usepackage{enumitem}
\usepackage{graphicx}
\usepackage{parskip}

\everymath{\displaystyle}

\title{Statistics \\ Chapter 20 Homework}
\date{\today}
\author{}

\begin{document}

  \maketitle

  \section{Homework}
  Chapter 20: 17--21, 23--28, 32, 35

  \ifprintanswers{}
    \section{Solutions}
    \begin{description}

      \item[17] 
        \begin{enumerate}[label = ({\alph*})]
          \item Yes---we have a large sample size.

          \item 
            \begin{align*}
              \hat{p}_{y} - \hat{p}_{o} & = 0.2087 \pm 0.0887 \\
                                        & = (0.1200, 0.2975) \\
            \end{align*}
        \end{enumerate}

      \item[18]
        \begin{enumerate}[label = ({\alph*})]
          \item There were only 7 positive responses from one of the schools and we
            need at least 10 from both.

          \item 9/139 and 29/145

          \item 
            \begin{align*}
              \hat{p}_{ut} - \hat{p}_{t} & = 0.1374 \pm 0.0760 \\
                                         & = (0.0614, 0.2134) \\
            \end{align*}
        \end{enumerate}

      \item[19]
        \begin{enumerate}[label = ({\alph*})]
          \item One of the samples has zero successes.

          \item 25/37 and 2/22

          \item 
            \begin{align*}
              \hat{p}_{a} - \hat{p}_{n} & = 0.6357 \pm 0.2379 \\
                                        & = (0.3978, 0.8737) \\
            \end{align*}
        \end{enumerate}

      \item[20]
        The hypotheses are:
        \begin{itemize}[label = {}, itemsep = 0pt]
          \item $H_0$: $\hat{p}_{ut} = \hat{p}_{t}$
          \item $H_a$: $\hat{p}_{ut} > \hat{p}_{t}$
        \end{itemize}

        The z-statistic is 3.5285. 
        
        $P < 0.0005$ and there is a statistically significant difference. 

      \item[21]
        The hypotheses are:
        \begin{itemize}[label = {}, itemsep = 0pt]
          \item $H_0$: $\hat{p}_{h} = \hat{p}_{nh}$
          \item $H_0$: $\hat{p}_{h} \ne \hat{p}_{nh}$
        \end{itemize}

        The z-statistic is 3.39.
        
        $P < 0.001$ and there is a statistically significant difference. 

      \item[23]
        \begin{align*}
          \hat{p}_{nh} - \hat{p}_{h} & = 0.1405 \pm 0.0774 \\
                                     & = (0.0631, 0.2179) \\
        \end{align*}
        
      \item[24] This should be a matched pair study rather than a proportion study.
        The two samples aren't independent since they are using the same facilities.

      \newpage

      \item[25]
        \begin{enumerate}[label = ({\alph*})]
          \item
            \begin{align*}
              p_m &= \sfrac{15}{106} \approx 0.1415 \\
              p_w &= \sfrac{7}{42} \approx 0.1667 \\
            \end{align*}
            With these proportions:

            The 2-sided hypotheses are:
            \begin{itemize}[label = {}, parsep = 0pt]
              \item $H_0$: $p_m = p_w$
              \item $H_a$: $p_m \ne p_w$
            \end{itemize}

            The z statistic is $z = 0.3879$ making $p > 0.5$. There isn't a
            statistically significant difference.

          \item With the new sample size: $z = 2.144$ and $0.02 < p < 0.04$

          \item 
            The original confidence interval is:
            \begin{align*}
              p_w - p_m & = 0.0252 \pm 0.1308 \\
                        & = (-0.1056, 0.1559) \\
            \end{align*}

            With the larger sample, the confidence interval is:
            \begin{align*}
              p_w - p_m & = 0.0252 \pm 0.0239 \\
                        & = (0.0013, 0.0490) \\
            \end{align*}

            The new confidence interval is smaller by a factor of $\sqrt{30}$. 
            
            We can be confident the women-run business fail more frequently, but the
            difference is somewhere between 0.1 and 4.9 percentage points.

        \end{enumerate}

      \item[26]
        We don't know which group will do better, so a 2-sided hypothesis makes
        sense: 
        \begin{itemize}[label = {}, parsep = 0pt]
          \item $H_0$: $p_u = p_r$
          \item $H_a$: $p_u \ne p_r$
        \end{itemize}

        The z-statistic is: $z = 2.9867$ and $0.001 < p < 0.0025$. 

        There is a statistically significant difference, $p < 0.0025$, between the
        two groups of students.

      \item[27]
        We don't know which group will do better, so a 2-sided hypothesis makes
        sense: 
        \begin{itemize}[label = {}, parsep = 0pt]
          \item $H_0$: $p_w = p_m$
          \item $H_a$: $p_w \ne p_m$
        \end{itemize}

        The z-statistic is: $z = 0.0245$ and $p > 0.5$. There's no evidence of a
        difference.

      \item[28]
        \begin{align*}
          p_u - p_r & = 0.2545 \pm 0.1373 \\
                    & = (0.1172, 0.3919) \\
        \end{align*}

      \item[32]
        \begin{enumerate}[label = {(\alph*)}]
          \item This is an observational study. The observers didn't randomly assign
            people to one of two groups. They just observed the drivers on the road.

          \item 
            We don't know which group will do better, so a 2-sided hypothesis makes
            sense: 
            \begin{itemize}[label = {}, parsep = 0pt]
              \item $H_0$: $p_b = p_{ny}$
              \item $H_a$: $p_b \ne p_{ny}$
            \end{itemize}

            The z-statistic is: $z = 5.025$ and $p < 0.001$. Either the seat belt law
            is working or the New Yorkers are buckling up for some other reason.
        \end{enumerate}

      \item[35]
        The 95\% confidence interval for the people who use a credit card is:
        \begin{align*}
          p & = 0.4948 \pm 0.0995 \\
            & = (0.3953, 0.5943) \\
        \end{align*}

        The difference in credit card use for planned vs.\ impulse purchases is:
        \begin{align*}
          p_{p} - p_i & = 0.1109 \pm 0.2114 \\
                       & = (-0.1004, 0.3223) \\
        \end{align*}

        The 2-sided hypotheses are:
        \begin{itemize}[label = {}, parsep = 0pt]
          \item $H_0$: $p_p = p_i$
          \item $H_a$: $p_p \ne p_i$
        \end{itemize}

        $z = 1.0192$ which means $0.2 < z < 0.3$ and the difference is not
        statistically significant.

  \end{description}

  \else
    \vspace{12 cm}
    \begin{quote}
      \begin{em}
        The degree of civilization in a society can be judged by entering its prisons. 
      \end{em}
    \end{quote}
    \hspace{1 cm}--Fyodor Dostoyevsky
  \fi

\end{document}

